\documentclass[a4paper,11pt]{article}
\usepackage[utf8]{inputenc}
\usepackage[english,francais]{babel}
\usepackage{amsmath}
\usepackage{bm}
\usepackage{amsfonts}
\usepackage{graphicx}
\usepackage{geometry}
\usepackage[utf8]{inputenc}
\usepackage{titlesec}
\usepackage{enumitem}
\usepackage{tikz}
\usepackage{calc}
\usepackage{caption}
\usepackage{easy-todo}
\usepackage{comment}
\usepackage{lipsum}
\usepackage{bbm}
\usepackage{fancyhdr}
\usepackage{tikz}
\usepackage{pdfpages}
\newcommand{\Var}{\mathbb{V}\text{ar}}
\newcommand{\Ex}{\mathbb{E}}
\newcommand{\Prob}{\mathbb{P}}
\DeclareMathOperator*{\argmin}{arg\,min \,}
\DeclareMathOperator*{\argmax}{arg\,max \,}
\usepackage{hyperref}
\usepackage{booktabs}

\pagestyle{fancy}
% \rhead{Victor Trappler}
% \lhead{}
% \graphicspath{{./Figures/}}
\begin{document}


\title{Bayesian approach of the parameter inverse problem under uncertainties}

\author{Victor Trappler \\[1cm]
  \begin{tabular}{lr}
    Directeurs de Thèse: & Arthur VIDARD (Inria) \\
                        & Élise ARNAUD (UGA)\\
                        & Laurent DEBREU (Inria)
  \end{tabular}
}

\maketitle
% \vspace{3cm}
% \includegraphics[scale=0.3]{/home/victor/logo_UGA}
% \hfill
% \includegraphics[scale=0.3]{/home/victor/ljk}
% \hfill
% \includegraphics[scale=0.3]{/home/victor/inria}
% \thispagestyle{empty} 
% \clearpage
\tableofcontents
\section{(Joint) Posterior formulation}
\subsection{Priors}
\begin{align*}
  K \sim \mathcal{U}(\mathbb{K}), \quad p(k) \\
  U \sim \mathcal{U}(\mathbb{U}), \quad p(u)
\end{align*}
\subsection{Likelihood model}
\begin{align*}
  p(y \mid k, u, \sigma^2) &= \frac{1}{\sqrt{2\pi}\sigma}\exp\left[-\frac{1}{2\sigma^2}SS(k,u)\right] \\
                         &= \frac{1}{\sqrt{2\pi}\sigma}\exp\left[-\frac{1}{2\sigma^2} \|\mathcal{M}(k,u) - y \|^2_{\Sigma}\right]
\end{align*}
Now to Bayes' theorem
\begin{align*}
  p(k,u \mid y,\sigma^2) = \frac{p(y \mid k, u, \sigma^2) p(k,u)}{\iint_{\mathbb{K}\times\mathbb{U}}p(y \mid k, u, \sigma^2) p(k,u) \, \mathrm{d}(k,u)}
\end{align*}
Let us assume an hyperprior for $\sigma^2$: $p(\sigma^2)$

\end{document}

%%% Local Variables:
%%% mode: latex
%%% TeX-master: t
%%% End:
