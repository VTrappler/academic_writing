%%%%%%%%%%%%%%%%%%%%%%%%%%%%%%%%%%%%%%%%%
% a0poster Portrait Poster
% LaTeX Template
% Version 1.0 (22/06/13)
%
% The a0poster class was created by:
% Gerlinde Kettl and Matthias Weiser (tex@kettl.de)
% 
% This template has been downloaded from:
% http://www.LaTeXTemplates.com
%
% License:
% CC BY-NC-SA 3.0 (http://creativecommons.org/licenses/by-nc-sa/3.0/)
%
%%%%%%%%%%%%%%%%%%%%%%%%%%%%%%%%%%%%%%%%%

%----------------------------------------------------------------------------------------
%	PACKAGES AND OTHER DOCUMENT CONFIGURATIONS
%----------------------------------------------------------------------------------------

\documentclass[a0,portrait, 30pt]{a0poster}

\usepackage{multicol} % This is so we can have multiple columns of text side-by-side
\columnsep=100pt % This is the amount of white space between the columns in the poster
\columnseprule=3pt % This is the thickness of the black line between the columns in the poster

\usepackage[svgnames]{xcolor} % Specify colors by their 'svgnames', for a full list of all colors available see here: http://www.latextemplates.com/svgnames-colors
% \usepackage{times} % Use the times font
\usepackage{palatino} % Uncomment to use the Palatino font 
\usepackage[utf8]{inputenc}
\usepackage{graphicx} % Required for including images
\graphicspath{{../Slides/Figures/}} % Location of the graphics files
\usepackage{booktabs} % Top and bottom rules for table
\usepackage[font=small,labelfont=bf]{caption} % Required for specifying captions to tables and figures
\usepackage{subfig}
\usepackage{pgfplots}
\pgfplotsset{compat=newest}
%\usepackage{booktabs}
\usepackage{amsfonts, amsmath, amsthm, amssymb} % For math fonts, symbols and environments
\usepackage{wrapfig} % Allows wrapping text around tables and figures
\usepackage{pgfplots}
\usepackage{bm}
\newcommand{\Ex}{\mathbb{E}}
\newcommand{\Var}{\mathbb{V}\mathrm{ar}}
\newcommand{\Prob}{\mathbb{P}}
\DeclareMathOperator*{\argmin}{arg\,min}
\DeclareMathOperator*{\argmax}{arg\,max}
\newcommand{\Cov}{\textsf{Cov}}
\newcommand{\tra}{\mathrm{tr}}
\newcommand{\yobs}{\bm{y}^{\mathrm{obs}}}
\newcommand{\kest}{\hat{\bm{k}}}
\DeclareMathOperator*{\KL}{\textsf{KL}}
\begin{document}
\definecolor{blkcol}{HTML}{E1E1EA}
\definecolor{blkcol2}{RGB}{209, 224, 224}
%----------------------------------------------------------------------------------------
%	POSTER HEADER 
%---------------------------------------------------------------------------------------

% The header is divided into two boxes:
% The first is 75% wide and houses the title, subtitle, names, university/organization and contact information
% The second is 25% wide and houses a logo for your university/organization or a photo of you
% The widths of these boxes can be easily edited to accommodate your content as you see fit
{\veryHuge \color{NavyBlue} \textbf{Parameter estimation in the presence of uncertainties} \color{Black} \\\Huge\textit{Robust estimation of bottom friction}\\[2cm]} % Subtitle}

\begin{minipage}[b]{0.5\linewidth}

\huge \textbf{Victor Trappler\\ Élise Arnaud, Laurent Debreu, Arthur Vidard}\\[0.5cm] % Author(s)
\huge Inria / LJK, Univ. Grenoble Alpes\\[0.4cm] % University/organization
\Large \texttt{victor.trappler@univ-grenoble-alpes.fr}
\end{minipage}
%
\begin{minipage}[t]{0.5\linewidth}
\includegraphics[width = 0.45\textwidth]{../Slides/Figures/INRIA_SCIENTIFIQUE_UK_CMJN}
\includegraphics[width= 0.45\textwidth]{../Slides/Figures/ljk}
\end{minipage}

\vspace{1cm} % A bit of extra whitespace between the header and poster content

%----------------------------------------------------------------------------------------

\begin{minipage}{0.48\linewidth} % This is how many columns your poster will be broken into, a portrait poster is generally split into 2 columns

%----------------------------------------------------------------------------------------
%	ABSTRACT
%----------------------------------------------------------------------------------------

\color{Navy} % Navy color for the abstract

\begin{abstract}
Many physical phenomena are modelled numerically in order to better understand and/or to predict their behaviour. However, some complex and small scale phenomena can not be fully represented in the models. The introduction of ad-hoc correcting terms is usually the solution to represent these unresolved processes, but those need to be properly estimated.
A good example of this type of problem is the estimation of bottom friction parameters of the ocean floor. This task is further complicated by the presence of uncertainties in certain other characteristics linking the bottom and the surface (eg boundary conditions).
  Classical methods of parameter estimation usually imply the minimisation of an objective function, that measures the error between some observations and the results obtained by a numerical model. The optimum is directly dependent on the fixed nominal value given to the uncertain parameter ; and therefore may not be relevant in other conditions.
  Strategies taking into account those uncertainties will be presented and applied on an academic model of a coastal area, in order to find an optimal value in a robust sense.
\end{abstract}



%----------------------------------------------------------------------------------------
%	INTRODUCTION
%----------------------------------------------------------------------------------------

% \color{SaddleBrown} % SaddleBrown color for the introduction

\section*{Introduction}
\large
\begin{itemize}
\item Numerical models of physical systems cannot represent fully the reality


  $\rightarrow$ Subgrid phenomena need to be parametrized
\item How to choose the parameter value in a robust way ?
\end{itemize}
\subsection*{Classical deterministic inverse problem}
\begin{center}
  % \resizebox{\linewidth}{!}{\input{./inverse_problem.tikz}}
  \input{./inverse_problem.tikz}
\end{center}
\subsection*{Introducing randomness}
% \resizebox{\linewidth}{!}{\input{./inverse_problem.tikz}}
\begin{center}
  \input{./inverse_problem_rnd.tikz}
\end{center}
%----------------------------------------------------------------------------------------
%	OBJECTIVES
%----------------------------------------------------------------------------------------

\color{DarkSlateGray} % DarkSlateGray color for the rest of the content

\section*{Main Objectives}

\begin{enumerate}
\item Define a relevant definition of robustness
\item Nullam at mi nisl. Vestibulum est purus, ultricies cursus volutpat sit amet, vestibulum eu.
\item Praesent tortor libero, vulputate quis elementum a, iaculis.
\item Phasellus a quam mauris, non varius mauris. Fusce tristique, enim tempor varius porta, elit purus commodo velit, pretium mattis ligula nisl nec ante.
\item Ut adipiscing accumsan sapien, sit amet pretium.
\item Estibulum est purus, ultricies cursus volutpat
\item Nullam at mi nisl. Vestibulum est purus, ultricies cursus volutpat sit amet, vestibulum eu.
\item Praesent tortor libero, vulputate quis elementum a, iaculis.
\end{enumerate}

%----------------------------------------------------------------------------------------
%	MATERIALS AND METHODS
%----------------------------------------------------------------------------------------

\section*{Case}
\begin{itemize}
  \item (Deterministic) Computer code: Maps bottom friction to the observation of the sea surface height
\begin{equation*}
  \begin{array}{rcl}
    M: \mathcal{K} \times \mathcal{U} & \longrightarrow & \mathcal{Y} \\
    (\bm{k},\bm{u}) & \longmapsto & M(\bm{k},\bm{u})
    \end{array} 
  \end{equation*}
\item Uncertainty quantification (UQ): Model the uncertainties with a random variable
  \begin{equation*}
    \bm{U} \quad \text{has density} \quad \bm{u} \mapsto p_U(\bm{u})
  \end{equation*}
\item Compare observations with output of the model, using a precision matrix $\bm{\Sigma}^{-1}$:
  \begin{equation*}
   \begin{array}{rcl}
    j: \mathcal{K} \times \mathcal{U} & \longrightarrow & \mathcal{Y} \\
    (\bm{k},\bm{u}) & \longmapsto & j(\bm{k},\bm{u}) = \|M(\bm{k},\bm{u}) - \yobs \|^2_{\bm{\Sigma}^{-1}}
    \end{array} 
  \end{equation*}
\end{itemize}

  Two alternatives:
  \begin{itemize}
  \item  Consider the random variable indexed by $\bm{k} \in \mathcal{K}$:
    \begin{equation*}
      j(\bm{k},\bm{U}) \quad \text{random variable with sample space } \mathbb{R}^+ 
    \end{equation*}
    \item 
    \end{itemize}

    \section*{Bayesian framework}
%----------------------------------------------------------------------------------------
%	RESULTS 
%----------------------------------------------------------------------------------------
\end{minipage}
\begin{minipage}{0.49\linewidth}
  \large
\section*{Results}

Donec faucibus purus at tortor egestas eu fermentum dolor facilisis. Maecenas tempor dui eu neque fringilla rutrum. Mauris \emph{lobortis} nisl accumsan. Aenean vitae risus ante.
%
\begin{wraptable}{l}{12cm} % Left or right alignment is specified in the first bracket, the width of the table is in the second
\begin{tabular}{l l l}
\toprule
\textbf{Treatments} & \textbf{Response 1} & \textbf{Response 2}\\
\midrule
Treatment 1 & 0.0003262 & 0.562 \\
Treatment 2 & 0.0015681 & 0.910 \\
Treatment 3 & 0.0009271 & 0.296 \\
\bottomrule
\end{tabular}
\captionof{table}{\color{Green} Table caption}
\end{wraptable}
%
Phasellus imperdiet, tortor vitae congue bibendum, felis enim sagittis lorem, et volutpat ante orci sagittis mi. Morbi rutrum laoreet semper. Morbi accumsan enim nec tortor consectetur non commodo nisi sollicitudin. Proin sollicitudin. Pellentesque eget orci eros. Fusce ultricies, tellus et pellentesque fringilla, ante massa luctus libero, quis tristique purus urna nec nibh.

Nulla ut porttitor enim. Suspendisse venenatis dui eget eros gravida tempor. Mauris feugiat elit et augue placerat ultrices. Morbi accumsan enim nec tortor consectetur non commodo. Pellentesque condimentum dui. Etiam sagittis purus non tellus tempor volutpat. Donec et dui non massa tristique adipiscing. Quisque vestibulum eros eu. Phasellus imperdiet, tortor vitae congue bibendum, felis enim sagittis lorem, et volutpat ante orci sagittis mi. Morbi rutrum laoreet semper. Morbi accumsan enim nec tortor consectetur non commodo nisi sollicitudin.

\begin{center}\vspace{1cm}
%\includegraphics[width=0.8\linewidth]{placeholder}
\captionof{figure}{\color{Green} Figure caption}
\end{center}\vspace{1cm}

In hac habitasse platea dictumst. Etiam placerat, risus ac.

Adipiscing lectus in magna blandit:

\begin{center}\vspace{1cm}
\begin{tabular}{l l l l}
\toprule
\textbf{Treatments} & \textbf{Response 1} & \textbf{Response 2} \\
\midrule
Treatment 1 & 0.0003262 & 0.562 \\
Treatment 2 & 0.0015681 & 0.910 \\
Treatment 3 & 0.0009271 & 0.296 \\
\bottomrule
\end{tabular}
\captionof{table}{\color{Green} Table caption}
\end{center}\vspace{1cm}

Vivamus sed nibh ac metus tristique tristique a vitae ante. Sed lobortis mi ut arcu fringilla et adipiscing ligula rutrum. Aenean turpis velit, placerat eget tincidunt nec, ornare in nisl. In placerat.

\begin{center}\vspace{1cm}
%\includegraphics[width=0.8\linewidth]{placeholder}
\captionof{figure}{\color{Green} Figure caption}
\end{center}\vspace{1cm}

%----------------------------------------------------------------------------------------
%	CONCLUSIONS
%----------------------------------------------------------------------------------------

\color{SaddleBrown} % SaddleBrown color for the conclusions to make them stand out

\section*{Conclusions}

\begin{itemize}
\item Pellentesque eget orci eros. Fusce ultricies, tellus et pellentesque fringilla, ante massa luctus libero, quis tristique purus urna nec nibh. Phasellus fermentum rutrum elementum. Nam quis justo lectus.
\item Vestibulum sem ante, hendrerit a gravida ac, blandit quis magna.
\item Donec sem metus, facilisis at condimentum eget, vehicula ut massa. Morbi consequat, diam sed convallis tincidunt, arcu nunc.
\item Nunc at convallis urna. isus ante. Pellentesque condimentum dui. Etiam sagittis purus non tellus tempor volutpat. Donec et dui non massa tristique adipiscing.
\end{itemize}

\color{DarkSlateGray} % Set the color back to DarkSlateGray for the rest of the content

%----------------------------------------------------------------------------------------
%	FORTHCOMING RESEARCH
%----------------------------------------------------------------------------------------

\section*{Forthcoming Research}

Vivamus molestie, risus tempor vehicula mattis, libero arcu volutpat purus, sed blandit sem nibh eget turpis. Maecenas rutrum dui blandit lorem vulputate gravida. Praesent venenatis mi vel lorem tempor at varius diam sagittis. Nam eu leo id turpis interdum luctus a sed augue. Nam tellus.

 %----------------------------------------------------------------------------------------
%	REFERENCES
%----------------------------------------------------------------------------------------

\nocite{*} % Print all references regardless of whether they were cited in the poster or not
\bibliographystyle{plain} % Plain referencing style
\bibliography{sample} % Use the example bibliography file sample.bib

%----------------------------------------------------------------------------------------
%	ACKNOWLEDGEMENTS
%----------------------------------------------------------------------------------------

\section*{Acknowledgements}

Etiam fermentum, arcu ut gravida fringilla, dolor arcu laoreet justo, ut imperdiet urna arcu a arcu. Donec nec ante a dui tempus consectetur. Cras nisi turpis, dapibus sit amet mattis sed, laoreet.

%----------------------------------------------------------------------------------------

\end{minipage}
\end{document}
%%% Local Variables:
%%% mode: latex
%%% TeX-master: t
%%% End:
