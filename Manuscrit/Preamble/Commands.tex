%%%%%%%%%%%%%%%%%%%%%%%%%%%%%%%%%%%%%
% Miscs Cmds
%%%%%%%%%%%%%%%%%%%%%%%%%%%%%%%%%%%%%
\newcommand{\quadn}{\!\!\!\!}
\newcommand{\todoinl}{\todo[inline]{}}
\newcommand{\paragraphTitle}[1]{
%\begin{center}
%\vspace{-5mm}
%\item \paragraph{#1}~\\
%\end{center}
\subsubsection*{\centering{#1}}
}

%%%%%%%%%%%%%%%%%%%%%%%%%%%%%%%%%%%%%
%% Référencer des tables, figures
%%%%%%%%%%%%%%%%%%%%%%%%%%%%%%%%%%%%%
% \newcommand\figref{Fig.~\ref}
% \newcommand\tabref{Table~\ref}
% \newcommand\secref{Sec.~\ref}
\renewcommand\eqref{\cref}

%%%%%%%%%%%%%%%%%%%%%%%%%%%%%%%%%%%%%
%% Norme
%%%%%%%%%%%%%%%%%%%%%%%%%%%%%%%%%%%%%
\newcommand{\norm}[1]{\left\lVert #1 \right\rVert} 
% \newcommand{\innerprod}[2]{\left\langle #1, #2 \right\rangle}
\newcommand{\innerprod}[2]{#1^T #2}

\newcommand{\pfrac}[2]{\frac{\partial #1}{\partial #2}}
%%%%%%%%%%%%%%%%%%%%%%%%%%%%%%%%%%%%%
%% Tableaux
%%%%%%%%%%%%%%%%%%%%%%%%%%%%%%%%%%%%%
% Laisser de l'espace avant ou après une ligne horizontale 
\newcommand\Tstrut{\rule{0pt}{2.6ex}}         % = `top' strut
\newcommand\Bstrut{\rule[-0.9ex]{0pt}{0pt}}   % = `bottom' strut
% Utilisation : 
%	\begin{tabular}{|l|}
%		\hline
%		TEXT\Tstrut\Bstrut\\ % top *and* bottom struts
%		\hline
%		TEXT \Tstrut\\       % top strut only
%		TEXT \Bstrut\\       % bottom strut only
%		\hline
%	\end{tabular}

% Epaisseur des traits de toprule et bottomrule
\setlength\heavyrulewidth{0.25ex}

% Différents types de colonnes à mettre dans tabular
\newcolumntype{C}{>{\centering\arraybackslash}X} % Centered
\newcolumntype{D}{>{\RaggedLeft}X} % Aligné à droite

% Ecrire plusieurs ligne dans une cell de tableau, en précisant où on veut les mettre
\newcommand{\specialcell}[2][c]{ \begin{tabular}[#1]{@{}c@{}}#2\end{tabular}} 


%%%%%%%%%%%%%%%%%%%%%%%%%%%%%%%%%%%%%%%%%%%%%%%%%%
%% 				 SAUTER DES PAGES				%%
%%%%%%%%%%%%%%%%%%%%%%%%%%%%%%%%%%%%%%%%%%%%%%%%%%
% Force que la suite de cette commande soit écrite sur une page impaire (à droite), et qu'au moins une page complètement blanche la précède (donc 2 si nécessaire pour être à droite)
% A utiliser avant un nouveau chapitre par exemple
\newcommand{\fullclear}{\clearpage\phantom{h}\pagestyle{empty}\newpage\cleardoublepage\pagestyle{fancy}}
% La page sera écrite à gauche
\newcommand*\cleartoleftpage{%
	\clearpage
	\ifodd\value{page}\phantom{h}\pagestyle{empty}\newpage\fi
}
\newcommand{\specialchapter}[1]{\fullclear\chapter{#1}}
\newcommand{\specialchapterstar}[1]{\fullclear\chapter*{#1}}


  
%%%%%%%%%%%%%%%%%%%%%%%%%%%%%%%%%%%%%
%% Mise en page
%%%%%%%%%%%%%%%%%%%%%%%%%%%%%%%%%%%%%
% Dans le cas où on utilise la mise en page Conny de fncychap
% Plutôt que d'avoir "chapitre 1 -- Intro", on veut que Intro soit directement écrit entre les lignes
% Pour la page de garde du chapitre, 
% On utilise cette commande (peut aussi être utile pour conclusion, annexes)
\makeatletter
	\newcommand\TitleBtwLines{%% 
		
		\ChTitleVar{\centering\Huge\rm\bfseries}
		\renewcommand{\DOCH}{%
			\mghrulefill{3\RW}\par\nobreak
			\vskip -0.5\baselineskip
			\mghrulefill{\RW}\par\nobreak
			
		}
		\renewcommand{\DOTI}[1]{%
			\CTV\FmTi{##1}\par\nobreak
			\vskip -0.5\baselineskip
			\mghrulefill{\RW}\par\nobreak
			\vskip 60\p@
		}
		\renewcommand{\DOTIS}[1]{%
			\mghrulefill{\RW}\par\nobreak
			\CTV\FmTi{##1}\par\nobreak
			\vskip 60\p@
		}
	
	} 
\makeatother
	
% Ligne avec 3 étoiles au milieu -> Très bien pour séparer des paragraphes
\newcommand*{\etoile}{
	\begin{center}
		* \hspace{2ex}* \hspace{2ex}*
	\end{center}
}


%================================================
%% Faciliter l'écriture
%================================================
\DeclareMathOperator*{\argmin}{arg\,min \,}
\DeclareMathOperator*{\argmax}{arg\,max \,}
\newcommand{\Ex}{\mathbb{E}}
\newcommand{\Prob}{\mathbb{P}}
\newcommand{\ProbGP}{\mathcal{P}}
\newcommand{\Var}{\mathbb{V}\mathrm{ar}}
\newcommand{\Cov}{\mathbb{C}\mathrm{ov}}
\newcommand{\Kspace}{\mathbb{K}}
\newcommand{\Uspace}{\mathbb{U}}
\newcommand{\Xspace}{\mathbb{X}}
\newcommand{\Yspace}{\mathbb{Y}}
\newcommand{\estimtxt}[2]{\hat{#1}_{\mathrm{#2}}}
\newcommand{\DKL}[2]{D_{\mathrm{KL}}\left(#1 \| #2\right)}
\newcommand{\kk}{k}
\newcommand{\uu}{u}
\newcommand{\UU}{U}
%%%%%%%%%%%%%%%%%%%%%%%%%%%%%%%%%%%%% 
%% TikZ Cmds
%%%%%%%%%%%%%%%%%%%%%%%%%%%%%%%%%%%%%
\newcommand{\tikzmark}[1]{\tikz[overlay,remember picture] \node (#1) {};}
\newcommand{\tikzdrawbox}[3][(0pt,0pt)]{%
    \tikz[overlay,remember picture]{
    \draw[#3]
      ($(left#2)+(-0.3em,0.9em) + #1$) rectangle
      ($(right#2)+(0.2em,-0.4em) - #1$);}
}


%%%%%%%%%%%%%%%%%%%%%%%%%%%%%%%%%%%%%
%% -- Package TODO enrichi
%%%%%%%%%%%%%%%%%%%%%%%%%%%%%%%%%%%%%
% Pavé coloré sur toute la ligne pour signaler ce qu'on doit faire
\newcommand{\todol}[1]{\todo[linecolor=red,backgroundcolor=red!25,bordercolor=red,inline]{!! #1 !!}}
\newcommand{\unsure}[1]{\todo[linecolor=yellow,backgroundcolor=yellow!25,bordercolor=yellow,inline]{?? #1 ??}}
\newcommand{\change}[1]{\todo[linecolor=blue,backgroundcolor=blue!25,bordercolor=blue,inline]{#1}}
\newcommand{\info}[1]{\todo[linecolor=OliveGreen,backgroundcolor=OliveGreen!25,bordercolor=OliveGreen,inline]{#1}}
\newcommand{\improvement}[1]{\todo[linecolor=Plum,backgroundcolor=Plum!25,bordercolor=Plum,inline]{#1}}
\newcommand{\thiswillnotshow}[1]{\todo[disable,inline]{#1}}
\newcommand{\reformuler}{\change{Reformuler}}

% Pavé coloré petit dans le texte pour les citations à ajouter
\newcommand{\aciter}[1]{\todo[linecolor=violet,backgroundcolor=violet!25,bordercolor=violet, inline]{~ #1 ~}}

% Texte coloré dans les paragraphes, pour signaler que c'est des infos/des formulations à vérifier
\newcommand{\tocheck}[1]{{\color{red}{#1}} \addcontentsline{tdo}{todo}{#1}}

% Note to self : n'apparaît pas dans la liste des todos à faire, c'est pour avoir des infos complémentaires pour soi qui apparaissent dans le pdf
\newcommand{\notetoself}[1]{\todo[linecolor=orange,backgroundcolor=orange!25,bordercolor=orange,nolist,inline]{Note to self: #1}}

%% Liste de todos à cocher
\newlist{todolist}{itemize}{2}
\setlist[todolist]{label=$\square$}
\usepackage{pifont}
\newcommand{\cmark}{\ding{51}}%
\newcommand{\xmark}{\ding{55}}%
\newcommand{\done}{\rlap{$\square$}{\raisebox{2pt}{\large\hspace{1pt}\cmark}}%
	\hspace{-2.5pt}}
\newcommand{\wontfix}{\rlap{$\square$}{\large\hspace{1pt}\xmark}}
