\documentclass[11pt, oneside, a4paper]{book} % one-sided par défaut. Si on veut du twosided (pour la jolie version fnale) aller dans Config.txt et changer CfgPrintedVersion en true

\usepackage{etoolbox}
\newbool{CfgPrintedVersion}

%%%%%%%%%%%%%%%%%%%%%%%%%%%%%%%%%%%%%%%%%%%%%%%%%%
%% 					PACKAGE SUBFILE				%%
%%%%%%%%%%%%%%%%%%%%%%%%%%%%%%%%%%%%%%%%%%%%%%%%%%
%% => Permettre de compiler un chapitre seulement à la fois
\usepackage{subfiles}
\newcommand{\subfileLocal}[1]{#1}
\newcommand{\subfileGlobal}[1]{}
\newcommand{\RootDir}[1]{#1}
\usepackage{xr}

%%%%%%%%%%%%%%%%%%%%%%%%%%%%
%% Config keys
%% Document main infos
\newcommand{\ThesisTitle}{Parameter control in the presence of uncertainties}
\newcommand{\ThesisAuthor}{Victor Trappler}
\newcommand{\ThesisSubject}{This thesis deals with the problem of calibration of a numerical model under uncertainty, by introducing a new criterion based on the regret.}

%% Set true for a paper printed version (twosided document)
%% and false for a numeric version (onesided document)
\setbool{CfgPrintedVersion}{true} 


%%%%%%%%%%%%%%%%%%%%%%%%%%%%%%%%%%%%%%%%%%%%%%%%%%
%% 		PACKAGE FONT LANGUAGE & ENCODING		%%
%%%%%%%%%%%%%%%%%%%%%%%%%%%%%%%%%%%%%%%%%%%%%%%%%%
%\usepackage[LGR]{fontenc}

\usepackage[utf8]{inputenc}
\usepackage[T1]{fontenc}
\usepackage{natbib}
%\usepackage{cmap}
%\usepackage{ucs}
\usepackage[english]{babel} % if babel greek : probleme avec la liste de nomenclature
\usepackage{alphabeta}
%\frenchbsetup{AutoSpacePunctuation=false}
\usepackage[babel=true]{microtype}
\microtypecontext{kerning=english} 	% Set microtype to french


%%%%%%%%%%%%%%%%%%%%%%%%%%%%%%%%%%%%%%%%%%%%%%%%%%
%% 					PACKAGE GEOMETRY			%%
%%%%%%%%%%%%%%%%%%%%%%%%%%%%%%%%%%%%%%%%%%%%%%%%%%
%\ifbool{CfgPrintedVersion}{
%	\newcommand{\CfgBookLayout}{true}
%}{
%	\newcommand{\CfgBookLayout}{false}
%}
\usepackage[
	twoside=true, % always true for page numbering consistency
	left=32mm,
	right=32mm,
	top=40mm,
	bottom=35mm,
	headheight=18.1pt
]{geometry}


%%%%%%%%%%%%%%%%%%%%%%%%%%%%%%%%%%%%%%%%%%%%%%%%%%
%% 					PACKAGES MATHS				%%
%%%%%%%%%%%%%%%%%%%%%%%%%%%%%%%%%%%%%%%%%%%%%%%%%%
\usepackage{amsmath}
\usepackage{amssymb}
\usepackage{amsthm}
\usepackage{mathrsfs}
\usepackage{bbm}
\usepackage{interval}

%%%%%%%%%%%%%%%%%%%%%%%%%%%%%%%%%%%%%%%%%%%%%%%%%%
%% 				PACKAGES GRAPHIQUES				%%
%%%%%%%%%%%%%%%%%%%%%%%%%%%%%%%%%%%%%%%%%%%%%%%%%%
\usepackage{array}
\usepackage{float}
\usepackage{xcolor}
\usepackage{colortbl}
\usepackage{wrapfig}
\usepackage{caption}
\usepackage{subcaption}
\usepackage{graphicx} % Graphics
\usepackage{rotating}


%%%%%%%%%%%%%%%%%%%%%%%%%%%%%%%%%%%%%%%%%%%%%%%%%%
%% 				PACKAGES POLICES				%%
%%%%%%%%%%%%%%%%%%%%%%%%%%%%%%%%%%%%%%%%%%%%%%%%%%
%% Examples:
%% Font 1
%\usepackage{fouriernc}
\usepackage[scaled=0.875]{helvet}
%\usepackage{courier}
%
%% Font 2
% \usepackage{palatino} % Text font
%\usepackage{xcharter}
%
%% Font 3
%\usepackage{newpxtext,newpxmath}
%
%% Font 4
%\usepackage[osf,sc]{mathpazo}
%
%%
% \usepackage{lmodern}
%\usepackage{palatino}
%\usepackage{newpxmath}


%%%%%%%%%%%%%%%%%%%%%%%%%%%%%%%%%%%%%%%%%%%%%%%%%%
%% 				PACKAGE LIENS REFERENCES		%%
%%%%%%%%%%%%%%%%%%%%%%%%%%%%%%%%%%%%%%%%%%%%%%%%%%
\usepackage{hyperref}
\hypersetup{colorlinks, breaklinks,
	urlcolor  = cfgUrlColor,
	linkcolor = cfgLinkColor,
	citecolor = cfgCiteColor,
	pdftitle    = {\ThesisTitle},
	pdfauthor   = {\ThesisAuthor},
	pdfsubject  = {\ThesisSubject},
	%pdfkeywords = {keyword1, key2, key3},
}
\urlstyle{same}
\usepackage[capitalize, nameinlink]{cleveref}


\usepackage{setspace}

%%%%%%%%%%%%%%%%%%%%%%%%%%%%%%%%%%%%%%%%%%%%%%%%%%
%% 					PACKAGE TABLEAUX			%%
%%%%%%%%%%%%%%%%%%%%%%%%%%%%%%%%%%%%%%%%%%%%%%%%%%
\usepackage{makecell} 		% Pour les tableaux, diviser du texte dans une cellule, 
							% pour qu'il passe à la ligne suivante (\thead)
\usepackage{ragged2e}						
\usepackage{booktabs}
\usepackage[export]{adjustbox}	% Les images dans les tableaux sont automatiquement en bas; 
   
\usepackage{changepage}	% Quand les figures/tableaux dépassent des marges, 
						% Qu'ils dépassent également des deux cotés et pas que à droite
\usepackage{multirow}	% Pour les tableaux, une case composée de plusieurs lignes
\usepackage{hhline}		% Les lignes de tableaux plus évoluées


%%%%%%%%%%%%%%%%%%%%%%%%%%%%%%%%%%%%%%%%%%%%%%%%%%
%% 					PACKAGE TODO				%%
%%%%%%%%%%%%%%%%%%%%%%%%%%%%%%%%%%%%%%%%%%%%%%%%%%
\usepackage{xkeyval}
\setlength{\marginparwidth }{2cm} % A enlever pour l'impression finale? C'est pour que les todo dans les marges ne posent pas de problèmes
\usepackage[colorinlistoftodos,prependcaption,textsize=small, textwidth=2.7cm]{todonotes}


%%%%%%%%%%%%%%%%%%%%%%%%%%%%%%%%%%%%%%%%%%%%%%%%%%
%% 					PACKAGES DESSIN				%%
%%%%%%%%%%%%%%%%%%%%%%%%%%%%%%%%%%%%%%%%%%%%%%%%%%
\usepackage{pgf}
\usepackage{tikz}
\usetikzlibrary{calc, fit, matrix, arrows, automata, positioning, shapes.arrows, fadings, patterns,decorations.markings}


%%%%%%%%%%%%%%%%%%%%%%%%%%%%%%%%%%%%%%%%%%%%%%%%%%
%% 					PACKAGES DIVERS				%%
%%%%%%%%%%%%%%%%%%%%%%%%%%%%%%%%%%%%%%%%%%%%%%%%%%
\usepackage[shortlabels]{enumitem}
\usepackage{xargs} % Utilisation de + de 1 paramètre optionnel dans les nouvelles commandes
\usepackage{lipsum} % générer du texte 
\usepackage[final]{pdfpages} % inclure une page pdf


%%%%%%%%%%%%%%%%%%%%%%%%%%%%%%%%%%%%%%%%%%%%%%%%%%
%% 					MISE EN FORME				%%
%%%%%%%%%%%%%%%%%%%%%%%%%%%%%%%%%%%%%%%%%%%%%%%%%%

%% Saute une ligne avant chaque nouveau paragraphe et garde l'alinea
\edef\restoreparindent{\parindent=\the\parindent\relax}
\usepackage{parskip}
\restoreparindent
\usepackage{indentfirst}

% Pas de ligne seule en début ou fin de page
\usepackage[all]{nowidow}

%% Saute une ligne après le titre d'un paragraphe
\makeatletter
\renewcommand\paragraph{%
	\@startsection{paragraph}{4}{0mm}%
	{-\baselineskip}%
	{.5\baselineskip}%
	{\normalfont\normalsize\bfseries}}
\makeatother

%\setlength{\parindent}{3pt}

%%%%%%%%%%%%%%%%%%%%%%%%%%%%%%%%%%%%%%%%%%%%%%%%%%
%% 					COULEURS					%%
%%%%%%%%%%%%%%%%%%%%%%%%%%%%%%%%%%%%%%%%%%%%%%%%%%
\definecolor{darkcyan}{rgb}{0.0, 0.55, 0.55}
\definecolor{halfgray}{gray}{0.55}
\definecolor{webgreen}{rgb}{0,.5,0}
\definecolor{webbrown}{rgb}{.6,0,0}
\definecolor{Maroon}{cmyk}{0, 0.87, 0.68, 0.32}
\definecolor{RoyalBlue}{cmyk}{1, 0.50, 0, 0}
\definecolor{Black}{cmyk}{0, 0, 0, 0}
\definecolor{blueGreen}{HTML}{4CA6A7}
\definecolor{blueGreenN2}{HTML}{3964B4}
% Using colorbrewer, 1 is lightest colour, 4 is darkest
\definecolor{brewsuperlight}{HTML}{F0F9E8}
\definecolor{brewlight}{HTML}{BAE4BC}
\definecolor{brewdark}{HTML}{7BCCC4}
\definecolor{brewsuperdark}{HTML}{2B8CBE}


\colorlet{cfgHeaderBoxColor}{blueGreen} %{green!50!blue!70}
\colorlet{cfgCiteColor}{blueGreen} % vLionel : RoyalBlue
\colorlet{cfgUrlColor}{blueGreen}  % vLionel : RoyalBlue
\colorlet{cfgLinkColor}{blueGreen} % vLionel : blueGreenN2

%%% VICTOR COLORS ----------------------------------------
\colorlet{cfgHeaderBoxColor}{brewsuperdark} %{green!50!blue!70}
\colorlet{cfgCiteColor}{RoyalBlue} % vLionel : RoyalBlue
\colorlet{cfgUrlColor}{RoyalBlue}  % vLionel : RoyalBlue
\colorlet{cfgLinkColor}{RoyalBlue} % vLionel : blueGreenN2


%%%%%%%%%%%%%%%%%%%%%%%%%%%%
%% Chapter pages
\usepackage[Conny]{fncychap}



%%%%%%%%%%%%%%%%%%%%%%%%%%%%%%%%%%%%%%%%%%%%%%%%%%
%% 				HEADERS et FOOTERS				%%
%%%%%%%%%%%%%%%%%%%%%%%%%%%%%%%%%%%%%%%%%%%%%%%%%%
\usepackage{fancyhdr}
\pagestyle{fancy}

% Header horizontal line
\def\headrulefill{  \leaders \hrule width 0pt height 3pt depth -2.8pt \hfill}
% Header title
\renewcommand\chaptermark[1]{\markboth{\enskip#1}{}}

% Create style for content
\fancypagestyle{contentStyle}{%
	% clear header and footer
	\fancyhf{} 
	\renewcommand*{\headrulewidth}{0pt}
	%\renewcommand*{\footrulewidth}{0.2pt}
	\renewcommand{\chaptername}{Chapter \ }
	
	% Numéro de page en bas
	\cfoot{\thepage}
	
	% Configure header
\ifbool{CfgPrintedVersion}{
	%% Header for twosided document
	\fancyhead[LE]{% LE : Left side of Even pages
  		\makebox[0pt][r]{
    	\colorbox{cfgHeaderBoxColor}{\makebox[\textwidth][r]{\textcolor{white}{\chaptername\ \thechapter}\enskip}}\hspace*{1em}}%
		{\itshape\leftmark}\hspace*{1em}\headrulefill%\hspace*{1.4em} si numéro de page en haut
	}
	\fancyhead[RO]{% RO: Right side of Odd pages
  		%\hspace*{1.4em} si numéro de page en haut
  		\mbox{}\headrulefill \hspace*{1em}{\itshape\leftmark}%
  		\makebox[0pt][l]{%
    		\hspace*{1em}\colorbox{cfgHeaderBoxColor}{\makebox[\textwidth][l]{\enskip\textcolor{white}{\chaptername\ \thechapter}}}}
	}
	%% Numéro de page en haut
%	\fancyhead[LO]{
%		 \thepage 
%	}
%	\fancyhead[RE]{
%		 \thepage 
%	}
}{
	%% Header of onesided document
	\fancyhead[L]{%
  		\makebox[0pt][r]{
    	\colorbox{cfgHeaderBoxColor}{\makebox[\textwidth][r]{\textcolor{white}{\chaptername\ \thechapter}\enskip}}\hspace*{1em}}%
		{\itshape\leftmark}\hspace*{1em}\headrulefill%
	}
%	\fancyhead[R]{
%		 \thepage 
%	}
}
}


% Create style for content
\fancypagestyle{introStyle}{%
	% clear header and footer
	\fancyhf{} 
	\renewcommand*{\headrulewidth}{0pt}
	% Numéro de page en bas
	\cfoot{\thepage}
	% Configure header
	\ifbool{CfgPrintedVersion}{
		%% Header for twosided document
		\fancyhead[LE]{% LE : Left side of Even pages
			\makebox[0pt][r]{
				\colorbox{cfgHeaderBoxColor}{\makebox[\textwidth][r]{\textcolor{white}{Introduction}\enskip}}\hspace*{1em}}%
			{\itshape}\hspace*{1em}\headrulefill%
		}
		\fancyhead[RO]{% RO: Right side of Odd pages
			\mbox{}\headrulefill \hspace*{1em}{\itshape}%
			\makebox[0pt][l]{%
				\hspace*{1em}\colorbox{cfgHeaderBoxColor}{\makebox[\textwidth][l]{\enskip\textcolor{white}{Introduction}}}}
		}
	
	}{
		%% Header of onesided document
		\fancyhead[L]{%
			\makebox[0pt][r]{
				\colorbox{cfgHeaderBoxColor}{\makebox[\textwidth][r]{\textcolor{white}{Introduction}\enskip}}\hspace*{1em}}%
			{\itshape}\hspace*{1em}\headrulefill%
		}
	}
}



% Create style for content
\fancypagestyle{conclusionStyle}{%
	%% Configuration du header
	% clear header and footer
	\fancyhf{} 
	
	% NUméro de page en bas
	\cfoot{\thepage}
	
	\renewcommand*{\headrulewidth}{0pt}
	% Configure header
	\ifbool{CfgPrintedVersion}{
		%% Header for twosided document
		\fancyhead[LE]{% LE : Left side of Even pages
			\makebox[0pt][r]{
				\colorbox{cfgHeaderBoxColor}{\makebox[\textwidth][r]{\textcolor{white}{Conclusion}\enskip}}\hspace*{1em}}%
			{\itshape}\hspace*{1em}\headrulefill%
		}
		\fancyhead[RO]{% RO: Right side of Odd pages
			\mbox{}\headrulefill \hspace*{1em}{\itshape}%
			\makebox[0pt][l]{%
				\hspace*{1em}\colorbox{cfgHeaderBoxColor}{\makebox[\textwidth][l]{\enskip\textcolor{white}{Conclusion}}}}
		}

	}{
		%% Header of onesided document
		\fancyhead[L]{%
			\makebox[0pt][r]{
				\colorbox{cfgHeaderBoxColor}{\makebox[\textwidth][r]{\textcolor{white}{Conclusion}\enskip}}\hspace*{1em}}%
			{\itshape}\hspace*{1em}\headrulefill%
		}

	}
}
	



% Create style for content
\fancypagestyle{appendixStyle}{%
	% clear header and footer
	\fancyhf{} 
	\renewcommand*{\headrulewidth}{0pt}
	% NUméro de page en bas
	\cfoot{\thepage}
	% Configure header
	\ifbool{CfgPrintedVersion}{
		%% Header for twosided document
		\fancyhead[LE]{% LE : Left side of Even pages
			\makebox[0pt][r]{
				\colorbox{cfgHeaderBoxColor}{\makebox[\textwidth][r]{\textcolor{white}{Appendix}\enskip}}\hspace*{1em}}%
			{\itshape}\hspace*{1em}\headrulefill%
		}
		\fancyhead[RO]{% RO: Right side of Odd pages
			\mbox{}\headrulefill \hspace*{1em}{\itshape}%
			\makebox[0pt][l]{%
				\hspace*{1em}\colorbox{cfgHeaderBoxColor}{\makebox[\textwidth][l]{\enskip\textcolor{white}{Appendix}}}}
		}

	}{
		%% Header of onesided document
		\fancyhead[L]{%
			\makebox[0pt][r]{
				\colorbox{cfgHeaderBoxColor}{\makebox[\textwidth][r]{\textcolor{white}{Appendix}\enskip}}\hspace*{1em}}%
			{\itshape}\hspace*{1em}\headrulefill%
		}
	}
}




%% Macro that forces a command to be excuted with plain style set to empty style. This is useful for \tableofcontents and \listoffigures command which issue a \thispagestyle{plain} command.
\makeatletter
\newcommand{\ForceEmptyStyle}[1]{%
  \cleardoublepage
  \begingroup
  \let\ps@plain\ps@empty
  \pagestyle{empty}
  #1
  \cleardoublepage
  }
\makeatletter

%%%%%%%%%%%%%%%%%%%%%%%%%%%%%%%%%%%%%%%%%%%%%%%%%%
%% 			ENVIRONNEMENTS DES THEOREMES		%%
%%%%%%%%%%%%%%%%%%%%%%%%%%%%%%%%%%%%%%%%%%%%%%%%%%

\usepackage[framemethod=tikz]{mdframed}
\newtheoremstyle{defi}
  {1em}%\topsep}%
  {\topsep}%
  {\normalfont}%
  {}%
  {\bfseries}% 
  {:}%
  {.5em}%
  {\thmname{#1~}\thmnumber{#2}\thmnote{ -- #3}}
  %
  \theoremstyle{defi}
\mdfdefinestyle{theoremStyle}{
	hidealllines=true,
	leftline=true,
	bottomline=true,
	innertopmargin=2pt,
	innerbottommargin=6pt,
	linewidth=2.5pt,
	linecolor=gray!40,
	innerrightmargin=0pt,
}
\newcounter{thmCounter}[section]
\numberwithin{thmCounter}{section}
\newmdtheoremenv[style=theoremStyle]{theorem}[thmCounter]{Theorem}
\newmdtheoremenv[style=theoremStyle]{definition}[thmCounter]{Definition}
\newmdtheoremenv[style=theoremStyle]{conjecture}[thmCounter]{Conjecture}
\newmdtheoremenv[style=theoremStyle]{lemma}[thmCounter]{Lemma}
\newmdtheoremenv[style=theoremStyle]{remark}[thmCounter]{Remark}
\newmdtheoremenv[style=theoremStyle]{example}[thmCounter]{Example}
\newmdtheoremenv[style=theoremStyle]{proposition}[thmCounter]{Proposition}
\newmdtheoremenv[style=theoremStyle]{corollary}[thmCounter]{Corollary}

\mdfdefinestyle{alertStyle}{
backgroundcolor=red!27,
linecolor=gray,
roundcorner=6pt,
middlelinewidth=0pt
}

% Repetable theorem env
\usepackage{thmtools}
\usepackage{thm-restate}
\declaretheoremstyle
[
    preheadhook={\begin{mdframed}[style=theoremStyle]},
    postfoothook=\end{mdframed},
]{framedThmStyle}
\declaretheorem[style=framedThmStyle, title=Théorème, numberlike=thmCounter]{repenvThm}
\declaretheorem[style=framedThmStyle, title=Conjecture, numberlike=thmCounter]{repenvConj}


%%%%%%%%%%%%%%%%%%%%%%%%%%%%%%%%%%%%%%%%%%%%%%%%%%
%% 			SOMMAIRES et NUMEROTATION			%%
%%%%%%%%%%%%%%%%%%%%%%%%%%%%%%%%%%%%%%%%%%%%%%%%%%
% Profondeur du sommaire principal
\setcounter{tocdepth}{2} % Va jusqu'à 1.1.1 si depth=2 (subsection)

%% Sommaires intermédiaires
\usepackage{minitoc}
% Profondeur des sommaires des chapitres
\setcounter{minitocdepth}{4} % Si depth = 4 : paragraphes inclus; depth=3 : subsubsections incluses

% Numérotation des subsubsections 
\setcounter{secnumdepth}{3}

\renewcommand{\thesubsubsection}{\thesubsection.\alph{subsubsection}}

%\dominitoc
%\renewcommand{\mtctitle}{Sommaire}
%\usepackage[explicit]{titlesec} % Non compatible avec minitoc



%%%%%%%%%%%%%%%%%%%%%%%%%%%%%%%%%%%%%%%%%%%%%%%%%%
%% 			LISTES DES ABBREVIATIONS			%%
%%%%%%%%%%%%%%%%%%%%%%%%%%%%%%%%%%%%%%%%%%%%%%%%%%

\usepackage[intoc, english]{nomencl}
\makenomenclature
\makeindex
\usepackage{ifthen}
\renewcommand{\nomgroup}[1]{
	\ifthenelse{\equal{#1}{B}}{\item[\textbf{Bone name}]}{
	{}}
	\ifthenelse{\equal{#1}{C}}{\item[\textbf{Joint name}]}{
	{}}
	\ifthenelse{\equal{#1}{I}}{\item[\textbf{Image Acquisition Technique}]}{
	{}}
	\ifthenelse{\equal{#1}{M}}{\item[\textbf{Method name}]}{
	{}}
}
%% Abbréviations en 2 colonnes
\usepackage{multicol}
\renewcommand{\nompreamble}{\begin{multicols}{2}}
\renewcommand{\nompostamble}{\end{multicols}}


\setlength{\nomlabelwidth}{1.5cm}

%
%\usepackage{xpatch}
%% initial definitions for storing the section info (name and number)
%\def\thissectiontitle{}
%\def\thissectionnumber{}
%
%\newtoggle{noTodos}
%
%\titleformat{\section}
%{\normalfont\Large\bfseries\sffamily}
%{\thesection}
%{0.5em}
%{\gdef\thissectiontitle{#1}\gdef\thissectionnumber{\thesection}#1}
%
%\pretocmd{\section}{\global\toggletrue{noTodos}}{}{}
%
%% the \todo command does the job: the first time it is used after a \section command, 
%% it writes the information of the section to the list of todos
%\AtBeginDocument{%
%	\xpretocmd{\todo}{%
%		\iftoggle{noTodos}{
%			\addtocontents{tdo}{\protect\contentsline{section}%
%				{\protect\numberline{\thissectionnumber}{\thissectiontitle}}{}{} }
%			\global\togglefalse{noTodos}
%		}{}
%	}{}{}%
%}



%%%%%%%%%%%%%%%%%%%%%%%%%%%%
%% Custom commands definitions
%%%%%%%%%%%%%%%%%%%%%%%%%%%%%%%%%%%%%
% Miscs Cmds
%%%%%%%%%%%%%%%%%%%%%%%%%%%%%%%%%%%%%
\newcommand{\quadn}{\!\!\!\!}
\newcommand{\todoinl}{\todo[inline]{}}
\newcommand{\paragraphTitle}[1]{
%\begin{center}
%\vspace{-5mm}
%\item \paragraph{#1}~\\
%\end{center}
\subsubsection*{\centering{#1}}
}

%%%%%%%%%%%%%%%%%%%%%%%%%%%%%%%%%%%%%
%% Référencer des tables, figures
%%%%%%%%%%%%%%%%%%%%%%%%%%%%%%%%%%%%%
% \newcommand\figref{Fig.~\ref}
% \newcommand\tabref{Table~\ref}
% \newcommand\secref{Sec.~\ref}
\renewcommand\eqref{\cref}

%%%%%%%%%%%%%%%%%%%%%%%%%%%%%%%%%%%%%
%% Norme
%%%%%%%%%%%%%%%%%%%%%%%%%%%%%%%%%%%%%
\newcommand{\norm}[1]{\left\lVert #1 \right\rVert} 
% \newcommand{\innerprod}[2]{\left\langle #1, #2 \right\rangle}
\newcommand{\innerprod}[2]{#1^T #2}

\newcommand{\pfrac}[2]{\frac{\partial #1}{\partial #2}}
%%%%%%%%%%%%%%%%%%%%%%%%%%%%%%%%%%%%%
%% Tableaux
%%%%%%%%%%%%%%%%%%%%%%%%%%%%%%%%%%%%%
% Laisser de l'espace avant ou après une ligne horizontale 
\newcommand\Tstrut{\rule{0pt}{2.6ex}}         % = `top' strut
\newcommand\Bstrut{\rule[-0.9ex]{0pt}{0pt}}   % = `bottom' strut
% Utilisation : 
%	\begin{tabular}{|l|}
%		\hline
%		TEXT\Tstrut\Bstrut\\ % top *and* bottom struts
%		\hline
%		TEXT \Tstrut\\       % top strut only
%		TEXT \Bstrut\\       % bottom strut only
%		\hline
%	\end{tabular}

% Epaisseur des traits de toprule et bottomrule
\setlength\heavyrulewidth{0.25ex}

% Différents types de colonnes à mettre dans tabular
\newcolumntype{C}{>{\centering\arraybackslash}X} % Centered
\newcolumntype{D}{>{\RaggedLeft}X} % Aligné à droite

% Ecrire plusieurs ligne dans une cell de tableau, en précisant où on veut les mettre
\newcommand{\specialcell}[2][c]{ \begin{tabular}[#1]{@{}c@{}}#2\end{tabular}} 


%%%%%%%%%%%%%%%%%%%%%%%%%%%%%%%%%%%%%%%%%%%%%%%%%%
%% 				 SAUTER DES PAGES				%%
%%%%%%%%%%%%%%%%%%%%%%%%%%%%%%%%%%%%%%%%%%%%%%%%%%
% Force que la suite de cette commande soit écrite sur une page impaire (à droite), et qu'au moins une page complètement blanche la précède (donc 2 si nécessaire pour être à droite)
% A utiliser avant un nouveau chapitre par exemple
\newcommand{\fullclear}{\clearpage\phantom{h}\pagestyle{empty}\newpage\cleardoublepage\pagestyle{fancy}}
% La page sera écrite à gauche
\newcommand*\cleartoleftpage{%
	\clearpage
	\ifodd\value{page}\phantom{h}\pagestyle{empty}\newpage\fi
      }


% \renewcommand{\specialchapter}[1]{\fullclear\chapter{#1}}
% \renewcommand{\specialchapterstar}[1]{\fullclear\chapter*{#1}}

% \makeatletter
% \let\stdchapter\chapter
% \renewcommand*\chapter{%
%   \@ifstar{\starchapter}{\@dblarg\nostarchapter}}
% \newcommand*\starchapter[1]{\stdchapter*{#1}}
% \def\nostarchapter[#1]#2{\stdchapter[{#1}]{#2}}
% \makeatother

\renewcommand\chapter{\if@openright\cleardoublepage\else\clearpage\fi
                    \thispagestyle{plain}%
                    \global\@topnum\z@
                    \@afterindentfalse
                    \secdef\@chapter\@schapter}
  
%%%%%%%%%%%%%%%%%%%%%%%%%%%%%%%%%%%%%
%% Mise en page
%%%%%%%%%%%%%%%%%%%%%%%%%%%%%%%%%%%%%
% Dans le cas où on utilise la mise en page Conny de fncychap
% Plutôt que d'avoir "chapitre 1 -- Intro", on veut que Intro soit directement écrit entre les lignes
% Pour la page de garde du chapitre, 
% On utilise cette commande (peut aussi être utile pour conclusion, annexes)
\makeatletter
	\newcommand\TitleBtwLines{%% 
		
		\ChTitleVar{\centering\Huge\rm\bfseries}
		\renewcommand{\DOCH}{%
			\mghrulefill{3\RW}\par\nobreak
			\vskip -0.5\baselineskip
			\mghrulefill{\RW}\par\nobreak
			
		}
		\renewcommand{\DOTI}[1]{%
			\CTV\FmTi{##1}\par\nobreak
			\vskip -0.5\baselineskip
			\mghrulefill{\RW}\par\nobreak
			\vskip 60\p@
		}
		\renewcommand{\DOTIS}[1]{%
			\mghrulefill{\RW}\par\nobreak
			\CTV\FmTi{##1}\par\nobreak
			\vskip 60\p@
		}
	
	} 
\makeatother
	
% Ligne avec 3 étoiles au milieu -> Très bien pour séparer des paragraphes
\newcommand*{\etoile}{
	\begin{center}
		* \hspace{2ex}* \hspace{2ex}*
	\end{center}
}


%================================================
%% Faciliter l'écriture
%================================================
\DeclareMathOperator*{\argmin}{arg\,min \,}
\DeclareMathOperator*{\argmax}{arg\,max \,}
\DeclareMathOperator{\IMSE}{\mathrm{IMSE}}
\DeclareMathOperator{\GP}{\mathrm{GP}}

\newcommand{\Ex}{\mathbb{E}}
\newcommand{\Prob}{\mathbb{P}}
\newcommand{\ProbGP}{\mathcal{P}}
\newcommand{\Var}{\mathbb{V}\mathrm{ar}}
\newcommand{\Cov}{\mathbb{C}\mathrm{ov}}
\newcommand{\Kspace}{\Theta}
\newcommand{\Uspace}{\mathbb{U}}
\newcommand{\Xspace}{\mathbb{X}}
\newcommand{\Yspace}{\mathbb{Y}}
\newcommand{\estimtxt}[2]{{#1}_{\mathrm{#2}}}
\newcommand{\DKL}[3][]{D^{\mathrm{#1}}_{\mathrm{KL}}\left(#2 \| #3\right)}

\newcommand{\kk}{\theta}
\newcommand{\KK}{\theta}
\newcommand{\uu}{u}
\newcommand{\UU}{U}
%%%%%%%%%%%%%%%%%%%%%%%%%%%%%%%%%%%%% 
%% TikZ Cmds
%%%%%%%%%%%%%%%%%%%%%%%%%%%%%%%%%%%%%
\newcommand{\tikzmark}[1]{\tikz[overlay,remember picture] \node (#1) {};}
\newcommand{\tikzdrawbox}[3][(0pt,0pt)]{%
    \tikz[overlay,remember picture]{
    \draw[#3]
      ($(left#2)+(-0.3em,0.9em) + #1$) rectangle
      ($(right#2)+(0.2em,-0.4em) - #1$);}
}


%%%%%%%%%%%%%%%%%%%%%%%%%%%%%%%%%%%%%
%% -- Package TODO enrichi
%%%%%%%%%%%%%%%%%%%%%%%%%%%%%%%%%%%%%
% Pavé coloré sur toute la ligne pour signaler ce qu'on doit faire
\newcommand{\todol}[1]{\todo[linecolor=red,backgroundcolor=red!25,bordercolor=red,inline]{!! #1 !!}}
\newcommand{\unsure}[1]{\todo[linecolor=yellow,backgroundcolor=yellow!25,bordercolor=yellow,inline]{?? #1 ??}}
\newcommand{\change}[1]{\todo[linecolor=blue,backgroundcolor=blue!25,bordercolor=blue,inline]{#1}}
\newcommand{\info}[1]{\todo[linecolor=OliveGreen,backgroundcolor=OliveGreen!25,bordercolor=OliveGreen,inline]{#1}}
\newcommand{\improvement}[1]{\todo[linecolor=Plum,backgroundcolor=Plum!25,bordercolor=Plum,inline]{#1}}
\newcommand{\thiswillnotshow}[1]{\todo[disable,inline]{#1}}
\newcommand{\reformuler}{\change{Reformuler}}

% Pavé coloré petit dans le texte pour les citations à ajouter
\newcommand{\aciter}[1]{\todo[linecolor=violet,backgroundcolor=violet!25,bordercolor=violet, inline]{~ #1 ~}}

% Texte coloré dans les paragraphes, pour signaler que c'est des infos/des formulations à vérifier
\newcommand{\tocheck}[1]{{\color{red}{#1}} \addcontentsline{tdo}{todo}{#1}}

% Note to self : n'apparaît pas dans la liste des todos à faire, c'est pour avoir des infos complémentaires pour soi qui apparaissent dans le pdf
\newcommand{\notetoself}[1]{\todo[linecolor=orange,backgroundcolor=orange!25,bordercolor=orange,nolist,inline]{Note to self: #1}}

%% Liste de todos à cocher
\newlist{todolist}{itemize}{2}
\setlist[todolist]{label=$\square$}
\usepackage{pifont}
\newcommand{\cmark}{\ding{51}}%
\newcommand{\xmark}{\ding{55}}%
\newcommand{\done}{\rlap{$\square$}{\raisebox{2pt}{\large\hspace{1pt}\cmark}}%
	\hspace{-2.5pt}}
\newcommand{\wontfix}{\rlap{$\square$}{\large\hspace{1pt}\xmark}}

