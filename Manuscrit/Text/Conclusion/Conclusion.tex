\documentclass[../../Main_ManuscritThese.tex]{subfiles}

\subfileGlobal{
\renewcommand{\RootDir}[1]{./Text/Chapter4/#1}
}

% For cross referencing
\subfileLocal{
\externaldocument{../../Text/Introduction/build/Introduction}
\externaldocument{../../Text/Chapter2/build/Chapter2}
\externaldocument{../../Text/Chapter3/build/Chapter3}
\externaldocument{../../Text/Chapter4/build/Chapter4}
\externaldocument{../../Text/Chapter5/build/Chapter5}
}



%%%%%%%%%%%%%%%%%%%%%%%%%%%%%%%%%%%%%%
%% CHAPTER TITLE
%%%%%%%%%%%%%%%%%%%%%%%%%%%%%%%%%%%%%%

\begin{document}

\relax

\begingroup

%% ---- On veut que "conclusion" soit entre les trait au début du chapitre

%% ---- On veut que ce soit le chapitre numéro 3 en notation alphabétique pour avoir un C
% \clearpage
% \setcounter{chapter}{2}
% \renewcommand{\thechapter}{\Alph{chapter}}%
\TitleBtwLines

\chapter*{Conclusion and perspectives}
\phantomsection
\addstarredchapter{Conclusion and perspectives}

\label{chap:Conclusion}
\pagestyle{conclusionStyle}
\renewcommand{\thesection}{}

\section{Wrapping-up}
In~\cref{chap:inverse_problem}, after having detailed the common notions of we introduced the calibration from a probabilistic point of view
\section{Limitations}

\section{Perspectives}


%%%%%%%%%%%%%%%%%%%%%%%%%%%%%%%%%%%%%%%%%%%%%%%%%%%%%%%
%%%%%%%%%%%%%%%%%%%%%%%%%%%%%%%%%%%%%%%%%%%%%%%%%%%%%%%
% \todo{Wrapping up}
% \todo{Limitations}
% \todo{Perspectives}
%%%%%%%%%%%%%%%%%%%%%%%%%%%%%%%%%%%%%%%%%%%%%%%%%%%%%%%
%%%%%%%%%%%%%%%%%%%%%%%%%%%%%%%%%%%%%%%%%%%%%%%%%%%%%%%
%%%%%%%%%%%%%%%%%%%%%%%%%%%%%%%%%%%%%%
%% BIB
%%%%%%%%%%%%%%%%%%%%%%%%%%%%%%%%%%%%%%
\subfileLocal{
	\pagestyle{empty}
	\bibliographystyle{alpha}
        \bibliography{/home/victor/acadwriting/bibzotero}
}
\relax

\endgroup
\end{document}


%%% Local Variables:
%%% mode: latex
%%% TeX-master: "../../Main_ManuscritThese"
%%% End:
