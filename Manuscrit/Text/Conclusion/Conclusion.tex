\documentclass[../../Main_ManuscritThese.tex]{subfiles}

\subfileGlobal{
\renewcommand{\RootDir}[1]{./Text/Chapter4/#1}
}

% For cross referencing
\subfileLocal{
\externaldocument{../../Text/Introduction/build/Introduction}
\externaldocument{../../Text/Chapter2/build/Chapter2}
\externaldocument{../../Text/Chapter3/build/Chapter3}
\externaldocument{../../Text/Chapter4/build/Chapter4}
\externaldocument{../../Text/Chapter5/build/Chapter5}
}



%%%%%%%%%%%%%%%%%%%%%%%%%%%%%%%%%%%%%%
%% CHAPTER TITLE
%%%%%%%%%%%%%%%%%%%%%%%%%%%%%%%%%%%%%%

\begin{document}
\pagestyle{conclusionStyle}

% \relax

% \begingroup
%% ---- On veut que "conclusion" soit entre les trait au début du chapitre

%% ---- On veut que ce soit le chapitre numéro 3 en notation alphabétique pour avoir un C
% \clearpage
% \setcounter{chapter}{2}
% \renewcommand{\thechapter}{\Alph{chapter}}%
\TitleBtwLines
\chapter*{Conclusion and perspectives}
\phantomsection
\addstarredchapter{Conclusion and perspectives}
\label{chap:Conclusion}
\renewcommand{\thesection}{} % In this thesis, we studied the problem of
% the calibration of a numerical model under uncertainties, by proposing a new criterion based on the regret, relative or additive

\paragraph{Summary}
In~\cref{chap:inverse_problem}, after having detailed common notions
of probabilities and statistical inference, we presented the
calibration problem as an optimisation problem, by introducing an
objective function, which we can see as a \emph{loss} we wish to
minimise. However, due to the presence of the environmental variable
in the study, a plain minimisation of the objective function is not
possible: by taking into account the random nature of the
\emph{nuisance parameters}, the calibration can be seen as a problem
of \emph{optimisation under uncertainties}, and specific methods and
criterion can be defined to treat accordingly this new problem.


Some classical criteria are first introduced
in~\cref{chap:robust_estimators}, leading to Bayesian or frequentist
estimates if we keep an inference framework, or estimates such as the
minimiser of the mean value of the objective function. In this thesis,
we introduce a new estimate based on the regret. Instead of comparing
directly the values of the objective function for different situations
corresponding to different values of the uncertain variable, the
regret allows the modeller to compare the value of the objective with
the best attainable performance given a specific environmental
variable. This allows to put less emphasis on configurations which
already lead to bad performances.

Moreover, the user can adjust a
parameter in order to reflect either a risk-adverse behaviour, by
favourising a control of the regret with high probability, or a
risk-seeking one, by favourising an estimate that will yield values of
the regret closer to its optimum, albeit with lower probability.


In general, criteria of robust optimisation require a global
knowledge of the function, since they often involve the evaluation of
expectations and probabilities with respect to $\UU$. In addition to
that, regret-based criteria we introduced depend directly on the
conditional minimum and minimiser. In \cref{chap:robust_estimators},
we proposed to use Gaussian Processes in order to compute the
quantities associated with regret-based estimators. More precisely, we
proposed a few methods which aim at improving this estimation by
choosing iteratively a new, or a batch of new points to evaluate and
to add to the design.

Finally in~\cref{chap:croco}, we looked to apply some of the methods
we proposed previously on an academic problem of calibration of a
coastal model based on CROCO. After having reduced the input space
based on the sediment type at the bottom, \todo{finir}

\paragraph{Limitations and perspectives}
The new criteria introduced in this thesis rely on an additional
parameter, the maximal threshold or the wanted probability, that
controls the deviation with respect to the conditional optimal
value. Setting one of those parameters can lead to an unsatisfactory
counterpart as mentioned~\cref{chap:robust_estimators}.  In terms of
numerical methods, Gaussian Processes are not well suited for problems
of dimension larger than about \num{10}: when too large designs are
considered, fitting the GP can also be problematic, as large matrices
need to be inversed, and the optimisation of the hyperparameters is
difficult.  When looking to apply methods based on GP, a first
reduction of the input space may then be necessary.

%\todo{Wrapping up}
% \todo{Limitations}
% \todo{Perspectives}
%%%%%%%%%%%%%%%%%%%%%%%%%%%%%%%%%%%%%%%%%%%%%%%%%%%%%%%
%%%%%%%%%%%%%%%%%%%%%%%%%%%%%%%%%%%%%%%%%%%%%%%%%%%%%%%
%%%%%%%%%%%%%%%%%%%%%%%%%%%%%%%%%%%%%%
%% BIB
%%%%%%%%%%%%%%%%%%%%%%%%%%%%%%%%%%%%%%
\subfileLocal{
	\pagestyle{empty}
	\bibliographystyle{alpha}
        \bibliography{/home/victor/acadwriting/bibzotero}
}
% \relax

% \endgroup
\end{document}


%%% Local Variables:
%%% mode: latex
%%% TeX-master: "../../Main_ManuscritThese"
%%% End:
