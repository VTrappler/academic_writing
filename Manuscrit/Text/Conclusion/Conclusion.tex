\documentclass[../../Main_ManuscritThese.tex]{subfiles}

\subfileGlobal{
\renewcommand{\RootDir}[1]{./Text/Chapter4/#1}
}

% For cross referencing
\subfileLocal{
\externaldocument{../../Text/Introduction/build/Introduction}
\externaldocument{../../Text/Chapter2/build/Chapter2}
\externaldocument{../../Text/Chapter3/build/Chapter3}
\externaldocument{../../Text/Chapter4/build/Chapter4}
\externaldocument{../../Text/Chapter5/build/Chapter5}
}



%%%%%%%%%%%%%%%%%%%%%%%%%%%%%%%%%%%%%%
%% CHAPTER TITLE
%%%%%%%%%%%%%%%%%%%%%%%%%%%%%%%%%%%%%%

\begin{document}
\pagestyle{conclusionStyle}

% \relax

% \begingroup
%% ---- On veut que "conclusion" soit entre les trait au début du chapitre

%% ---- On veut que ce soit le chapitre numéro 3 en notation alphabétique pour avoir un C
% \clearpage
% \setcounter{chapter}{2}
% \renewcommand{\thechapter}{\Alph{chapter}}%
\TitleBtwLines
\chapter*{Conclusion and perspectives}
\phantomsection
\addstarredchapter{Conclusion and perspectives}
\label{chap:Conclusion}
\renewcommand{\thesection}{} % In this thesis, we studied the problem of
% the calibration of a numerical model under uncertainties, by proposing a new criterion based on the regret, relative or additive

\paragraph{Summary}
In~\cref{chap:inverse_problem}, after having detailed common notions
of probabilities and aspects of statistical inference, we presented
the calibration problem as an optimisation problem, by introducing an
objective function, that we wish to minimise with resepct to the
control parameter $\kk$.

However, due to the presence of some random environmental variables in
the study $\UU$, a plain minimisation of the objective function is not
completely relevant, as it will depend on the realisation of this
random variable.

By taking into account the random nature of the {environmental
  parameters}, the calibration can be seen as a problem of
\emph{optimisation under uncertainties}, and many specific methods and
criteria can be defined to treat accordingly this new problem of
\emph{robust} calibration.


Some classical criteria are first introduced
in~\cref{chap:robust_estimators}, leading to Bayesian or frequentist
estimates when keeping an probabilistic inference framework, or
estimates such as the minimiser of the mean value of the objective
function, the minimiser of what is sometimes called the \emph{expected loss} for instance.

In this thesis, we focused on estimates based on a notion of regret:
instead of comparing directly the values taken by the objective
function for different configurations given by $\uu$ sampled from
$\UU$, the regret allows the modeller to compare the value of the
objective with the best attainable performance given this specific
environmental variable.

This allows to put less emphasis on configurations which already lead
to bad performances, and to focus more on ``salveagable'' situations.

Moreover, the user can adjust a parameter in order to reflect either a
risk-adverse behaviour, by favourising a control of the regret with
high probability, or a risk-seeking one, by favourising an estimate
that will yield values of the regret closer to its optimum, albeit
with lower probability.


In general, criteria of robust optimisation require a global knowledge
of the function, since they often involve several evaluations of
expectations and probabilities with respect to $\UU$. In addition to
that, The family of regret-based estimates we introduced depends
directly on the conditional minimum and
minimiser. In~\cref{chap:adaptative_design_gp}, we proposed to use
Gaussian Processes (GP) in order to compute the quantities associated with
regret-based estimators. More precisely, we proposed a few methods
which aim at improving this estimation by choosing iteratively a new,
or a batch of new points to evaluate and to add to the design.

Finally in~\cref{chap:croco}, we studied an academic problem of
calibration of a coastal model based on CROCO.\@ After having reduced
the dimension of the input space based on the sediment type at the
bottom, we enriched the design in order to improve the estimation of
the functions that define the regret-based estimates, which are then
optimised.

\paragraph{Limitations and perspectives}


% Calibration using a Bayesian approach
In this thesis we focused on the variational formulation of the
calibration problem, \emph{i.e.} by defining an objective function,
akin to the negative log-likelihood, that is then optimised. A
Bayesian method could be performed in order to estimate the posterior
distribution of the parameter $\kk$ given the observations. Moreover,
this task could be performed using adapted sampling schemes, such as
Hamiltonian Monte-Carlo~\cite{betancourt_conceptual_2017}, in order to
use the gradient that may be available using adjoint method.

More generally, we did not take much advantage of the potential
availability of the gradient with respect to the control parameter: We
can incorporate this additional knowledge in the mdoelling of the GP,
so that the predictions are improved, as done
in~\cite{bouhlel_gradient-enhanced_2019,laurent_overview_2019}.
However, GP are flexible and useful tools but are not well suited for
modelling problems with dimensions higher than about 10: when the
designs considered are too large, fitting the GP can also be
problematic, as large matrices have to be inverted, and the
optimisation of the hyperparameters can become difficult. Reducing the
dimension of the input spaces is then often necessary.

The segmentation we performed in~\cref{chap:croco} for instance is
rather coarse and based on external information. A finer dimension
reduction method could be done, without prior information, using the
gradient for instance, as done in~\cite{benameur_refinement_2002} or
in~\cite{zahm_certified_2018}.



% Choice of quantile ? to remove maybe
In this thesis, we proposed a family of criteria that depends on the
probability of exceedance. However, we do not control the values of
the function, when exceeding this threshold. A similar approach could
be developed analogous to the CVaR (Conditional Value at
Risk)~\cite{rockafellar_conditional_2002}., also called \emph{expected
  shortfall}, in order to control the values of the objective function
when they do exceed the threshold.

% Horsetail matching, comparison J*(U) and J(k, U)
We introduced the random variable that corresponds to the conditional
minimum and the conditional minimisers. The distribution of the
conditional minimum $J^*(\UU)$ can be seen as a distribution for the
objective (\emph{i.e.} the distribution that one could get if the
calibrated parameter had always been chosen optimally for all
realisations of the uncertain variable). The minimisation of a measure
of misfit between the distribution of $J^*(\UU)$ and the distribution
of $J(\hat{\kk},\UU)$ (the objective function of the model calibrated
with $\hat{\kk}$) may be worth exploring. Horsetail matching for
instance~\cite{cook_extending_2017,cook_horsetail_2018} could be
used so that the a metric between cdf of the two distribution is minimised.

% Choice of alpha or p
The new criteria introduced in this thesis rely on an additional
parameter, the maximal threshold or alternatively, the targeted level
of confidence, that controls the deviation with respect to the
conditional optimal value. Setting one of those parameters can lead to
an unsatisfactory counterpart as mentioned
in~\cref{chap:robust_estimators}.




% 2-stage enrichment strategies
Using the GP, we developed criteria which aim improving the
estimations of the functions $\Gamma_\alpha$ and $q_p$. In order to
get estimators, those functions had to be optimised, in a sample
average approximation (SAA) fashion. We could instead develop methods
which aim at optimising directly those functions, along with their
estimations. This could for instance be done in a $2$-stage enrichment
strategy. First, a $\tilde{\kk}$ is chosen, with a ``high'' potential to be
the optimiser (quite similarly as the EI criterion), and then the
couple $(\kk_{n+1}, \uu_{n+1})$ to evaluate is chosen in order to
reduce a measure of uncertainty associated with the space
$\{\tilde{\kk}\} \times \Uspace$ (\emph{e.g.} the IMSE integrated over this
space). This would focus the enrichment in regions of interest, and
also reduce the dimension of the integral to evaluate.

% Gradient enhanced kriging
More generally, we did not use the gradient of the objective function:




% Sampling problem in AK-MCS


\etoile
\vfill

\subfileLocal{
	\pagestyle{empty}
	\bibliographystyle{alpha}
        \bibliography{/home/victor/acadwriting/bibzotero}
}
% \relax

% \endgroup
\end{document}


%%% Local Variables:
%%% mode: latex
%%% TeX-master: "../../Main_ManuscritThese"
%%% End:
