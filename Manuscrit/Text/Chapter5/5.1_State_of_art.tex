\newpage
\section{State-of-the-art}
\label{sec:5_State_of_art}


For some simple joints, such as the knee, \textit{in vivo} measures can be taken using non-invasive surface markers placed on the skin, and it can be deduced what happens at the bone level of the joint. It is not the case for the wrist due to its complex bone structure. Therefore studies of wrist articulations for carpal bones kinematics understanding have rested for a long time on invasive \textit{in vitro} examinations. The analysis of cadaveric kinematics typically combines markers fixed on the bones and medical imaging such as biplanar radiographs like in \cite{ruby_1988_relative, horii_1991_kinematic}. Non-invasive \textit{in vivo} studies were performed such as in \cite{palmer_1985_functional}, in which wrist kinematics are being characterized using an electrogoniometer to measure the articulation degrees of freedom and their ranges. However the acquired information are insuficient to characterize the inside bone kinematics. 

Non-invasive \textit{in vivo} analysis of the carpal bones were made possible with the utilization of 3D CT scans or MRI instead of biplanar radiographs. It enabled the analysis of images with real 3 dimensions without overlapping shadows and unreliable angle measurements \cite{belsole_1991_carpal}. At first they were mostly used for carpal bone geometry analysis \cite{belsole_1988_mathematical, patterson_1995_carpal, viegas_1993_measurement}, but also normal carpal orientation and position characterization \cite{belsole_1991_carpal}.
The first non-invasive methods to quantify \textit{in vivo} 3D wrist bones kinematics was proposed by Crisco et al. in \cite{crisco_1999_noninvasive}. They were inspired by similar existing methods for large joints such as the knee, that could not directly be applied to the wrist due to its complex small bones and narrow articular spaces. The method proposed rests on the imaging of the joint at multiple positions using CT scans. The bones are segmented based on a threshold value determined from the image histogram, manual intervention may be required to correct some results. An algorithm considering the centroid of each contour is used to label the bones. 3D motion is described by a rotation matrix and a translation vector in the space. Other studies later analyzed \textit{in vivo} 3D kinematics of carpal bones including a series of papers by Moojen, Snel et al. \cite{snel_2000_quantitative, moojen_2001_pisiform, moojen_2002_scaphoid, moojen_2002_three} and a comparison of their results with the literature \cite{moojen_2003_vivo}. Foumani et al. for their part compared carpal bone kinematics results during dynamic motion and step-wise static poses, and concluded that no significant difference exists \cite{foumani_2009_vivo}. 
These analysis mostly rest on deformable models for prior shape information to segment the images at various poses.

The thumb kinematics at wrist level, that is mostly the TMC joint kinematics, are often studied separately from the wrist kinematics. Due to the importance of the thumb for every day life and the effects of osteoarthritis on its mobility, the characterization of healthy motion is important. Late works on thumb kinematics include a study of Crisco et al. \cite{crisco_2015_vivo} to characterize the TMC joint during flexion-extension and adduction-abduction of the thumb. D'Agostino et al. proposed a similar study for the same thumb poses, but including the analysis of the whole chain of movement of the thumb in the wrist, including the radius and the scaphoid in addition to the trapezium and the \first* metacarpal \cite{dagostino_2017_vivo}. 

When the kinematics of the carpal bones have been quantified, it is interesting to be able to model healthy motion. Concerning the wrist kinematics, Chen et al. \cite{chen_2011_inferring, chen_2012_automatic} and Anas et al. \cite{anas_2014_statistical, anas_2016_automatic} propose two models, both based on a principal component analysis of the rotations and translations of the bones between poses, the radius being the unmoving reference. Chen et al. propose an inter-subjects model, based on a PCA on all poses of all wrists in \cite{chen_2011_inferring, chen_2012_automatic} which they use to infer carpal kinematics from single view fluoroscopic sequences. Anas et al. construct two models from the translations and rotations of the carpal bones at various poses of multiple subjects: one intra-subject model represents all possible poses of one person, while an inter-subject model describes the differences for one pose between the different wrists \cite{anas_2014_statistical, anas_2016_automatic}. The combination of these two models is used for the segmentation of new images at any wrist pose. 
Concerning the thumb kinematics at the wrist level, Crisco et al. \cite{crisco_2015_envelope} proposed a mathematical model describing the envelope of physiological motion of the TMC joint. 

The few existing parametric models are based on a PCA, the principal components can describe any combination of motion between the different poses, which make them hard to interpret. We want to create a model of both wrist and thumb motion, adjustable with predictor variables that are observable and make sense for a human user. Therefore we propose a linear model whose parameters are the degrees of flexion-extension and adduction-abduction of the wrist and of the thumb. Additionally, we add a parameter describing whether the wrist is loaded or not. 


