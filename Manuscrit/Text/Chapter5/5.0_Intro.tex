\section{Introduction}
\label{sec:5_Intro}


In this chapter, we are interested in the modeling of the wrist motion. More particularly, we observed that very few models of the wrist movement were proposed in the literature, and all of them are PCA-based. Similarly to the observation that lead to the GPMMs, we note that such models can only lead to fully automatic utilizations. The PCA-based modes are likely to blend multiple simple transformations  such as flexion or radial deviation at once, which makes them unusable for a human user. We propose to use linear regression to build a parametric model described by meaningful predictors: the wrist and thumb degrees of freedom. 


Human interaction with a model could have various applications. For instance a diagnosis tool could be composed of a model whose parameters are interpretable and easily measurable. An individual could be easily lead to take a few characterized poses, and the scans could be compared to the theoretical healthy wrist pattern defined by the model. An unhealthy wrist bone pattern could happen to be closely approximated by an all-automatic model, while the actual pose characteristics are different, which would be avoided with a manual model. 


At first we present a brief state-of-the-art of existing works about carpal bones movement modeling. Then we introduce the mathematical theory behind the model, that is multiple regression analysis. Using the \db*, we present a parametric model based on five predictors. We validate the definition of these parameters: they should be correlated to the poses and the model should have predictable changes when their values are modified. We test the generalization capacity of the model with the poses available. In the conclusion we discuss the need of another more complete database to build an exhaustive model and further test its capacities. The exploratory study presented in this chapter can integrate more data and suggests that the new proposed approach is a promising alternative to efficiently model carpal bones movement.