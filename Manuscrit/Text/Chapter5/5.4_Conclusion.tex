
\clearpage
\section{Conclusion}
\label{sec:5_Conclusion}

In this chapter we have presented the work done about wrist motion modeling. A parametric model based on a linear regression analysis is proposed. The parameters were chosen so they make sense to a human user, on the opposite of PCA modes which are likely to blend several meaningful transformations. They were also determined according to the ease with which these values can be measured, both for a real flesh wrist and for a computerized bony one. The degrees of freedom of the modeled joints have been selected to be the parameters, along with a boolean describing if force is put in the pose. Such a model would be useful for diagnosis, to compare a healthy modeled movement and the one of a subject. We have proven that the parameters seem to be acceptable predictors: the model reacts as expected to the changes of value of the parameters. For instance when the predictors supposed to describe the angles between the \mct* and the \rad* are changed, the global wrist is indeed moved. We have also tested the generalization capacity of the model by measuring the difference between poses computed from the predictors values and the experimental poses as captured by CT scans. 

The modeling is promising, however further analysis of the pertinence of such a model would require another database incorporating more poses per individual. Motion of the wrist alone without thumb activity are necessitated for a more complete description of the motion space. Intermediary poses would also be required both to decompose the path between two extreme complementary poses such as flexion and extension, and to further explore combinations of simple movements. Additionally these intermediary poses are necessary to further validate the model by measuring its generalization capacity while using extreme poses to define the space of possible poses. Further testing of the model on a more complete database would also enable optimization of the functions and parameters, and underline the necessity of yet another predictor if one is actually needed. However this preliminary work is interesting, it confirms the feasibility of the method. 


We have restricted our model to define the motion space of one wrist at a time, on the opposite of the PCA-based model in \cite{chen_2012_automatic}. 
Something we did not address but is interesting is the influence of bones shapes on the wrist movements. %, even if we haven't proved it yet. Therefore learning motion patterns from another individual makes no sense. 
It would be very interesting once a more complete model of the wrist movement is achieved, to analyze similarities and differences of individual motions with respect to the bone shapes, to confirm or disprove this hypothesis. The bone shapes are not the only factors of influence on the carpal wirst motion, the ligaments for instance can also have an impact on the bones displacement patterns. For the moment, the movement model is still uncorrelated to the shape one. However, a future complete model, incorporating all subjects' bone shapes and motion could be computed. It would ideally adapt the modeled motion to the wrist anatomy. 
