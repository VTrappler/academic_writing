\documentclass[../../Main_ManuscritThese.tex]{subfiles}

\subfileGlobal{
\renewcommand{\RootDir}[1]{./Text/Chapter5/#1}
}
\newcommand{\CROCO}{CROCO\,}
\newcommand{\zob}{z_{0,b}}


\subfileLocal{
\externaldocument{../../Text/Introduction/build/Introduction}
\externaldocument{../../Text/Chapter2/build/Chapter2}
\externaldocument{../../Text/Chapter3/build/Chapter3}
\externaldocument{../../Text/Chapter4/build/Chapter4}
\externaldocument{../../Text/Conclusion/build/Conclusion}
}

%%%%%%%%%%%%%%%%%%%%%%%%%%%%%%%%%%%%%%
%% CHAPTER TITLE
%%%%%%%%%%%%%%%%%%%%%%%%%%%%%%%%%%%%%%

\newcommand\imgpath{/home/victor/acadwriting/Manuscrit/Text/Chapter5/img/} 


\begin{document}
% \subfileLocal{\dominitoc}
% \subfileLocal{\setcounter{chapter}{4}}
% \subfileLocal{\chapter{Application to the numerical coastal model CROCO}}
\chapter{Application to the numerical coastal model CROCO}
\label{chap:croco}
\minitoc
% \newpage
\subfileLocal{\pagestyle{contentStyle}}
%%%%%%%%%%%%%%%%%%%%%%%%%%%%%%%%%%%%%%%%%%%%%%%%%%%%%%%
%%%%%%%%%%%%%%%%%%%%%%%%%%%%%%%%%%%%%%%%%%%%%%%%%%%%%%%
In this chapter, we will study the problem of calibration of a
specific model under uncertainties. We will first address the problem
of the high dimensionality of the input space by segmentating it in
arbitrary way. Then, after applying some of the methods introduced in
the previous chapter.

\cite{boutet_estimation_2015}

\section{The ocean modelling system \CROCO}
\CROCO\footnote{\CROCO and CROCO\_TOOLS are provided by
  \url{https://www.croco-ocean.org}} is a new oceanic modeling system
built upon ROMS\_AGRIF and the non-hydrostatic kernel of SNH (under
testing), gradually including algorithms from MARS3D (sediments) and
HYCOM (vertical coordinates). An important objective for \CROCO is to
resolve very fine scales (especially in the coastal area), and their
interactions with larger scales. It is the oceanic component of a
complex coupled system including various components, e.g., atmosphere,
surface waves, marine sediments, biogeochemistry and ecosystems.
\cite{mcwilliams_irreducible_2007,zanna_ocean_2011}

\subsection{Numerical setting of the model}
\label{sec:geographical_setting}

The model used in this thesis is roughly the same
as~\cite{boutet_estimation_2015}. The spatial domain ranges from
\ang{9}W to \ang{1}E and from \ang{43}N to \ang{51}N, and spans most
of the Bay of Biscay, the English Channel and the eastern part of the
Celtic Sea.  The resolution is \SI{1/14}{\degree}, which leads to a
mesh size between \SI{5}{\kilo\metre} and \SI{6}{\kilo\metre}. The bathymetry map is
shown~\cref{fig:depth_maps}. We can notice roughly two regions based
on the depth map: the region near the coasts which correspond to the
continental shelf, where the water depth is less than
\SI{200}{\meter}, and the offshore region of the Bay of Biscay, where
the depth is closer to \SI{5000}{\meter}.
\begin{figure}[ht]
  \centering
  \includegraphics{\imgpath depth_maps_log.png}
  \caption{\label{fig:depth_maps} Bathymetry used in \CROCO, and geographical landmarks. The continental shelf correspond roughly to the area with depth less than \SI{200}{\meter} (green hue), while the abyssal plain has a depth larger than \SI{4000}{\meter} (blue hue)}
\end{figure}.
\CROCO can solve 3D fluid motions equations, but in this configuration
is used only with one vertical level and thus solves numerically the
shallow water equations:
\begin{align}
  \left\{
  \begin{array}{rcl}
    \frac{\partial u}{\partial t} + \nabla \cdot \left(\vec{v}u\right) - fv &=& -\frac{\partial \phi}{\partial x} + \mathcal{F}_{u} + \mathcal{D}_u \\
    \frac{\partial v}{\partial t} + \nabla \cdot \left(\vec{v}v\right) + fv &=& -\frac{\partial \phi}{\partial y} + \mathcal{F}_{v} + \mathcal{D}_v
  \end{array}
       \right.
\end{align}
\subsection{Modelling of the bottom friction}
In \CROCO, the bottom friction is modelled using a quadratic friction
drag coefficient, as described
\cref{eq:bottom_stress_tau,eq:quadratic_friction_vonkarman},
expression which can be derived by assuming a logarithmic profile of the velocity
at the bottom.
\begin{align}
  \label{eq:bottom_stress_tau}
  \bm{\tau}_b &= C_d \|\mathbf{u}_b\|\mathbf{u}_b \\  
  C_d &= \left(\frac{\kappa}{\log\left(\frac{H}{z_{0,b}}\right)}\right)^2 \text{for } C_d \in [C_d^{\min}, C_d^{\max}]   \label{eq:quadratic_friction_vonkarman}
\end{align}
where $\kappa$ is the Von K\'arm\'an constant, equal to $0.41$.

Based on the documentation of the SHOM, the \cref{fig:sediments_reduced} shows a simplified version of the map of the repartition of tthe different types of sediments. Most of the ocean floor is here composed of sand. There is also very few areas where Siltic soil is listed.    

\begin{figure}[ht]
  \centering
  \includegraphics{\imgpath sediments_reduced.png}
  \caption{\label{fig:sediments_reduced} Repartition of the sediments on the ocean floor.}
\end{figure}


\begin{table}[!ht]
  \centering
  \begin{tabular}{rrrl} \toprule
    Code  & Description & Size of the majority of particles                              & $\zob^{\mathrm{truth}}$ \\ \midrule
    Roche & Rock        & Larger                                                         & \SI{50}{\milli\meter}       \\
    C     & Pebble      & $>$\SI{20}{\milli\metre}                                       & \SI{25}{\milli\meter}     \\
    G     & Gravel      & $\interval{\SI{20}{\milli\metre}}{\SI{2}{\milli\metre}}$       & \SI{7}{\milli\meter}       \\
    S     & Sand        & $ \interval{\SI{2}{\milli\metre}}{\SI{0.5}{\milli\metre}}$     & \SI{1}{\milli\meter}       \\
    SF    & Fine Sand   & $ \interval{\SI{0.5}{\milli\metre}}{\SI{0.05}{\milli\metre}}$  & \SI{1.5e-1}{\milli\meter}     \\
    Si    & Silt        & $ \interval{\SI{0.05}{\milli\metre}}{\SI{0.01}{\milli\metre}}$ & \SI{2e-2}{\milli\meter}       \\
    V     & Muds        & $< \SI{0.05}{\milli\metre}$                                    & \SI{2e-2}{\milli\meter}       \\ \bottomrule
    % A   & Clay        & $< \SI{0.01}{\milli\metre}$                                    & \bottomrule
  \end{tabular}
  \caption{\label{tab:size_sediments} Type of sediments and size of the majority of particles for each type of sediment}
\end{table}
\Cref{fig:sediments_reduced} shows that the largest sediments are
rocks but are mostly located in the Bay of Biscay, near the boundary
of the continental shelf. Pebbles however are mostly located in the
shallow region in the English Channel, thus it may be expected that
controlling the size of pebbles is highly influencal in a calibration
context.


On~\cref{fig:cd_zob} is shown the drag coefficient $C_d$ as a function
of the roughness $\zob$ of the ocean floor, for different height of
the water column $H$.
\begin{figure}[ht]
  \centering \input{\imgpath cd_zob.pgf}
  \caption{\label{fig:cd_zob} Drag coefficient $C_d$ as a function of the column water height and the roughness at the bottom}
\end{figure}
We can see that the higher the water column height, the less
variation appears when adjusting the bottom roughness $\zob$.
Considering the physical properties of the bottom friction and the
types of sediments, it can be expected that the English Channel, and at a lesser extent the rest of the continental shelf
are the areas which are the most influential for the calibration.


\section{Deterministic calibration of the bottom friction}
\label{sec:deterministic_calibration_bott}
\subsection{Twin experiments setup}
In a twin experiment setup, the observation $y$ are generated using
the numerical model, and thus can be compared directly 
\subsection{Cost function definition}
Let $y \in \mathbb{R}^{N_{\mathrm{mesh}} \cdot N_{\mathrm{time}}}$ be
the observations, generated using a particular configuration of the
model.
We define $J$ as
\begin{align}
  J(\kk) &= \sum_{t=1}^{ N_{\mathrm{time}}}\sum_{i=1}^{N_{\mathrm{mesh}}}  \left(\eta_{t,i}(\kk) - y_{t, i}\right)^2 \\
         &= \|\mathcal{M}(\kk) - y\|_2^2
\end{align}
Equivalently, as mentioned~\cref{chap:inverse_problem}, by assuming
that the distribution of the (random) observation vector is known and
$Y \mid \kk \sim \mathcal{N}(\mathcal{M}(\kk), I)$ where $I$ is the
identity matrix, $J$ is proportional to the negative log-likelihood of the data.

\subsection{Optimisation without uncertainties}
The optimisation is first carried using M1QN3, a version of a
gradient-descent procedure, as described
in~\cite{gilbert_numerical_1989}. We can first look to control $\zob$
at every cell of the mesh: $\kk = (\kk_1,\cdots, \kk_p)$ where
$\kk_i = \zob^i$ and $p=\num{15684}$. Due to the large number of
points whose friction can be controlled., a finite difference method
to get the gradient is unfeasible. Instead,
Tapenade~\citep{hascoet_tapenade_2013}, an Automatic Differentiation
tool has been used in order to get the gradient of the cost function
$J$ using the adjoint method, as
described~\cref{sec:calibration_adjoint_optimization}. The
optimisation procedure is stopped after \num{400} iterations, and the
estimated controlled parameter is
shown~\cref{fig:optimization_map_399}.
\Cref{fig:ctrl_true} shows the
evolution of the cost function and the squared norm of the gradient
during the optimisation procedure.

 \begin{figure}[ht]
  \centering
  \includegraphics{/home/victor/optimisation_dahu/optim_sediments/map_399.png% /home/victor/optimisation_dahu/optim_true/map_150.png
  }
  \caption{\label{fig:optimization_map_399} Optimization of $\zob$ on the whole space using gradient obtained via adjoint method, after $400$ iterations TODO: {Taille et colorbar}}
\end{figure}

 \begin{figure}[ht]
  \centering
  \input{/home/victor/optimisation_dahu/optim_sediments/ctrl_true.pgf}
  \caption{\label{fig:ctrl_true} Evolution of the cost function and the squared normed of the gradient}
\end{figure}
By comparing \cref{fig:optimization_map_399} with
\cref{fig:sediments_reduced} and \cref{fig:depth_maps}, we can have a
first overview on the links between sediments type, depth, and
\emph{identifiability}, which can be understood as the ability of the
optimisation procedure to retrieve a truth value.

On a first look, we can see that the abyssal plain (the deep region
off the Bay of Biscay) remains mostly untouched by the optimisation,
while the continental shelf, except for the English Channel, is
retrieved. In terms of sediments, the bottom of the continental shelf
is largely composed of sand.
On~\cref{fig:optimisation_type_sediments}, we can see that indeed,
points of the mesh corresponding to sands have seen their $\zob$ value
shifted toward the truth, while for silts and muds, the procedure has
not been able to identify their roughness. This is probably due to the
fact that those sediments lay at great depth, and thus have little
effect on the circulation per~\cref{eq:quadratic_friction_vonkarman}.
%
%
\begin{figure}[ht]
  \centering
  \includegraphics{\imgpath optimisation_type_sediments.pdf}
  \caption{\label{fig:optimisation_type_sediments} Results of the optimisation procedure, depending on the type of sediments. The initial value is \SI{5e-3}{\meter}}
\end{figure}

In the English Channel, similar conclusions can be drawn: the size of
the pebbles is well retrieved, but the control variable of points
mapped to gravel do seem to compensate: on the northern part of the
channel the size of the gravel is overestimated, while it is
underestimated on the southern part.  Finally the rocks appear to be
hard to capture: their assumed size, \emph{i.e.} their truth value is
significantly larger than the rest of the sediments, and they are
sparsely distributed.
%  \begin{figure}[ht]
%   \centering
%   \input{/home/victor/optimisation_dahu/optim_0_001/ctrl_0_001.pgf}
%   \caption{\label{fig:ctrl_0_001} Gradient descent procedure in misspecified case}
% \end{figure}
% \clearpage

\subsection{Tidal modelling}
\cite{egbert_efficient_2002} TPX model of tides

\label{sec:tidal_modelling}
\begin{table}[!h]
  \centering % % Chose order from the rank in the TPXO file :
% "M2 S2 N2 K2 K1 O1 P1 Q1 Mf Mm"
% " 1  2  3  4  5  6  7  8  9 10"
  \begin{tabular}{rrr}\toprule
    Darwin Symbol & Period (h)   & Species                           \\ \midrule
    $M_2$         & 12.4206      & Principal Lunar Semidiurnal       \\
    $S_2$         & 12           & Principal solar Semidiurnal       \\
    $N_2$         & 12.65834751  & Larger Lunar Elliptic Semidiurnal \\
    $K_2$         & 11.96723606  & Lunisolar Semidiurnal             \\
    $K_1$         & 23.93447213  & Lunar Diurnal                     \\\midrule
    $O_1$         & 25.81933871  & Lunar Diurnal                     \\ 
    $P_1$         & 24.06588766  & Solar Diurnal                     \\
    $Q_1$         & 26.868350    & Larger Lunar Elliptic Diurnal     \\
    $M_f$         & 13.660830779 & Lunisolar Fortnightly             \\
    $M_m$         & 27.554631896 & Lunar Monthly                     \\
    \bottomrule
  \end{tabular}
  \caption{Tide components}
  \label{tab:tides_components}
\end{table}
The uncertainties in this configuration represent an error on the amplitude of the tide:
\begin{equation}
  \tilde{A}_i(\uu_i) = A_i (1 + 0.01(2u - 1))
\end{equation}
so $\tilde{A}_i(0) = 0.99A_i$, $\tilde{A}_i(0.5) = A_i$ and $\tilde{A}_i(1) = 1.01A_i$

\section{Sensitivity analysis}
\label{sec:sensitivity-analysis}
Sensitivity analysis (often abbreviated as \emph{SA}), aims at
quantififying the effect of the variation of some input variable to
the output of the model~\cite{iooss_revue_2011,janon_analyse_2012}.
Intuitively, \emph{SA} ties the variation of the input to the
variation of the output. It can then be approached at two different
scales: around a nominal value, using the gradient, and at a global
scale, by considering the inputs as random variable, and by measuring
the variance of the output.

\subsection{Methods of Sensitivity analysis}
\label{sec:methods_SA}
\subsubsection{Local sensitivity analysis}
\label{sec:loca_SA}
Local sensitivity analysis~\cite{morio_global_2011} refers to the
study of how a small perturbation $\delta \kk$ of a nominal value
$\kk$ affects the output of the numerical model. As we assume that the
numerical model is accessible through the cost function $J$, a
straightforward way to quantify this perturbation is to consider the
partial derivative of $J$, with respect to each component of the
control variable $\kk=(\kk_1,\dots,\kk_p)$:
\begin{equation}
  \frac{\partial J}{\partial \kk_i}(\kk)
\end{equation}
The normalized local sensitivity at $\kk$  associated with the $i$-th component is then 
\begin{equation}
  \frac{{\Delta J}/{J}}{{\Delta \kk_i}/{\kk_i}} = \frac{\kk_i}{J(\kk)} \frac{\partial J}{\partial \kk_i}
\end{equation}
\subsubsection{Global Sensitivity Analysis: Sobol' indices}
\label{sec:sobol-indices}
\cite{janon_analyse_2012}
The $i$-th Sobol' indice of order $1$ is defined as 
\begin{equation}
  S_i = \frac{\Var_{X_i}\left[\Ex_{Y}\left[Y \mid X_i\right]\right]}{\Var_{Y}\left[Y\right]}
\end{equation}
where $Y = J(X)$ is the random variable. These are computed using a
replicated
method~\cite{gilquin_making_2019,gilquin_echantillonnages_2016}, in
order to get bootstrap confidence interval for the first and second
order effects, and total effects
\begin{equation}
 S_{T_i} = 1 - \frac{\Var_{X_{-i}}\left[\Ex_{Y}\left[Y \mid X_{-i}\right]\right]}{\Var_{Y}\left[Y\right]}
\end{equation}
where $X_{-i} = (X_1,\dots X_{i-1},X_{i+1},\dots,X_p)$ is random vector of $p-1$ components.

\subsection{Application to CROCO}
The bottom friction affects the ocean circulation through two factors,
as shown \cref{eq:quadratic_friction_vonkarman}, first by the bottom
roughness, $\zob$, and by the ocean depth. We are first going to
perfrom a sensitivity analysis, in order to quantify the role of each
sediment based region, without incorporating the knowledge on the
typical size of the sediment there.

\subsubsection{SA on the class of sediments}

\subsubsection{SA on the tide components}
\begin{figure}[ht]
  \centering
  \input{\imgpath SA_tides.pgf}
  \caption{\label{fig:SA_tides} Global SA on the different components of the tide }
\end{figure}

\cref{fig:SA_tides} shows the Sobol indices of order 1 (left), 2 (right), and the total effect indices, along with bootstrap confidence intervals. The variation of the variable associated with the $M_2$ component of the tide has the most impact on the cost function.


% \begin{figure}[ht]
%   \centering
%   \input{\imgpath SA_croco.pgf}
%   \caption{\label{fig:sobol_indices} Sobol indices obtained using replicated methods and bootstrap CI}
% \end{figure}

% \begin{figure}[ht]
%   \centering
%   \input{\imgpath distribution_minimizers.pgf}
%   \caption{\label{fig:dist_minimizers} Distribution of the minimizers, estimated using the GP constructed on $J$, and enriched using the PEI criterion}
% \end{figure}







%%%%%%%%%%%%%%%%%%%%%%%%%%%%%%%%%%%%%%%%%%%%%%%%%%%%%%%
%%%%%%%%%%%%%%%%%%%%%%%%%%%%%%%%%%%%%%%%%%%%%%%%%%%%%%%
%%%%%%%%%%%%%%%%%%%%%%%%%%%%%%%%%%%%%%
%% BIB
%%%%%%%%%%%%%%%%%%%%%%%%%%%%%%%%%%%%%%
\subfileLocal{
	\pagestyle{empty}
	\bibliographystyle{alpha}
	\bibliography{../../Bibliography}
}
\end{document}

%%% Local Variables:
%%% mode: latex
%%% TeX-master: "../../Main_ManuscritThese"
%%% End:
