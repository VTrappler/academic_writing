\documentclass[../../Main_ManuscritThese.tex]{subfiles}

\subfileGlobal{
\renewcommand{\RootDir}[1]{./Text/Chapter5/#1}
}
\newcommand{\CROCO}{CROCO\,}
\newcommand{\zob}{z_{0,b}}


\subfileLocal{
\externaldocument{../../Text/Introduction/build/Introduction}
\externaldocument{../../Text/Chapter2/build/Chapter2}
\externaldocument{../../Text/Chapter3/build/Chapter3}
\externaldocument{../../Text/Chapter4/build/Chapter4}
\externaldocument{../../Text/Conclusion/build/Conclusion}
}

%%%%%%%%%%%%%%%%%%%%%%%%%%%%%%%%%%%%%%
%% CHAPTER TITLE
%%%%%%%%%%%%%%%%%%%%%%%%%%%%%%%%%%%%%%

\newcommand\imgpath{/home/victor/acadwriting/Manuscrit/Text/Chapter5/img/} 


\begin{document}
% \subfileLocal{\dominitoc}
% \subfileLocal{\setcounter{chapter}{4}}
% \subfileLocal{\chapter{Application to the numerical coastal model CROCO}}
\chapter{Application to the numerical coastal model \CROCO}
\label{chap:croco}
\minitoc
% \newpage
\subfileLocal{\pagestyle{contentStyle}}
%%%%%%%%%%%%%%%%%%%%%%%%%%%%%%%%%%%%%%%%%%%%%%%%%%%%%%%
%%%%%%%%%%%%%%%%%%%%%%%%%%%%%%%%%%%%%%%%%%%%%%%%%%%%%%%

\section{Introduction}
\label{sec:intro_croco}

% \todo{\cite{mcwilliams_irreducible_2007,zanna_ocean_2011}}

In this chapter, we will study the problem of calibration under
uncertainties of the friction of the ocean bed of a realistic
numerical model of the atlantic french coast based on \CROCO. 

The bottom friction depends directly on the size of the asperities on
the ocean bed, thus it is a subgrid phenomenon: the program will not
try solve the equations of motions of fluid around those asperities,
but instead, a parametrization will be introduced in

The bottom friction parameter has been identified as a crucial
parameter that limits the accuracy of the modelling
systems~\cite{boutet_estimation_2015,das_variational_1992,das_estimation_1991}.
We will detail how bottom friction affects the oceanic circulation
in~\cref{ssec:modelling_bottom}, in order to get a first insight on
the regions that may influence the most the calibration.

The deterministic problem of calibration will then be addressed in
\cref{sec:deterministic_calibration_bott}, by first defining the cost
function and the input space. We will then calibrate the model
without uncertainties using adjoint-based method.

However for such problems, the large dimension of the input space may
be problematic, as the parameter may be chosen for each cell of the
mesh. Instead of carrying the optimisation over the whole space, we
will segment the geographical input space in different independent
regions, which are based on the type of sediments listed at the bottom
of the water. In order to quantify the influence of each of those
regions, we will carry a sensitivity analysis, in \cref{sec:methods_SA}.


Finally, we are going to apply methods introduced in the previous
chapters, in order to get a robust estimation of the bottom friction,
using Gaussian processes, as addressed in the previous chapter.



\section{\CROCO\ and bottom friction modelling}
\label{sec:croco_bottom_fr}
\CROCO{}\footnote{\CROCO\ and CROCO\_TOOLS are provided by
  \url{https://www.croco-ocean.org}} (Coastal and Regional Ocean
COmmunity model) is a program written in Fortran that describes the
motion of the ocean by solving the \emph{primitive equations}, which
are simplified versions of the Navier-Stokes equations, in order to
take into account the particular scales at play at the surface of the
Earth. \CROCO{} has been developed upon ROMS\_AGRIF, and is developed
to be coupled with other modelling systems, such as atmospheric,
biological or ecosystem models.

% gradually including algorithms from MARS3D (sediments) and
% HYCOM (vertical coordinates). An important objective for \CROCO\ is to
% resolve very fine scales (especially in the coastal area), and their
% interactions with larger scales. It is the oceanic component of a
% complex coupled system including various components, e.g., atmosphere,
% surface waves, marine sediments, biogeochemistry and ecosystems.


\subsection{Parameters and configuration of the model}
\label{sec:geographical_setting}

The model used in this thesis is roughly the same
as~\cite{boutet_estimation_2015}. The spatial domain ranges from
\ang{9}W to \ang{1}E and from \ang{43}N to \ang{51}N, and spans most
of the Bay of Biscay, the English Channel and the eastern part of the
Celtic Sea.  The resolution is \SI{1/14}{\degree}, which leads to a
mesh size between \SI{5}{\kilo\metre} and \SI{6}{\kilo\metre}. The
bathymetry map is shown~\cref{fig:depth_maps}. We can notice roughly
two regions based on the depth map: the region near the coasts which
correspond to the continental shelf, where the water depth is less
than \SI{200}{\meter}, and the offshore region of the Bay of Biscay,
where the depth is closer to \SI{5000}{\meter}.
\begin{figure}[ht]
  \centering
  \includegraphics{\imgpath depth_maps_log.png}
  \caption{\label{fig:depth_maps} Bathymetry used in \CROCO, and
    geographical landmarks. The continental shelf correspond roughly
    to the area with depth less than \SI{200}{\meter} (green hue),
    while the abyssal plain has a depth larger than \SI{4000}{\meter}
    (blue hue)}
\end{figure}
\CROCO{} can solve 3D fluid motions equations, but in this configuration
is used only with one vertical level and thus solves numerically the
shallow water equations:
% \begin{align}
%   \left\{
%   \begin{array}{rcl}
%     \frac{\partial u}{\partial t} + \nabla \cdot \left(\vec{v}u\right) - fv &=& -\frac{\partial \phi}{\partial x} + \mathcal{F}_{u} + \mathcal{D}_u \\
%     \frac{\partial v}{\partial t} + \nabla \cdot \left(\vec{v}v\right) + fv &=& -\frac{\partial \phi}{\partial y} + \mathcal{F}_{v} + \mathcal{D}_v
%   \end{array}
%        \right.
% \end{align}
\begin{align}
  \left\{
  \begin{array}{lll}
     \pfrac{\mathbf{v}}{t} + (\mathbf{v} \cdot \nabla )\mathbf{v} + 2 \bm{\Omega} \wedge \mathbf{v} &=& -g\nabla H + \frac{\bm{\tau}_b}{\rho H} \\
    \pfrac{H}{t} + \nabla (H \cdot \mathbf{v}) &=& 0
  \end{array}
   \right.
\end{align}
where $\mathbf{v} = (v_x,v_y)$ is the velocity field of the fluid,
$\bm{\Omega}$ is the rotational angular vector of Earth, $H$ is the
water column height, $g$ is the gravitational constant, $\rho$ is the
fluid density, and $\bm{\tau}_b$ is the shear stress at the
bottom. The bottom friction affects the circulation through
$\bm{\tau}_b$, and different parameterizations of this stress can be
derived.

\subsection{Modelling of the bottom friction}
\label{ssec:modelling_bottom}
In \CROCO, the bottom friction is modelled using a quadratic drag
coefficient $C_d$:
\begin{equation}
    \label{eq:bottom_stress_tau}
    \bm{\tau}_b= -C_d \|\mathbf{v}_b\|\mathbf{v}_b 
  \end{equation}
  where $\mathbf{v}_b$ is the velocity at the bottom, so in the case
  of the Shallow Water equations, $\mathbf{v} = \mathbf{v}_b$.  The
  drag coefficient can also be formulated as a function of the water
  column height and and the \emph{bottom roughness} $\zob$ by assuming a
  logarithmic profile of the velocity at the bottom~\cite{boutet_estimation_2015}
  \begin{equation}
    \label{eq:quadratic_friction_vonkarman}
  C_d = \left(\frac{\kappa}{\log\left(\frac{H}{\zob}\right)}\right)^2% \text{for } C_d \in [C_d^{\min}, C_d^{\max}]
\end{equation}
where $\kappa$ is the Von K\'arm\'an constant, that we will take equal
to $0.41$.  The bottom roughness $\zob$, or \emph{rugosity} in this
document, can be interpreted as the size of the turbulent layer at the
bottom, induced by the asperities of the sediments that lie at the
bottom.
\cite{boutet_estimation_2015} shows that in a calibration context,
controlling the rugosity $\zob$ yields better result that controlling
the drag coefficient $C_d$, thus we are going to directly control $\zob$ or $\log(\zob)$.


On~\cref{fig:cd_zob} is shown the drag coefficient $C_d$ as a function
of the roughness $\zob$ of the ocean floor, for different height of
the water column $H$.
\begin{figure}[ht]
  \centering \input{\imgpath cd_zob.pgf}
  \caption{\label{fig:cd_zob} Drag coefficient $C_d$ as a function of
    the column water height and the roughness at the bottom}
\end{figure}
We can see that the higher the water column height, the less variation
appears when adjusting the bottom roughness $\zob$.  Considering the
physical properties of the bottom friction and the types of sediments,
it can be expected that the English Channel, and at a lesser extent
the rest of the continental shelf are the areas which are the most
influential for the calibration.



This rugosity is directly linked to the type of sediment found on the
ocean bed, thus we can first study the type of materials that are
found near the french coasts. \Cref{tab:size_sediments} a coarse
classification, along with typical size of the sediments that can be
found, that will serve as a reference value, or \emph{truth value}
$\zob^{\mathrm{truth}}$.

\begin{table}[!ht]
  \centering
  \begin{tabular}{rrrl} \toprule
    Code  & Description & Size of the majority of particles                              & $\zob^{\mathrm{truth}}$ \\ \midrule
    Roche & Rock        & Larger                                                         & \SI{50}{\milli\meter}       \\
    C     & Pebble      & $>$\SI{20}{\milli\metre}                                       & \SI{25}{\milli\meter}     \\
    G     & Gravel      & $\interval{\SI{20}{\milli\metre}}{\SI{2}{\milli\metre}}$       & \SI{7}{\milli\meter}       \\
    S     & Sand        & $ \interval{\SI{2}{\milli\metre}}{\SI{0.5}{\milli\metre}}$     & \SI{1}{\milli\meter}       \\
    SF    & Fine Sand   & $ \interval{\SI{0.5}{\milli\metre}}{\SI{0.05}{\milli\metre}}$  & \SI{1.5e-1}{\milli\meter}     \\
    Si    & Silt        & $ \interval{\SI{0.05}{\milli\metre}}{\SI{0.01}{\milli\metre}}$ & \SI{2e-2}{\milli\meter}       \\
    V     & Muds        & $< \SI{0.05}{\milli\metre}$                                    & \SI{2e-2}{\milli\meter}       \\ \bottomrule
    % A   & Clay        & $< \SI{0.01}{\milli\metre}$                                    & \bottomrule
  \end{tabular}
  \caption{\label{tab:size_sediments} Type of sediments and size of the majority of particles for each type of sediment}
\end{table}

Based on the documentation of the SHOM, the
\cref{fig:sediments_reduced} shows a simplified version of the map of
the repartition of the different types of sediments. Most of the ocean
floor in this region is composed of sand. Even though siltic soil is
listed, it is only scarcely present. The figure also shows that the
largest sediments are rocks but are mostly located in the Bay of
Biscay, near the boundary of the continental shelf. Pebbles however
are mostly located in the shallow region in the English Channel, thus
it may be expected that controlling the size of pebbles is highly
influencal in a calibration context.

\begin{figure}[ht]
  \centering
  \includegraphics{\imgpath sediments_reduced.png}
  \caption{\label{fig:sediments_reduced} Repartition of the sediments on the ocean floor.}
\end{figure}


\subsection{Modelling of the uncertainties: tidal amplitude}
\cite{egbert_efficient_2002} TPX model of tides

\label{sec:tidal_modelling}
\begin{table}[!h]
  \centering % % Chose order from the rank in the TPXO file :
% "M2 S2 N2 K2 K1 O1 P1 Q1 Mf Mm"
% " 1  2  3  4  5  6  7  8  9 10"
  \begin{tabular}{rrr}\toprule
    Darwin Symbol & Period (h)   & Species                           \\ \midrule
    $M_2$         & 12.4206      & Principal Lunar Semidiurnal       \\
    $S_2$         & 12           & Principal Solar Semidiurnal       \\
    $N_2$         & 12.65834751  & Larger Lunar Elliptic Semidiurnal \\
    $K_2$         & 11.96723606  & Lunisolar Semidiurnal             \\
    $K_1$         & 23.93447213  & Lunar Diurnal                     \\
    % \midrule
    % $O_1$         & 25.81933871  & Lunar Diurnal                     \\ 
    % $P_1$         & 24.06588766  & Solar Diurnal                     \\
    % $Q_1$         & 26.868350    & Larger Lunar Elliptic Diurnal     \\
    % $M_f$         & 13.660830779 & Lunisolar Fortnightly             \\
    % $M_m$         & 27.554631896 & Lunar Monthly                     \\
    \bottomrule
  \end{tabular}
  \caption{Tide components}
  \label{tab:tides_components}
\end{table}
The uncertainties in this configuration represent an error on the amplitude of the tide:
\begin{equation}
  \tilde{A}_i(\uu_i) = A_i (1 + 0.01(2u - 1))
\end{equation}
so $\tilde{A}_i(0) = 0.99A_i$, $\tilde{A}_i(0.5) = A_i$ and $\tilde{A}_i(1) = 1.01A_i$


\section{Deterministic calibration of the bottom friction}
\label{sec:deterministic_calibration_bott}
The calibration of the bottom friction will be studied in a twin
experiment setup. In that setting, the observations
$y \in \mathbb{R}^{N_{\mathrm{obs}}}$ will be generated using the
numerical model, and some reference parameters, that will be qualified
as the \emph{truth}. The output of the model we are going to consider
is the sea surface height $\eta$ at every point of the mesh, and at every
time step, so
$N_{\mathrm{obs}} = N_{\mathrm{mesh}}\cdot N_{\mathrm{time}}$

We are going to use the notation
introduced in~\cref{sec:parameter_inference}, for the model.  Let us
define $\mathcal{M}$ as the numerical model.
\begin{equation}
  \begin{array}{rcl}
    \mathcal{M}: \Kspace \times \Uspace &\longrightarrow& \mathbb{R}^{N_{\mathrm{obs}}} \\
    (\kk, \uu)& \longmapsto & \mathcal{M}(\kk, \uu) = \left(\eta_{i,t}(\kk, \uu)\right)_{\substack{1 \leq i \leq N_{\mathrm{mesh}} \\ 1 \leq t \leq N_{\mathrm{time}}}} \\ 
  \end{array}
\end{equation}

\subsection{Twin experiment setup}
In a twin experiment setup, the observation $y$ are
generated using the numerical model, and a predefined truth value $\kk^{\mathrm{truth}}$.
This means that the ``physical model'' is defined using the forward operator
$\mathscr{M}$, based on the forward numerical model, evaluated with a
certain uncertain parameter $\uu^{\mathrm{truth}}$
\begin{equation}
  \label{eq:twin_exp}
  \begin{array}{rcl}
    \mathscr{M}: \Kspace &\longrightarrow & \mathbb{R}^{N_{\mathrm{obs}}} \\
      \kk & \longmapsto & \mathscr{M}(\kk) = \mathcal{M}(\kk, \uu^{\mathrm{truth}})
  \end{array}
\end{equation}
 the observations are generated using $\vartheta=\kk^{\mathrm{truth}}$:
\begin{equation}
  y = \mathscr{M}(\vartheta) = \mathcal{M}(\kk^{\mathrm{truth}}, \uu^{\mathrm{truth}})
\end{equation}


\subsection{Cost function definition}
Once we have access to the observations
$y \in \mathbb{R}^{N_{\mathrm{mesh}} \cdot N_{\mathrm{time}}}$, we can define a cost function $J$ as
\begin{align}
  J(\kk) &= \sum_{t=1}^{ N_{\mathrm{time}}}\sum_{i=1}^{N_{\mathrm{mesh}}}  \left(\eta_{t,i}(\kk) - y_{t, i}\right)^2 \\
         &= \|\mathcal{M}(\kk) - y\|_2^2
\end{align}
Equivalently, as mentioned~\cref{chap:inverse_problem}, by assuming
that the distribution of the (random) observation vector is known and
$Y \mid \kk \sim \mathcal{N}(\mathcal{M}(\kk), I)$ where $I$ is the
identity matrix, $J$ is proportional to the negative log-likelihood of the data.

\subsection{Optimisation without uncertainties}
\label{ssec:optim_gradient}
The optimisation is first carried using M1QN3, a version of a
gradient-descent procedure, as described
in~\cite{gilbert_numerical_1989}. We can first look to control $\zob$
at every cell of the mesh: $\kk = (\kk_1,\cdots, \kk_p)$ where
$\kk_i = \zob^i$ and $p=\num{15684}$. Due to the large number of
points whose friction can be controlled., a finite difference method
to get the gradient is unfeasible. Instead,
Tapenade~\citep{hascoet_tapenade_2013}, an Automatic Differentiation
tool has been used in order to get the gradient of the cost function
$J$ using the adjoint method, as
described~\cref{sec:calibration_adjoint_optimization}. The
optimisation procedure is stopped after \num{400} iterations, and the
estimated controlled parameter is
shown~\cref{fig:optimization_map_399}.
\Cref{fig:ctrl_true} shows the
evolution of the cost function and the squared norm of the gradient
during the optimisation procedure.

 \begin{figure}[ht]
  \centering
  \includegraphics{/home/victor/optimisation_dahu/optim_sediments/map_399.png% /home/victor/optimisation_dahu/optim_true/map_150.png
  }
  \caption{\label{fig:optimization_map_399} Optimization of $\zob$ on
    the whole space using gradient obtained via adjoint method, after
    $400$ iterations TODO: {Taille et colorbar} }
\end{figure}

 \begin{figure}[ht]
  \centering
  \input{/home/victor/optimisation_dahu/optim_sediments/ctrl_true.pgf}
  \caption{\label{fig:ctrl_true} Evolution of the cost function and the squared normed of the gradient}
\end{figure}
By comparing \cref{fig:optimization_map_399} with
\cref{fig:sediments_reduced} and \cref{fig:depth_maps}, we can have a
first overview on the links between sediments type, depth, and
\emph{identifiability}, which can be understood as the ability of the
optimisation procedure to retrieve a truth value.

On a first look, we can see that the abyssal plain (the deep region
off the Bay of Biscay) remains mostly untouched by the optimisation,
while the continental shelf, except for the English Channel, is
retrieved. In terms of sediments, the bottom of the continental shelf
is largely composed of sand.
On~\cref{fig:optimisation_type_sediments}, we can see that indeed,
points of the mesh corresponding to sands have seen their $\zob$ value
shifted toward the truth, while for silts and muds, the procedure has
not been able to identify their roughness. This is probably due to the
fact that those sediments lay at great depth, and thus have little
effect on the circulation per~\cref{eq:quadratic_friction_vonkarman}.
%
%
\begin{figure}[ht]
  \centering
  \includegraphics{\imgpath optimisation_type_sediments.pdf}
  \caption{\label{fig:optimisation_type_sediments} Results of the optimisation procedure, depending on the type of sediments. The initial value is \SI{5e-3}{\meter}}
\end{figure}

In the English Channel, similar conclusions can be drawn: the size of
the pebbles is well retrieved, but the control variable of points
mapped to gravel do seem to compensate: on the northern part of the
channel the size of the gravel is overestimated, while it is
underestimated on the southern part.  Finally the rocks appear to be
hard to capture: their assumed size, \emph{i.e.} their truth value is
significantly larger than the rest of the sediments, and they are
sparsely distributed.
%  \begin{figure}[ht]
%   \centering
%   \input{/home/victor/optimisation_dahu/optim_0_001/ctrl_0_001.pgf}
%   \caption{\label{fig:ctrl_0_001} Gradient descent procedure in misspecified case}
% \end{figure}
% \clearpage


\section{Sensitivity analysis}
\label{sec:sensitivity-analysis}
Sensitivity analysis (often abbreviated as \emph{SA}), aims at
quantififying the effect of the variation of some input variable to
the output of the model~\cite{iooss_revue_2011,janon_analyse_2012}.
Intuitively, \emph{SA} ties the variation of the input to the
variation of the output. It can then be approached at two different
scales: around a nominal value, using the gradient, and at a global
scale, by considering the inputs as random variable, and by measuring
the variance of the output.

% \subsection{Methods of Sensitivity analysis}
% \label{sec:methods_SA}
% \subsubsection{Local sensitivity analysis}
% \label{sec:loca_SA}
% Local sensitivity analysis~\cite{morio_global_2011} refers to the
% study of how a small perturbation $\delta \kk$ of a nominal value
% $\kk$ affects the output of the numerical model. As we assume that the
% numerical model is accessible through the cost function $J$, a
% straightforward way to quantify this perturbation is to consider the
% partial derivative of $J$, with respect to each component of the
% control variable $\kk=(\kk_1,\dots,\kk_p)$:
% \begin{equation}
%   \frac{\partial J}{\partial \kk_i}(\kk)
% \end{equation}
% The normalized local sensitivity at $\kk$  associated with the $i$-th component is then 
% \begin{equation}
%   \frac{{\Delta J}/{J}}{{\Delta \kk_i}/{\kk_i}} = \frac{\kk_i}{J(\kk)} \frac{\partial J}{\partial \kk_i}
% \end{equation}
\subsection{Global Sensitivity Analysis: Sobol' indices}
\label{sec:sobol-indices}
\cite{janon_analyse_2012}
The $i$-th Sobol' indice of order $1$ is defined as 
\begin{equation}
  S_i = \frac{\Var_{X_i}\left[\Ex_{Y}\left[Y \mid X_i\right]\right]}{\Var_{Y}\left[Y\right]}
\end{equation}
where $Y = J(X)$ is the random variable. These are computed using a
replicated
method~\cite{gilquin_making_2019,gilquin_echantillonnages_2016}, in
order to get bootstrap confidence interval for the first and second
order effects, and total effects
\begin{equation}
 S_{T_i} = 1 - \frac{\Var_{X_{-i}}\left[\Ex_{Y}\left[Y \mid X_{-i}\right]\right]}{\Var_{Y}\left[Y\right]}
\end{equation}
where $X_{-i} = (X_1,\dots X_{i-1},X_{i+1},\dots,X_p)$ is random vector of $p-1$ components.

% \subsection{Application to CROCO}
The bottom friction affects the ocean circulation through two factors,
as shown \cref{eq:quadratic_friction_vonkarman}, first by the bottom
roughness, $\zob$, and by the ocean depth. We are first going to
perfrom a sensitivity analysis, in order to quantify the role of each
sediment based region, without incorporating the knowledge on the
typical size of the sediment there.

\subsection{SA on the regions defined by the sediments}
\label{ssec:SA_sediments}
\begin{figure}[ht]
  \centering
  \input{\imgpath SA_sediments.pgf}
  \caption{\label{fig:SA_sediments} Global SA on the regions defined by the sediment type}
\end{figure}


\subsection{SA on the tide components}
\label{ssec:SA_tide}
\begin{figure}[ht]
  \centering
  \input{\imgpath SA_tides.pgf}
  \caption{\label{fig:SA_tides} Global SA on the different components of the tide }
\end{figure}

\cref{fig:SA_tides} shows the Sobol indices of order 1 (left), 2
(right), and the total effect indices, along with bootstrap confidence
intervals. The variation of the variable associated with the $M_2$
component of the tide has the most impact on the cost function.



\section{Robust Calibration of the bottom friction}

% \begin{figure}[ht]
%   \centering
%   \input{\imgpath SA_croco.pgf}
%   \caption{\label{fig:sobol_indices} Sobol indices obtained using replicated methods and bootstrap CI}
% \end{figure}

% \begin{figure}[ht]
%   \centering
%   \input{\imgpath distribution_minimizers.pgf}
%   \caption{\label{fig:dist_minimizers} Distribution of the minimizers, estimated using the GP constructed on $J$, and enriched using the PEI criterion}
% \end{figure}







%%%%%%%%%%%%%%%%%%%%%%%%%%%%%%%%%%%%%%%%%%%%%%%%%%%%%%%
%%%%%%%%%%%%%%%%%%%%%%%%%%%%%%%%%%%%%%%%%%%%%%%%%%%%%%%
%%%%%%%%%%%%%%%%%%%%%%%%%%%%%%%%%%%%%%
%% BIB
%%%%%%%%%%%%%%%%%%%%%%%%%%%%%%%%%%%%%%
\subfileLocal{
	\pagestyle{empty}
	\bibliographystyle{alpha}
	\bibliography{../../Bibliography}
}
\end{document}

%%% Local Variables:
%%% mode: latex
%%% TeX-master: "../../Main_ManuscritThese"
%%% End:
