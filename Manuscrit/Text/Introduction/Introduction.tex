\documentclass[../../Main_ManuscritThese.tex]{subfiles}

\subfileGlobal{
\renewcommand{\RootDir}[1]{./Text/Introduction/#1}
}

% \subfileLocal{
% \externaldocument{../../Text/Chapter2/build/Chapter2}
% \externaldocument{../../Text/Chapter3/build/Chapter3}
% \externaldocument{../../Text/Chapter4/build/Chapter4}
% \externaldocument{../../Text/Chapter5/build/Chapter5}
% \externaldocument{../../Text/Conclusion/build/Conclusion}
% }

%%%%%%%%%%%%%%%%%%%%%%%%%%%%%%%%%%%%%%
%% CHAPTER TITLE
%%%%%%%%%%%%%%%%%%%%%%%%%%%%%%%%%%%%%%

\begin{document}

\chapter*{Introduction}
\TitleBtwLines

\phantomsection
\addstarredchapter{Introduction}
\label{chap:Introduction}
% \newpage
%\minitoc
\pagestyle{introStyle}

% \subfileLocal{\pagestyle{contentStyle}}
\todo{\cite{mcwilliams_irreducible_2007,zanna_ocean_2011}}


%%%%%%%%%%%%%%%%%%%%%%%%%%%%%%%%%%%%%%%%%%%%%%%%%%%%%%%
%%%%%%%%%%%%%%%%%%%%%%%%%%%%%%%%%%%%%%%%%%%%%%%%%%%%%%%

Numerical models are widely used to study or forecast natural
phenomena and improve industrial processes. However, by essence models
only partially represent reality and sources of uncertainties are
ubiquitous (discretisation errors, missing physical processes, poorly
known boundary conditions).  Moreover, such uncertainties may be of
different nature. \cite{walker_defining_2003} proposes to consider two
categories of uncertainties:
\begin{itemize}
\item Aleatoric uncertainties, coming from the inherent variability of
a phenomenon, \emph{e.g.} intrinsic randomness of some environmental
variables
\item Epistemic uncertainties coming from a lack of knowledge about
the properties and conditions of the phenomenon underlying the
behaviour of the system under study
\end{itemize} The latter can be accounted for through the introduction
of ad-hoc correcting terms in the numerical model, that need to be
properly estimated. Thus, reducing the epistemic uncertainty can be
done through parameters estimation approaches. This is usually done
using optimal control techniques, leading to an optimisation of a well
chosen cost function which is typically built as a comparison with
reference observations.
  %
  An application of such an approach, in the context of ocean
circulation modeling, is the estimation of ocean bottom friction
parameters in~\cite{das_estimation_1991}
and~\cite{boutet_estimation_2015}.

  If parameters to be estimated are not the only source of
uncertainties, their optimal control is doomed to overfit the data,
\emph{e.g} to artificially introduce errors in the controlled
parameter to compensate for other sources. If such uncertainties are
of aleatoric nature, then the parameter estimation is only optimal for
the observed situation, and may be very poor in other configurations,
phenomenon coined as \textit{localized optimisation}
in~\cite{huyse_free-form_2001}.
  
  The calibration often takes the form of the minimisation of a
function $J$, that describes a distance between the output of the
numerical model and some given observed data, plus generally some
regularization terms.  In our study, this cost function takes two
types of arguments: $\kk\in\Kspace$ that represents the parameters to
calibrate, and $\uu\in\Uspace$, that represents the environmental
conditions.  We assume that the environmental conditions are uncertain
by nature, and thus will be modelled with a random variable $U$, to
account for these aleatoric uncertainties.  This is then the random
variable $J(k,U)$ that we want to minimize ``in some sense'' with
respect to $k$.

  Some of the optimisation under uncertainties methods rely on the
optimisation of the moments of $ k\mapsto J(k,U)$
(in~\cite{lehman_designing_2004,janusevskis_simultaneous_2010}), while
other methods are based on multiobjective problems, such as
in~\cite{baudoui_optimisation_2012,ribaud_krigeage_2018}.  These
approaches may compensate some bad performances by some very good
ones, as we are averaging over $\mathbb{U}$.

  
  We propose to compare the value of the objective function to the
best value attainable given the environmemtal conditions at this
point, with the idea that we want to be as close as possible, and as
often as possible, to this optimal value. Introducing the relative
regret, that is the ratio of the objective function by its conditional
optimum, we can define a new family of robust estimators.

  Within this family, choosing an estimator consists in favouring
either its robustness, \emph{e.g} its ability to perform well under
all circumstances, or on the contrary favour near-optimal
performances, transcribing a risk-averse or a risk-seeking behaviour
from the user.
 


%%%%%%%%%%%%%%%%%%%%%%%%%%%%%%%%%%%%%%
%% BIB
%%%%%%%%%%%%%%%%%%%%%%%%%%%%%%%%%%%%%%
\subfileLocal{
	\pagestyle{empty}
	\bibliographystyle{alpha}
	\bibliography{../../bibzotero}
}
\end{document}



%%% Local Variables:
%%% mode: latex
%%% TeX-master: "../../Main_ManuscritThese"
%%% End:
