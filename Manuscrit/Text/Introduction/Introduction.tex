\documentclass[../../Main_ManuscritThese.tex]{subfiles}

\subfileGlobal{
\renewcommand{\RootDir}[1]{./Text/Introduction/#1}
}

% \subfileLocal{
% \externaldocument{../../Text/Chapter2/build/Chapter2}
% \externaldocument{../../Text/Chapter3/build/Chapter3}
% \externaldocument{../../Text/Chapter4/build/Chapter4}
% \externaldocument{../../Text/Chapter5/build/Chapter5}
% \externaldocument{../../Text/Conclusion/build/Conclusion}
% }

%%%%%%%%%%%%%%%%%%%%%%%%%%%%%%%%%%%%%%
%% CHAPTER TITLE
%%%%%%%%%%%%%%%%%%%%%%%%%%%%%%%%%%%%%%

\begin{document}
\pagestyle{introStyle}
\chapter*{Introduction}
\TitleBtwLines

\phantomsection
\addstarredchapter{Introduction}
\label{chap:Introduction}



% \subfileLocal{\pagestyle{contentStyle}}
\todo{\cite{mcwilliams_irreducible_2007,zanna_ocean_2011}}


To understand and to be able to forecast natural phenomena is crucial
for many applications, with high social, environmental and economic
stakes.  In earth sciences especially, the modelling of the ocean and
the atmosphere is important for day to day weather forecasts,
hurricanes tracking, pollutant dispersion, or biological monitoring.

Those natural phenomena are then modelled mathematically, usually by
modelling the physical reality with the more general equations, and by
making successive reasonable assumptions and simplifications in order
to account for the relative scales of the processes involved. Indeed,
some small scales processes, such as turbulences, are notoriously hard
to model and a fine knowledge of those may not be completely relevant
for the foreseen application of the model. Such effects are not
necessarily completely overlooked, but can be parametrized, \textit{i.e.}
additional parameters can be introduced.

However, as this modelling is supposed to represent the reality, the
prediction should be compared with some data acquired through
observations. This comparison usually takes the form of the definition
of a misfit function that measures the error between the forecast and
the reality. This objective function is then minimized with respect to
some chosen
parameters~\cite{das_estimation_1991,das_variational_1992,boutet_estimation_2015}
in order to get a calibrated model.

Those calibrated models are often used to make decisions afterward,
such as emergency evacuations plans, or optimisation of the position
of turbines, or for forecasts. However, the parameters introduced are
not the only source of errors in the modelling.

As mentioned above, models are only a partial representation of the
reality and can be structurally unstable as they model chaotic
systems. Some uncertainties cannot be resolved at the model scale.
To deal with irreducible errors, \emph{i.e.} aleatoric uncertainties, ensemble simulations \cite{mcwilliams_irreducible_2007}. random errors still a \cite{zanna_ocean_2011}
%  If parameters to be estimated are not the only source of
% uncertainties, their optimal control is doomed to overfit the data,
% \emph{e.g} to artificially introduce errors in the controlled
% parameter to compensate for other sources. If such uncertainties are
% of aleatoric nature, then the parameter estimation is only optimal for
% the observed situation, and may be very poor in other configurations,
% phenomenon coined as \textit{localized optimisation}
% in~\cite{huyse_free-form_2001}.

% This can be seen as a layer of
% uncertainties which is added to the modelling.


The thesis is organized as follows
\begin{itemize}
\item in~\cref{chap:inverse_problem}, we introduce the notions of statistics and probabilities that we will use to define the calibration problem
\item in~\cref{chap:robust_estimators}, \cite{trappler_robust_2020-1}
\item in~\cref{chap:adaptative_design_gp}
\item in \cref{chap:croco} 
\end{itemize}



%%%%%%%%%%%%%%%%%%%%%%%%%%%%%%%%%%%%%%%%%%%%%%%%%%%%%%%
%%%%%%%%%%%%%%%%%%%%%%%%%%%%%%%%%%%%%%%%%%%%%%%%%%%%%%%

% Numerical models are widely used to study or forecast natural
% phenomena and improve industrial processes.
However, by essence models
only partially represent reality and sources of uncertainties are
ubiquitous (discretisation errors, missing physical processes, poorly
known boundary conditions). Moreover, such uncertainties may be of
different nature.~\cite{walker_defining_2003} proposes to consider two
categories of uncertainties:
\begin{itemize}
\item Aleatoric uncertainties, coming from the inherent variability of
a phenomenon, \emph{e.g.} intrinsic randomness of some environmental
variables
\item Epistemic uncertainties coming from a lack of knowledge about
the properties and conditions of the phenomenon underlying the
behaviour of the system under study
\end{itemize} The latter can be accounted for through the introduction
of ad-hoc correcting terms in the numerical model, that need to be
properly estimated. Thus, reducing the epistemic uncertainty can be
done through parameters estimation approaches. This is usually done
using optimal control techniques, leading to an optimisation of a well
chosen cost function which is typically built as a comparison with
reference observations.
  %
  An application of such an approach, in the context of ocean
circulation modeling, is the estimation of ocean bottom friction
parameters in~\cite{das_estimation_1991}
and~\cite{boutet_estimation_2015}.

 
  
  The calibration often takes the form of the minimisation of a
function $J$, that describes a distance between the output of the
numerical model and some given observed data, plus generally some
regularization terms.  In our study, this cost function takes two
types of arguments: $\kk\in\Kspace$ that represents the parameters to
calibrate, and $\uu\in\Uspace$, that represents the environmental
conditions.  We assume that the environmental conditions are uncertain
by nature, and thus will be modelled with a random variable $\UU$, to
account for these aleatoric uncertainties.  This is then the random
variable $J(\kk,\UU)$ that we want to minimize ``in some sense'' with
respect to $\kk$.

  Some of the optimisation under uncertainties methods rely on the
optimisation of the moments of $ \kk\mapsto J(\kk,\UU)$
(in~\cite{lehman_designing_2004,janusevskis_simultaneous_2010}), while
other methods are based on multiobjective problems, such as
in~\cite{baudoui_optimisation_2012,ribaud_krigeage_2018}.  These
approaches may compensate some bad performances by some very good
ones, as we are averaging with respect to  $\UU$.

  
  We propose to compare the value of the objective function to the
best value attainable given the environmemtal conditions at this
point, with the idea that we want to be as close as possible, and as
often as possible, to this optimal value. Introducing the relative
regret, that is the ratio of the objective function by its conditional
optimum, we can define a new family of robust estimators.

  Within this family, choosing an estimator consists in favouring
either its robustness, \emph{e.g} its ability to perform well under
all circumstances, or on the contrary favour near-optimal
performances, transcribing a risk-averse or a risk-seeking behaviour
from the user.
 


%%%%%%%%%%%%%%%%%%%%%%%%%%%%%%%%%%%%%%
%% BIB
%%%%%%%%%%%%%%%%%%%%%%%%%%%%%%%%%%%%%%
\subfileLocal{
	\pagestyle{empty}
	\bibliographystyle{alpha}
	\bibliography{../../bibzotero}
}
\end{document}



%%% Local Variables:
%%% mode: latex
%%% TeX-master: "../../Main_ManuscritThese"
%%% End:
