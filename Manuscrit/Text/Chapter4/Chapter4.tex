\documentclass[../../Main_ManuscritThese.tex]{subfiles}

\subfileGlobal{
\renewcommand{\RootDir}[1]{./Text/Chapter4/#1}
}
% \DeclareMathOperator{\IMSE}{\mathrm{IMSE}}
% \newcommand{\IMSE}{\mathop*{\mathrm{IMSE}}}

% For cross referencing
\subfileLocal{
\externaldocument{../../Text/Introduction/build/Introduction}
\externaldocument{../../Text/Chapter2/build/Chapter2}
\externaldocument{../../Text/Chapter3/build/Chapter3}
\externaldocument{../../Text/Chapter5/build/Chapter5}
\externaldocument{../../Text/Conclusion/build/Conclusion}
}
\newcommand\imgpath{/home/victor/acadwriting/Manuscrit/Text/Chapter4/img/} 

%%%%%%%%%%%%%%%%%%%%%%%%%%%%%%%%%%%%%%
%% CHAPTER TITLE
%%%%%%%%%%%%%%%%%%%%%%%%%%%%%%%%%%%%%%

\begin{document}
% \dominitoc
% \faketableofcontents
% \subfileLocal{\setcounter{chapter}{2}}
\chapter{Adaptative design enrichment for calibration using Gaussian Processes}
\label{chap:adaptative_design_gp}
\minitoc
\newpage
\subfileLocal{\pagestyle{contentStyle}}
%%%%%%%%%%%%%%%%%%%%%%%%%%%%%%%%%%%%%%%%%%%%%%%%%%%%%%%
%%%%%%%%%%%%%%%%%%%%%%%%%%%%%%%%%%%%%%%%%%%%%%%%%%%%%%%


%%%%%%%%%%%%%%%%%%%%%%%%%%%%%%%%%%%%%%%%%%%%%%%%%%%%%%
%%% SECTION 0 : Introduction	   %%%

\section{Introduction}

In the previous chapter, we introduced different notions of robustness for the estimation of a calibration parameter. Those objectives usually require the evaluation of various integrals, and the perform several optimizations. A large number of model evaluations is then needed, but it may bring some practical issues, as numerical models are usually very expensive to run in terms of computer resources. Indeed, for most realistic physical simulations, the programs have to solve systems of PDEs over large discretized domains. Even though the programs are optimized and parallelized to take best advantage of high-performance computers, the time required to compute the quantities of interest may range from a few seconds to days. Because of that, methods requiring a large number of runs of the model for exhaustivity should be avoided.

% On the other hand, one other issue commonly encountered is the dimensionality of the parameter spaces. The number of samples required to get the fixed coverage of a space scales exponentially with the number of dimensions: this is the \emph{curse of dimensionality}. This motivates the ongoing research on dimension reduction. Some classical techniques, such as Principal Component analysis (PCA) \cite{jolliffe_principal_2002}, and Singular Value Decomposition (SVD) look . \cite{zahm_certified_2018} look to find the subspaces where the Bayesian update has the most effects.\cite{blanchet-scalliet_specific_2017,ribaud_krigeage_2018}

% \subsubsection{Sensitivity analysis}

 In this chapter, we will focus on the use of \emph{surrogate models} to solve robust optimization problems, according to some of the criteria introduced in the previous chapter.
\begin{definition}[Surrogate model]
  Let $f: \Xspace \rightarrow \mathbb{R}$ be a function representing the computation of a quantity of interest. A \emph{surrogate}, or \emph{metamodel}, or \emph{emulator} of $f$, say $g$, is a function from $\Xspace$ to $\mathbb{R}$ which possesses two main properties:
  \begin{itemize}
  \item $g$ is an approximation of $f$
  \item $g$ is cheaper to evaluate than $f$ 
  \end{itemize}
\end{definition}
We will focus exclusively on Kriging \cite{krige_statistical_1951,matheron_traite_1962}, or Gaussian Process regression, but other methods can be used to solve such problems. Polynomial Chaos regression for instance \cite{wiener_homogeneous_1938,xiu_wiener--askey_2002,sudret_polynomial_2015,miranda_adjoint-based_2016}.
In this chapter, after defining the usual kriging equations for Gaussian Process Regression, we are going to introduce a few classic and useful criteria in \cref{sec:enrichment_strategies} for global optimization and/or exploration of an unknown function $f$. Afterwards, we are going to focus on robust optimization, by splitting the input space in $\Kspace$ and $\Uspace$, and introduce other strategies to take advantage of the nature of the surrogate in~\cref{sec:robust_criteria_gp}.

% Surrogate models are often built using initial evaluations of $f$.
\section{Gaussian process regression}
In the following, we will introduce a generic function $f$, that maps a space $\Xspace$ to $\mathbb{R}$. Depending on the application, $\Xspace = \Kspace$ or $\Xspace = \Kspace \times \Uspace$. This function is unknown, and supposedly expensive to evaluate.

\subsection{Random processes}
Let us assume that we have a map $f$ from a $p$ dimensional space to $\mathbb{R}$:
\begin{align}
  \begin{array}{rrcl}
    f: & \mathbb{X} \subset \mathbb{R}^p& \longrightarrow & \mathbb{R} \\
       & x & \longmapsto & f(x)
  \end{array}
\end{align}

This function is assumed to have been evaluated on a design of $n$ points, $\mathcal{X} = \left\{ (x_i, f(x_i) \right)\}_{1\leq i\leq n}$, called the \emph{initial design}. For notational simplicity, we write $x\in \mathcal{X}$ if $(x, f(x)) \in \mathcal{X}$.
As this function is unknown, there is uncertainty on the values outside of the initial design, which can be classified as \emph{epistemic} since it can be reduced by directly evaluating the function.
We can model this uncertainty on functions by defining random processes:
\begin{definition}[Random process]
  Let $(\Omega, \mathcal{F}, \Prob)$ be a probability space, and $\Xspace\subset \mathbb{R}^p$.
  A random process $Z$ is a collection of random variables indexed on $\Xspace$, so for each $x \in \Xspace$, $Z(x)$ is a random variable:
 \begin{equation}
  \begin{array}{rcl}
    Z: \mathbb{X} & \longrightarrow & \left(\Omega \rightarrow \mathbb{R} \right)\\
    x& \longmapsto & Z(x)
  \end{array}
\end{equation}
A sample from this random process, that is $Z(\cdot)(\omega)$ for $\omega \in \Omega$ will be shortened as $Z(\cdot, \omega)$ for notational purpose, and is called a \emph{sample trajectory}, or a \emph{sample path}.
\end{definition}
From a Bayesian point of view, such a random process can act $Z$ as a prior on the function $f$, or in other words, $f$ can be thought as a particular sample path of $Z$.
Evaluating the function at an additional point $x \notin \mathcal{X}$ provides new information on the random process, and we can update our belief on $f$.


In this work, we are going to focus exclusively on a specific type of random process, namely, the Gaussian process (abbreviated as GP), but other types of random process can be encountered in the literature: Student t-processes in~\cite{shah_student-t_2014} are introduced as alternatives to GP, or various graphical models such as Gaussian and Markov Random Fields in~\cite{bishop_pattern_2006,li_markov_2009}.

\begin{definition}[Gaussian process]
  Let $Z$ be a random process on $\Xspace$, i.e. a collection of random variables indexed by $\Xspace$. $Z$ is a Gaussian process (GP) if any finite number of those random variables have a multivariate joint Gaussian distribution.  A more thorough description of Gaussian processes and their applications can be found in~\cite{rasmussen_gaussian_2006}.
  In that case, $Z$ is uniquely defined by its mean function $m_Z:\Xspace \rightarrow \mathbb{R}$ and its covariance function $C_Z:\Xspace \times \Xspace \rightarrow \mathbb{R}$:
  \begin{align}
    m_Z(x) &= \Ex\left[Z(x)\right] \\
    C_Z(x, x^{\prime}) &= \Cov[Z(x), Z(x^{\prime})]
  \end{align}
  and we write $Z \sim \GP(m_Z, C_Z)$.
  Due to the definition of a GP, we also have for all $x\in \Xspace$
  \begin{equation}
    Z(x) \sim \mathcal{N}(m_Z(x), \sigma^2_Z(x)) 
  \end{equation}
  with $\sigma_Z^2(x) = C_Z(x, x)$
\end{definition}

Let $Z$ be a GP with a known covariance function. The construction of a covariance function verifying certain properties will be discussed~\cref{sec:cov_fun}.
Based on the initial design $\mathcal{X}$, we can construct a surrogate for the unknown function $f$ by conditioning the GP on the design which comprise some evaluations of $f$.   

\subsection{Kriging equations}
\label{sec:linear_estimation}
Given a Gaussian process $Z$ as a prior on $f$, and by conditioning it by the initial design $\mathcal{X} = \{(x_i, f(x_i)\}_{1 \leq i \leq n}$, the conditioned random process is still a GP:
\begin{equation}
  \label{eq:GP_Z_cond}
  Z \mid \mathcal{X} \sim \GP(m_{Z\mid \mathcal{X}}, C_{Z\mid \mathcal{X}})
\end{equation}
Given the nature of $Z$ and the initial design, the joint distribution of the GP at the points of the design and the GP at an unobserved point $x \in \Xspace$ is
\begin{equation}
  \label{eq:GP_joint_distrib}
  \begin{pmatrix}
    Z(\mathbf{x}) \\
    Z(x)
  \end{pmatrix} \sim
  \mathcal{N}\left(
    \begin{pmatrix}
      \mu_Z \\
      m_{Z}(x)
    \end{pmatrix} ;
    \begin{pmatrix}
      \mathbf{K}_{\mathcal{X}} & K_{\mathcal{X}}(x) \\
       K_{\mathcal{X}}(x)^T & C_Z(x, x)
    \end{pmatrix}
\right)
\end{equation}
where
\begin{align}
  K_{\mathcal{X}}(x) &= \left(C_Z(x, x_1),C_Z(x, x_2),\dots,C_Z(x,x_n)\right)^T \\
  \mathbf{K}_{\mathcal{X}} &= \left(C_Z(x_i, x_j)\right)_{1 \leq i,j \leq n}
\end{align}
and $\mu_Z = \Ex\left[Z(x)\right]$ is the mean of the unconditionned GP. When this mean is assumed to be known, the Kriging procedure is qualified of \emph{Simple Kriging}, otherwise, we often talk about \emph{Ordinary Kriging}. Finally, when this mean is a deterministic function (that may be estimated), we talk about \emph{Universal Kriging}~\cite{le_riche_introduction_2014}. In the following, we consider Simple Kriging.

Using the properties of the multivariate distribution, the GP conditioned on the observations of~\eqref{eq:GP_Z_cond} has mean and covariance function defined by
\begin{align}
  m_{Z \mid \mathcal{X}}(x) &= m_Z(x) + K_{\mathcal{X}}(x)^T \mathbf{K}_{\mathcal{X}}^{-1}(f(\mathbf{x}) - \mu_Z ) \\
  C_{Z\mid \mathcal{X}}(x, x^\prime) &= C_Z(x, x^\prime)  - K_{\mathcal{X}}(x)^T\mathbf{K}_{\mathcal{X}}^{-1}K_{\mathcal{X}}(x^\prime)
\end{align}
These are called the \emph{Kriging equations}.


Given the fact that a conditioned GP is still a GP, at a point $x \in \Xspace$, we have
\begin{equation}
  \label{eq:cond_gp}
  Z(x) \mid \mathcal{X} \sim \mathcal{N}\left(m_{Z\mid \mathcal{X}}(x), \sigma^2_{Z\mid \mathcal{X}}(x)\right) \quad \text{ with } \quad \sigma^2_{Z\mid \mathcal{X}}(x) = C_{Z \mid \mathcal{X}}(x, x)
\end{equation}
The function $m_{Z \mid \mathcal{X}}: \Xspace \rightarrow \mathbb{R}$ can then be defined as a surrogate for $f$. This surrogate provides an interpolation of the unkown function $f$, since for $x \in \mathcal{X}$, $f(x) = m_{Z \mid \mathcal{X}}(x)$.
In addition to that interpolation property, for points not in the design, we have a measure of the epistemic uncertainty modelled by the normal r.v.\ in~\eqref{eq:cond_gp}.
For a few training points and their evaluations, we represented the GP regression of an unknown function $f$ \cref{fig:example_GP}.

\begin{figure}[ht]
  \centering
  \input{\imgpath example_GP.pgf}
  \caption[Illustration of GP regression]{\label{fig:example_GP} Example of a Gaussian Process, as a surrogate for a function $f$ evaluated at 4 inputs. The shaded regions correspond to the regions $m_Z \pm i \cdot \sigma_Z$ for $i=1, 2,3$.}
\end{figure}

From a computational point of view, GP are relatively cheap to handle:
\begin{itemize}
\item If the covariance function $C_Z$ is known, the most expensive step in the estimation of $m_{Z\mid \mathcal{X}}$ and $C_{Z \mid \mathcal{X}}$ is the inversion of the matrix $\mathbf{K}_{\mathcal{X}}$. This matrix does not depend on the points $x$ upon which we wish to apply $m_{Z\mid \mathcal{X}}$, thus its inversion need only to be performed once.
\item In order to get a sample path from this GP, one need to be able to sample from a multivariate normal distribution, and thus only need to compute the Cholesky decomposition of the conditioned covariance matrix.
\end{itemize}
However, if some points of the design are very close to each other, the covariance matrix can be ill-conditioned. In this case,
for numerical stability, we can add \emph{nugget} effect to $\mathbf{K}_{\mathcal{X}}$.~\eqref{eq:GP_joint_distrib_nugget} becomes:
\begin{equation}
  \label{eq:GP_joint_distrib_nugget}
  \begin{pmatrix}
    Z(\mathbf{x}) \\
    Z(x)
  \end{pmatrix} \sim
  \mathcal{N}\left(
    \begin{pmatrix}
      \mu_Z \\
      m_{Z}(x)
    \end{pmatrix} ;
    \begin{pmatrix}
      \mathbf{K}_{\mathcal{X}} + \varsigma^2\mathrm{Id}& K_{\mathcal{X}}(x) \\
       K_{\mathcal{X}}(x)^T & C_Z(x, x)
    \end{pmatrix}
\right)
\end{equation}
From a probabilistic point of view, we can see the nugget effect as an added Gaussian noise when evaluating the function,\emph{ i.e.} that we are observing $f(x_i)+\epsilon$ with $\epsilon \sim \mathcal{N}(0, \varsigma^2)$. When $\varsigma^2 \neq 0$, the GP is no longer interpolant, as $\Var\left[Z(x) \mid \mathcal{X}\right] \neq 0$ for $x\in \mathcal{X}$.

\subsection{Covariance functions}
\label{sec:cov_fun}
In the previous section, we described the equations to solve to get the surrogate $m_{Z\mid \mathcal{X}}$ of $f$ based on GP.
The kriging equations are based on the covariance function $C_Z$.
A covariance is said to be stationary, if for all $x, x^{\prime} \in \Xspace$, the covariance of the GP between those two points depends only on the  difference $h = x-x^{\prime}=(h_1,\cdots, h_{\dim\Xspace})$. In that case, we will write $C_Z(x, x^{\prime}) = C_Z(h)$.
For multidimensional problems, covariance functions are usually chosen as the product of one dimensional covariance functions:
\begin{equation}
  C_Z(h) = s^2\prod_{i=1}^{\dim \Xspace} C_i(h_i;l_i)
\end{equation}
These covariance functions introduce an additional parameter $l$ of dimension $\dim \Xspace$, and a variance parameter $s^2$.
$l$ called the \emph{length scale}, that measure the length of the influence of each variable on its surroundings. If the length scales are all equals, the covariance kernel is said \emph{isotropic}. Otherwise, the kernel is \emph{anisotropic}.

A few common stationary $1D$-covariance functions are introduced~\cref{tab:common_cov_fc}.

  \begin{table}[ht]
    \centering
    \begin{tabular}{lrc}
      \toprule
      Name & $C(h;l)$ & Regularity of sample paths\\ \midrule
      Gaussian & $\exp\left(- \frac{h^2}{2 l^2}\right)$ & $C^{\infty}$\\
      Exponential &$\exp\left(- \frac{\lvert h \rvert}{l}\right)$ & $C^0$  \\
      Matérn 3/2 & $\left(1 + \sqrt{3}\frac{h}{l}\right)\exp\left(-\sqrt{3}\frac{h}{l}\right)$ & $C^1$\\
      Matérn 5/2 & $\left(1+ \sqrt{5}\frac{h}{l} + \frac{5}{3}\frac{h^2}{l^2}\right) \exp\left(-\sqrt{5}\frac{h}{l}\right)$ & $C^2$\\ \bottomrule
    \end{tabular}
    \caption{\label{tab:common_cov_fc} Common stationary covariance functions}
  \end{table}

  The difference between those covariance functions that motivates the choice of a particular covariance function is the assumption upon the regularity of the sample paths. For example, if the unknown function $f$ is assumed to be infinitely differentiable, a Gaussian kernel is suited for the modelling. One common choice is the Matérn kernel of order $5/2$, so that the samples paths are twice-differentiable.

  
\begin{figure}[ht]
  \centering
  \input{\imgpath covariance_functions.pgf}
  \caption[Common covariance functions for GP]{\label{fig:cov_fc_examples} Common covariance functions for GP regression. The right plots show (unconditioned) sample paths for those different covariance functions with same length scale.}
\end{figure}

\subsection{Initial design and validation}

Those $(\dim \Xspace + 1)$ hyperparameters have to be estimated based on the training set $\mathcal{X}$. This is usually done by MLE~(see for instance~\cite{ribaud_robustness_2019}), or by cross-validation (see~\cite{ginsbourger_note_2009}).
In order to construct a Gaussian process, or a general metamodel for that matter, the model must first be evaluated on an initial design which should present good space-filling properties. An usual choice is to use Latin Hypercube Sampling (LHS), and an usual rule of thumb for the initial number of points of $\Xspace$ to be evaluated is about $10\cdot\dim \Xspace $. In order to validate the GP, \emph{i.e.} to verify that the GP does not overfit the data, one can use cross-validation as described in~\cite{dubrule_cross_1983}.



\section{Enrichment strategies for Gaussian Processes}
\label{sec:enrichment_strategies}
\subsection{1-step lookahead strategies}
For a unknown function $f$, a GP is initially constructed based on a design $\mathcal{X} = \left\{\left(x_1,f(x_1)\right), \dots, \left(x_n, f(x_n)\right)\right\}$, that consists of $n$ points of $\Xspace$, and their corresponding evaluations. This GP is denoted $Z \mid \mathcal{X}$ and is defined as
\begin{equation}
  \label{eq:ZgivenXGP}
  Z\mid \mathcal{X} \sim \mathcal{N}(m_{Z\mid\mathcal{X}}(x),\sigma^2_{Z\mid\mathcal{X}}(x))
\end{equation}
For notational convenience, the conditioning with respect to $\mathcal{X}$ will be omitted if the experimental design is clear from the context.
Stepwise Uncertainty Reduction is based on the definition of a criterion, say $\kappa_n$, that measures in a way the uncertainty upon a certain objective associated with the GP and with $f$. This criterion is then maximized in order to select the next point to evaluate on the true function $f$.
\begin{equation}
  x_{n+1} = \argmax_{x\in\Xspace} \kappa_n(x) = \argmax_{x\in\Xspace} \kappa(x; Z \mid \mathcal{X})
\end{equation}
The GP is updated according to the new evaluation which is added to the design. The general principle of these strategies is summarized \cref{alg:SUR_strat}
% This new point is then evaluated by $f$, and the pair is added to the design: $\mathcal{X} \gets \mathcal{X} \cup \{(x_{n+1}, f(x_{n+1}))\}$.

\begin{algorithm}
  \caption{\label{alg:SUR_strat} SUR strategy}
\begin{algorithmic}
\REQUIRE Initial design $\mathcal{X}_0$, criterion function $\kappa$
\STATE Fit $Z$, a GP using the design $\mathcal{X}_0$
\STATE $n \leftarrow 0$
\WHILE{Stopping criterion not met or max budget evaluation not reached}
\STATE $x_{n+1} \leftarrow \argmax_{x \in \Xspace} \kappa(x; Z \mid \mathcal{X}_{n})$
\STATE Evaluate $f(x_{n+1})$
\STATE $\mathcal{X}_{n+1} \leftarrow \mathcal{X}_n \cup \left\{\left(x_{n+1}, f(x_{n+1})\right)\right\}$
\STATE Update the GP using $\mathcal{X}_{n+1}$
\STATE $n \leftarrow n + 1$
\ENDWHILE
\end{algorithmic}
\end{algorithm}


\subsection{Batch selection}
Due to the structure of many computer codes, it may sometimes be beneficial to evaluate a batch of points, instead of selecting a unique point at each step of the procedure.
\paragraph{Adaptation of 1-step criteria}

\cite{ginsbourger_kriging_2010} introduces a few ways to  make the best use of parallelism in enrichment strategies based on GP. The expected improvement for instance can be adapted to select $q$ points at each iteration, by the optimization of a modified criterion $q-\mathrm{EI}$ on $\Xspace^q$. More generally, using a greedy approach, we can modify $1$-step criteria to $q-$step criteria: let $\kappa$ be an enrichment criterion, $\mathcal{X}_n$ the design at the iteration $n$,  and we define for $1\leq i \leq q$, 
\begin{equation} 
 x^{n+i} = \argmax_{x \in \Xspace} \kappa\left(x; Z \mid \mathcal{X}_n\cup\left\{(x^{n+1}, z^{n+1}), \dots ,( x^{n+i-1}, z^{n+i-1})\right\} \right)
\end{equation}
The points $(x^{n+1},\dots x^{n+q})$ and their evaluations are then added to the design, the GP is updated.


In this case, for the $i$-th point to select, we maximize the criterion constructed using the augmented design $\mathcal{X}_n \cup  \left\{(x^{n+1}, z^{n+1}), \dots ,( x^{n+i-1}, z^{n+i-1})\right\}$. Depending on the choice of the realizations $y^{n},\dots,z^{n+i-1}$. We can augment the design in different ways:
\begin{itemize}
\item $\left(x^j, z^j\right) = \left(x^j, m_Z(x^j)\right)$ is the \emph{Kriging believer} (KB) method, as we the observation is replaced by the Kriging prediction
\item $\left(x^j, z^j\right) = \left(x^j, L\right)$ leads to the \emph{Constant liar} (CL) method, as we replace the observation with an arbitrary choice of $L$, such as $\min f$, $\max f$, $\mathrm{mean} f$.
% \item  $\left(x^j, z^j\right) = \left(x^j, Z(x^j)\right)$ assumes unknown the realisation. , require the evaluation of several expensive integrals in order to evaluate $\kappa$, so it may not be suited for realistic applications
\end{itemize}

\paragraph{Sampling-based methods}
\label{sec:sampling_based_criterion}
Instead of adapting 1-step criteria, we can rely on sampling in order to get $q$ points of $\Xspace$ to evaluate.
This technique is described in~\cite{dubourg_reliability-based_2011}, and use the criterion $\kappa$ as a pdf, in order to sample points according to their interest.
Let $\kappa$ be a 1-step criterion, and let $\bar{\kappa}(x) = \frac{\kappa(x)}{\int_{\Xspace}\kappa(u)\,\mathrm{d}u}$ be the normalized criterion, so that $\int_{\Xspace} \bar{\kappa}=1$. In this case, $\bar{\kappa}$ can be seen as a density.  Using an appropriate sampler, we can generate $N$ iid samples from this criterion $S=\{x_i\}_{1\leq i \leq N}$. 
  
However, as $N$ should be large, there is no point in evaluating all the samples in $S$. Using statistical reduction of the samples, we can find a reduced number of points, which are the most representative of the $N$ samples. This is usually done by clustering algorithms, especially those who have a fixed number of clusters such as the KMeans algorithm \cite{macqueen_methods_1967}, in order to select precisely $q$ points, number chosen in compliance with the wanted batch size. We then look for the closest points to the centroids, among those sampled originally, as the centroid can be located in a region of low or even zero $\bar{\kappa}$.
Such a procedure is described~\cref{alg:sampling_enrichment}.
\begin{algorithm}
  \caption{Enrichment of the design using sampling}
  \label{alg:sampling_enrichment}
\begin{algorithmic}
  \REQUIRE Initial design $\mathcal{X}_0$, normalized criterion function $\bar{\kappa}$
  \REQUIRE number of points to evaluate each iterations $q$, number of samples $N$
\STATE Fit $Z$, a GP using the design $\mathcal{X}_0$
\STATE $n \leftarrow 0$
\WHILE{Stopping criterion not met or max budget evaluation not reached} 
\STATE Sample $N$ points $S = \{x^s_i\}_{1 \leq i \leq N}$ according to the distribution with pdf $\bar{\kappa}$
\STATE Get the $q$ centroids $\{x_i^{\mathrm{center}}\}_{1 \leq i \leq q}$ using KMeans on $S$
\FOR{$j=1$ to $q$} % \COMMENT{Find the $N_s$ closest points}
\STATE $x_{n+j}=\argmin_{x \in S \setminus \{x_{n+1}, \dots, x_{n+j-1}\}} \|x - x_j^{\mathrm{center}} \|$
\ENDFOR
\STATE Evaluate $f(x_{n+j})$
\STATE $\mathcal{X}_{n+q} \leftarrow \mathcal{X}_n \cup \left\{\left(x_{n+1}, f(x_{n+1})\right),\dots,  \left(x_{n+q}, f(x_{n+q})\right)\right\}$
\STATE Condition the GP according to $\mathcal{X}_{n+q}$
\STATE $n \leftarrow n + q$
\ENDWHILE
\end{algorithmic}
\end{algorithm}
One central sampling based method is sampling in the region of points which cannot be classified with high enough probability

We are going to introduce a few criteria which are used for three different problems involving an unknown function $f$:
\begin{itemize}
\item Exploration of the input space~\cref{sec:exploration_criteria}
\item Global optimization of the function~\cref{sec:GP_optimization_criteria}
\item Estimation of volumes, and level-set~\cref{sec:GP_vol_estim} 
\end{itemize}

\subsection{Criteria for exploration of the input space}
\label{sec:exploration_criteria}
We are going to introduce first some common criteria of enrichment, that aim at exploring the input space $\Xspace$.

\paragraph{Maximum of variance}
A measure of uncertainty on the GP is $\max_{x\in\Xspace} \sigma_{Z \mid \mathcal{X}}^2(x)$, the maximum value of the prediction variance on the space.
A simple criterion is to select and evaluate the point corresponding to this maximum of variance:
\begin{equation}
  x_{n+1} = \argmax_{x\in \Xspace} \kappa_n(x) = \argmax_{x \in \Xspace} \sigma^2_{Z \mid \mathcal{X}}(x)
\end{equation}
This criterion by its simplicity is easy to implement, as the prediction variance is cheap to compute given a GP, and does not depend directly on the evaluations of the function $f(x_i)$, uniquely on the distance between the inputs points and the covariance parameters.

\paragraph{Integrated Mean Square Error}
The prediction variance is directly given by $\sigma^2_{Z\mid \mathcal{X}}$ and represents the uncertainty on the Gaussian regression. To summarize this uncertainty on the whole space $\mathcal{X}$, we define the Integrated Mean Square Error (IMSE)\cite{sacks_designs_1989}
 as
\begin{equation}
  \IMSE(Z \mid \mathcal{X}) = \int_{\Xspace} \sigma_{Z\mid\mathcal{X}}^2(x)\,\mathrm{d}x
\end{equation}
For practical reasons, we can consider to integrate the MSE only on a subset $\mathfrak{X}\subset \mathcal{X}$ that yields
\begin{equation}
  \IMSE_{\mathfrak{X}}\left(Z \mid \mathcal{X} \right) = \int_{\mathcal{X}} \sigma^2_{Z\mid \mathcal{X}}(x)  \mathbbm{1}_{\mathfrak{X}}(x)\,\mathrm{d}x = \int_{\mathfrak{X}} \sigma^2_{Z\mid \mathcal{X}}(x)\,\mathrm{d}x
\end{equation}

Unfortunately, exact evaluation of this integral is impossible, so it needs to be approximated using numerical integration, such as Monte-carlo or quadrature rules
\begin{equation}
  \IMSE(Z\mid \mathcal{X}) \approx \sum_{i=1}^{n_{\mathrm{quad}}} w_i \sigma^2_{Z\mid\mathcal{X}}(x_i)
\end{equation}
for $n_{\mathrm{quad}}$ quadrature points $x_i$ associated with weights $w_i$.


For a given $x\in \Xspace$ and an outcome $z=f(x)\in\Yspace$, the augmented design is defined as $\mathcal{X} \cup \left\{(x, z)\right\}$, and the IMSE of the augmented design is $\IMSE\left(Z \mid \mathcal{X} \cup \left\{(x, z)\right\} \right)$.
Before the actual experiment though, $z$ is unknown, but we can model it by its distribution given by the GP (per~\cref{eq:ZgivenXGP}). So for a given candidate $x$, the IMSE when evaluating the point $x$ will be on average
\begin{equation}
  \label{eq:IMSE_augmented}
  \Ex_{Z(x)}\Big[\IMSE\big(Z \mid \mathcal{X} \cup \left\{(x, Z(x))\right\} \big)\Big]
\end{equation}
where the expectation is to be taken with respect to the random variable $Z(x)$. As each scenario requires to fit a GP, and to compute the IMSE, a precise evaluation is quite expensive. A strategy found for instance in~\cite{villemonteix_informational_2006} is to take $M$ possible outcomes for $Z(x)$, corresponding to evenly spaced quantiles of its distribution, or using Gauss-Hermite quadratures~\cite{bernard_methodes_2019} in order to take advantage of the Gaussian nature of $Z(x)$.

The hyperparameters of the GP should not be reevaluated when augmenting the design, in order to get comparable values for the IMSE.\@

Finally, we can maximize a criterion which is the opposite of the expression in~\cref{eq:IMSE_augmented} to enrich the design:
\begin{equation}
  x_{n+1} = \argmax_{x\in \Xspace}-\Ex_{Z(x)}\Big[\mathrm{IMSE} \big(Z \mid \mathcal{X} \cup \left\{(x, Z(x))\right\} \big)\Big]
\end{equation}

% \tocheck{Et si au lieu d'estimer l'espérance, on choisit un échantillon. Stochastic simulation ?}

\subsection{Optimization oriented criteria}
\label{sec:GP_optimization_criteria}
The criteria we detailed above aim at reducing the epistemic uncertainty modelled through the Gaussian Process. In other words, we try to improve our knowledge on the unknown function globally. We are now going to evoke a few criteria which are driven by the global optimization of the function.

Those methods usually aim at striking a balance between \emph{exploration}, exploring the whole space $\Xspace$, and \emph{intensification}, evaluating the function near its optimum.
Let $f$ be the unknown function, and $Z$ be a GP constructed based on an initial design $\mathcal{X} = \{(x_i, f(x_i))\}$.
\paragraph{Probability of improvement}
We are first going to introduce the probability of improvement $\mathrm{PI}$, which is the probability that the GP is smaller than a threshold $f_{\min}$. Due to the Gaussian nature of $Z(x)$, this probability can be written in closed form using $\Phi = F_{\mathcal{N}(0, 1)}$ the cdf of a centered standard Gaussian r.v.\
\begin{align}
  \mathrm{PI}(x) &= \Prob% _{Z(x)}
                   \left[Z(x) < f_{\min}\right] \\
                 &= \Phi\left(\frac{m_Z(x) - f_{\min}}{\sigma_Z(x)}\right)
\end{align}
This threshold can have different forms
\begin{itemize}
\item $f_{\min} = \min_{i} f(x_i)$: the GP is compared with the current minimal value reached by the function
\item $f_{\min} = \min_i f(x_i) + \epsilon$: by introducing a small tolerance $\epsilon$, we encourage exploration instead of intensification.
\end{itemize}
Using the probability of improvement tends to select points quite close to the point evaluated so far, thus does favor intensification at the expense of exploration.
\begin{equation}
  x_{n+1} = \argmax_{x\in \Xspace} \mathrm{PI}(x)
\end{equation}
\paragraph{Expected improvement and EGO}
One of the most common criteria for global optimization is the \emph{Expected Improvement} (EI)~\cite{mockus_bayesian_1974}, and the SUR strategy using it as a enrichment criterion is called \emph{Efficient Global Optimization} (EGO)~\cite{jones_efficient_1998}.
Analogously to the probability of improvement, we define the improvement $I(x)$ as the random variable defined as
\begin{equation}
  \label{eq:def_improvement}
  I(x) = {\left[f_{\min} - Z(x)\right]}_+
\end{equation}
where $[y]_+ = \max(y, 0)$.
The Expected Improvement $\mathrm{EI}$ is 
\begin{align}
  \label{eq:def_ei}
  \mathrm{EI}(x) = \Ex[I(x)]  = \Ex\left[\left[f_{\min} - Z(x) \right]_+\right]
\end{align}
Again, a closed form is available to compute the expected improvement, that does not require the direct evaluation of the expectation \cref{eq:def_ei}.
\begin{equation}
  \mathrm{EI}(x) = \left(f_{\min} - m_Z(x)\right) \Phi\left(\frac{f_{\min} - m_Z(x)}{\sigma_Z(x)}\right) + \sigma_Z(x) \phi\left(\frac{f_{\min} - m_Z(x)}{\sigma_Z(x)}\right)
\end{equation}
The EI is then quite easy to evaluate, and furthermore, it is possible to compute the gradient of the EI and used them for the optimization as done in~\cite{pardalos_differentiating_2015}.
\begin{equation}
  x_{n+1} = \argmax_{x\in \Xspace} \mathrm{EI}(x)
\end{equation}
\paragraph{IAGO}
\label{ssec:IAGO} Another criterion worth mentioning is a criterion based on the distribution of the minimizers~\cite{villemonteix_informational_2006,hennig_entropy_2011}.
Let $y_i$ be a sample path of $Z$, and let $x_i^*$ the global minimizer of $y_i$.
We denote then $X^*$ the random variable corresponding to the global minimizer of $Z$.
We consider the differential entropy of $X^*$ introduced \cref{def:KL_entropy}, given the augmented design $\mathcal{X} \cup \left\{\left(x,Z(x)\right)\right\}$: $H[X^*\mid \mathcal{X} \cup \left\{(x, Z(x))\right\}]$.
 So at each step, we choose the point that gives the smallest expected uncertainty on the location of the global minimizers of the sample paths.
The criterion can then be written as
\begin{equation}
  \kappa_{\mathrm{IAGO}}(x \mid \mathcal{X})=-\Ex_{Z(x)}\Big[H\left[X^* \mid \mathcal{X} \cup \left\{(x, Z(x))\right\} \big)\right]\Big]
\end{equation}


\Cref{fig:example_optimization_criteria} shows different criteria introduced before for an unknown function.
\begin{figure}[ht!]
  \centering
  \input{\imgpath example_optimization_criteria.pgf}
  \caption[Optimization criteria for GP]{\label{fig:example_optimization_criteria} Example of optimization criteria. At this iteration, $\mathrm{EI}$ and $\mathrm{PI}$ aim toward intensification, while $\mathrm{IAGO}$ and the maximum of variance favorize exploration}
\end{figure}


 \clearpage
\subsection{Contour and volume estimation}
\label{sec:GP_vol_estim}

\subsubsection{Probability of coverage}
We can be interested in the estimation of a volume given by an unknown function. Given $\Prob_{X}$ a probability measure on $\Xspace$, and $B\subset \mathbb{R}$, we can compute $V=\Prob_{X}\left[f^{-1}(B)\right]$. For $B = [T, \infty[$, this is equivalent to compute the volume of the \emph{excursion set} of $f$ above $T$: $V=\Prob_{X}[f(X) \leq T]$.


% Let us start by introducing diverse tools based around Vorob'ev expectation of closed sets (\cite{el_amri_analyse_2019,heinrich_level_2012,vorobyev_new_2003}). 
Let $Z$ be a GP, indexed by $\Xspace$ with continuous sample paths. For a measurable set $B\subset \mathbb{R}$, $A = Z^{-1}(B)$ is a random closed set.
We define $\pi_A$ its probability of coverage:
\begin{equation}
  \pi_A(x) = \Prob\left[x\in A\right] = \Prob[Z(x) \in B]
\end{equation}
For a given $x\in\Xspace$, the event ``$x$ belongs to $A$'' happens with probability $\pi_A(x)$, thus has variance
\begin{equation}
\mathscr{V}_A(x)=\pi_A(x)(1 - \pi_A(x))
\end{equation}
We can also define the probability of missclassification, which is
\begin{equation}
\mathscr{P}_{\mathrm{mis}}(x) = \min(\pi_A(x), 1-\pi_A(x))
\end{equation}
For both $\mathscr{P}_{\mathrm{mis}}$ and $\mathscr{V}_A$, the maximum is reached for $\pi_A(x) = 1/2$.

As we want to classify each point either \emph{in} $A$ or \emph{out of} $A$, we can look for the different level-sets of the probability of coverage: for $\eta \in [0, 1]$, we define the $\eta$-level set of $\pi_A$, also called \emph{Vorob'ev quantiles} (see~\cite{vorobyev_new_2003})
\begin{equation}
  Q_{\eta} = \{x\in\Xspace \mid \pi_A(x) \geq \eta \}
\end{equation}
Those sets are decreasing (with respect to the inclusion) when $\eta$ increases:
\begin{equation}
  0\leq \eta \leq \xi \leq 1 \implies Q_{\xi} \subseteq Q_{\eta}
\end{equation}

For a function $f$, let us assume that we are interested in the estimation of the estimation of the set $f^{-1}(B) = \{x \in \Xspace \mid f(x) \leq T\}$, with  $B = ]-\infty, T]$.
Let $Z$ be a GP, constructed using an initial design $\mathcal{X}$ on $\Xspace$ for the function $f$, $Z\sim \GP(m_Z, C_Z)$
The inverse image of $B$ through $Z$ gives the random set $A = Z^{-1}(B) = \{x \in \Xspace \mid Z(x) \leq T\}$.
Using the properties of the GP, we can express the probability of coverage of the random set $A$ using $\Phi$, the cdf of the centered standard Gaussian distribution:
\begin{align}
  \pi_A(x) &= \Prob_{Z(x)}\left[Z(x) \leq T\right] = \Phi\left(-\frac{m_Z(x) - T}{\sigma_Z(x)}\right) \\
  \mathscr{V}_A(x) &= \pi_A(x)(1 - \pi_A(x)) \\
  \mathscr{P}_{\mathrm{mis}}(x) &= \Phi\left(- \frac{\lvert m_Z(x) - T \rvert}{\sigma_Z(x)} \right) \label{eq:prob_missclassification_GP}
\end{align}
since for $x\in\Xspace$, $Z(x)\sim\mathcal{N}\left(m_Z(x), \sigma^2_Z(x)\right)$.
\cite{echard_ak-mcs_2011} defines the argument of the cdf \cref{eq:prob_missclassification_GP} as the \emph{reliability index} $\rho$ (denoted $U$ in its original definition \cite{echard_ak-mcs_2011})):
\begin{equation}
  \rho(x) = \frac{\lvert m_Z(x) - T \rvert}{\sigma_Z(x)}
\end{equation}

We can define then two criteria. First, we can minimize the point which has the maximal probability of missclassification, or equivalently since $\Phi$ is monotonously increasing, maximize the criterion $\kappa_{\rho}$ given by
\begin{equation}
  \kappa_{\rho}(x) = -\rho(x) = -\frac{\lvert m_Z(x) - T \rvert}{\sigma_Z(x)}
\end{equation}

We can also consider the variance of the coverage over the whole space $\mathcal{X}$, as the criteria introduced~\cite{bect_sequential_2012}.
In a similar fashion as the $\IMSE$, we can define the integrated variance of the probability of coverage
\begin{equation}
  \label{eq:IVPC_def}
\mathrm{IVPC}(\mathcal{X}) =  \int_{\Xspace} \mathscr{V}_A(x)\,\mathrm{d}x
\end{equation}
and define a criterion based on an augmented design:
\begin{equation}
  \label{eq:expected_aIVPC}
  \kappa(x) = -\Ex_{Z(x)}\Big[\mathrm{IVPC}(\mathcal{X} \cup \{   \left(x, Z(x) \right)  \}     \Big]
\end{equation}

Other criterion have also been developed, such as the utilisation of the theory of random sets~\cite{el_amri_data-driven_2019}, in order to define a measure of uncertainty based on Vorob'ev mean and deviation \cite{vorobyev_new_2003}.


\subsubsection{Margin of uncertainty}
\label{sec:margin_of_uncertainty}

Using the level sets, we can construct the $\eta$-margin of uncertainty, as introduced in~\cite{dubourg_reliability-based_2011}, that is the set of points $x \in \Xspace$ that we cannot classify in or out of $A$ with high enough probability.
Setting the classical level $\eta=0.05$ for instance, $Q_{1-\frac{\eta}{2}}=Q_{0.975}$ is the set of points whose probability of coverage is higher than $0.975$, while $Q_{\frac{\eta}{2}}=Q_{0.025}$ is the set of points whose probability of coverage is higher than $0.025$, thus its complement in $\Xspace$, denoted by $Q_{\frac{\eta}{2}}^C$ is the set of points whose probability of coverage is lower than $0.025$. Obviously, $Q_{1-\frac{\eta}{2}} \subset Q_{\frac{\eta}{2}}$.

The $\eta$-margin of uncertainty $\mathbb{M}_{\eta}$ is defined as the sets of points whose coverage probability is between $0.025$ and $0.975$.
\begin{align}
  \label{eq:margin_unc}
  \mathbb{M}_{\eta} &= \left(Q_{1-\frac{\eta}{2}} \cup Q^C_{\frac{\eta}{2}} \right)^C = Q_{1-\frac{\eta}{2}}^C \cap Q_{\frac{\eta}{2}} = Q_{\frac{\eta}{2}} \setminus Q_{1-\frac{\eta}{2}} \\
                    &= \left\{x \in \Xspace \mid \frac{\eta}{2} \leq \pi_A(x) \leq 1 - \frac{\eta}{2} \right\}
\end{align}
We can then construct easily a sampling criterion: $\kappa_{\mathbb{M}}(x) = \mathbbm{1}_{\mathbb{M}_{\eta}}(x)$.
Evaluating points in this region allows to reduce the uncertainty at $x$ (completely if the GP is interpolant, as $\sigma_Z(x)$ becomes $0$ after evaluating $f(x)$) on the classification of the point $x$.
\todo{ajouter figure}

\section{Robust criteria and enrichment of GP}
\label{sec:robust_criteria_gp}
So far, we introduced strategies for either optimization or exploration to be applied on a generic space $\Xspace$. From a robust point of view, we are going to consider the objective function $J$ on a joint space $\Xspace = \Kspace \times \Uspace$:

We assume that we constructed a GP $Z$ on the joint space $\Kspace \times \Uspace$, based on a design of $n$ evaluated points $\mathcal{X}_n = \left\{\left((\kk_i,\uu_i),J(\kk_i, \uu_i) \right)\right\}_{1\leq i \leq n}$, denoted as $(\kk,\uu)\mapsto Z(\kk,\uu)$. If the design used to construct the GP is clear from the context, it is omitted in the notation.

As a GP, $Z$ is described by its mean function $m_{Z}$ and a covariance function $C(\cdot, \cdot)$, while $\sigma^2_Z(\kk,\uu) = C\left((\kk,\uu), (\kk,\uu)\right)$
\begin{equation}
  Z(\kk,\uu) \sim \mathcal{N}\left(m_{Z}(\kk,\uu), \sigma^2_Z(\kk,\uu) \right)
\end{equation}
A surrogate of $J$ based on $Z$ is then $m_Z$.

One main challenge when making the distinction $\Xspace = \Kspace \times \Uspace$ is that the objectives on the two spaces are not the same.
While some methods introduced here involve the computation and the maximization of a criterion on the joint space directly, following the method of~\cref{alg:SUR_strat},
some others use the GP $Z$ to remove the dependence on $\UU$, by projecting the GP for instance in~\cref{ssec:expected_loss_GP_projection}.
We will detail hereafter a 1-step criterion in order to estimate $\estimtxt{\kk}{mean}$, defined as the minimum of the expected value of the objective function, and a criterion in order to reconstruct the conditional minimisers and conditional minimum, in order to compute $\estimtxt{\kk}{MPE}$.
\subsection{Expected loss}
\label{ssec:expected_loss_GP_projection}
Recalling the definition of $\estimtxt{\kk}{mean} = \argmin_{\kk \in \Kspace} \Ex_{\UU}\left[J(\kk, \UU)\right]$, we can look to minimize the expected loss. 
In~\cite{janusevskis_simultaneous_2010}, the authors define the \emph{projected process} $W$, the stochastic process $\kk \rightarrow \left(\Omega \rightarrow \mathbb{R}\right)$ which verifies
\begin{equation}
  W(\kk) = \Ex_U[Z(\kk, \UU)] = \int_{\Uspace} Z(\kk, \uu) p_{\UU}(\uu)\,\mathrm{d}\uu
\end{equation}
$W$ is then a Gaussian process on $\Kspace$. As done before, we want to minimize the unknown function $\Ex_U[J(\kk, \UU)]$ using $W$, thus we can maximize the expected improvement~\cref{eq:def_ei} to find a potentially interesting point $\tilde{\kk}$.
Then, we will look for the pair $(\kk, \uu) \in \Kspace\times \Uspace$ which minimizes the variance of the augmented projected process at this candidate.
\begin{align}
  \tilde{\kk} &= \argmax_{\kk \in \Kspace} \mathrm{EI}_W(\kk) \\
  (\kk_{n+1}, \uu_{n+1}) &= \argmin_{(\kk, \uu)\in\Kspace\times \Uspace} \Var\left[W(\tilde{\kk}) \mid \mathcal{X} \cup (\kk,\uu) \right]
\end{align}
The variance of the augmented projected process does not depend on the realization of r.v.\ $Z(\kk, \uu)$, but only on the distance of $(\kk, \uu)$ to the current experimental design.
\todo{ajouter figure}
\subsection{Profile expected improvement}
In the previous chapter, we introduced the conditional minimum $\uu\mapsto J^*(\uu)$ and the conditional minimizers $\uu \mapsto \kk^*(\uu)$. These require an optimization procedure, thus can be expensive from a computational point of view.~\cite{ginsbourger_bayesian_2014} proposes a criterion, adapted from the EI, which aims at solving such a problem: the \emph{Profile Expected Improvement} $\mathrm{PEI}$, which is defined as
\begin{equation}
  \label{eq:def_PEI}
  \mathrm{PEI}(\kk, \uu)= \Ex\left[[f_{\min}(\uu) - Z(\kk, \uu)]_+\right] \text{ with } f_{\min} = \max(\min_i f(x_i), \min_{\kk \in \Kspace} m_Z(\kk, \uu))
\end{equation}

This writing allow us to see the similarity with the $\mathrm{EI}$ criterion: instead of having a fixed threshold, the $\mathrm{PEI}$ introduces a criterion that depends on $\uu$. \Cref{fig:example_PEI} shows an example of GP enriched using the $\mathrm{PEI}$ criterion.
The criterion can then be used in a classical SUR strategy introduced \cref{alg:SUR_strat}, where at each iteration,
\begin{equation}
  \label{eq:PEI_criterion_SUR}
  \kappa_{\mathrm{PEI}}((\kk, \uu) ; Z \mid \mathcal{X}_n) = \mathrm{PEI}(\kk, \uu)
\end{equation}
is maximized over $\Kspace \times \Uspace$.
\begin{figure}[ht]
  \centering
  \input{\imgpath PEI_example.pgf}
  \caption[Illustration of enrichment using the PEI criterion]{\label{fig:example_PEI} GP after 30 additional iterations chosen using PEI. The GP prediction using the initial design is represented on the top left plot, and the PEI criterion for this GP is evaluated over the whole space is plotted top right. Bottom left plot is the GP prediction after 30 iterations. Bottom right is the surrogate estimation of the conditional minimum. The shaded region corresponds to the initial surrogate conditional minimum, $\pm$ one standard deviation. After the iterations, the surrogate estimate and the true value coincide}
\end{figure}
Once a stopping criterion is met, such as the maximal variance attained at the conditional minimum is below a specified threshold $T$, 
\begin{equation}
  \max_{\uu\in\Uspace} \sigma^2_Z(\argmin_{\kk\in\Kspace}m_Z(\kk,\uu), \uu) \leq T
\end{equation}
we can construct accurate surrogates of $J^*(\uu)$ and $\kk^*(\uu)$, using the GP in a plug-in approach:
\begin{equation}
  \min_{\kk\in\Kspace} m_Z(\kk,\uu) = m^*_Z(\uu) \quad \text{ and } \quad \argmin_{\kk\in\Kspace} m_Z(\kk, \uu) = \kk_Z^*(\uu)
\end{equation}
and with a set of i.i.d.\ samples $\{\uu_i\}_i$ of $\UU$, use the surrogates to find the MPE, based on the distribution of $\{\kk_Z^*(\uu_i)\}$.


\subsection{GP of the penalized objective function}
\label{ssec:gp_delta_alpha}

We are now going to detail how Gaussian processes can help to estimate regret-based estimators:
\begin{align}
  \{ \estimtxt{\kk}{AR,\beta} &= \argmax_{\kk\in\Kspace} \Gamma_\beta(\kk) = \argmax_{\kk\in\Kspace} \Prob_{\UU}\left[J(\kk, \UU) \leq J^*(\UU) + \beta \right] \mid \beta \geq 0 \} \\
\{ \estimtxt{\kk}{RR,\alpha} &= \argmax_{\kk\in\Kspace} \Gamma_\alpha(\kk) = \argmax_{\kk\in\Kspace}\Prob_{\UU}\left[J(\kk, \UU) \leq \alpha J^*(\UU)\right] \mid \alpha \geq 1 \}
\end{align}
We first need to detail some notations with respect to the conditional minimum and conditional minimizers, and their equivalent for the GP.
% Let us first consider the conditional minimiser:
% \begin{align}
%   J^*(\uu) = J(\kk^*(\uu),\uu) = \min_{\kk\in\Kspace} J(\kk,\uu)
% \end{align}

We define $Z^*$ as
\begin{equation}
  Z^*(\uu) \sim \mathcal{N}\left(m^*_Z(\uu), \sigma^{2}_{Z^*}(\uu)\right)
\end{equation}
where
\begin{align}
  m^*_Z(\uu) = \min_{\kk\in\Kspace} m_Z(\kk,\uu) \\
  \kk_Z^*(\uu)=\argmin_{\kk\in\Kspace} m_Z(\kk, \uu) \\
  \sigma^{2}_{Z^*}(\uu) = \sigma^{2}_Z(\kk^*_Z(\uu), \uu) 
\end{align}
We use then the minimizer and the minimum of the GP prediction as estimates of the conditional minimum and minimiser, similarly as in~\cite{ginsbourger_bayesian_2014}. % for instance, but other choices could be considered, such as $m_Z(\kk^*_Z(\uu)) - \gamma \sigma^{2}_{Z^*}(\kk_Z^*(\uu))$. This choice would lead to be more ``optimistic'' in the estimation of the minimum (i.e.\ a lower minimum), and in turn, would have a tendency to overestimate the estimated value of $\alpha$, thus trending toward more conservative estimates.
To generalize notions of the additive and relative regret, we can define a GP based on $J - \alpha J^* - \beta$: which is $\Delta_{\alpha, \beta} = Z - \alpha Z^* - \beta$.
This difference $\Delta_{\alpha,\beta}$ is a linear combination of correlated Gaussian processes, thus is a GP as well and its distribution can be derived by first considering the joint distribution of $Z(\kk,\uu)$ and $Z^*(\uu) = Z(\kk^*(\uu), \uu)$:
\begin{equation}
  \begin{bmatrix}
    Z(\kk,\uu) \\
    Z^*(\uu)
  \end{bmatrix}
  \sim \mathcal{N}\left(
    \begin{bmatrix}
      m_Z(\kk,\uu) \\
      m_Z^*(\uu)
    \end{bmatrix}
    ;\,
    \begin{bmatrix}
      C\left((\kk,\uu),(\kk,\uu)\right) & C\left((\kk,\uu),(\kk^*(\uu),\uu)\right) \\
      C\left((\kk,\uu),(\kk^*(\uu),\uu)\right) & C\left((\kk^*(\uu),\uu),(\kk^*(\uu),\uu)\right)
    \end{bmatrix}
\right)
\end{equation}

Multiplying by the matrix $\begin{pmatrix}1 & -\alpha \end{pmatrix}$ and translating by $-\beta$, yields
\begin{align}
  \Delta_{\alpha,\beta}(\kk,\uu) &\sim \mathcal{N}\left(m_{\Delta}(\kk,\uu); \sigma^2_{\Delta}(\kk,\uu)\right)  \label{eq:delta_GP_ab}\\
  m_{\Delta}(\kk,\uu) &= m_Z(\kk,\uu) - \alpha m_Z^*(\uu) - \beta \label{eq:mu_delta_GP_ab}\\
  \sigma^2_{\Delta}(\kk,\uu) &= \sigma_Z^2(\kk,\uu) + \alpha^2 \sigma_{Z^*}^2(\uu) - 2\alpha C\left((\kk,\uu),(\kk^*(\uu),\uu)\right) \label{eq:variance_delta_GP_ab}
\end{align}

We are then interested in the set $\{(\kk, \uu) \mid \Delta_{\alpha, \beta}(\kk, \uu) \leq 0\}$. The different value of $\alpha$ and $\beta$ dictate if we are either interested in the additive or the relative regret:
\begin{itemize}
\item $(\alpha, \beta) = (1, \beta)$ corresponds to the additive regret
\item $(\alpha, \beta) = (\alpha, 0)$ corresponds to the relative regret
\end{itemize}



Decomposing the variance $\sigma^2_{\Delta}$ in~\cref{eq:variance_delta_GP_ab}, 3 sources of uncertainty appear:
\begin{itemize}
\item $\sigma^2_{Z}$ is the prediction variance of the GP on $J$, that is directly reduced when additional points are evaluated
\item $\sigma^2_{Z^*}$ is the variance of the predicted value of the minimizer.
\item Assuming a stationary form of the covariance, the third term is dependent only on the distance between $\kk$ and $\kk^*(\uu)$. By separating the $\kk$ and $\uu$ components, the covariance term can be written $C((\kk,\uu), (\kk^{\prime},\uu^{\prime})) = s \mathcal{K}_{\kk}(\|\kk - \kk^{\prime}\|) \mathcal{K}_{\uu}(\|\uu - \uu^{\prime}\|)$, and
  % $C((\kk,\uu), (\kk',\uu')) = s \prod_{i\in\mathcal{I}_{\kk}}\rho_{\theta_i}(\|k_i - k'_i\|) \prod_{j\in\mathcal{I}_{\uu}} \rho_{\uu_j}(\|u_j - u'_j\|)$
  substituting $\kk^*(\uu)$ for $\kk^{\prime}$ gives
\begin{align}
  C\left((\kk,\uu),(\kk^*(\uu),\uu)\right) &= s  \mathcal{K}_{\kk}(\|\kk - \kk^*(\uu)\|) \mathcal{K}_{\uu}(0) \\
                                           &= s  \mathcal{K}_{\kk}(\|\kk - \kk^*(\uu)\|)
\end{align}
\end{itemize}
This decomposition highlights the fact that the prediction error $\sigma_{\Delta}^2$ of the difference $\Delta_{\alpha}$ measured at a point $(\kk, \uu)$ will not be reduced completely by evaluating the function $J$ at this point, as only the prediction variance $\sigma_Z^2$ will be significantly affected in general. In other words, reducing the uncertainty at a point $(\kk, \uu)$ may require the evaluation of another point $(\tilde{\kk},\tilde{\uu}) \neq (\kk, \uu)$.


\subsection{2 stages enrichment strategies}
A measure of the uncertainty on this derived quantity is evaluated and used to choose a ``candidate'' $\tilde{\kk}$, so that we can reduce the uncertainty on $\{\tilde{\kk}\} \times \Uspace$.
The next point $(\kk_{n+1}, \uu_{n+1})$ to evaluate is then chosen in $\Kspace \times \Uspace$, where $\kk_{n+1}$ may not be equal to $\tilde{\kk}$.
A general 2-stage SUR strategy is described~\cref{alg:2stage_SUR}.


\begin{algorithm}
  \caption{\label{alg:2stage_SUR} Two-stages SUR strategies for robust optimization problem}
\begin{algorithmic}
\REQUIRE Initial design $\mathcal{X}_0$, criterion function $\kappa$
\STATE Fit $Z$, a GP using the design $\mathcal{X}_0$
\STATE $n \leftarrow 0$
\WHILE{Stopping criterion not met or max budget evaluation not reached} 
\STATE $\tilde{\kk} \leftarrow \argmax_{\kk \in \Kspace} \kappa(\kk; Z \mid \mathcal{X}_{n})$
\STATE $(\kk_{n+1},\uu_{n+1}) \leftarrow \argmax_{(\kk,\uu) \in \Kspace\times\Uspace} \kappa((\kk,\uu); \tilde{\kk}, Z \mid \mathcal{X}_{n})$
\STATE Evaluate $J(\kk_{n+1}, \uu_{n+1})$
\STATE $\mathcal{X}_{n+1} \leftarrow \mathcal{X}_n \cup \left\{\left((\kk_{n+1},\uu_{n+1}), J(\kk_{n+1}, \uu_{n+1})\right)\right\}$
\STATE $n \leftarrow n + 1$
\ENDWHILE
\end{algorithmic}
\end{algorithm}


\subsection{Evaluation and optimization of $\Gamma$}
Let consider $\alpha\geq 1$ fixed. In order to compute $\estimtxt{\kk}{rel, \alpha}$, we need to estimate and optimize the function $\Gamma_{\alpha}$. For that purpose, we can first explore the space to improve the estimation of $\Gamma_{\alpha}$, and once sufficient knowledge is acquired, use the plug-in estimate $\hat{\Gamma}_{\alpha}$ for the optimization, to get the wanted estimator.
\subsubsection{Improving the estimation of $\Gamma_{\alpha}$}

For a given $\kk\in\Kspace$, the coverage probability of the $\alpha$-acceptable region, i.e.\ the probability for $\kk$ to be $\alpha$-acceptable is
\begin{align}
  \Gamma_{\alpha}(\kk) &= \Prob_{U}\left[J(\kk,\UU) \leq \alpha J^*(\UU)\right] \\
                              & =\Ex_{U}\left[\mathbbm{1}_{J(\kk,\UU) \leq \alpha J^*(\UU)}\right]
\end{align}
As $J$ is not known perfectly, it can be seen as a classification problem.
This classification problem can be approached with a plug-in approach in~\cref{eq:plugin_indicator}, or a probabilistic one in~\cref{eq:prob_indicator}:
\begin{align}
  \mathbbm{1}_{J(\kk,\uu) \leq \alpha J^*(\uu)} &\approx   \mathbbm{1}_{m_Z(\kk,\uu) \leq \alpha m_Z^*(\uu)} \label{eq:plugin_indicator} \\
  \mathbbm{1}_{J(\kk,\uu) \leq \alpha J^*(\uu)} &\approx   \ProbGP\left[ \Delta_{\alpha}(\kk,\uu) \leq 0 \right] = \pi_{\alpha}(\kk,\uu) \label{eq:prob_indicator}
\end{align}

Based on those two approximation, we can define two different estimations of $\Gamma_\alpha$, namely $\hat{\Gamma}_\alpha^{\mathrm{PI}}$ with the plug-in approach, and $\Gamma_{\alpha}^{\pi}$ for the probabilistic one.
\begin{itemize}
\item For $\hat{\Gamma}_{\alpha}^{\mathrm{PI}}$, we are going to reduce the expected augmented $\IMSE$ of the GP $Z - \alpha Z^*$.
\item For $\hat{\Gamma}_{\alpha}^{\pi}$, we are going to reduce the expected augmented integrated variance of probability of coverage.
\end{itemize}

\subsubsection{Plug-in approach}
For the plug-in approach, the chosen estimator is defined \cref{eq:def_gamma_PI}:
  \begin{equation}
    \label{eq:def_gamma_PI}
    \hat{\Gamma}_{\alpha}^{\mathrm{PI}}(\kk) = \Prob_U\left[m_Z(\kk,\uu) \leq \alpha m_Z^*(\uu) \right]
  \end{equation}
  The outer expectation operator is to be computed numerically, using quadrature rule, or Monte-carlo methods. In general, from a set of samples $\{\uu_i\}_{1\leq i \leq n_{\uu}}$,
  \begin{equation}
    \Gamma_{\alpha}(\kk) \approx \frac{1}{n_{\uu}}\sum_{i=1}^{n_{\uu}} \mathbbm{1}_{m_Z(\kk, \uu_i) - \alpha m^*(\uu_i)\leq 0}
  \end{equation}

  Due to the fact that the GP surrogate is cheap to evaluate, the computation of the outer expectation with respect to $\UU$ is assumed to be performed without too much problems.

  In order to improve the accuracy of this estimator, one need to improve the GP prediction $m_Z$ of the objective function $J$.
  In this case, we propose to reduce the IMSE of the GP $Z - \alpha Z^*$. The choice of the IMSE (instead of choosing the point of maximal variance for instance) comes from the decomposition of the variance~\cref{eq:variance_delta_GP_ab}.
  Let $\hat{\Gamma}_{\alpha,n}$ be the plug-in approximation of $\Gamma_\alpha$, constructed using the Gaussian Process surrogate with $n$ points added, according to the augmented IMSE. \Cref{fig:IMSE_enrichment} illustrates the $L^2$ and $L^{\infty}$ between the truth $\Gamma_\alpha$ and the estimation $\hat{\Gamma}_{\alpha,n}$.

\begin{figure}[ht]
  \centering
  \input{\imgpath IMSE_enrichment.pgf}
  \caption{\label{fig:IMSE_enrichment} Evolution of the $L^2$ and $L^\infty$ error of the estimation of $\Gamma_\alpha$ and IMSE of the successive constructed GP. The points added are chosen by the augmented expected IMSE}
\end{figure}

  
\subsubsection{Improving the estimation of the probability of coverage}
\cite{echard_ak-mcs_2011,schobi_rare_2017,razaaly_quantile-based_2020}

Recalling the definition of the probabilistic approach \cref{eq:prob_indicator}, given the GP $Z$, we can write an estimator of $\Gamma_{\alpha}$:
\begin{align}
  \hat{\Gamma}_{\alpha}^{\pi}(\kk) &= \Ex_U\left[ \Prob_{Z}\left[ \Delta_{\alpha}(\kk,\UU) \leq 0\right]\right] = \Ex_U\left[ \Prob_{Z}\left[ Z(\kk, \UU) - \alpha Z^*(\UU) \leq 0\right]\right] \\ &= \Ex_U\left[\pi_{\alpha}(\kk,\uu)\right]
\end{align}

  
The set $\{(\kk, \uu) \mid Z(\kk, \uu) - \alpha Z^*(\uu)\leq 0\}$ has a probability of coverage written $\pi_{\alpha}$, which can be computed using the CDF of the standard normal distribution $\Phi$, because $\Delta_{\alpha}$ is a GP, as defined~\cref{eq:delta_GP_ab,eq:mu_delta_GP_ab,eq:variance_delta_GP_ab}:
\begin{equation}
  \label{eq:def_pialphaku}
  \pi_{\alpha}(\kk,\uu) = \Phi\left(-\frac{m_{\Delta_\alpha}(\kk,\uu)}{\sigma_{\Delta_\alpha}(\kk,\uu)}\right)
\end{equation}
\begin{equation}
  P_\mathrm{mis}(\kk, \uu) = \Phi\left(-\frac{\lvert m_{\Delta_\alpha}(\kk,\uu)\rvert}{\sigma_{\Delta_\alpha}(\kk,\uu)}\right)
\end{equation}
Finally, averaging the coverage probability over $\uu$ yields
\begin{equation}
  \hat{\Gamma}_{\alpha}^{\pi}(\kk) = \Ex_U\left[\pi_{\alpha}(\kk,\uu)\right]=\int_{\Uspace}\pi_{\alpha}(\kk,\uu)p_{\UU}(\uu) \,\mathrm{d}\uu = \int_{\Uspace}\Phi\left(-\frac{m_{\Delta_\alpha}(\kk,\uu)}{\sigma_{\Delta_\alpha}(\kk,\uu)}\right)p_{\UU}(\uu) \,\mathrm{d}\uu
\end{equation}

The variance of the probability of coverage is $\pi_\alpha(\kk, \uu) \left(1 - \pi_{\alpha}(\kk,\uu)\right)$.
Integrating this variance over the whole space $\Kspace \times \Uspace$ gives the integrated variance of the probability of coverage $\mathrm{IVPC}$:
\cite{bect_sequential_2012}
\begin{equation}
  \label{eq:IVPC_def}
\mathrm{IVPC}(\mathcal{X}) =  \int_{\Kspace \times \Uspace} \pi_{\alpha}(\kk, \uu) \left(1 - \pi_{\alpha}(\kk, \uu)\right)p_{\UU}(\uu)\,\mathrm{d}\kk \,\mathrm{d}\uu
\end{equation}

Instead of evaluating this integrated variance, on the current design $\mathcal{X}$, we can once again, augment the design at the (unevaluated) point $(\kk, \uu)$, assuming that its evaluation is the random variable $Z(\kk, \uu)$:
\begin{equation}
  \mathrm{IVPC}(\mathcal{X} \cup \{((\kk, \uu), Z(\kk, \uu))\}
\end{equation}

Finally, we can define a new learning function, which is the expected $\mathrm{IVPC}$ with respect to the random variable $Z(\kk, \uu)$:
\begin{equation}
  \label{eq:expected_aIVPC}
  \kappa(\kk, \uu) = \Ex_{Z(\kk, \uu)}\Big[\mathrm{IVPC}(\mathcal{X} \cup \{   \left(\left(\kk, \uu \right), Z(\kk, \uu) \right)  \}     \Big]
\end{equation}
\Cref{fig:IVPC_enrichment} shows the evolution of the error in the estimation of $\Gamma_{\alpha}$, with respect to the $L^2$ and $L^{\infty}$ norm.
\begin{figure}[ht]
  \centering
  \input{\imgpath IVPC_enrichment.pgf}
  \caption{\label{fig:IVPC_enrichment} Enriching the design according to the criterion of \cref{eq:expected_aIVPC}}
\end{figure}

\paragraph{Sampling in the margin of uncertainty}
We can also enrich the design by means of sampling within the margin of uncertainty defined~\cref{sec:margin_of_uncertainty}. In this case, the probability of coverage is defined as
\begin{equation}
  \pi_{\alpha,\beta}(\kk, \uu) = \ProbGP\left[\Delta_{\alpha,\beta} \leq 0\right] = \Phi\left(-\frac{m_{\Delta}}{\sigma_{\Delta}}\right)
\end{equation}
and the margin of uncertainty of level $\eta$ is
\begin{equation}
  \mathbb{M}_{\eta} = \left\{ (\kk, \uu) \mid \frac{\eta}{2} \leq \pi_{\alpha,\beta}(\kk, \uu) \leq 1 - \frac{\eta}{2}\right\}
\end{equation}

Let us assume that we obtained $N_s$ points which represent statistically the margin of uncertainty. As mentioned before, the uncertainty on those points to be acceptable can be reduced by either evaluating the function at this point (thus reducing $\sigma_Z$), or by improving the knowledge of the conditional minimiser (reducing $\alpha^2 \sigma_Z^*$). 


% \subsubsection{Vorob'ev expectation as placeholder}
% We can also consider the set $Z - \alpha Z^* \leq 0$ from a random set perspective:
% For each $\kk\in\Kspace$, we can consider the random closed set $A_\kk =  \{\uu \in \Uspace \mid Z(\kk, \uu) - \alpha Z^*(\uu) \leq 0\}$.
% In this case, we can set $\mu = \Prob_{\UU}$, so that $\mu(A_{\kk}) = \Prob_{\UU}[A_{\kk}]$.
% For a given $\kk$, the probability of coverage of the set $A_{\kk}$
% is
% \begin{equation}
%   \label{eq:prob_cov_randclosedset}
%   \pi_{A_{\kk}}(\uu) = \frac{\pi_{\alpha}(\kk, \uu)}{\int_{\Uspace} \pi_{\alpha}(\kk, \uu)p_{\UU}(\uu)\,\mathrm{d}\uu}
% \end{equation}
% The normalization constant of \cref{eq:prob_cov_randclosedset} can be obtained quite easily numerically, as $\pi_{\alpha}$ can be computed in a closed from~\cref{eq:def_pialphaku}.

% The Vorob'ev mean is defined as the $\eta^*$-level set of $\pi_{A_\kk}(\uu)$, that verifies
% \begin{equation}
%   \Prob_{\UU}[Q_{\eta^*}(\kk)] = \Ex_{Z}\left[\Prob_{\UU}\left[A_{\kk}\right]\right] %= \int_{\Zspace} \int_{\Uspace} \mathbbm{1}_{\{(y(\kk, \uu) - \alpha y^*(\uu)\leq 0\}} \,\mathrm{d}u \mathrm{d}y
% \end{equation}
% We can then define a new estimator of $\Gamma$:
% \begin{equation}
%   \Gamma_{\alpha}^{\mathrm{Voro}}(\kk) = \Prob_{\UU}\left[Q_{\eta^*}(\kk)\right]
% \end{equation}
% % \tocheck{Compare Vorob'ev deviation with pi(1-pi) ?}




\subsection{Estimation of $\alpha_p$ based on GP}
Instead of choosing a fixed threshold $\alpha \geq 0$, we can instead look for a threshold such that $\max \Gamma_{\alpha}$ is large enough.
For a level $p$, we define $\alpha_p$ as the smallest threshold giving a maximal probability of acceptability above $p$.
\begin{equation}
  \alpha_p = \inf_{\alpha \geq 1}\{\max_{\kk \in \Kspace}\Gamma_{\alpha}(\kk)\geq p \}
\end{equation}

As $J^*(\uu) \neq 0$ for all $\uu$, we can define $q_{p}(\kk)$ as the quantile of level $p$ of the ratio $J(\kk, \UU) / J^*(\UU)$:
\begin{align}
       & q_{p}(\kk) = Q_{\UU}\left(\frac{J(\kk, \UU)}{J^*(\UU)};p\right) \\
  \iff &p          = \Prob_{\UU}\left[\frac{J(\kk, \UU)}{J^*(\UU)} \leq q_{p}(\kk)\right]
\end{align}
$\alpha_p$ verifies then
\begin{equation}
  \label{eq:alpha_p_min}
\alpha_p = \min_{\kk} q_p(\kk)
\end{equation}
and 
\begin{equation}
  \label{eq:alpha_p_quantile_max}
  p = \Prob_{\UU}\left[\max_{\kk \in \Kspace}\frac{J(\kk, \UU)}{J^*(\UU)} \leq \alpha_p\right] \iff \alpha_p = Q_{\UU}\left(\max_{\kk\in\Kspace} \frac{J(\kk,\UU)}{J^*(\UU)};p\right)
  ????
\end{equation}

Those two formulations \cref{eq:alpha_p_min}, and \cref{eq:alpha_p_quantile_max} shows two ways of getting to $\alpha_p$.
\paragraph{Plug-in approach}
Again, the plug-in approach is to replace $J(\kk, \uu)/J^*(\uu)$ with $m_Z(\kk, \uu) / m^*_Z(\uu)$, and to compute the associated estimates.
\begin{equation}
  q_{p}^{\mathrm{PI}}(\kk) = Q_{U}\left(\frac{m_Z(\kk, \UU)}{m^*_Z(\UU)};p\right)
\end{equation}
the estimation of the relaxation value $\hat{\alpha}_p$ is then the minimal value of the quantiles with respect to $\kk$:
\begin{equation}
    \label{eq:def_plugin_alpha}
  \hat{\alpha}^{\mathrm{PI}}_p = \min_{\kk \in \Kspace} Q_U\left(\frac{m_Z(\kk, \UU)}{m^*_Z(\UU)};p\right) %=^{?} Q_{\UU}\left(\max_{\kk\in\Kspace} \frac{m_Z(\kk,\UU)}{m_Z^*(\UU)};p\right)
\end{equation}

\paragraph{Monte-Carlo approach}
Another approach relies on the sample paths of $Z$. Let $y$ a sample path of $Z$. We can then compute $y(\kk, \uu)/y^*(\uu)$ and get an estimate based on this sample. 
Using the random nature of $Z$, we can compute measures of uncertainty on the estimation, and ultimately, reduce this uncertainty by choosing the next point to evaluate accordingly.

Before sampling trajectories however, one should first explore the whole space $\Kspace \times \Uspace$ sufficiently. Indeed, in order to get the prediction $m_Z$ and the prediction of the conditional minimum $m^*_{Z}$ of the GP should be strictly larger than $0$. This condition applies to the trajectories too.

Let us say that we sampled $N$ function from $Z$, namely $y^{(i)}$ for $1 \leq i \leq N$. For each of these samples, we can get $q^{(i)}_p(\kk)$. Using Monte-Carlo, we can get a Monte-Carlo estimation of $q_p$:.
\begin{equation}
 \frac{1}{N} \sum_{i=1}^N q_p^{(i)}(\kk) = q_p^{\mathrm{MC}}(\kk)
\end{equation}

and finally, minimizing the value of the estimated quantile leads to $\hat{\alpha}_p^{\mathrm{MC}}$, the MC approximation of $\alpha_p$: 
\begin{equation}
  \hat{\alpha}_p^{\mathrm{MC}} = \min_{\kk \in \Kspace} q_p^{\mathrm{MC}}(\kk)
\end{equation}

\subsubsection{Enriching the design for the estimation}
\paragraph{Reducing the augmented IMSE of the original GP}

\begin{figure}[ht]
  \centering
  \input{\imgpath IMSE_enrichment_alpha.pgf}
  \caption{\label{fig:IMSE_enrichment_alpha} Enrichment according to the augmented IMSE of the gp $Z$}
\end{figure}

However, so far, we only tried to reduce the uncertainty on the whole space $\Kspace \times \Uspace$, in order to have a better approximation of $q_p$ and $\Gamma_{\alpha}$. We can adopt an approach more focused on their optimization. To do so, the principle stays the same: first, we derive a quantity of interest, that lead us to select a certain candidate $\tilde{\kk} \in \Kspace$, then we look to reduce the uncertainty for this point:

\paragraph{Two-stages IVPC}
We are first going to detail a two-stage approach to improve the maximization of $\Gamma_\alpha$ using the probabilistic approach based on the probability of coverage.
We define first 
\begin{equation}
  \mathrm{IVPC}(\kk\mid \mathcal{X}) = \int_{\Uspace} \pi_{\alpha}(\kk, \uu)\left(1 - \pi_{\alpha}(\kk, \uu)\right)p_{\UU}(\uu)\, \mathrm{d}\uu
\end{equation}
that is the $\mathrm{IVPC}$ when $\kk$ is fixed, or in other words, the uncertainty associated with the estimation of $\hat{\Gamma}_{\alpha}$ at a point $\kk \in \Kspace$. We have also that $\mathrm{IVPC}(\mathcal{X}) = \int_{\Kspace} \mathrm{IVPC}(\kk \mid \mathcal{X}) \,\mathrm{d}\kk$
On a computational note,  $\mathrm{IVPC}(\kk\mid \mathcal{X})$ is easier to compute than the ``full'' $\mathrm{IVPC}$ defined \cref{eq:IVPC_def}, as the integration has to be performed on $\Uspace$ only, instead of the joint space.


Once a candidate $\tilde{\kk}$  which maximizes a specified criterion has been obtained (\cref{eq:acq_candidate_IVPC}), we can attempt to reduce the uncertainty associated at this point, by finding the point that reduces at most the expected augmented IVPC (for instance) on $\{\tilde{\kk}\}\times \Uspace$.

\begin{align}
  \tilde{\kk} &= \argmax_{\kk\in \Kspace} \kappa_n(\kk) \label{eq:acq_candidate_IVPC}\\
  (\kk_{n+1}, \uu_{n+1}) & = \argmin_{(\kk,\uu) \in \Kspace \times \Uspace} \Ex_{Z(\kk, \uu)}\left[ \mathrm{IVPC}\left(\tilde{\kk} \mid \mathcal{X}_n \cup\left\{\left((\kk,\uu), Z(\kk, \uu)\right)\right\} \right)  \right]
\end{align}



% \begin{table}[ht]
%   \centering
%   \begin{tabular}{lll}
%     \toprule
%     Objective name & Objective to minimize wrt $\kk$  & Computational solution \\ \midrule
%     Profile Likelihood & $-\log \max_{\uu \in \Uspace} p_{Z \mid \KK, \UU}(y \mid \kk, \uu)$  \\
%     Integrated Likelihood & $-\log \int_{\Uspace} p_{Z \mid \KK, \UU}(y \mid \kk, \uu) \,\mathrm{d}\uu=-\log p_{Z\mid \KK}(y\mid \kk)$ \\
%     Marginal maximum a posteriori & $-\log p_{\KK\mid }(\kk \mid y)$  & MCMC based sampling (\cite{doucet_marginal_2002})\\ \midrule
%     Global Optimum & $\min_{\uu\in\Uspace} J(\kk, \uu)$ & EGO (\cite{jones_efficient_1998})\\
%     Worst-case & $\max_{\uu \in \Uspace} J(\kk, \uu)$ \\
%     Regret worst-case & $\max_{\uu\in\Uspace}\left\{J(\kk, \uu) - \min_{\kk^{\prime}\in\Kspace} J(\kk^{\prime}, \uu)\right\}$ \\ \midrule
%     Mean & $\Ex_{\UU}[J(\kk, \UU)]$ & Projected GP \\
%     Mean and variance & $ \lambda \Ex_{\UU}[J(\kk, \UU)] + (1-\lambda) \sqrt{\Var_{\UU}[J(\kk, \UU)]}$ & Projected GP \\ \bottomrule
%   \end{tabular}
%   \caption{Summary of single objective robust estimators}
% \end{table}



%%%%%%%%%%%%%%%%%%%%%%%%%%%%%%%%%%%%%%%%%%%%%%%%%%%%%%%
%%%%%%%%%%%%%%%%%%%%%%%%%%%%%%%%%%%%%%%%%%%%%%%%%%%%%%%
%%%%%%%%%%%%%%%%%%%%%%%%%%%%%%%%%%%%%%
%% BIB
%%%%%%%%%%%%%%%%%%%%%%%%%%%%%%%%%%%%%%
\subfileLocal{
	\pagestyle{empty}
	\bibliographystyle{alpha}
	\bibliography{/home/victor/acadwriting/bibzotero}
}
\end{document}


%%% Local Variables:
%%% mode: latex
%%% TeX-master: "../../Main_ManuscritThese"
%%% End:
