\documentclass[../../Main_ManuscritThese.tex]{subfiles}

\subfileGlobal{
\renewcommand{\RootDir}[1]{./Text/Chapter4/#1}
}

% For cross referencing
\subfileLocal{
\externaldocument{../../Text/Introduction/build/Introduction}
\externaldocument{../../Text/Chapter2/build/Chapter2}
\externaldocument{../../Text/Chapter3/build/Chapter3}
\externaldocument{../../Text/Chapter5/build/Chapter5}
\externaldocument{../../Text/Conclusion/build/Conclusion}
}

%%%%%%%%%%%%%%%%%%%%%%%%%%%%%%%%%%%%%%
%% CHAPTER TITLE
%%%%%%%%%%%%%%%%%%%%%%%%%%%%%%%%%%%%%%

\begin{document}

\subfileLocal{\setcounter{chapter}{3}}
\specialchapter{Adaptative design enrichment for calibration using Gaussian Processes}
\label{chap:adaptative_design_gp}
\minitoc
\newpage
\subfileLocal{\pagestyle{contentStyle}}
%%%%%%%%%%%%%%%%%%%%%%%%%%%%%%%%%%%%%%%%%%%%%%%%%%%%%%%
%%%%%%%%%%%%%%%%%%%%%%%%%%%%%%%%%%%%%%%%%%%%%%%%%%%%%%%


%%%%%%%%%%%%%%%%%%%%%%%%%%%%%%%%%%%%%%%%%%%%%%%%%%%%%%
%%% 				SECTION 0 : Introduction	   %%%

\section{Gaussian Processes}
\label{sec:GP}

Let us assume that we have a map $f$ from a $p$ dimensional space to $\mathbb{R}$:
\begin{align}
  \begin{array}{rrcl}
    f: & \mathbb{X} \subset \mathbb{R}^p& \longrightarrow & \mathbb{R} \\
       & x & \longmapsto & f(x)
  \end{array}
\end{align}
This function is assumed to have been evaluated on a design of $n$ points, $\mathcal{X} \subset \mathbb{X}^n$. 
We wish to have a probabilistic modelling of this function
We introduce random processes as way to have a prior distribution on function
This uncertainty on $f$ is modelled as a random process:
\begin{equation}
  \begin{array}{rcl}
    Z: \mathbb{X} \times \Omega& \longrightarrow & \mathbb{R} \\
    (x,\omega) & \longmapsto & Z(x,\omega)
  \end{array}
\end{equation}
The $\omega$ variable will be omitted next.
\subsection{Linear Estimation}
\label{sec:linear_estimation}
A linear estimation $\hat{Z}$ of $f$ at an unobserved point $x\notin \mathcal{X}$ can be written as
\begin{equation}
  \label{eq:lin_est}
  \hat{Z}(x) =
  \begin{bmatrix}
    w_1 \dots w_n
    \end{bmatrix}
    \begin{bmatrix}
      f(x_1) \\ \vdots \\ f(x_n)
    \end{bmatrix} = \mathbf{W}^Tf(\mathcal{X}) = \sum_{i=1}^n w_i(x) f(x_i)
\end{equation}
Using those kriging weights $\mathbf{W}$, a few additional conditions must be added, in order to obtain the Best Linear Unbiased Estimator:
\begin{itemize}
\item Non-biased estimation: $\Ex[\hat{Z}(x) - Z(x)]=0$
\item Minimal variance: $\min~\Ex[(\hat{Z}(x) - Z(x))^2]$
\end{itemize}
Translating using Eq.\eqref{eq:lin_est}:
\begin{equation}
  \Ex[\hat{Z}(x) - Z(x)]=0 \iff m(\sum_{i=1}^n w_i(x)-1) = 0 \iff \sum_{i=1}^n w_i(x) = 1 \iff \mathbf{1}^T \mathbf{W} = 1
\end{equation}
For the minimum of variance, we introduce the augmented vector $\mathbf{Z}_n(x) = [Z(x_1),\dots Z(x_n), Z(x)]$, and
the variance can be expressed as:
\begin{align}
  \Ex[(\hat{Z}(x) - Z(x))^2] &= \Cov\left[[\mathbf{W}^T, -1] \cdot \mathbf{Z}_n(x) \right] \\
                             &= [\mathbf{W}^T, -1] \Cov\left[\mathbf{Z}_n(x) \right] [\mathbf{W}^T, -1]^T
\end{align}
In addition, we have
\begin{equation}
  \Cov\left[\mathbf{Z}_n(x) \right] =
  \begin{bmatrix}
    \Cov\left[
      \begin{bmatrix}
        Z(x_1) \dots Z(x_n)
      \end{bmatrix}^T\right]
    & \Cov\left[
      \begin{bmatrix}
        Z(x_1) \dots Z(x_n)
      \end{bmatrix}^T, Z(x) \right]
  \\
  \Cov\left[
    \begin{bmatrix}
      Z(x_1) \dots Z(x_n)
    \end{bmatrix}^T, Z(x) \right]^T & \Var\left[Z(x)\right]
  \end{bmatrix}
\end{equation}
Once expanded, the kriging weights solve then the following optimisation problem:
\begin{align}
  \min_{\mathbf{W}} ~&\mathbf{W}^T \Cov\left[Z(x_1) \dots Z(x_n)\right] \mathbf{W}\\ &-\Cov\left[
    \begin{bmatrix}
      Z(x_1) \dots Z(x_n)
    \end{bmatrix}^T, Z(x) \right]^T \mathbf{W}\\ &- \mathbf{W}^T\Cov\left[
    \begin{bmatrix}
      Z(x_1) \dots Z(x_n)
    \end{bmatrix}^T, Z(x) \right] \\ &+ \Var\left[Z(x)\right] \\
  \text{s.t.}& \mathbf{W}^T \mathbf{1} = \mathbf{1}
\end{align}
This leads to
\begin{align}
  \begin{bmatrix}
    \mathbf{W} \\ m
  \end{bmatrix}
  &=
  \begin{bmatrix}
    \Cov\left[Z(x_1) \dots Z(x_n)\right] & \mathbf{1} \\
  \mathbf{1}^T & 0
\end{bmatrix}^{-1}
                 \begin{bmatrix}
                  \Cov\left[
    \begin{bmatrix}
      Z(x_1) \dots Z(x_n)
    \end{bmatrix}^T, Z(x) \right]^T \\ 1 
\end{bmatrix}
  \\ &=
    \begin{bmatrix}
      C(x_1, x_1) & \cdots & C(x_1, x_n) & 1 \\
      C(x_2, x_1) & \cdots & C(x_2, x_n) & 1 \\
      \vdots & \ddots & \vdots & \vdots \\
      C(x_n, x_1) & \cdots & C(x_n, x_n)& 1 \\
      1 & \cdots & 1 & 0
    \end{bmatrix}^{-1}
                       \begin{bmatrix}
                         C(x_1, x) \\
                         C(x_2, x) \\
                         \vdots \\
                         C(x_n, x) \\
                         1
                       \end{bmatrix}
\end{align}

\subsection{Covariance functions}
\label{sec:cov_fun}







%%%%%%%%%%%%%%%%%%%%%%%%%%%%%%%%%%%%%%%%%%%%%%%%%%%%%%%
%%%%%%%%%%%%%%%%%%%%%%%%%%%%%%%%%%%%%%%%%%%%%%%%%%%%%%%
%%%%%%%%%%%%%%%%%%%%%%%%%%%%%%%%%%%%%%
%% BIB
%%%%%%%%%%%%%%%%%%%%%%%%%%%%%%%%%%%%%%
\subfileLocal{
	\pagestyle{empty}
	\bibliographystyle{alpha}
	\bibliography{../../bibzotero}
}
\end{document}


%%% Local Variables:
%%% mode: latex
%%% TeX-master: "../../Main_ManuscritThese"
%%% End:
