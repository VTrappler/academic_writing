%%%%%%%%%%%%%%%%%%%%%%%%%%%%%%%%%%%%%%%%%%%%%%%%
\section{NIH reparameterization}
\label{sec:5_NIH}


In \secref{sec:5_SSM} were computed Statistical Shape Models using PCA. The registration results of the SSMs were promising, particularly the individual models which could approximate the \db* meshes with a mean distance below $0.3$ mm. These models were used to register the only publicly available wrist database: the NIH database \cite{moore_2007_digital}. 

Correspondence for the NIH database was not computed by using the same method than the one used for the \db* introduced in Chapter \ref{chap:Method}. Indeed, this database present additional difficulties: the radii and metacarpals are cut, and only small and variable portions of these bones are visible. It makes the alignment between the bones complicated, and if templates describe common information to all subjects, it significantly limits knowledge to be learned from these bones, the carpal bones only are entirely known. 

On the other hand, using the PCA-based SSMs computed with the \db* to register the NIH database offers several interests. At first, as previously mentioned, a model quality is dependent on its training data diversity, and its assessment relies on the diversity of the test set. Therefore to have the possibility to test the model on more data from another database is an opportunity to further validate the model. Moreover if the results are good enough, the registered models can serve as a reparameterization of the NIH database. In this case the 60 wrists would be in dense correspondence with the \db*, which would enable the use of both databases at once for the calculation of more complete models for instance. Or deeper shape analysis could be carried out due to the larger amount of data, since there is high chances that a larger span of wrists types are covered. Moreover, additional information can be learned from the models adjustments to the NIH bones: the CMC-based models encode complete metacarpals, and can be used as hints of what the whole bone shapes are like.


\subsection{Dense correspondence mapping of the NIH using the SSMs}

Similarly to the original meshes of the \db*, the vertices of the raw meshes of the NIH database are irregularly spread along the bones surface (see \figref{im:2_CT2mesh}). This uneven distribution can skew the distance measures between meshes. A first step consists in resampling the bones to describe the shapes with the same number of vertices, but with an homogeneous distribution and a stable edge length. As previously, this resampling is computed using the graphite software \cite{graphite}, based on a Centroidal Voronoi Tesselation. These resampled homogeneous meshes are later on used as reference shapes. On the opposite of the post-segmentation processing that was performed on the first database, we did not need to remove coarse errors of segmentation.

To be able to adapt the statistical models to the bones, the latter have to be aligned to the former. Our optimization algorithm includes refinement of the rigid and scale alignment to the model at every iterations. However shapes of short radii are too different to the associated model, which makes basic shape alignment impossible, preprocessing is required. Six feature points were manually selected on all radii and metacarpals of the NIH database and on the mean models. The bones were then roughly aligned to the models by minimizing the distances between sets of feature points. Additional adjustments could be manually done if needed. For each bone, the plane along which it had been cut was computed, and any vertex within one millimeter of the plane or beyond it was listed. A customized model was created for each bone by removing all recorded vertices, so both shapes would be similar. Vertices within one millimeter of the cut plane were also removed from the database meshes, these points being absent from the associated model. Carpal bones were roughly aligned by using the transformation matrix of the radii, it is enough to guarantee the convergence of the alignment algorithm. 

Afterward, the models are being registered to the database bones. The latter are individual SSMs of each bone, computed with a training set of 42 wrists from the \db* (please refer to \secref{subsec:4_PCA}). According to the mean shape, the eigenvectors and their associated eigenvalues, the best parameters to approximate each NIH bone are calculated. It was decided that 39 vectors would be used for the models, which is almost all available vectors. Indeed throughout the leave-one-out tests of the SSMs, it was proven that the more vectors are used, the better the results. 

The SSMs of the radius and metacarpals describe the entire bones, however a portion only of the bone can be registered to the targets. All vertices of the SSM previously listed as beyond the cut plane for the current subject are therefore ignored during the distance computation. Moreover the refinement of the bone alignment by the optimization algorithm uses the customized model associated to the target, deformed in the same way than the SSM. Two radii of the NIH database had such a small part of them captured that they were discarded from the calculations. 

In order to use the NIH database for further studies, we want to perfectly capture the details of the bones. Yet, if the results of the SSMs are comparable with the ones obtained with the leave-one-out method on the \db*, we expect to correctly approximate the global shapes of the bones, but details should be missed by the registration. Therefore, in addition to the model registration via optimization of the parameters, a second step of projection along the normals is added. The same process was used when the \db* was reparameterized (please refer to \secref{ssubsec:projection}). The shapes being already really close from each other, the points are not projected far, which prevents the meshes from having crossed faces or wrongly projected points. It allows to refine the registration by capturing the sharp details of the targets. All points listed as beyond the cut plane of the target mesh are removed from the final meshes, no corresponding target surface exists to project them onto. 

Finally the resulting meshes are moved back to their target bones initial positions and orientations, to reconstruct the wrists. Two types of final wrists are available: the ones described by a deformation of the SSMs that include the whole metacarpals and a large portion of the radii, thanks to the extra information brought by the SSMs. The second type is composed of more precise meshes, that were projected onto the target surfaces, but only the portions of bones visible in the initial data are apprehended by the new meshes. Both types of wrists are in correspondence with the \db* by construction. Only the similarity of shape between the database bones and the resulting meshes need to be validated, which is analyzed in the next section. 


\subsection{Numerical Validation}

In the previous section was described the correspondence mapping of the NIH database with the \db*. It was performed by optimizing individual SSMs previously created based on the second database to register the new one. In addition to the optimization of the SSMs, a second set of resulting meshes was created by additionally projecting the deformed SSMs towards the target bones. By construction all meshes of bone $b$ for any subject are in correspondence with the ones of the \db*, whether resulting from SSM deformation of further projection along the normals. To assess the quality of the correspondence mapping meshes, and to prove that they can be used instead of the original raw meshes, only the similarity of the encoded shapes needs to be evaluated. 


\paragraph{NIH resampling}

At the very first, the NIH database has been resampled, to be described by an homogeneous distribution of points over the surface. The treatment is the same than the one applied to the \db*. Similarly, this step is very important, since the resulting meshes are used as reference for the rest of the work. If differences are introduced during this step, they will remain in the data. Therefore particular attention is paid to achieve proper outcomes. 

In \tabref{tab:dist_nih_raw_resampled} the results of the resampling step are introduced. Dissimilarity between meshes is computed using two mesh-to-mesh distances: a mean distance from a vertex to its closest point on the paired surface \eqref{eq:mesh_dist} and a max distance, which corresponds to the maximal distance over all vertices of both meshes of a vertex to its closest point \eqref{eq:mesh_hausdorff}. The mean distance between a vertex and its closest point is in average between 0.002 and 0.004 mm and the maximal values are included in range $[0.004; 0.007]$ mm. The Hausdorff distance is in average between 0.026 and 0.078 mm, while its maximal values are under 1.5mm, that is below the acquisition precision. Only one \first* metacarpal has a maximal value of 0.201mm, slightly above the 0.2 mm threshold. After verification, this higher value is due to small holes in the original mesh next to the cut plane that are mishandled by the algorithm. However since the holes are really close to the cut plane, they will not be taken into account in the following processes, and the distal side of the bone is not affected, so it was kept in the database

The results are in the same order of magnitude as the resampling results of the \db*, perhaps even slightly better (please refer to \secref{subsec:3:Results_preprocessing} for the \db* results). The precision of the initial data is between $0.2\times 0.2 \times 1$ and $0.3\times 0.3 \times 1$ mm. The distance between a vertex and the paired surface is in average largely below the precision of the raw data. Even in maxima, this distance is always below the initial precision, but for one bone. Given these distances, we can conclude that the resampled data perfectly characterize the original shapes of the NIH database.



%%%%%%         TABLEAU DISTANCE MESHES RAW 2 REMESHED NIH 
% Pour chaque patient, chaque os, de la base NIH 
% On a mesuré notre distance et la distance de Hausdorff entre le maillage original
% et le maillage remeshé une première fois avec Graphite. 
% On mesure la distance entre ces deux maillages, pour montrer qu'on peut
% utiliser les deux maillages indifféremment, 
% ils décrivent la même forme 3D. 

% Généré à partir du script : D:/users/emilie/chirocap/prog/cmc_database/nih/measure_res_nih/distance_meshes_raw2remeshed.py

\begin{table}[ht]
	\centering
	\begin{tabular}{>{\RaggedRight}p{3cm} %centré gauche
			>{\centering\arraybackslash}p{1.3cm}
			>{\centering\arraybackslash}p{1.3cm}
			>{\centering\arraybackslash}p{1.3cm}
			p{0.7cm}
			>{\centering\arraybackslash}p{1.3cm}
			>{\centering\arraybackslash}p{1.3cm}
			>{\centering\arraybackslash}p{1.3cm}}
		\toprule
		& \multicolumn{3}{c}{\textbf{Mean dist. \eqref{eq:mesh_dist}} \small{(mm)}} & & \multicolumn{3}{c}{\textbf{Hausdorff dist. \eqref{eq:mesh_hausdorff}} \small{(mm)}} \\
		& mean & max & std & & mean & max & std  \Tstrut \Bstrut \\
		\midrule \ \vspace{-2.5mm} & & & & & & & \\
		Radius		 	 & \textbf{0.004} & 0.005 & \footnotesize{0.001} & 		& \textbf{0.078} & 0.133 & \footnotesize{0.020}\\
		Scaphoid		 & \textbf{0.003} & 0.004 & \footnotesize{0.000} & 		& \textbf{0.043} & 0.068 & \footnotesize{0.009}\\
		Lunate		 	 & \textbf{0.003} & 0.004 & \footnotesize{0.000} & 		& \textbf{0.038} & 0.069 & \footnotesize{0.009}\\
		Triquetrum		 & \textbf{0.002} & 0.005 & \footnotesize{0.000} & 		& \textbf{0.037} & 0.073 & \footnotesize{0.010}\\
		Pisiform		 & \textbf{0.002} & 0.004 & \footnotesize{0.001} & 		& \textbf{0.026} & 0.069 & \footnotesize{0.009}\\
		Trapezoid		 & \textbf{0.003} & 0.004 & \footnotesize{0.000} & 		& \textbf{0.039} & 0.070 & \footnotesize{0.009}\\
		Trapezium		 & \textbf{0.003} & 0.004 & \footnotesize{0.000} & 		& \textbf{0.040} & 0.057 & \footnotesize{0.008}\\
		Capitate		 & \textbf{0.004} & 0.005 & \footnotesize{0.001} & 		& \textbf{0.051} & 0.084 & \footnotesize{0.010}\\
		Hamate		 	 & \textbf{0.003} & 0.004 & \footnotesize{0.000} & 		& \textbf{0.050} & 0.069 & \footnotesize{0.008}\\
		Metac. 1		 & \textbf{0.003} & 0.006 & \footnotesize{0.001} & 		& \textbf{0.060} & 0.201 & \footnotesize{0.029}\\
		Metac. 2		 & \textbf{0.004} & 0.007 & \footnotesize{0.001} & 		& \textbf{0.064} & 0.125 & \footnotesize{0.018}\\
		Metac. 3		 & \textbf{0.004} & 0.007 & \footnotesize{0.001} & 		& \textbf{0.060} & 0.101 & \footnotesize{0.017}\\
		Metac. 4		 & \textbf{0.003} & 0.006 & \footnotesize{0.001} & 		& \textbf{0.048} & 0.078 & \footnotesize{0.012}\\
		Metac. 5		 & \textbf{0.003} & 0.006 & \footnotesize{0.001} & 		& \textbf{0.052} & 0.104 & \footnotesize{0.018}\\
		\bottomrule
	\end{tabular}
	\caption[Distance between original and resampled meshes of the NIH database]{Distances between the original meshes and a first resampling to regularize the vertices and edges distribution on the surface of the NIH bones. Both mean and Hausdorff distances \eqref{eq:mesh_dist} and \eqref{eq:mesh_hausdorff} are computed, the results are in mm. }
	\label{tab:dist_nih_raw_resampled}
\end{table}


\paragraph{Initial comparison to SSMs mean shapes}

The initial similarity between SSMs mean shapes and the NIH bones shapes is measured, in order to be compared with later model registration results, to evaluate the model efficiency.
In \tabref{tab:dist_nih_tpl2resampled} are presented the distances between the individual SSMs mean shapes and the meshes of the NIH bones, when the bones have been rigidly aligned and scaled. These distances are for comparison with later results of the SSMs registration. It enables to measure the quality of the registration compared to the initial situation. 

The average distance between a vertex and the paired surface is included in range $[0.311; 0.747]$ mm, and the worst values go as high as 0.591 to 1.673 mm. In average the surfaces are far from each other, the global shapes aren't captured by the mean shape. It proves the variability of shape among the same class of bones. The Hausdorff distances are in average between 1.218 and 2.779 mm, while the maximal values of the Hausdorff distance are for some bones higher than 5 mm, even up to 7.038 mm for a \first* metacarpal. Compared to the size of the bones, these distances are really high. 
These results are in the same order of magnitude as the difference between the \db* meshes and the initial templates visible in \ref{tab:dist_resampled_template}. They are slightly worse for the metacarpals and the radii, due to the cut ends of the bones, although the distance functions have been modified to take these into account. 




%%%%%%         TABLEAU DISTANCE NIH TEMPLATE 2 REMESHED 
% Pour chaque patient, chaque os, de la base NIH 
% On a mesuré notre distance et la distance de Hausdorff entre le maillage remeshé une première fois avec Graphite. 
% et les templates 
% On mesure la distance entre ces deux maillages, pour montrer qu'on peut
% utiliser les deux maillages indifféremment, 
% ils décrivent la même forme 3D. 

% Généré à partir du script : D:/users/emilie/chirocap/prog/cmc_database/nih/measure_res_nih/distance_meshes_mean_pca_final2remeshed.py

\begin{table}[ht]
	\centering
	\begin{tabular}{>{\RaggedRight}p{3cm} %centré gauche
			>{\centering\arraybackslash}p{1.3cm}
			>{\centering\arraybackslash}p{1.3cm}
			>{\centering\arraybackslash}p{1.3cm}
			p{0.7cm}
			>{\centering\arraybackslash}p{1.3cm}
			>{\centering\arraybackslash}p{1.3cm}
			>{\centering\arraybackslash}p{1.3cm}}
		\toprule
		& \multicolumn{3}{c}{\textbf{Mean dist. \eqref{eq:mesh_dist}} \small{(mm)}} & & \multicolumn{3}{c}{\textbf{Hausdorff dist. \eqref{eq:mesh_hausdorff}} \small{(mm)}} \\
		& mean & max & std & & mean & max & std  \Tstrut \Bstrut \\
		\midrule \ \vspace{-2.5mm} & & & & & & & \\
		Radius		 & \textbf{0.747} & 1.673 & \footnotesize{0.275} & 		& \textbf{2.779} & 7.038 & \footnotesize{1.003}\\
		Scaphoid	 & \textbf{0.373} & 1.022 & \footnotesize{0.123} & 		& \textbf{1.758} & 4.431 & \footnotesize{0.700}\\
		Lunate		 & \textbf{0.411} & 1.142 & \footnotesize{0.184} & 		& \textbf{1.557} & 2.943 & \footnotesize{0.482}\\
		Triquetrum	 & \textbf{0.320} & 0.624 & \footnotesize{0.082} & 		& \textbf{1.280} & 1.936 & \footnotesize{0.286}\\
		Pisiform	 & \textbf{0.311} & 0.591 & \footnotesize{0.089} & 		& \textbf{1.218} & 2.129 & \footnotesize{0.295}\\
		Trapezoid	 & \textbf{0.345} & 0.884 & \footnotesize{0.120} & 		& \textbf{1.416} & 2.421 & \footnotesize{0.416}\\
		Trapezium	 & \textbf{0.362} & 0.875 & \footnotesize{0.127} & 		& \textbf{1.508} & 4.221 & \footnotesize{0.553}\\
		Capitate	 & \textbf{0.417} & 1.162 & \footnotesize{0.167} & 		& \textbf{2.011} & 4.489 & \footnotesize{0.771}\\
		Hamate		 & \textbf{0.382} & 1.202 & \footnotesize{0.171} & 		& \textbf{1.656} & 4.846 & \footnotesize{0.665}\\
		Metac. 1	 & \textbf{0.566} & 1.354 & \footnotesize{0.219} & 		& \textbf{2.726} & 7.459 & \footnotesize{1.582}\\
		Metac. 2	 & \textbf{0.556} & 1.372 & \footnotesize{0.180} & 		& \textbf{2.070} & 4.064 & \footnotesize{0.642}\\
		Metac. 3	 & \textbf{0.457} & 0.964 & \footnotesize{0.143} & 		& \textbf{1.939} & 5.195 & \footnotesize{0.717}\\
		Metac. 4	 & \textbf{0.448} & 1.153 & \footnotesize{0.158} & 		& \textbf{1.806} & 5.448 & \footnotesize{0.939}\\
		Metac. 5	 & \textbf{0.422} & 0.796 & \footnotesize{0.110} & 		& \textbf{1.549} & 3.570 & \footnotesize{0.514}\\
		\bottomrule
	\end{tabular}
	\caption[Distance between resampled meshes of the NIH database and the SSMs mean shapes]{Distances between the NIH database meshes and the SSMs mean shapes. Both mean and Hausdorff distances \eqref{eq:mesh_dist} and \eqref{eq:mesh_hausdorff} are computed, the results are in mm. }
	\label{tab:dist_nih_tpl2resampled}
\end{table}




%%%%%%         TABLEAU DISTANCE NIH PCA 2 REMESHED 
% Pour chaque patient, chaque os, de la base NIH 
% On a mesuré notre distance et la distance de Hausdorff entre le maillage remeshé une première fois avec Graphite. 
% et les résultats de la pca 
% On mesure la distance entre ces deux maillages, pour montrer qu'on peut
% utiliser les deux maillages indifféremment, 
% ils décrivent la même forme 3D. 

% Généré à partir du script : D:/users/emilie/chirocap/prog/cmc_database/nih/measure_res_nih/distance_meshes_mean_pca_final2remeshed.py

\begin{table}[ht]
	\centering
	\begin{tabular}{>{\RaggedRight}p{3cm} %centré gauche
			>{\centering\arraybackslash}p{1.3cm}
			>{\centering\arraybackslash}p{1.3cm}
			>{\centering\arraybackslash}p{1.3cm}
			p{0.7cm}
			>{\centering\arraybackslash}p{1.3cm}
			>{\centering\arraybackslash}p{1.3cm}
			>{\centering\arraybackslash}p{1.3cm}}
		\toprule
		& \multicolumn{3}{c}{\textbf{Mean dist. \eqref{eq:mesh_dist}} \small{(mm)}} & & \multicolumn{3}{c}{\textbf{Hausdorff dist. \eqref{eq:mesh_hausdorff}} \small{(mm)}} \\
		& mean & max & std & & mean & max & std  \Tstrut \Bstrut \\
		\midrule \ \vspace{-2.5mm} & & & & & & & \\
		Radius		 	 & \textbf{0.195} & 0.304 & \footnotesize{0.037} & 		& \textbf{1.039} & 2.526 & \footnotesize{0.283}\\
		Scaphoid		 & \textbf{0.137} & 0.191 & \footnotesize{0.020} & 		& \textbf{0.670} & 0.999 & \footnotesize{0.123}\\
		Lunate		 	 & \textbf{0.123} & 0.191 & \footnotesize{0.019} & 		& \textbf{0.638} & 1.013 & \footnotesize{0.129}\\
		Triquetrum		 & \textbf{0.123} & 0.165 & \footnotesize{0.015} & 		& \textbf{0.622} & 0.879 & \footnotesize{0.108}\\
		Pisiform		 & \textbf{0.117} & 0.208 & \footnotesize{0.040} & 		& \textbf{0.575} & 1.533 & \footnotesize{0.207}\\
		Trapezoid		 & \textbf{0.144} & 0.305 & \footnotesize{0.039} & 		& \textbf{0.744} & 1.320 & \footnotesize{0.195}\\
		Trapezium		 & \textbf{0.152} & 0.197 & \footnotesize{0.019} & 		& \textbf{0.766} & 1.328 & \footnotesize{0.155}\\
		Capitate		 & \textbf{0.173} & 0.236 & \footnotesize{0.022} & 		& \textbf{0.908} & 1.409 & \footnotesize{0.158}\\
		Hamate		 	 & \textbf{0.167} & 0.250 & \footnotesize{0.027} & 		& \textbf{0.910} & 1.583 & \footnotesize{0.198}\\
		Metac. 1		 & \textbf{0.139} & 0.211 & \footnotesize{0.030} & 		& \textbf{0.710} & 1.293 & \footnotesize{0.152}\\
		Metac. 2		 & \textbf{0.175} & 0.227 & \footnotesize{0.024} & 		& \textbf{0.996} & 2.599 & \footnotesize{0.292}\\
		Metac. 3		 & \textbf{0.151} & 0.202 & \footnotesize{0.023} & 		& \textbf{0.783} & 1.214 & \footnotesize{0.179}\\
		Metac. 4		 & \textbf{0.133} & 0.182 & \footnotesize{0.017} & 		& \textbf{0.690} & 1.222 & \footnotesize{0.142}\\
		Metac. 5		 & \textbf{0.121} & 0.165 & \footnotesize{0.017} & 		& \textbf{0.611} & 0.905 & \footnotesize{0.127}\\
		\bottomrule
	\end{tabular}
	\caption[Distance between resampled meshes of the NIH database and the registered SSMs]{Distances between the NIH database meshes and the registered SSMs. Both mean and Hausdorff distances \eqref{eq:mesh_dist} and \eqref{eq:mesh_hausdorff} are computed, the results are in mm. }
	\label{tab:dist_nih_pca2resampled}
\end{table}



\paragraph{SSMs registration}

When the bones are aligned and scaled to the individual SSMs, the latter are being deformed following their principal modes of variations. They are being registered to the database meshes by optimization of a mean distance function. The resulting registered SSMs are being compared to the database meshes. All vertices of the models that have been identified as having no equivalent target surface are ignored in the distance computation. The average mean distances are included between $0.117$ and $0.195$ mm, while the highest mean distances are in range $[0.165; 0.305]$ mm. The Hausdorff distances are included between $0.575$ and $1.039$ mm in average and between $0.879$ and $2.599$ mm for the maximal values.

When the registered SSMs are being compared to the mean shapes, we can observe that they are strictly closer to the target meshes, as expected. We can also note that in average the mean distance between a vertex and the paired surface is smaller than the initial precision of the data, though some of the highest mean distances are above the 0.2 mm threshold, which is the best original precision of some of the original data. We can however conclude that the global shape is well captured by the deformed models. However the Hausdorff distances for their part are quite high, details are definitely missed. These distances can be compared to the distances obtained between individual models registered using 39 principal modes to the \db* bones, in \figref{im:4_mean_one_nbVec} and \figref{im:4_hausdorff_one_nbVec}. The results are slightly better for the NIH registration than they were with the leave-one-out method for the \db* registration. 

We can note that the SSMs of the metacarpals, created based on the \db*, represent the whole bones, while they are being registered to the proximal end of the bones only. This has both upsides and downsides. The model is not exactly adapted to the data it must describe. Some of its modes could characterize shape variations in the distal end of the bone only, which could therefore give absurd values associated to these modes. Since the models are not perfectly fitting the data, it is one of the reason why the results are somewhat worse for these bones. The same apply to the radius, which portion of diaphysis available in the model is far longer than the ones in the NIH database radii. We could imagine to learn the models only for the length of the bone visible in the target mesh. The model would therefore be more adapted. However, in exchange of a model less customized to the target shapes, we gain information about the distal end of the bone that is not initially present in the database. The models bring additional knowledge.  


%%%%%%         TABLEAU DISTANCE NIH SNAP 2 REMESHED 
% Pour chaque patient, chaque os, de la base NIH 
% On a mesuré notre distance et la distance de Hausdorff entre le maillage remeshé une première fois avec Graphite. 
% et les résultats de la pca + snap 
% On mesure la distance entre ces deux maillages, pour montrer qu'on peut
% utiliser les deux maillages indifféremment, 
% ils décrivent la même forme 3D. 

% Généré à partir du script : D:/users/emilie/chirocap/prog/cmc_database/nih/measure_res_nih/distance_meshes_mean_pca_final2remeshed.py

\begin{table}[ht]
	\centering
	\begin{tabular}{>{\RaggedRight}p{3cm} %centré gauche
			>{\centering\arraybackslash}p{1.3cm}
			>{\centering\arraybackslash}p{1.3cm}
			>{\centering\arraybackslash}p{1.3cm}
			p{0.7cm}
			>{\centering\arraybackslash}p{1.3cm}
			>{\centering\arraybackslash}p{1.3cm}
			>{\centering\arraybackslash}p{1.3cm}}
		\toprule
		& \multicolumn{3}{c}{\textbf{Mean dist. \eqref{eq:mesh_dist}} \small{(mm)}} & & \multicolumn{3}{c}{\textbf{Hausdorff dist. \eqref{eq:mesh_hausdorff}} \small{(mm)}} \\
		& mean & max & std & & mean & max & std  \Tstrut \Bstrut \\
		\midrule \ \vspace{-2.5mm} & & & & & & & \\
		Radius		 & \textbf{0.015} & 0.019 & \footnotesize{0.002} & 		& \textbf{0.287} & 0.565 & \footnotesize{0.096}\\
		Scaphoid	 & \textbf{0.020} & 0.025 & \footnotesize{0.002} & 		& \textbf{0.221} & 0.458 & \footnotesize{0.067}\\
		Lunate		 & \textbf{0.018} & 0.021 & \footnotesize{0.001} & 		& \textbf{0.199} & 0.315 & \footnotesize{0.047}\\
		Triquetrum	 & \textbf{0.023} & 0.028 & \footnotesize{0.002} & 		& \textbf{0.230} & 0.347 & \footnotesize{0.047}\\
		Pisiform	 & \textbf{0.019} & 0.024 & \footnotesize{0.002} & 		& \textbf{0.185} & 0.359 & \footnotesize{0.050}\\
		Trapezoid	 & \textbf{0.020} & 0.025 & \footnotesize{0.002} & 		& \textbf{0.255} & 0.517 & \footnotesize{0.081}\\
		Trapezium	 & \textbf{0.029} & 0.035 & \footnotesize{0.003} & 		& \textbf{0.295} & 0.533 & \footnotesize{0.071}\\
		Capitate	 & \textbf{0.019} & 0.023 & \footnotesize{0.002} & 		& \textbf{0.222} & 0.362 & \footnotesize{0.048}\\
		Hamate		 & \textbf{0.022} & 0.028 & \footnotesize{0.002} & 		& \textbf{0.283} & 0.431 & \footnotesize{0.069}\\
		Metac. 1	 & \textbf{0.013} & 0.019 & \footnotesize{0.002} & 		& \textbf{0.174} & 0.423 & \footnotesize{0.071}\\
		Metac. 2	 & \textbf{0.018} & 0.024 & \footnotesize{0.003} & 		& \textbf{0.216} & 0.363 & \footnotesize{0.062}\\
		Metac. 3	 & \textbf{0.020} & 0.030 & \footnotesize{0.004} & 		& \textbf{0.257} & 0.417 & \footnotesize{0.067}\\
		Metac. 4	 & \textbf{0.020} & 0.031 & \footnotesize{0.004} & 		& \textbf{0.208} & 0.360 & \footnotesize{0.058}\\
		Metac. 5	 & \textbf{0.018} & 0.029 & \footnotesize{0.003} & 		& \textbf{0.177} & 0.340 & \footnotesize{0.049}\\
		\bottomrule
	\end{tabular}
	\caption[Distance between resampled meshes of the NIH database and the final resulting deformed templates]{Distances between the NIH database meshes and the final resulting meshes. Both mean and Hausdorff distances \eqref{eq:mesh_dist} and \eqref{eq:mesh_hausdorff} are computed, the results are in mm. }
	\label{tab:dist_nih_snap2resampled}
\end{table}


\paragraph{Projection along the normals towards the target meshes}

As previously analyzed, the deformed SSMs capture the global shapes of the bones, but miss details. If we want to enrich our data by mixing both NIH and \db*, we need to fully trust the shapes characterized by the corresponding meshes. Therefore the shapes describing the NIH bones were refined by projecting the vertices towards the target surface, as previously done with the CMC data. 

In \tabref{tab:dist_nih_snap2resampled}, the distances between target bones and final meshes are presented. The mean distances between the shapes are smaller than 0.035 mm for all bones. This is largely below the initial precision if the data. The Hausdorff distances are included in range $[0.174; 0.295]$ mm with highest values up to 0.565 mm. The highest distances between vertices and the paired surface are a little higher than the original precision of the data. However, these distances are small nonetheless, half a millimeter at worst, and concern a few vertices at max. The data can be trusted to be used instead of the initial meshes. 


\subsection{Discussion}

In this section, by using the statistical shape models constructed based on the \db*, we have been able to reparameterize a second database and map it to be in dense correspondence with the first one.
We have proven that the shapes of the raw meshes and the final projected ones are very similar and can be used instead of each other. We haven't tested the quality of the correspondence between the vertices. However, the meshes were constructed by deformation of statistical shape models following their principal modes. By construction, they are in dense correspondence. 
It should also be noted that during the registration of the SSMs of complete bones to the only proximal end of these bones, we added information about the distal end of the metacarpals that is absent from the initial data. However, during the projection along the normal step, only vertices having close neighbors were projected. The parts with no initial information (distal end of the metacarpals, proximal end of the radii) could obviously not be refined. This step loses the extra knowledge brought by the former process. In the case where this info would be needed, the meshes output by the SSMs registration have to be used. 


 Since the SSMs were able to register properly the target meshes, it proves that they are not too specific to the training set. An additional step of projection along the normals of the vertices towards the target surface was used to refine the results. The final shapes are very close to the target ones, the average Hausdorff distance is below $0.3$ mm. We can use them instead of the original meshes. Additionally, by construction, the correspondence is guaranteed, the NIH database is now in correspondence with the \db*. In future studies it will enable the use of all subjects from both databases, which doubles the number of wrists considered. Issues had to be faced about the metacarpals and radii bones due to a variable length of bones visible in the NIH database, while longer portions were encoded in the SSMs. 

The two first sections of this chapter are dedicated to te construction of statistical models, and their registration to 3D shapes. This is mostly aimed at testing the quality of the models before using them for registration to images. In the following section, a completely different domain of application is proposed: biomechanics and definition of system of coordinates for joint movement measurement. 

