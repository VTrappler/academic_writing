\section{Conclusion}
\label{sec:4_Conclusion}

In this chapter we have proposed different applications of the database that was previously put in dense correspondence: the meshes can be used to create statistical models which can in turn be used to compute correspondence for new instances of the shapes, even when the latter are incomplete. The correspondence relations can also be used to transfer landmarks or coordinate systems from one or a few examples to all other occurrences of the database. 



First we have studied the accuracy level that could be reached by statistical models for 3D shape registration. PCA-based models representing one bone have proven to be able to reach a good similarity with the target mesh, the mean error is below the accuracy of the original data. They have been further tested on a new database and enabled to reparameterize the latter to be in correspondence with the meshes of the first database. We have chosen to use an additional step of projection along the normals of the optimized models towards the target meshes. It was aimed at warping as well as possible the NIH meshes in order to use them later as training shapes with certainty that they faithfully represent the bones. 


PCA-based statistical models were defined both for all wrist bones at once and for each bone separately. While the individual models have been shown to reach a good level of accuracy for 3D shape registration, the global model wasn't precise enough for the whole wrist registration. It is interesting to consider all bones at once, as they are really close from each other and the shape of one bone necessarily influences its neighbors shapes. However, the model has proved to need more principal modes than were available from the training set formed with the \db*. Now that the NIH database is in correspondence with the CMC one, more data are available, a possible perspective would be to compute a new model for the whole wrist using both databases together as prior information. In addition to a larger span of shapes learned, more modes would be available, which could enable the model to reach a higher level of registration accuracy. 


A second type of models were used: Gaussian Processes Morphable Models. Shapes are modeled as vertex deformations undergone by the mean shape. These models are promising: depending on the covariance function chosen, they can have various combined properties, and are not necessarily linear. Moreover, they adapt to posterior information, which allows human intervention in the results. A completely user-based registration approach was tested and was successful, really high similarity levels between shapes that can be reached as long as enough points are given by the user. However, we have concluded that in practice the algorithm cannot be used as is, it would require too much points hence too much time from a qualified person. In a second phase an all-automatic registration method based on a parametric approximation was tested, and achieved good results. They were not as good as the SSMs ones, but lack of time prevented us from optimizing the model parameters, and we believe that better similarity can be reached. We believe GPMMs to be promising and next steps would be to further test the model and maybe mix the all-automatic and all-user based methods. 

Statistical models can have various utilizations. For example they can be employed for a quantitative analysis of bone shapes and allow the identification of phenotypes \cite{chaudhari_2014_global}. They are often used as prior knowledge for image segmentation \cite{chen_2012_automatic, anas_2014_statistical}: the models are being registered to the 2D or 3D images and provide simultaneously a segmentation of the object in the image and a 3D representation of it. Such applications are dependent on the quality of the model and its capacity to represent new shapes. Now that it has been assessed that the SSMs are quite precise, these applications could be implemented. 


Finally a biomechanics oriented application was proposed. We have shown that based on the correspondence relations previously established, landmarks and orientations can be easily transferred from one or a few instances of the database to all others. We have proven that the reliability of our method is similar to another method proposed in the litterature. This method happens to be to the best of our knowledge the only automatic method of joint coordinate system definition for the TMC joint, except for the method proposed in \cite{coburn_2007_coordinate} which doesn't follow the ISB prescriptions. Our method is less specialized than the one it was compared to and can therefore be employed on any other joint of the wrist. It can also be used for any other application requiring the transfer of points to other instances of a same shape. Additionally the reliability of the method was a proof of the quality of the correspondence relations defined in the previous chapter. 

This chapter and the last one's purpose was to study the wrist bone shapes. In the next chapter we will present the work we have done about wrist movement. The ultimate goal is to be able to take the influence of wrist bones shapes into account when the movement is modeled. 