\documentclass[../../Main_ManuscritThese.tex]{subfiles}

\subfileGlobal{
\renewcommand{\RootDir}[1]{./Text/Resumes/#1}
}

% \subfileLocal{
% \externaldocument{../../Text/Chapter2/build/Chapter2}
% \externaldocument{../../Text/Chapter3/build/Chapter3}
% \externaldocument{../../Text/Chapter4/build/Chapter4}
% \externaldocument{../../Text/Chapter5/build/Chapter5}
% \externaldocument{../../Text/Conclusion/build/Conclusion}
% }

%%%%%%%%%%%%%%%%%%%%%%%%%%%%%%%%%%%%%%
%% CHAPTER TITLE
%%%%%%%%%%%%%%%%%%%%%%%%%%%%%%%%%%%%%%

\begin{document}

\chapter*{Résumé Français}
\TitleBtwLines

\phantomsection
\addstarredchapter{Résumé Français}
\label{chap:resume_fr}
%\newpage
%\minitoc
\pagestyle{resumeStyle}

                                % \subfileLocal{\pagestyle{contentStyle}}
\subsection*{Présentation générale et contexte}
De nombreux phénomènes naturels sont modélisés afin de mieux connaître
leurs comportements et de pouvoir les prévoir.  Cependant, lors du
processus de modélisation, de nombreuses sources d'erreurs sont
introduites. Elles proviennent par exemple des paramétrisations qui
rendent compte des phénomènes sous-mailles, ou bien de l'ignorance des
conditions environnementales réelles dans lesquelles le phénomène est
observé.

De manière plus formelle, on peut distinguer deux types d'incertitudes
dans ces modèles, comme évoqué dans~\cite{walker_defining_2003}.
\begin{itemize}
\item les incertitudes dites \emph{épistémiques}, qui proviennent d'un
  manque de connaissance sur des charactéristiques du phénomène
  étudié, mais qui pourraient être réduites
\item les incertitudes dites \emph{aléatoires}, qui proviennent
  directement de la variabilité intrinsèque du phénomène étudié.
\end{itemize}

Dans le cadre de la thèse, les incertitudes épistémiques prennent la
forme de la méconnaissance de la valeur d'un paramètre
$\kk \in \Kspace$, que l'on va chercher à calibrer. Les incertitudes
aléatoires sont par exemple les conditions environnementales, comme le
forçage d'un modèle ou les conditions aux bords. Elles vont être
représentées par une variable aléatoire $\UU$, de réalisation
$\uu\in \Uspace$.  Cette calibration est effectuée à l'aide d'une
fonction $J$, dite fonction coût. Cette fonction prendra donc en
entrée le paramètre à estimer $\kk$, ainsi que la variable
environementale $\uu$:

\begin{equation}
  \label{eq:def_J}
  \begin{array}{rccc}
   J: & \Kspace\times\mathbb{U}& \rightarrow& \mathbb{R}_+ \\
   &(\kk,\uu)& \mapsto& J(\kk,\uu)
  \end{array}
\end{equation}

Ne pas prendre en compte les incertitudes aléatoires dans l'estimation
de $\kk$ peut amener à compenser de manière artificielle l'erreur
aléatoire, et donc amener une situation similaire à du
\emph{sur-apprentissage} (overfitting). On cherche donc à définir une
valeur de $\kk$, notée $\hat{\kk}$ de manière à ce que
$J(\hat{\kk}, \uu)$ reste \emph{acceptable} lorsque $\uu$ varie.


En prenant en compte le caractère aléatoire de la variable
environnementale, pour un $\kk$ donné, la fonction coût peut être vue
comme une variable aléatoire: $J(\kk,\UU)$, que l'on va chercher
intuitivement à ``minimiser'' dans un certain sens qui reste à
définir.

Cette problématique porte différents noms, comme l'optimisation
robuste, où robuste doit être compris comme l'insensibilité aux
variations de $\UU$, optimisation sous incertitudes (Optimisation
under Uncertainty ou OUU), ou encore d'optimisation stochastique. Une
nomenclature prenant en compte les différences notamment sur les
contraintes potentiellement présentes peut être trouvé
dans~\cite{lelievre_consideration_2016}

Un exemple de ce genre de problème est l'estimation de la friction
dans les modèles océaniques.  En effet, la friction de fond est dûe à
la rugosité du plancher océanique, provoquant de la dissipation
d'énergie à cause des turbulences engendrées. L'estimation de la
friction de fond est un problème qui a déjà été traité dans un cadre
d'assimilation de données avec des méthodes variationelles comme
dans~\cite{das_estimation_1991,das_variational_1992} sur un cas
simplifié, ou dans un cas plus réaliste
dans~\cite{boutet_estimation_2015}, avec une méthode de gradient
stochastique, permettant de se passer du calcul du gradient.


L'objectif de la thèse est d'établir différents critères de
robustesse, et d'appliquer des méthodes adaptées permettant d'estimer
un paramètre en présence d'incertitudes. Cette estimation se réalise
dans un premier temps dans des cas simples (fonctions analytiques,
problèmes simplifiés de faibles dimensions), puis sur des problèmes
plus complexes d'estimation de la friction de fond (modèles réalistes
coûteux en temps de calcul, dimension élevée).

\subsection*{Critères basés sur le regret additif et relatif}

Un certain nombre des méthodes d'optimisation sous incertitudes se
basent sur la minimisation des moments de la variable aléatoire
$J(\mathbf{\cdot}, \UU)$ comme
dans~\cite{lehman_designing_2004,janusevskis_simultaneous_2010}, ou
bien se basent sur la résolution d'un problème
multiobjectifs~\cite{baudoui_optimisation_2012,ribaud_krigeage_2018}.

Dans le cadre de cette thèse, nous proposons une approche basée sur le
regret, qui consiste à comparer les valeurs de la fonction $J$ avec le
\emph{minimum conditionel}, qui est le minimum de la fonction
$J(\cdot, \uu)$, où $\uu$ est une réalisation de la variable aléatoire
$\UU$. Le minimum conditionel est donc défini par
\begin{equation}
  \label{eq:Jstar}
  J^*(\uu) = \min_{\kk\in\Kspace} J(\kk,\uu)
\end{equation}
et le \emph{minimiseur conditionel} associé est
\begin{equation}
  \label{eq:Kstar}
  \kk^*(\uu) = \argmin_{\kk\in \Kspace}J(\kk,\uu)
\end{equation}

\cite{trappler_robust_2020-1}
Ces quantités permettent de définir des critères de robustesse qui
prennent en compte la plus petite valeur atteignable pour des
conditions environnementales données $\uu$.  D'un point de vue
pratique, ces fonctions peuvent être estimées à l'aide de
méta-modèles, comme dans~\cite{ginsbourger_bayesian_2014}.
 

$\Prob_{\uu}\left[J(\kk,\uu) \leq \alpha J^*(\uu) \right]$, où
$\alpha \geq 1$ est un paramètre bien choisi. Ce critère peut être
rapproché de la Value-at-Risk~\cite{rockafellar_deviation_2002}, qui
est utilisé notamment dans le domaine de la finance.

\subsection*{Optimisation robuste et processus Gaussiens}
\todo{écrire}
\subsection*{Application au code de calcul CROCO}
\todo{écrire}


% Pour le futur et la fin de la thèse, plusieurs points sont à considérer:
% \begin{itemize}
% \item Des méthodes permettant un contrôle plus fin sur les propriétés recherchées du paramètre (aversion/recherche du risque) sont à explorer, comme le ``horsetail matching''~\cite{cook_horsetail_2018}, en prenant notamment pour cible la distribution des minimums.
% \item L'application à l'estimation de la friction de fond dans le cadre du modèle CROCO est à envisager. CROCO\footnote{\url{https://www.croco-ocean.org/}} (Coastal and Regional Ocean COmmunity model) est un modèle régional d'océan, notamment conçu pour des applications de modélisation côtières. La région étudiée dans le cadre de la thèse est la façade atlantique de la France. 
%   \begin{itemize}
%     \item Premièrement les incertitudes du modèle sont à spécifier. Une première piste envisagée est introduire des incertitudes sur les composantes de marée à ajouter dans le forçage.
%   \item La friction de fond, que l'on cherche à estimer, est un paramètre variant dans l'espace, et donc peut potentiellement être défini en chaque point du maillage d'un modèle. Ceci force à prendre en compte la possibilité d'une grande dimension de $\Kspace$. On peut donc s'interroger sur la façon dont les méthodes et les critères décrits vont se mettre à l'échelle, et sur la possibilité d'appliquer des procédures de réduction de dimension.
%   \item En plus de la possible grande dimension du problème, les modèles réalistes sont souvent très coûteux en terme de temps de calcul. Il serait donc intéressant de pouvoir réduire au plus les évaluations du code, en adoptant par exemples des méthodes basées sur l'utilisation de méta-modèles. 
%   \end{itemize}
% \end{itemize}

%%%%%%%%%%%%%%%%%%%%%%%%%%%%%%%%%%%%%%%%%%%%%%%%%%%%%%%
%%%%%%%%%%%%%%%%%%%%%%%%%%%%%%%%%%%%%%%%%%%%%%%%%%%%%%%


%%%%%%%%%%%%%%%%%%%%%%%%%%%%%%%%%%%%%%
%% BIB
%%%%%%%%%%%%%%%%%%%%%%%%%%%%%%%%%%%%%%
\subfileLocal{
	\pagestyle{empty}
	\bibliographystyle{alpha}
	\bibliography{../../bibzotero}
}
\end{document}



%%% Local Variables:
%%% mode: latex
%%% TeX-master: "../../Main_ManuscritThese"
%%% End:
