% 4e de couverture
\pagestyle{empty}


\small
\begin{center}
  \bf Abstract
\end{center}
\vspace{0.5cm}
Many physical phenomena are modelled numerically in
order to better understand and/or to predict their behaviour. However,
some complex and small scale phenomena can not be fully represented in
the models. The introduction of parametrization terms is usually the
solution to represent these unresolved processes, but those need to be
properly estimated.

Classical methods of parameter estimation usually imply the minimisation of an objective function, that measures the error between some observations and the results obtained by a numerical model. The optimum is directly dependent on the fixed nominal value given to the uncertain parameter ; and therefore may not be relevant in other conditions. 

 A good example of this type of problem is the estimation of bottom friction parameters of the ocean floor. This task is further complicated by the presence of uncertainties in certain other characteristics linking the bottom and the surface (eg boundary conditions).

In this work, we propose a new criterion, based on the relative-regret, in order to control in probability the deviation of the relative-regret above a specified threshold. We introduce also iterative methods based on Gaussian Processes, in order to compute efficiently quantities linked to this estimation.
Finally, we will apply some of those techniques on an academic model of a coastal area, in order to find a robust value of the bottom friction.

\vspace{1cm}
\etoile
\vspace{1cm}

\begin{center}
  \bf Résumé
\end{center}
\vspace{0.5cm}
De nombreux phénomènes physiques sont modélisés afin d'en mieux connaître les comportements ou de pouvoir les prévoir. Cependant, certains paramètres ne sont connus qu'ap\-proxi\-mati\-vement d'où la nécessité d'ajouter des termes correctifs. 

Les techniques utilisées actuellement pour ajuster de tels paramètres se basent sur la minimisation d'une fonction mesurant l'erreur entre les observations réelles, et les prévisions obtenues grâce au modèle numérique. Cette estimation ne prend absolument pas en comptes les incertitudes présentes dans les caractéristiques du modèle, augmentant d'autant l'erreur sur cette estimation. Estimer ces paramètres en présence d'incertitudes nécessite donc un traitement particulier.

Dans ce travail, nous proposons un nouveau critère d'optimisation sous incertitudes, basé sur le regret-relatif. Nous cherchons donc à maximiser la probabilité avec laquelle le regret-relatif reste sous un certain seuil, ou inversement, nous cherchons à minimiser le seuil atteint par le regret-relatif, à un niveau de confiance donné.  D'un point de vue pratique, nous proposons aussi des méthodes itératives basées sur les processus Gaussiens, permettant de calculer efficacement les quantités liées à ce nouveau critère.  Enfin, nous appliquons ces nouvelles méthodes à la calibration sous incertitudes d'un modèle océanique côtier, pour trouver un paramètre de friction de fond robuste.

