% 4e de couverture
%\pagestyle{empty}
\vfill

\newgeometry{top=3cm, left=3cm, right=3cm, bottom=3cm}
\scriptsize
% \small
\phantomsection
\addstarredchapter{Abstracts}
\begin{center}
 \small \bf Abstract
\end{center}
\vspace{0.3cm}

To understand and to be able to forecast natural phenomena is
increasingly important nowadays, as those predictions are often the
basis of many decisions, whether economical or ecological. In order to
do so, mathematical models are introduced to represent the reality at
a specific scale, and are then implemented numerically. However in
this process of modelling, many complex and subscale phenomena have to
be simplified, often through the introduction of additional parameters
in order to account for those unresolved processes, but they need to
be properly estimated. Classical methods of estimation usually involve
an objective function, that measures the distance between the
simulations and some observations, which is then optimised. This
function requires a run of the numerical model, thus can be expensive
to evaluate computational-wise.

However, some other uncertainties can also be present, which represent
some uncontrollable and external factors that affect the
modelling. Those variables will be qualified as
\emph{environmental}. By modelling them with a random variable, the
objective function is then a random variable as well, that we wish to
minimise in some sense. Omitting the random nature of the
environmental variable can lead to localised optimisation, and thus a
value of the parameter that is optimal only for the fixed nominal
value.  To overcome this, the minimisation of the expected value of
the objective function is often considered in the field of
optimisation under uncertainty for instance.

In this thesis, we focus instead on the notion of regret, that
measures the deviation of the objective function from its optimal
value for the same environmental variable. This regret (either
additive or relative) translates a notion of robustness through its
probability of exceeding a specified threshold. So, by either
controlling the threshold or the probability, we can define a family
of estimators based on this regret.

The regret can quickly become expensive to evaluate since it requires
an optimisation of the objective for every realisation of the
environmental variable. We then propose to use Gaussian Processes (GP)
in order to reduce the computational burden of this evaluation. In
addition to that, we propose a few adaptive methods in order to
improve the estimation: the next points to evaluate are chosen
sequentially according to a specific criterion, in a Stepwise
Uncertainty Reduction (SUR) strategy.


Finally, we are going to apply some of the methods introduced in this
thesis on an academic problem of parameter estimation. We will study
the calibration of the bottom friction of a model of the Atlantic
ocean near the French coasts, while introducing some uncertainties in
the forcing of the tide, and get a robust estimation of this friction
parameter in a twin experiment setting.



%\todo{réécrire complétement}
%Many physical phenomena are modelled numerically in
%order to better understand and/or to predict their behaviour. However,
%some complex and small scale phenomena can not be fully represented in
%the models. The introduction of parametrization terms is usually the
%solution to represent these unresolved processes, but those need to be
%properly estimated.
%
%Classical methods of parameter estimation usually imply the
%minimisation of an objective function, that measures the error between
%some observations and the results obtained by a numerical model. The
%optimum is directly dependent on the fixed nominal value given to the
%uncertain parameter ; and therefore may not be relevant in other
%conditions.
%
%A good example of this type of problem is the estimation of bottom
%friction parameters of the ocean floor. This task is further
%complicated by the presence of uncertainties in certain other
%characteristics linking the bottom and the surface (eg boundary
%conditions).
%
%In this work, we propose a new criterion, based on the
%relative-regret, in order to control in probability the deviation of
%the relative-regret above a specified threshold. We introduce also
%iterative methods based on Gaussian Processes, in order to compute
%efficiently quantities linked to this estimation.  Finally, we will
%apply some of those techniques on an academic model of a coastal area,
%in order to find a robust value of the bottom friction.

\vspace{0.5cm}
\vfill
\etoile
\vfill
\vspace{0.5cm}
%\todo{à traduire}
\begin{center}
\small  \bf Résumé
\end{center}
\vspace{0.3cm} \todo{à traduire?}

De nombreux phénomènes physiques sont modélisés afin d'en mieux
connaître les comportements ou de pouvoir les prévoir. Cependant pour
représenter la réalité, de nombreux processus doivent être simplifiés,
car ils sont souvent trop complexes, ou apparaissent à échelle bien
inférieure à celle du maillage. Au lieu de complétement omettre, les
effets des processus sont souvent retranscrits dans les modèles à
l'aide de paramétrisations, c'est-à-dire en introduisant des termes
représentant des caractéristiques physiques, qui doivent être ensuite
estimées. Les méthodes classiques d'estimation se basent sur la
définition d'une fonction objectif qui mesure l'écart entre le modèle
numérique et la réalité, qui est ensuite optimisée.

Cependant, d'autres incertitudes peuvent aussi être
présentes. Celles-ci peuvent représenter des  incontrollables

% iLes techniques utilisées
% actuellement pour ajuster de tels paramètres se basent sur la
% minimisation d'une fonction mesurant l'erreur entre les observations
% réelles, et les prévisions obtenues grâce au modèle numérique. Cette
% estimation ne prend absolument pas en comptes les incertitudes
% présentes dans les caractéristiques du modèle, augmentant d'autant
% l'erreur sur cette estimation. Estimer ces paramètres en présence
% d'incertitudes nécessite donc un traitement particulier.
Dans cette thèse, nous nous intéressons plutôt à la notion de regret,
qui mesure l'écart entre la fonction objectif et la valeur optimale
qu'elle peut atteindre, étant donné la valeur de la variable
environementale.  Cette idée de regret, additif ou bien relatif, nous
pouvons proposer une notion de robustesse à travers l'étude de sa
probabilité de dépasser un certain seuil, ou inversement à travers le
calcul de ses quantiles. À l'aide de ce seuil, ou de l'ordre du
quantile choisi, on peut donc définir une famille d'estimateurs basés
sur le regret.

Néanmoins, le calcul du regret, et donc des quantités dérivées peut
vite devenir très coûteux, car il nécessite une optimisation par
rapport au paramètre de contrôle. Nous proposons donc d'utiliser des
processus Gaussiens (GP) afin de construire un modèle de substitution,
et donc de réduire cette contrainte en pratique. Nous proposons aussi
des méthodes itératives basées notamment sur la stratégie SUR
(Stepwise Uncertainty Reduction, Réduction d'incertitudes
séquentielle): le point à évaluer ensuite est choisi selon un critère
permettant d'améliorer au mieux des quantités associées au
regret-relatif.

Enfin, nous appliquerons les outils présentés dans cette thèse à un
problème académique d'estimation de paramètre. Nous étudierons ainsi
la calibration sous incertitudes du paramètre de friction de fond d'un
modèle océanique, représentant la façade atlantique des côtes
françaises, ainsi que la Manche.


% Dans ce travail, nous proposons un nouveau critère d'optimisation sous
% incertitudes, basé sur le regret-relatif. Nous cherchons donc à
% maximiser la probabilité avec laquelle le regret-relatif reste sous un
% certain seuil, ou inversement, nous cherchons à minimiser le seuil
% atteint par le regret-relatif, à un niveau de confiance donné.  D'un
% point de vue pratique, nous proposons aussi des méthodes itératives
% basées sur les processus Gaussiens, permettant de calculer
% efficacement les quantités liées à ce nouveau critère.  Enfin, nous
% appliquons ces nouvelles méthodes à la calibration sous incertitudes
% d'un modèle océanique côtier, pour trouver un paramètre de friction de
% fond robuste.
%\restoregeometry
\vfill