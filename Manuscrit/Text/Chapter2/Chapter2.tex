\documentclass[../../Main_ManuscritThese.tex]{subfiles}

\subfileGlobal{
\renewcommand{\RootDir}[1]{./Text/Chapter2/#1}
}

% For cross referencing
\subfileLocal{
\externaldocument{../../Text/Introduction/build/Introduction}
\externaldocument{../../Text/Chapter3/build/Chapter3}
\externaldocument{../../Text/Chapter4/build/Chapter4}
\externaldocument{../../Text/Chapter5/build/Chapter5}
\externaldocument{../../Text/Conclusion/build/Conclusion}
}
\newcommand\imgpath{/home/victor/acadwriting/Manuscrit/Text/Chapter2/img/} 

%%%%%%%%%%%%%%%%%%%%%%%%%%%%%%%%%%%%%%
%% CHAPTER TITLE
%%%%%%%%%%%%%%%%%%%%%%%%%%%%%%%%%%%%%%

\begin{document}
\chapter{Inverse Problem and calibration}
\label{chap:inverse_problem}
\minitoc
% \listoftodos
\newpage
\subfileLocal{\pagestyle{contentStyle}}
%%%%%%%%%%%%%%%%%%%%%%%%%%%%%%%%%%%%%%%%%%%%%%%%%%%%%%%
%%%%%%%%%%%%%%%%%%%%%%%%%%%%%%%%%%%%%%%%%%%%%%%%%%%%%%%

\section{Introduction}
\label{sec:intro_chap2}
In this chapter we will first lay the ground for developing the
general ideas behind calibration, by introducing the notions of
models, and forward and inverse problems
in~\cref{sec:forward_inverse_problem_proba_theory}. This implies also
a short review of notions of probability theory. Calibration will be
defined in \cref{sec:parameter_inference} as the optimization of an
objective function: Maximum likelihood estimation in a frequentist
setting, or posterior maximization using Bayes' theorem. In practice,
for large-scale applications, the optimization is performed using
gradient-descent, and the computational cost of gradient computation
can be overcome by adjoint method, as described
in~\cref{sec:calibration_adjoint_optimization}. Finally, we are going
to discuss two aspects related to calibration, namely model selection
in \cref{sec:model_selection} and the influence of nuisance parameters
and model misspecification in calibration in
\cref{sec:model_misspecification}.


\section{Forward, inverse problems and probability theory}
\label{sec:forward_inverse_problem_proba_theory}
Running a simulation using numerical tools is useful to grasp a better
understanding of the physical phenomena, or to forecast them. On the
other hand, when observing and comparing the measurements and the
output of the numerical simulation, we can quantify the mismatch
between the two and tune some parameters involved in the
computations. Indeed, these parameters represent different physical
quantities or processes that are for example unresolved at the model's
scale (such as the modelling of the
turbulences)% friction of the ocean bed, or $k-\epsilon$ turbulence models for instance \todo{trop précis, selon si évoqué dans intro générale})
, or ill-known. A proper estimation of these parameters has to be
performed in order to guarantee a meaningful output when evaluating
the model.

Model calibration or parameter estimation has been widely treated in
the literature, either from a statistical and probabilistic point of
view using likelihood-based methods and Bayesian inference, or from a
\emph{variational} point of view by defining proper objective
functions. To match those two approaches, we will first review the
problem from a probabilistic point of view, in order to define
properly some appropriate objective functions and introduce tools from
optimal control theory to optimize them.

 \subsection{Model space data space and forward problem}
\label{sec:model_space_data_space}
In order to describe accurately a physical system, we have to define
the notion of models, and will be
following~\cite{tarantola_inverse_2005} approach to define inverse
problems.  A model represents the link between some parameters and
some observable quantities. An example is a model that takes the form
of a system of ODEs or PDEs, maybe discretized, while the parameters
are the initial conditions and the output is one or several time
series, describing the time evolution of a quantity at one or several
spatial points. An important point is that a model is not only the
\emph{forward operator}, but must also include the parameter space.


 \begin{definition}[Model]
   \label{def:model}
  A model $\mathfrak{M}$ is defined as a pair composed of a \emph{forward operator} $\mathcal{M}$, and a \emph{parameter space} $\Theta$
  \begin{equation}
    \mathfrak{M} = (\mathcal{M}, \Theta)
  \end{equation}
  The forwarxd operator is the mathematical representation of the
  physical system, while the parameter space is chosen here to be a
  subset of a finite dimensional space, so usually $\Theta$ will be a
  subset of $\mathbb{R}^n$.
\end{definition}
As we will usually choose $\Theta$ as a subset of $\mathbb{R}^n$, for
$n\geq 1$, we can define the dimensionality of the model, based on the
number of \emph{degrees of freedom} available for the parameters.


\begin{remark}
  \label{rmk:model_dimension}
  The dimension of a model $\mathfrak{M}=(\mathcal{M},\Theta)$ is the number of parameters not reduced to a singleton, so if $\Theta \subset \mathbb{R}^n$, the dimension of $\mathfrak{M}$ is $d \leq n$. The dimension of a model $\mathfrak{M}$ is sometimes called the degrees of freedom of $\mathfrak{M}$.
  \end{remark}

\begin{example}
  A model with parameter space $\Theta = \mathbb{R}^2\times \interval{0}{1}$ has dimension $3$, while $\Theta = \mathbb{R}^2 \times \{1\}$ has dimension $2$.
\end{example}
Now that we have introduced the forward operator and the parameter
space, we will focus on the output of the model.  Ideally, the data
space $\Yspace$ consists in all the physically acceptable results of
the physical experiment.  Then, the forward operator $\mathcal{M}$
maps the parameter space $\Theta \subset \mathbb{R}^{n}$ to the data
space $\Yspace$, as one can expect that all models provide physically
acceptable outputs.

\subsection{Forward problem}
Given a model $(\mathcal{M}, \Theta)$, the \emph{forward problem}
consists in applying the forward operator to a given
$\theta \in \Theta$, in order to get the \emph{model prediction}. The
forward problem is then to obtain information on the result of the
experiment based on the parameters we chose as input, so deriving a
satisfying forward operator $\mathcal{M}$.
\begin{equation}
  \begin{array}{cccc}
    \mathcal{M}:& \Theta &\longrightarrow & \Yspace \\
                & \theta &\longmapsto     & \mathcal{M}(\theta)
  \end{array}
\end{equation}
As said earlier, the forward operator can be a set of ODEs or PDEs,
discretized or not. The forward problem is then the attempt to link
the causes, i.e.\ the parameters, to the consequences, i.e.\ the
output in the data space.

\subsection{Inverse Problem}
The inverse problem is the counterpart of the forward problem, and
consists in trying to gather more information on the parameters, based
on: the result of the experiment or the observation of the physical
process and on the knowledge of the forward operator, as illustrated
\cref{fig:inv_problem_pple}.

\begin{figure}[ht]
  \centering
  \input{\imgpath inv_problem_basic.pgf}
  \caption{Forward and Inverse problem diagram}
  \label{fig:inv_problem_pple}
\end{figure}

This is done by directly comparing the output of the forward operator,
and trying to reduce the mismatch between the observed data and the
model prediction.



However, a purely deterministic approach for the inverse problem is
doomed to underperform: as most physical processes are not perfectly
known, some uncertainties remain in the whole modelling process. Those
uncertainties are ubiquitous: the observations available may be
corrupted by a random noise coming from the measurement devices and
the model may not represent perfectly the reality, thus introducing a
systematic bias for instance. Taking into account those uncertainties
is crucial to solve the inverse problem.



In that perspective we are going to introduce briefly the usual
probabilistic framework, along with common notations that we will use
throughout this manuscript. Those notions are well established in the
scientific literature, and one can
read~\cite{billingsley_probability_2008} for a more thorough
description.




\subsection{Notions of probability theory}
\subsubsection{Probability measure, and random variables}
\label{sec:notion_prob_theory}

We are first going through some usual notions of probability theory. 
Let us consider the usual probabilistic space $(\Omega, \mathcal{F}, \Prob)$.
\begin{definition}[Event probability and conditioning]
  \label{def:prob_event}
  We call an outcome of a random experiment $\omega$ an element of the
  sample space $\Omega$, and an event $A$ is an element of the
  $\sigma$-algebra $\mathcal{F}$ ($\sigma$-algebra on the set
  $\Omega$). The probability of an event $A\in \mathcal{F}$ is defined
  as the Lebesgue integral
  \begin{equation}
    \Prob[A] = \int_{A} \,\mathrm{d}\Prob(\omega) = \Prob[\{\omega ; \omega \in A\}]
  \end{equation}
  Observing an event $B \in \mathcal{F}$ can bring information upon
  another event $A\in \mathcal{F}$. In that sense, we introduce the
  conditional probability of $A$ given $B$.
\label{def:cond_proba}
  Let $A$, $B \in \mathcal{F}$.
  The event $A$ given $B$ is written $A \mid B$ and its probability is
  \begin{equation}
    \Prob[A \mid B] = \frac{\Prob[A \cap B]}{\Prob[B]}
  \end{equation}
\end{definition}
Formally, an event can be seen as an outcome of some uncertain experiment, and its probability is ``how likely'' this event will happen.

Let us now introduce a measurable state (or sample) space $S$, that is the set of all possible outcomes we can observe (and upon which we can assign a probability).
\begin{definition}[Random Variable, Expectation]
  \label{def:random_variable}
  A random variable (abbreviated as r.v.) $X$ is a measurable function
  from $\Omega \longrightarrow S$. A random variable will usually be
  written with an upper case letter. A realisation or observation $x$
  of the r.v. $X$ is the actual image of $\omega\in\Omega$ under $X$:
  $x = X(\omega)$. If $S$ is countable, the random variable is said to
  be \emph{discrete}. When $S\subseteq \mathbb{R}^p$ for $p\geq 1$,
  $X$ is sometimes called a random vector
  
  \label{def:expectation}
  The expectation of a r.v. $X:\Omega \rightarrow S$ is defined as
  \begin{equation}
    \Ex[X] = \int_{\Omega} X(\omega) \,\mathrm{d}\Prob(\omega)
  \end{equation}
\end{definition}

  Using the \cref{def:expectation}, the probability of an event $A$ can be seen as the expectation of the indicator function of $A$:
  \begin{equation}
    \begin{array}{cccc}
      \mathbbm{1}_{A}:& \Omega& \longrightarrow& \{0,1\} \\
                      & \omega& \longmapsto & \begin{cases}
                        1\text{ if } \omega \in A \\
                        0 \text{ if } \omega \notin A
                                              \end{cases}
    \end{array}
  \end{equation}
  and it follows that
  \begin{equation}
    \Ex[\mathbbm{1}_A] = \int_{\Omega} \mathbbm{1}_A \, \mathrm{d}\Prob(\omega)= \int_{A} \, \mathrm{d}\Prob(\omega) = \Prob[A]
  \end{equation}

  
  As we defined the notion of a r.v.\ in \cref{def:random_variable} as
  a measurable function from $\Omega \to S$, we can now focus on the
  measurable sets through $X$, by using in a sense the change of
  variable $x = X(\omega)$.
\begin{definition}[Image (Pushforward) measure]
  \label{def:image_measure}
  Let $X:\Omega \rightarrow S$ be a random variable, and
  $A \subseteq S$. The image measure (also called pushforward measure)
  of $\Prob$ through $X$ is denoted by $\Prob_X = \Prob \circ
  X^{-1}$. This notation can differ slightly depending on the
  community, so one can find also
  $ \Prob_X = \Prob \circ X^{-1} = X_{\sharp}\Prob$, the latter
  notation being used in transport theory. The probability, for the
  r.v. $X$ to be in $A$ is equal to
  \begin{equation}
    \Prob[X \in A] = \Prob_X[A] = \int_{A}\,\mathrm{d}\Prob_X(\omega) =  \int_{X^{-1}(A)}\,\mathrm{d}\Prob(\omega) = \Prob[X^{-1}(A)] = \Prob\left[\{\omega\,;\,X(\omega) \in A\}\right]
  \end{equation}
  
Similarly, for any measurable function $h$, the expectation taken with respect to a specific random variable $X$ is 
\begin{equation}
  \Ex_{X}[h(X)] = \int_{\Omega} h(X(\omega)) \,\mathrm{d}\Prob_{X}(\omega)
\end{equation}
\end{definition}



In most of this thesis, the sample space will be
$S\subseteq \mathbb{R}^p$ for $p\geq 1$, so we are going to introduce
useful tools and notations to characterize these particular real
random variables.
\subsubsection{Real-valued random variables}
We are now going to focus on real-valued random variables, so
measurable function from $\Omega$ to the sample space
$S = \mathbb{R}$.
\begin{definition}[Distribution of a real-valued r.v.]
  \label{def:distribution}
  The distribution of a r.v.\ can be characterized by a few functions:
  \begin{itemize}
  \item The \emph{cumulative distribution function} (further
abbreviated as cdf) of a real-valued r.v. $X$ is defined as:
  \begin{equation} F_{X}(x) = \Prob\left[X \leq x\right] =
\Prob_X\big[\,\interval[open left]{-\infty}{x}\, \big]
  \end{equation} and $\lim_{-\infty}F_X = 0$ and $\lim_{+\infty} F_X
= 1$
If the cdf of a random variable is continuous, the r.v. is said to be \emph{continuous} as well.
  
\item The \emph{quantile function} $Q_X$ is the generalized inverse function
of the cdf:
  \begin{equation} Q_X(p) = \inf\{q:\, F_X(q)\geq p\}
  \end{equation}
\item If there exists a function $f: S\rightarrow \mathbb{R}^{+}$ such that
  for all measurable sets $A$
  \begin{equation} \Prob[X \in A] = \int_A \,\mathrm{d}\Prob_X(\omega) = \int_A f(x)\,\mathrm{d}x
\end{equation}
then $f$ is called the \emph{probability density function} (abbreviated pdf), or \emph{density} of $X$ and is denoted $p_X$.
As $\Prob[X \in S] = 1$, it follows trivially that $\int_{S}f(x)\,\mathrm{d}x=1$.
One can verify that if $F_X$ is derivable, then its derivative is the density of the r.v.\ :
\begin{equation}
  \frac{\mathrm{d}F_X}{\mathrm{d}x}(x) = p_X(x)
\end{equation}

  \end{itemize}
\end{definition}
\begin{remark}
  When restricting this search to ``classical'' functions, $p_X$ may
  not exist. However, allowing generalized functions such as the
  \emph{dirac delta function}, provides a way to consider
  simultaneously all types of real-valued random variables (continous,
  discrete, and mixture of both). Dirac's delta function can
  (in)formally be defined as
  \begin{equation}
    \label{eq:def_dirac_delta}
    \delta_{x_0}(x) = 
    \begin{cases}
      +\infty \text{ if } x=x_0 \\
      0 \text{ elsewhere}
    \end{cases} \quad \text{ and }
    \int_S \delta_{x_0}(x)\,\mathrm{d}x = 1
  \end{equation}
\end{remark}
\begin{example}
  \label{ex:X_rv}
  Let us consider the random variable $X$ that takes the value $1$ with probability $0.5$, and follows a uniform distribution with probability $0.5$ over $[2;4]$. Its cdf can be expressed as
  \begin{equation}
    F_X(x) =
    \begin{cases}
      0 \text{ if } x < 1 \\
      0.5 \text{ if } 1 \leq x < 2 \\
      0.5 + \frac{x-2}{8} \text{ if } 2 \leq x < 4 \\
      1 \text{ if } 4 \leq x
    \end{cases}
  \end{equation}
  and its pdf (as a generalized function)
  \begin{equation}
    p_X(x) = \frac{1}{2}\delta_{1}(x) + \frac{1}{4}\mathbbm{1}_{\{2\leq x < 4\}}(x) 
  \end{equation}
  The pdf and cdf are shown \cref{fig:example_pdf_cdf}.
\end{example}
\begin{figure}[!h]
  \centering
  \input{\imgpath cdf_pdf_example.pgf}
  \caption[Example of cdf and of pdf]{Cdf and Pdf of $X$ defined in \cref{ex:X_rv}. The arrow indicates Dirac's delta function}
  \label{fig:example_pdf_cdf}
\end{figure}

\begin{definition}[Moments of a r.v. and $L^s$ spaces]
  Let $X$ be a random variable.
  The moment of order $s$ is defined as $\Ex\left[X^s\right]$, and the centered moment of order $s$ is defined as
  \begin{equation}
    \Ex[(X-\Ex[X])^s]=\int \left(X(\omega) - \Ex[X]\right)^s \,\mathrm{d}\Prob(\omega) = \int (x-\Ex[X])^s\cdot p_X(x)\,\mathrm{d}x
  \end{equation}
  To ensure that those moments exists, let us define $L^s(\Prob)$ as the space of random variables $X$ such that $\Ex\left[\lvert X \rvert^s\right] < +\infty$.
  If $X\in L^2(\Prob)$, the centered moment of order $2$ is called the variance:
  \begin{equation}
    \label{eq:variance_def}
    \Ex\left[(X-\Ex[X])^2 \right] = \Var[X] \geq 0
  \end{equation}
\end{definition}

These definition above hold for real-valued random variables, so 1D
r.v., but can be extended for random vectors.
%%%%%%%%%%%%%%%%%%%%%%%%%%%%%%%%%%%%%%%%%%%%%%%%%% 

\subsubsection{Real-valued random vectors}
 Most of the definitions for a random variable extend component-wise to random vectors:

\begin{definition}[Joint, marginal and conditional densities]
 \label{def:joint_marginal_cond_densities}
 Let $X=[X_1,\cdots,X_p]$ be a random vector from $\Omega \rightarrow S\subseteq\mathbb{R}^p$
 The expected value of a random vector is the expectation of the components
 \begin{equation}
\Ex[X] = \left[\Ex[X_1],\dots,\Ex[X_p]\right]
\end{equation}

The cdf of $X$ at the point $x=[x_1,\dots x_p]$ is
   \begin{align}
     \label{eq:def_cdf_multi}
     F_{X}(x) = F_{X_1,\dots, X_p}(x_1,\dots, x_p) &= \Prob\left[X_1 \leq x_1, \cdots, X_p\leq x_p\right]\\
                                                   &= \Prob\left[\bigcap_{i=1}^p \{\omega;\,X_i(\omega) \leq x_i\}\right] \nonumber
  \end{align}
  Similarly as in the real-valued case, we can define the pdf of the
  random vector, or \emph{joint pdf} by derivating with respect to the
  variables:
  \begin{equation}
    p_{X}(x)= p_{X_1,\dots, X_p}(x_1,\dots, x_p) =\frac{\partial^p F_X}{\partial x_1 \cdots \partial x_p}(x)
  \end{equation}
  
  and $\int_{S}p_{X_1,\dots, X_p}(x_1,\dots, x_p)\,\mathrm{d}(x_1,\dots, x_p)=1$

For two random vectors $X$ and $Y$, the (cross-)covariance matrix of $X$ and $Y$ is defined as  
  \begin{equation}
    \Cov\left[X,Y\right] = \Ex\left[(X-\Ex[X])(Y - \Ex[Y])^T\right] = \Ex[XY^T] - \Ex[X]\Ex[Y]^T
  \end{equation}
and based on this definition, we can extend the notion of variance to vectors. The covariance matrix $\Sigma \in \mathbb{R}^{p\times p}$ of $X$, is defined to be
  \begin{equation}
    \Sigma = \Cov(X)=\Cov[X,X]= \Ex\left[\left(X - \Ex[X]\right)\left(X-\Ex[X]\right)^T\right] = \Ex[XX^T] - \Ex[X]\Ex[X]^T
  \end{equation}


  We can now define the \emph{marginal densities}.
  For notation clarity, we are going to set $X = [Y,Z]$: the marginal densities of $Y$ and $Z$ are
  \begin{equation}
    \label{eq:marginals_def}
    p_{Y}(y) = \int_{\mathbb{R}}p_{Y,Z}(y,z) \,\mathrm{d}z \quad \text{ and } \quad p_{Z}(z) = \int_{\mathbb{R}}p_{Y,Z}(y,z) \,\mathrm{d}y
  \end{equation}
  The random variable $Y$ given $Z$, denoted by $Y \mid Z$ has the conditional density
  \begin{equation}
    p_{Y \mid Z}(y \mid z) = \frac{p_{Y,Z}(y,z)}{p_Z(z)}
  \end{equation}
  allowing us to rewrite the marginals as
  \begin{align}
    \label{eq:marginal_conditioned}
        p_{Y}(y) = \int_{\mathbb{R}}p_{Y\mid Z}(y\mid z)p_Z(z) \,\mathrm{d}z=\Ex_Z\left[p_{Y\mid Z}(y\mid z)\right] \\ p_{Z}(z) = \int_{\mathbb{R}}p_{Z\mid Y}(z\mid y)p_Y(y) \,\mathrm{d}y = \Ex_{Y}\left[p_{Z\mid Y}(z\mid y)\right]
  \end{align}

\end{definition}

\subsubsection{Bayes' Theorem}
\label{ssec:bayes_theorem}

The classical Bayes' theorem is directly a consequence of the
definition of the conditional probabilities in \cref{def:cond_proba},
and for random variables admitting a density
in~\cref{def:joint_marginal_cond_densities}.

\begin{theorem}[Bayes' theorem]
  Let $A, B\in\mathcal{F}$. Bayes' theorem states that
  \begin{align*}
    \Prob[A\mid B]\cdot \Prob[B] = \Prob[B \mid A]\cdot\Prob[A] \\
    \Prob[A \mid B] = \frac{\Prob[B \mid A]\cdot\Prob[A]}{\Prob[B]} \text{ if } \Prob[B] \neq 0
  \end{align*}
 In terms of densities, the formulation is sensibly the same.
  Let $Y$ and $Z$ be two random variables. The conditional density of $Y$ given $Z$ can be expressed using the conditional density of $Z$ given $Y$.
  \begin{equation}
    p_{Y\mid Z}(y \mid z) = \frac{p_{Z\mid Y}(z\mid y) p_Y(y)}{p_Z(z)} = \frac{p_{Z\mid Y}(z\mid y) p_Y(y)}{\int p_{Z,Y}(z,y) \,\mathrm{d}y}  \propto p_{Z\mid Y}(z\mid y) p_Y(y)
  \end{equation}
\end{theorem}
Bayes' theorem is central as it links in a simple way conditional
densities. In the inverse problem framework, if $Y$ represents the
state of information on the parameter space, while $Z$ represents the
information on the data space, $Z\mid Y$ can be seen as the forward
problem. Bayes' theorem allow us to ``swap'' the conditioning, and get
information on $Y\mid Z$, that can be seen as the inverse problem.


The influence of one (or a set of) random variable(s) over another can
be measured with the conditional probabilities. Indeed, if the state
of information on a random variable does not change when observing
another one, the observed one carries no information on the
other. This notion of dependence (and independence) is first defined
on events and extended to random variables
\begin{definition}[Independence]
  Let $A,B\in \mathcal{F}$. Those two events are said independent if $\Prob[A \cap B] = \Prob[A]\Prob[B]$.
  Quite similarly, two real-valued random variables $Y$ and $Z$ are said to be independent if $F_{Y,Z}(y,z) = F_Y(y) F_Z(z)$ or equivalently, $p_{Y,Z}(y,z) = p_Y(y) p_Z(z)$.
  Speaking in terms of conditional probabilities, this can be written as $p_{Y|Z}(y,z) = p_{Y}(y)$.
  If $Y$ and $Z$ are independent, $\Cov[Y, Z] = 0$. The converse if false in general.
\end{definition}

We discussed so far the different quantities that characterize random
variables. Let us consider now two random variables which share the
same sample space: $X, X^{\prime}: \Omega \rightarrow S$. There exists
various way to compare those two random variables, usually by
quantifying some measure of distance between their pdf when they
exist. One of the most used comparison tool for random variables is
the Kullback-Leibler divergence.

\begin{definition}[KL--divergence and entropy]
   \label{def:KL_entropy}
  The Kullback-Leibler divergence, introduced in~\cite{kullback_information_1951} is a measure of dissimilarity between two distributions, based on information-theoretic considerations.
  Let $X$, $X'$ be r.v.\ with the same sample space $S$, and $p_X$ and $p_{X^{\prime}}$ their densities, such that $\forall A\in\mathcal{F}$, $\int_A p_X(x)\,\mathrm{d}x = 0 \implies \int_A p_{X^{\prime}}(x)\,\mathrm{d}x =0$. The KL-divergence is defined as
  \begin{align}
    % \DKL{p_X}{p_Y} = \int_{\Omega} p_X(X(\omega)) \log \frac{p_X(X(\omega)}{p_Y(X(\omega)} \,\mathrm{d}\Prob_X(\omega)
    \DKL{p_X}{p_{X^\prime}} &= \int_{S} p_X(x) \log \frac{p_X(x)}{p_{X^\prime}(x)} \,\mathrm{d}x \\ &= \Ex_X\left[-\log p_{X^\prime}(X)\right] - \Ex_X\left[-\log p_X(X)\right] \\
                                                                                                &= H\left[X^{\prime}, X\right] - H\left[X\right]
  \end{align}
  $H[X]$ is called the (differential) entropy of the random variable
  $X$, and $H[X^{\prime}, X]$ the cross-entropy of $X^{\prime}$ and
  $X$.  Using Jensen's inequality, one can show that for all $X$ and
  $X^{\prime}$ such that the KL-divergence exists,
  $\DKL{p_X}{p_{X^{\prime}}}\geq 0$ with equality iff they have the
  same distribution, a desirable property when measuring
  dissimilarity. However, the KL--divergence is not a distance
  function, as it is not symmetric in general, and it does not verify
  the triangle inequality.
\end{definition}

\subsubsection{Important examples of real random variables}
One of the most well known distribution is the normal distribution, also called Gaussian distribution, that appears in various situations, but most notably in the central limit theorem.
\begin{example}[The Normal Distribution]
  \label{ex:gaussian_distribution}
  Let $X$ be a r.v.\ from $\Omega$ to $\mathbb{R}$.
  If $X$ follow the normal distribution of mean $\mu \in \mathbb{R}$ and variance $\sigma^2>0$, we write $X \sim \mathcal{N}(\mu,\sigma^2)$, and its pdf is
  \begin{equation}
    p_X(x) = \phi(x) = \frac{1}{\sqrt{2\pi}\sigma}\exp\left(-\frac{1}{2}\frac{(x-\mu)^2}{\sigma^2}\right)
  \end{equation}

For the multidimensional case, let $X$ be a r.v.\ from $\Omega$ to $\mathbb{R}^p$, that follows a normal distribution of mean $\mu \in \mathbb{R}^p$ and covariance matrix $\Sigma \in \mathbb{R}^{p\times p}$, where $\Sigma$ is semi-definite positive.
In that case, $X\sim \mathcal{N}(\mu, \Sigma)$ the density of the random vector $X$ can be written as
\begin{equation}
    p_X(x) = (2\pi)^{-\frac{p}{2}}\lvert\Sigma\rvert^{-1}\exp\left(-\frac{1}{2}(x-\mu)^T\Sigma^{-1}(x-\mu)\right)
  \end{equation}
  where $|\Sigma|$ is the determinant of the matrix $\Sigma$, and
  $(\cdot)^T$ is the transposition operator.  As the covariance matrix
  appears through its inverse, another encountered parametrization is
  to use the precision matrix $\Sigma^{-1}$.  Examples of pdf of
  Gaussian normal distributions are
  displayed~\cref{fig:example_normal}.
\end{example}
\begin{figure}[!h]
  \centering
  \input{\imgpath gaussian_distribution_examples.pgf}
  \caption[Probability Density functions of 1D and 2D Gaussian r.v.]{Probability Density functions of 1D Gaussian distributed r.v.\ (left), and density of $Z$, a 2D Gaussian r.v.\ (right)}
  \label{fig:example_normal}
\end{figure}
When adding independent squared samples of the normal distribution,
the resulting random variable follows a $\chi^2$ distribution.
\begin{example}[The $\chi^2$ distribution]
  \label{ex:chi2}
  Let $X_1, X_2,\dots,X_{\nu}$ be $\nu$ independent random variables, such that for $1\leq i \leq \nu$, $X_i \sim \mathcal{N}(0,1)$
  We define the random variable $X$ as
  \begin{equation}
    X = \sum_{i=1}^\nu X_i^2
  \end{equation}
  By definition, the random variable $X$ follows a $\chi^2$ distribution with $\nu$ degrees of freedom: $X\sim \chi^2_{\nu}$. The quantile of order $\beta$ is written $\chi^2_{\nu}(\beta)$ and verifies
  \begin{equation}
    \Prob[X \leq \chi^2_{\nu}(\beta)] = \beta 
  \end{equation}
  The pdf of such a r.v. is displayed~\cref{fig:chi2_examples}, for different degrees of freedom.
\end{example}

\begin{figure}[ht]
  \centering
  \input{\imgpath chi2_distribution_examples.pgf}
  \caption[Probability density functions of $\chi^2$ r.v.]{Probability density functions of $\chi^2_{\nu}$ random variables, for different degrees of freedom}
  \label{fig:chi2_examples}
\end{figure}

\clearpage

\section{Parameter inference}
\label{sec:parameter_inference}
\subsection{From the physical experiment to the model}
\label{ssec:inv_problem}
We can represent both the reality and the computer simulation as
models.  The physical system (the reality) that is observed can be
represented by a model
$\mathfrak{M}=\left(\mathscr{M},\Theta_{\mathrm{real}}\right)$, so by
a forward operator $\mathscr{M}$, and a parameter space
$\Theta_{\mathrm{real}}$.  Observing the physical system means to get
access to $y\in \Yspace$ that is the image of an \emph{unknown}
parameter value $\vartheta \in \Theta_{\mathrm{real}}$ through the
forward operator, so
$y = \mathscr{M}(\vartheta)\in \Yspace \subseteq \mathbb{R}^p$.

On the other hand, let us assume that a numerical model of the reality
has been constructed, by successive various assumptions,
discretizations and simplifications giving $(\mathcal{M},\Theta)$. The
main objective of calibration is to find $\hat{\theta}$ such that the
forward operator applied to $\hat{\theta}$:
$\mathcal{M}(\hat{\theta})$ represents as accurately as possible the
physical system, and thus matches as closely the data
$\mathscr{M}(\vartheta)=y$. This is
illustrated~\cref{fig:inv_problem}.

\begin{figure}[ht]
  \centering
  \input{\imgpath inv_problem.pgf}
  \caption{Forward and inverse problem using models as defined \cref{def:model}}
  \label{fig:inv_problem}
\end{figure}

% \begin{equation}
%     \mathscr{M}(x, \vartheta) = \mathcal{M}(x, \theta) + \delta(x,\theta)
% \end{equation}
% When evaluated on a fixed vector , we will omit the input, and define
% 

For the sake of simplicity, let us assume that $\vartheta \in \Theta \subseteq \Theta_{\mathrm{real}}$. In~\cite{kennedy_bayesian_2001,higdon_combining_2004}, the authors rewrite the link between the reality and the model at this value as 
\begin{equation}
  \label{eq:vartheta_models}
    % \mathscr{M}(x_{\mathrm{grid}}, \vartheta) = 
    \mathscr{M}(\vartheta) = % \mathcal{M}(x_{\mathrm{grid}}, \theta) + \epsilon(x_{\mathrm{grid}},\theta) =
    \mathcal{M}(\vartheta) + \epsilon(\vartheta) \in \Yspace \subseteq \mathbb{R}^p
  \end{equation}
 
  The difference
  $\epsilon(\vartheta) = \mathscr{M}(\vartheta) -
  \mathcal{M}(\vartheta)$ is the error between the physical model and
  the model, called sometimes the misfit, or the residual error.  This
  error is unknown and encompasses different sources of uncertainties,
  such as measurement errors, or model bias (with respect to the
  reality). To deal with this unknown, we are going to model it as a
  sample of a random variable, leading us to treat the obtained data
  as a random sample as well.

  From the diverse assumptions we can make upon this sampled random
  variable, we can then treat the calibration procedure as a parameter
  estimation problem of a random variable.  The estimated parameter
  will be written $\hat{\kk}$, and the subscript will denote
  additional information on the estimator.  In this thesis, we focus
  on extremum estimators. Those estimators are defined as the
  optimizer of a given objective function $J$,
  $\hat{\kk} = \argmin J$. In the next sections, we will see the
  probabilistic origins of a few classical objective functions.
\subsection{Frequentist inference, MLE}
\label{sec:frequentist_inference_MLE}
As mentioned before, we can model the observations as a random
variable, say $Y$ (uppercase to highlight its random nature), and
assume that this r.v.\ belongs to a family of parametric
distributions, whose densities are
\begin{equation}
  \label{eq:family_pdf}
  \left\{y\mapsto p_{Y}(y; \theta) ; \theta\in\Theta \right\}
\end{equation}
This choice of notation has been made to keep explicit the dependency
on $\theta$. Assuming now that the residual are normally distributed
with a given covariance matrix $\Sigma$, and that
$\Yspace \subseteq \mathbb{R}^p$, $Y$ is a random vector distributed
as
\begin{equation}
  \label{eq:lik_gaussian}
  Y  \sim \mathcal{N}(\mathcal{M}(\theta), \Sigma)
\end{equation}
whose one sample is $y=\mathscr{M}(\vartheta)$.


Now, instead of looking at the densities of~\cref{eq:family_pdf} as
functions taking as arguments the samples in $\Yspace$, we may look at
it as a function of $\theta$, as the observations $y\in\Yspace$ do not
vary. We can then define the likelihood function and its associated
extremum estimator.
\begin{definition}[Likelihood function, MLE]
  \label{def:mle}
  The probability density function of the observations for a set of
  parameters is called the likelihood of those parameters given the
  observations, and is written $\mathcal{L}$. In the Gaussian case,
  this can be written as
  \begin{align}
    \label{eq:likelihood_definition}
    \mathcal{L}(\cdot ;y): \theta \mapsto p_{Y}(y;\theta) &= \mathcal{L}(\theta;y) \\
    &=(2\pi)^{-n/2}\lvert \Sigma \rvert^{-1/2}\exp\left(-\frac{1}{2}(\mathcal{M}(\theta) - y)^T\Sigma^{-1}(\mathcal{M}(\theta) - y)\right)
  \end{align}
  If $\Sigma = \mathrm{diag}(\sigma^2_1,\dots, \sigma^2_p)$, the likelihood can be written as the product of 1D Gaussians: 
  \begin{align}
    \mathcal{L}(\theta;y) &= \left(\prod_{i=1}^p\frac{1}{\sqrt{2\pi}\sigma_i}\right)\exp\left(\sum_{i=1}^p -\frac{(\mathcal{M}(\theta)_i - y_i)^2}{2\sigma^2_i}\right) \\
                          &= \prod_{i=1}^p\frac{1}{\sqrt{2\pi}\sigma_i}\exp\left(-\frac{(\mathcal{M}(\theta)_i - y_i)^2}{2\sigma^2_i}\right)
  \end{align}
  with $y = [y_1, \dots, y_p]$ and $\mathcal{M}(\theta) = [\mathcal{M}(\theta)_1,\dots \mathcal{M}(\theta)_p]$.
  Based on the likelihood function, we can define the \emph{Maximum Likelihood Estimator}, or \emph{MLE}, that maximizes the likelihood defined above:
  \begin{equation}
    \label{eq:def_MLE}
    \estimtxt{\theta}{MLE} = \argmax_{\theta\in\Theta}\mathcal{L}(\theta;y)
  \end{equation}

\end{definition}
  For practical and numerical reasons, the maximization of the likelihood is often replaced by the minimization of the negative log-likelihood:
  \begin{equation}
    \label{eq:MLE_iid}
    \estimtxt{\theta}{MLE} = \argmin_{\theta \in \Theta} -\log \mathcal{L}(\theta;y)= \argmin_{\theta \in \Theta} -\sum_{i=1}^p\log p_{Y_i\mid \theta}(y_i \mid \theta) 
  \end{equation} 
  where
  \begin{equation}
    -\log\mathcal{L}(\theta;y) = \frac{1}{2}(\mathcal{M}(\theta) - y)^T\Sigma^{-1}(\mathcal{M}(\theta) - y)+  \frac{n}{2}\log(2\pi) + \frac{1}{2}\log\lvert \Sigma \rvert
  \end{equation}
  % More generally, we can show that the MLE is also the minimizer of the KL-divergence between the sampling distribution:

As the optimization is performed on $\theta$, we can remove the constant terms of the objective function, and rewrite the objective function as a $L^2$ norm in~\cref{eq:MLE_L2norm}.  % By optimizing with respect to $\theta$, we can omit the constant terms, and
  \begin{align}
    \estimtxt{\theta}{MLE} &= \argmin_{\theta \in \Theta}\frac{1}{2}(\mathcal{M}(\theta) - y)^T\Sigma^{-1}(\mathcal{M}(\theta) - y) \nonumber\\
                           &= \argmin_{\theta \in \Theta}\frac12 \| \mathcal{M}(\theta) - y \|^2_{\Sigma^{-1}} \label{eq:MLE_L2norm}
  \end{align}

  Frequentist inference and Maximum Likelihood estimation boils down to Generalized non-linear least-square regression, that minimizes the squared Mahalanobis distance between $\mathcal{M}({\theta})$ and $y$. This is only true as we assumed a Gaussian form of the errors in~\cref{eq:lik_gaussian}. Other choices of the sampling distribution~\cref{eq:lik_gaussian} will result in different objective functions. To reduce the sensitivity on outliers, some authors such as~\cite{rao_robust_2015} introduce Student or Laplace distributed errors, or specifically designed norm such as the Huber norm~\cite{huber_robust_2011}.

  If the covariance matrix is diagonal, the residual errors are then uncorrelated, thus independent due to their Gaussian nature as defined in~\cref{eq:lik_gaussian}. The likelihood can be rewritten as the product of densities evaluated at the different samples $y_i$, obtained from their true distribution $Y$.
  A direct link can be written between the KL--divergence and the MLE. The KL--divergence between the true density $p_Y$ and the parametric sampling distribution $p_Y(\cdot;\theta)$ is
  \begin{equation}
  \DKL{p_Y}{p_Y(\cdot;\theta)} = \Ex_Y\left[\log p_Y(Y)\right]-\Ex_Y\left[\log p_Y(Y;\theta)\right]  
\end{equation}
As the first term does not depend on $\theta$, minimizing this expression is equivalent to minimizing the second part of the equation, so
\begin{align}
 \argmin_{\theta \in\Theta} \DKL{p_Y}{p_Y\left(\cdot;\theta \right)} = \argmin_{\theta \in \Theta} -\Ex_{Y}\left[\log p_{Y}(Y ; \theta)\right]
\end{align}
The true distribution of the observation is unknown, but samples $y_i$ are available. Using the empirical KL-divergence denoted $D^{\mathrm{empirical}}_{\mathrm{KL}}$,  and replacing the theoretical expectation with the empirical one, the equation above becomes:

\begin{equation}
  \argmin_{\theta \in\Theta} D^{\mathrm{empirical}}_{\mathrm{KL}}(p_Y \| p_Y\left(\cdot;\theta \right)) = \argmin_{\theta \in\Theta} \frac{1}{p} \sum_{i=1}^p - \log p_Y(y_i;\theta) = \estimtxt{\theta}{MLE}
\end{equation}
Thus, the MLE minimizes the empirical KL-divergence between the true distribution of the observations and the sampling distribution of the observation (that depends on $\theta$).

The MLE possesses desirable asymptotic properties, such as asymptotic normality when the number of observations grows large~\cite{reid_aspects_2013}. Those properties permit the construction of asymptotic confidence interval, and to perform hypothesis testing, especially for model selection. This aspect will be further developed in \cref{sec:model_selection}.

% \subsubsection{(Asymptotic) Properties of the MLE \todo{à garder/écrire ?}}
% \todo{y faire référence, pas de détails}
% In the following, the log-likelihood will be denoted $\ell(\theta) = \log \mathcal{L}(\theta;y)$, and the derivative of the log-likelihood with respect to $\theta$ is the score vector $s(\theta) = \pfrac{\ell(\theta)}{\theta}$.
% Implicitly, $s$ is a function of the observation $Y$ modelled by a random variable of density $p_Y(y;\vartheta)$, so by taking the expected value of the score wrt to this r.v.\, we can see that the score function at the true value $\vartheta$ has mean $0$.
% \begin{align}
%   \Ex\left[s(\vartheta)\right] &= \int_{\Yspace} p_{Y}(y ; \vartheta) s(\vartheta) \,\mathrm{d}y = \int_{\Yspace} p_{Y}(y ; \vartheta) \pfrac{\log p_Y(y ; \vartheta)}{\vartheta} \,\mathrm{d}y \\
%                             & = \int_{\Yspace} \pfrac{p_Y(y ; \vartheta)}{\vartheta} \,\mathrm{d}y = \frac{\partial}{\partial \vartheta} \int_{\Yspace} p_Y(y ; \vartheta) \,\mathrm{d}y \\
%                             &= 0
% \end{align}

% The covariance matrix of the score function is the \emph{Fisher Information Matrix}:
% \begin{equation}
%   \mathcal{I}(\theta) = \Cov\left[s(\theta)\right]=  \Ex\left[s(\theta) s(\theta)^T\right] = \Ex\left[\left(\pfrac{}{\theta}\log \mathcal{L}(\theta;y)\right) \left(\pfrac{}{\theta}\log \mathcal{L}(\theta;y)\right)^T\right]
% \end{equation}
% \begin{equation}
%   \mathcal{J}(\theta) = \Ex\left[-\nabla^2 \ell(\theta)\right]
% \end{equation}
 So far, the only information assumed on $\theta$ is its parameter space $\Theta$. In the case where some belief on $\theta$ is present before the calibration, we can incorporate this information through the Bayesian framework.
\subsection{Bayesian Inference}
\label{sec:bayesian_inference_MAP}
In Bayesian inference, the uncertainty present on $\theta$ is modelled by considering it as a random variable. Instead of having a precise value for $\theta$, albeit unknown, we assume that we have a \emph{prior distribution} on $\theta$, denoted $p_{\theta}$, that represents the initial state of belief upon the parameter, prior to any experiment and observations. The choice of this prior distribution will be discussed later.
Using the experiment, whose sampling distribution is given by the likelihood, the prior distribution is updated to reflect the new state of belief upon the parameter. 
The Gaussian likelihood in \cref{eq:lik_gaussian} for the frequentist approach can be almost be rewritten as is in the Bayesian setting, just by conditioning $Y$ with $\theta$.
\cref{eq:lik_gaussian} becomes
\begin{equation}
  Y \mid  \theta \sim \mathcal{N}(\mathcal{M}(\theta), \Sigma)
\end{equation}
and the likelihood is the pdf $\mathcal{L}(\theta;y) = p_{Y\mid \theta}(y \mid  \theta)$.
Using Bayes' theorem, the \emph{posterior distribution} of the parameters given the observed data is
\begin{equation}
  \label{eq:bayes_posterior}
  p_{\theta \mid Y}(\theta \mid y) = \frac{p_{Y\mid \theta}(y \mid  \theta)p_{\theta}(\theta)}{p_Y(y)} = \frac{\mathcal{L}(\theta;y)p_{\theta}(\theta)}{p_Y(y)}
\end{equation}
The denominator can be seen as a normalizing constant, ensuring that $\int_{\Theta} p_{\theta \mid Y} = 1$. But it can also be seen as a measure of how well does the model explain the data obtained. This interpretation will be extended in \cref{sec:model_selection}
\begin{definition}[Model Evidence]
\label{def:model_evidence}
  The model evidence, (or marginal likelihood, integrated likelihood) is defined as the distribution of the data marginalised over the parameters.
  \begin{equation}
    \label{eq:model_evidence}
    p_Y(y) = \int_{\Theta}p_{Y,\theta}(y,\theta)\,\mathrm{d}\theta = \int_{\Theta}p_{Y \mid \theta}(y \mid \theta)p_{\theta}(\theta)\,\mathrm{d}\theta
  \end{equation}
  This quantity depends implicitly on the underlying mathematical model $\mathfrak{M} = (\mathcal{M},\Theta)$. Comparing evidence of different models allows for the comparison of those different models. However, computing the model evidence requires the expensive evaluation of an integral over the whole parameter space, and no analytical form is available except for trivial cases. Specific techniques for this evaluation are reviewed in~\cite{friel_estimating_2011}.
\end{definition}
When the model $(\mathcal{M},\Theta)$ and the data $y$ is fixed, the model evidence is constant with respect to the calibration parameter $\theta$. The posterior distribution is thus often written and evaluated up to a multiplicative constant.
\begin{equation}
p_{\theta \mid Y}(\theta \mid y) \propto \mathcal{L}(\theta;y)p_{\theta}(\theta)
\end{equation}



\subsubsection{Posterior inference}
\label{sec:posterior_inference}
This posterior distribution is central in Bayesian analysis, as it gathers all the information we have on the parameter, given the observed data. Given~\cref{eq:bayes_posterior}, evaluating the posterior density at a point requires the evaluation of the model evidence, that is an expensive integral. To bypass this evaluation, several techniques have been developed to get samples from a unnormalized arbitrary function. One of the most well-known method is based on the construction of a Markov-chain whose stationary state is the searched posterior. Classical MCMC algorithms such as Metropolis-Hastings requires the use of a proposal density, and then to accept or reject the proposal based on the posterior distribution evaluated at the point.

A lot of refinement of these methods are available in the literature in order to better tackle the high-dimensionality of the parameter space, or to improve the mixing of the sampled MC chain. One important adaptation to mention is Hamiltonian Monte-Carlo~\cite{hanson_markov_2001,betancourt_conceptual_2017}, that improves the performance of the chain by using the value of the gradient of the log-posterior distribution. Obtaining this gradient (although for a different purpose) is discussed in~\cref{sec:calibration_adjoint_optimization}.


For time-dependent systems, Bayesian framework is particularly well-suited to treat observations sequentially, especially because Bayesian updating is done via multiplication. Bayes' theorem is the basis of many data assimilation methods, such as Kalman filter or various particle filters, that are often used for state estimation.

\subsubsection{Bayesian Point estimates}
\label{sec:bayes_point_estimates}
The whole posterior distribution aggregates a lot of information on the problem. However, as mentioned above, a certain work has to be done in order to get independent samples. Instead, one can try to find a point $\theta \in \Theta$ that summarizes % as best
this distribution. Consequently, the chosen estimate is often an indicator of the central tendency. In that sense, we wish to get a value that is quite close to all sampled values from the posterior~\cite{lehmann_theory_2006}.

Let us define a function $L$ that measures a distance in the parameter space: $L:\Theta\times \Theta$. For a candidate $\theta^{\prime}$, the measured risk with respect to a sample from the posterior $\theta_{\mathrm{sample}} \sim \theta \mid Y$ is $L(\theta^{\prime}, \theta_{\mathrm{sample}})$.
The \emph{Bayesian risk} for $\theta^{\prime}$ is then the expectation of this Bayesian loss functions $L$ under the posterior distribution: $\Ex_{\theta\mid Y}\left[L(\theta^{\prime},\theta)\mid y\right]$. A Bayesian point estimate is defined as a minimizer of the Bayesian risk:
\begin{equation}
  \hat{\theta}_{L} = \argmin_{\theta^{\prime} \in \Theta} \Ex_{\theta\mid Y}\left[L(\theta^{\prime}, \theta) \mid y\right]
\end{equation}

Obviously, different loss functions will lead to different Bayesian point estimates, and we are going to evoke two of them.
\paragraph{Posterior mean}
By defining $L$ as the squared error $L(\theta^{\prime}, \theta) = (\theta^{\prime} - \theta)^2$, we can define the Mean Squared Error (MSE) as $\mathrm{MSE}: \theta^{\prime}\mapsto\Ex_{\theta\mid Y}\left[(\theta^{\prime} - \theta)^2 \mid y\right]$. Finally, the value corresponding to the Minimum Mean Squared Error is
\begin{equation}
  \label{eq:MMSE_mini}
  \estimtxt{\theta}{MMSE} = \argmin_{\theta^{\prime}\in\Theta}\Ex_{\theta\mid Y}\left[(\theta^{\prime} - \theta)^2 \mid  y\right]
\end{equation}
Simple algebraic manipulations show that the minimizer is in fact the posterior mean:
\begin{equation}
  \label{eq:def_MMSE}
  \estimtxt{\theta}{MMSE} = \Ex_{\theta\mid Y}[\theta \mid  y] = \int_{\Theta}\theta\cdot p_{\theta\mid Y}(\theta \mid  y)\,\mathrm{d}\theta
\end{equation}
In order to compute $\estimtxt{\theta}{MMSE}$, it is easier to compute directly the mean of the posterior samples obtained via posterior inference, than to solve the minimization problem in~\cref{eq:MMSE_mini}.

\paragraph{Posterior Mode: the MAP}
Taking $L(\theta^{\prime},\theta) = -\delta_{\theta}(\theta^{\prime})$, the dirac delta function defined in~\cref{eq:def_dirac_delta}, one can show that the minimizer of $\Ex_{\theta\mid Y}\left[L(\theta^{\prime},\theta) \mid y\right]$ is the mode of the posterior distribution, and is called the \emph{Maximum A Posteriori} (MAP):
\begin{align}
  \label{eq:def_MAP}
  \estimtxt{\theta}{MAP} &= \argmin_{\theta^{\prime} \in \Theta}\Ex_{\theta\mid Y}\left[\delta_{\theta}(\theta^{\prime})\mid y\right] = \argmin_{\theta^{\prime} \in \Theta} -p_{\theta\mid Y}(\theta^{\prime}\mid y) \\
                         &= \argmax_{\theta^{\prime} \in \Theta} p_{\theta\mid Y}(\theta^{\prime} \mid y) = \argmax_{\theta^{\prime} \in \Theta} \mathcal{L}(\theta^{\prime} ;y)p_{\theta}(\theta^{\prime})
                           \nonumber
\end{align}
% We chose to use a generalized function as a tool to link the MAP to Bayesian point estimate, but it is sometimes introduced as the limit of $0-1$ loss functions. In~\cite{bassett_maximum_2019}, the authors shows that this claim does not always hold, unless some conditions are met. \todo{à enlever}
One interesting fact about the MAP, is that its evaluation does not require the full knowledge of the posterior distribution, nor samples to evaluate the integral of~\cref{eq:def_MMSE}. We can resort to classical optimization techniques for this evaluation. Similarly to the likelihood, taking the negative logarithm leads to the following minimization problem.
\begin{equation}
  \label{eq:minimisation_MAP_log}
  \estimtxt{\theta}{MAP} = \argmin_{\theta^{\prime}\in \Theta} -\log \mathcal{L}(\theta^{\prime};y) - \log p_{\theta}(\theta^{\prime})
\end{equation}

\subsubsection{Choice of a prior distribution}
\label{sec:choice_prior}
As seen in the application of Bayes' theorem
in~\cref{eq:bayes_posterior}, the prior has a preponderant role in the
formulation of the posterior distribution. Indeed, this prior
distribution represents the current state of knowledge on the value of
the parameter, before any experiment. This comes usually from an
expert opinion, or some reasonable assumptions about the nature of
$\theta$.

Let us assume for instance that we have a Gaussian prior for $\theta$:
$\theta \sim \mathcal{N}(\theta_{b},B)$ where $B$ is called the
background covariance error matrix and $\theta_b$ is called the
\emph{background value} that acts as a plausible reference
value. Assuming a Gaussian form for the errors as well with covariance
matrix $\Sigma$, the MAP can be written as
\begin{equation}
  \estimtxt{\theta}{MAP} = \argmin_{\theta \in \Theta} \frac{1}{2}\| \mathcal{M}(\theta) - y\|^2_{\Sigma^{-1}} + \frac{1}{2}\|\theta - \theta_b\|^2_{B^{-1}}
\end{equation}
Adding a Gaussian prior for the parameter comes down to adding a $L^2$
regularization term to the optimization problem, also called Tikhonov
regularization~\cite{tikhonov_solutions_1977}. This expression is very
analoguous to the state estimation in the 3D-Var method in Data
assimilation.  Other choices of priors lead to other regularizations,
such as the lasso regularization~\cite{tibshirani_regression_2011}
that is a consequence for choosing $\theta$ that follows a priori a
Laplace distribution of mean $0$.

The choice of a prior distribution has an influence on the inference
of the parameter and its point estimation. Where there is no knowledge
on the parameter beforehand, one can try to choose a non-informative
prior in order to try to mitigate its effect. One can for instance
choose a ``flat'' prior over the parameter space, but this can lead to
\emph{improper prior}, in the sense that they do not integrate to
$1$. However, improper priors do not necessarily lead to improper
posterior, allowing for the usual Bayesian analysis of the
quantity. For instance, if $\Theta=\mathbb{R}^n$, the prior
$p_{\theta}(\theta) \propto 1$ is improper, but the MAP estimation is
equivalent to the MLE.

% If $\Theta = \mathbb{R}^p$, an improper non-informative prior is . In this case, the MAP estimation is equivalent to the MLE, as the prior in~\cref{eq:minimisation_MAP_log} is constant with respect to $\theta^{\prime}$.
% When $\Theta = \mathbb{R}_+$, the prior distribution should be invariant by multiplication by a positive constant, so $p_{\theta}(\theta) \propto \frac{1}{\theta}\mathbbm{1}_{\theta > 0}$ (i.e.\ flat in the log scale), and that leads to a regularization term of the form $\log(\theta)$.\unsure{garder flat priors?}
% As mentioned before, the MAP does not require the full knowledge of the posterior distribution $p_{\theta \mid Y}$, as ``only'' an optimization is required.


All in all, when looking for the MAP or the MLE, parameter estimation
boils down to the minimization of a well chosen objective function,
that measures the misfit between the output of the numerical model and
the observations. This objective function will be written $J$ in the
following, to match the notation of data assimilation.  In this
context of calibration, we can then summarize the estimation as a
minimization problem, where $J$ represents some kind of distance
between $\mathcal{M}(\theta)$ and the observations.
\begin{equation}
  \estimtxt{\theta}{} = \argmin_{\theta\in \Theta} J(\theta)
\end{equation}
% If one is looking for the whole posterior distribution,

\section{Calibration using adjoint-based optimization}
\label{sec:calibration_adjoint_optimization}
Point estimates in this context take the form of extremum estimators,
that is an extremum of some given objective function $J$. This
function takes the form of the log-likelihood for the MLE, or the
log-posterior for the MAP, but other misfits can be considered, such
as optimal transport based metrics. The formulation is then quite
simple, but the problem of efficient optimization remains. For
differentiable problems, most of minimization instances are solved
using gradient based methods, such as gradient descent, quasi-newton
methods.  However, this implies to be able to compute efficiently the
gradient of the objective function $J$ with respect to the parameter:
$\nabla_{\theta} J$. The straightforward way, is to compute the
gradient using finite differences. Let us suppose that
$\theta = (\theta_1,\cdots \theta_n)$, and $e_i$ is 0 for all its
component except the $i$th one which is $1$. The gradient can be
approximated by the usual 1st order forward finite-difference scheme,
as displayed in \cref{eq:finite_diff}.
\begin{equation}
  \label{eq:finite_diff}
  \nabla_{\theta} J  \approx \left[\frac{J(\theta + \epsilon e_1) - J(\theta)}{\epsilon}, \frac{J(\theta + \epsilon e_2) - J(\theta)}{ \epsilon},\dots, \frac{J(\theta + \epsilon e_n)- J(\theta)}{\epsilon} \right] \quad \text{ for } \epsilon \ll 1
\end{equation}
In addition to the run of the model at $\theta$, we have to evaluate
the model $n$ times, for each one of the coordinate of $\theta$. If
this is feasible in practice for low dimensional problems, this is
impossible for large problems that cumulate more than hundreds of
parameters. Nevertheless, different methods can be used to compute the
gradient, at least approximately for optimization purpose: for
instance, \cite{boutet_estimation_2015} uses Simultaneous Perturbation
Stochastic Approximation to approximate the gradient using only one
additional run, indepedently on the number of parameters.

In geophysical applications, parameter estimation and the subsequent
optimization is usually performed by deriving the adjoint equation in
order to get the exact gradient for a relatively reasonable cost. This
gradient is used afterward in optimization methods such as conjuguate
gradient, or BFGS for instance. Adjoint methods are thus very popular
in large-scale optimization of Computational Fluid Dynamics codes, as
the additional cost of implementation is often worth the gain in the
short term. This situation is common in data assimilation, as shown
in~\cite{das_estimation_1991,das_variational_1992,honnorat_identification_2010,couderc_dassfow-shallow_2013},
or in shape optimization of airfoils in~\cite{huyse_free-form_2001}.

To derive the adjoint equations, we will first rewrite the objective function as a function of the forward operator and the parameter: $J(\theta) = J(\mathcal{M}(\theta),\theta)$:
The estimation of the parameter can be written as the following constrained optimisation problem:
\begin{equation}
  \begin{aligned}
  \min_{\theta \in \Theta} J(\theta) &= J(y, \theta) \label{eq:def_J_adjoint} \\
  \text{such that } &\mathcal{F}(y, \theta) = 0 % \label{eq:contraint_adjoint}
  \end{aligned}
\end{equation}
where the constraint on $\mathcal{F}$ signifies that the model is admissible, i.e.\ that $y = \mathcal{M}(\theta)\in \Yspace$.

 Differentiating the~\cref{eq:def_J_adjoint} with respect to $\theta$ using the chain rule gives
\begin{equation}
  \begin{aligned}
  \nabla_\theta J &= \pfrac{J}{y} \pfrac{y}{\theta} + \pfrac{J}{\theta}\\
    \nabla_\theta \mathcal{F} &= \pfrac{\mathcal{F}}{y} \pfrac{y}{\theta} + \pfrac{\mathcal{F}}{\theta}
  \end{aligned} \label{eq:aze}
\end{equation}
In those equations, the partial derivatives with respect to $\theta$ are quite easily obtainable, while the real challenge is to obtain the derivative with respect to the state variable: $\pfrac{}{y}$.


To treat the constrained optimization in~\cref{eq:def_J_adjoint}, let us introduce the Lagrange multiplier $\lambda \in \Yspace$, so that we can write the Lagrangian $\mathscr{L}$
\begin{equation}
  \mathscr{L}(\theta, y, \lambda) = J(y,\theta) - \innerprod{\lambda}{\mathcal{F}(y, \theta)}
\end{equation}
 is then
\begin{equation}
  \min_{\theta,y,\lambda} \mathscr{L}(\theta,y,\lambda)
\end{equation}
The first-order condition of optimality for the Lagrangian:
$\pfrac{\mathscr{L}}{\theta} = \pfrac{\mathscr{L}}{y} =
\pfrac{\mathscr{L}}{\lambda} = 0$ translates into the optimality
condition, adjoint equation and the state equation: When
differentiating with respect to the adjoint variable, we retrieve the
state equation:
\begin{equation}
  \pfrac{\mathscr{L}}{\lambda} = - \mathcal{F}(y,\theta) = 0 \tag{State equation}
\end{equation}
When differentiating with respect to the state variable, the equation that verifies the adjoint variable is called the adjoint equation
\begin{equation}
  \pfrac{\mathscr{L}}{y} = \pfrac{J}{y} - \innerprod{\lambda}{\pfrac{\mathcal{F}}{y}}= 0  \tag{Adjoint equation}
\end{equation}
Finally, when $\lambda$ verifies the adjoint equation: $ \left(\pfrac{\mathcal{F}}{y}\right)^T\lambda= \left(\pfrac{J}{y}\right)^T$, the gradient of the objective function can be expressed using the partial derivative \emph{with respect to $\theta$} of the objective function and of the forward model, and the adjoint variable:

\begin{equation}
    \pfrac{\mathscr{L}}{\theta} = \nabla_\theta J %  &= \pfrac{J}{\theta}(y,\theta) - \innerprod{\lambda}{\pfrac{\mathcal{F}}{\theta}(y,\theta)} \\
                               % &=\pfrac{J}{\mathcal{M}}\pfrac{\mathcal{M}}{\theta}+ \pfrac{J}{\theta} - \innerprod{\lambda}{\pfrac{\mathcal{F}}{\mathcal{M}} \pfrac{\mathcal{M}}{\theta} + \pfrac{\mathcal{F}}{\theta}} \\  
                               % &= \left(\pfrac{J}{\mathcal{M}} - \innerprod{\lambda}{\pfrac{\mathcal{F}}{\mathcal{M}}} \right) \pfrac{\mathcal{M}}{\theta} + \left( \pfrac{J}{\theta} - \innerprod{\lambda}{\pfrac{\mathcal{F}}{\theta}} \right) \\
                               = \pfrac{J}{\theta} - \innerprod{\lambda}{\pfrac{\mathcal{F}}{\theta}} = 0 \tag{Optimality condition}
\end{equation}

So, to get $\nabla_\theta J$, as the partial derivatives with respect
to the control variable are relatively easy to obtain, the challenge
lies in solving the adjoint equation. Albeit tedious, one can derive
those equations by writing the tangent linear model of the original
model, and implement a dedicated solver for the adjoint variables. A
more common and way simpler approach is to derive the adjoint
equations directly from the computer code implemented to solve the
model, by using Automatic differentiation tools, such as
\textsc{TAPENADE}~\cite{hascoet_tapenade_2013}. Those programs
directly translate the source code into a program that solves the
original model equations and the adjoint equations, and outputs the
gradient along with the objective function.


\section{Model selection}
\label{sec:model_selection}
So far, we have discussed the calibration of a specific model
$(\mathcal{M}, \Theta)$ given some observations, thus solving an
inverse problem and finding $\estimtxt{\theta}{}$ as an extremum of an
objective function. But different models may be considered to explain
the data. Those models may differ by their forward operator, by their
parameter space, or by both at the same time.

But changing models also means changing the potential ``best'' fit attainable, in terms of minimum reached by the objective function.
More complex models usually provide a better fit of the model but to the cost of a higher dimension in the parameter space. At the same time, more complex models may exhibit an overfitting behaviour.

\begin{example}
  \Cref{fig:overfitting} shows a curve-fitting problem using
  polynomial functions, where the $y_i$'s are realisations of
  $Y_i \sim \mathcal{N}(i, 1)$ for $i=0$ to $10$. The objective
  function associated is
  $J(\kk) = \sum_{i=0}^{10}\|P(i;\kk) - y_i\|^2$, where $P$ is a
  polynomial of degree $\dim \Kspace$, and whose coefficients are
  given by $\kk = (\kk_1,\dots,\kk_n)$.  For this problem of curve
  fitting, increasing the degree of the polynomial used (thus the
  dimensionality of the model) decreases the minimum value of $J$
  reached. However, the increase in the degree leads also to some
  oscillations between the sampled points, as the fitting procedure
  looks to account for the deviations due to the random origin of the
  $y_i$'s
\end{example}

\begin{figure}[ht]
  \centering
  \input{\imgpath overfitting.pgf}
  \caption[Example of overfitting phenomenon]{\label{fig:overfitting} Overfitting phenomenon, and reduction of the minimal value of the objective function}
\end{figure}


We can then look to reduce the complexity of the model, without decreasing significantly its performance.

We are first going to consider the case of nested models: models that share the same forward model, but whose parameter spaces are nested. Model selection in this case is a way to reduce the dimension of the model, by reducing the parameter space.

Finally, tools introduced in this section bring up model comparison. A calibrated model $(\mathcal{M},\{\hat{\theta}\})$ is optimal given an objective function, but we can show that values close to the calibrated value $\hat{\theta}$ may also be of interest, as the decrease of performance may not be statistically significant. The ``perturbed'' model $(\mathcal{M}, \{\hat{\theta} + \varepsilon\})$ for small $\varepsilon$ may accurately describe the data as well.

\subsection{Likelihood ratio test and relative likelihood}
\label{sec:likelihood_ratio_test}
Generally speaking, more complex models have a better ability to represent the data, but all the parameters included in the model may not be relevant for the modelling. It can be interesting to test if a ``simpler'' model would give similar performances, or at least show a decrease in performances that is not statistically significant.
One of the most well-known test is the Likelihood-ratio test, that tests if two \emph{nested models} are equivalent:
Let us consider two nested models: $\mathfrak{M}_1 = (\mathcal{M}_1, \Theta_1)$, $\mathfrak{M}_2= (\mathcal{M}_2,\Theta_2)$, such that $\mathcal{M}_1=\mathcal{M}_2=\mathcal{M}$ and $\Theta_2 \subsetneq \Theta_1$. In this case, $\mathfrak{M}_2$ represents the simpler model, with a reduced parameter space, while $\mathfrak{M}_1$ is the more general model. Recalling the notion of model dimension in~\cref{rmk:model_dimension},  $\mathfrak{M}_1$ has dimension $r$, and $\mathfrak{M}_2$ has dimension $d$ with $r>d$.
As $\mathfrak{M}_1$ is more general, one can expect better performances.

The likelihood ratio is defined as the ratio of the largest values taken by the likelihood on their respective parameter space, value that is assumed to be attained at $\hat{\theta}_1$ and $\hat{\theta}_2$.
% Under the null hypothesis, the two models are equivalent, that is that the ``smaller'' parameter space is enough to represent the inverse problem: $\theta \in \Theta_2$. The alternative hypothesis is that $\theta \in \Theta_1$.
\begin{equation}
  \label{eq:def_lik_ratio}
  \Lambda(y) = \frac{\sup_{\theta \in \Theta_2} \mathcal{L}(\theta ; y)}{\sup_{\theta \in \Theta_1} \mathcal{L}(\theta ; y)} = \frac{\mathcal{L}(\hat{\theta}_2 ; y)}{\mathcal{L}(\hat{\theta}_1 ; y)} \leq 1
\end{equation}
Based on this quantity, we can test whether the smaller model is sufficient to explain the data as good as the larger model. The two hypothesis for this test are
\begin{itemize}
\item The null hypothesis $\mathcal{H}_0$: The two models are statistically equivalent: the difference between the maximal values of the likelihood is not statistically significant. This corresponds to $\Lambda$ close to $1$
\item  The alternative hypothesis $\mathcal{H}_1$: the two models are statistically different: the larger model performs better than the reduced one. This corresponds to $\Lambda$ significantly smaller than $1$.
\end{itemize}
Under the null hypothesis  $\mathcal{H}_0$, $-2 \log \Lambda$ follows asymptotically (as the number of observations grows large) a $\chi^2$ distribution defined in~\cref{ex:chi2}, according to Wilks' theorem~\cite{wilks_large-sample_1938}. The number of degrees of freedom of the $\chi^2$ distribution is given by the difference of dimensionality between the two models:
\begin{equation}
  \label{eq:deviance_asymptotics}
  - 2 \log \Lambda(y) \xrightarrow[]{\mathrm{d}} \chi^2_{r-d}
\end{equation}
By denoting $\chi^2_{r-d}(1-\nu)$ the quantile of order $1-\nu$ of the $\chi^2$ distribution with $r-d$ degrees of freedom, the asymptotic rejection region of the null hypothesis at level $\nu$ is:
\begin{align}
  \mathrm{RejReg}_{\nu} &= \left\{y \mid -2 \log \Lambda(y) > \chi^2_{r-d}(1-\nu) \right\} \label{eq:LRT_rej_reg}
\end{align}
So if the data $y \in \mathrm{RejReg}_{\nu}$, we reject $\mathcal{H}_0$ at the $\nu$-level, thus we accept $\mathcal{H}_1$.
By reformulating using the log-likelihoods and objective functions $l(\theta;y) = \log \mathcal{L}(\theta;y) = - J(\theta)$
\begin{align}
  \mathrm{RejReg}_{\nu} &= \left\{ y \mid (\sup_{\theta\in\Theta_1} l(\theta;y) - \sup_{\theta\in\Theta_2} l(\theta;y)) > \frac12 \chi^2_{r-d}(1-\nu) \right\} \\
                             &= \left\{ y \mid J(\theta_2) - J(\theta_1) >  \frac12 \chi^2_{r-d}(1-\nu) \right\} \label{eq:rejreg_chi2}
\end{align}


\begin{example}[Likelihood Ratio test for the MLE]
  \label{ex:lrt_mle}
  Let $\Theta_1 = \Theta\subset \mathbb{R}$ and $\Theta_2 = \{\kk\}\subset \Theta_1$.
  We have then $\sup_{\Theta_1} \mathcal{L} = \mathcal{L}(\estimtxt{\kk}{MLE})$, and $\sup_{\Theta_2} \mathcal{L} = \mathcal{L}(\kk)$
 The difference of dimensionality is then $r-d=1$ and the .95 quantile of the $\chi^2$ r.v.\ is $\chi^2_1(1-0.05) =3.84$. When the data falls into the rejection region (\emph{i.e.} $J(\theta_2)$ significantly larger than $J({\theta}_1)$), the null hypothesis is rejected, and the models can be asserted significantly different. 
\end{example}



\label{sec:relative_likelihood}
Conversely, the rejection region defined above allows us to define an asymptotic confidence interval for the parameter. Let us consider $\Theta_1 = \Theta$, and $\Theta_2 = \{\kk\}$ as in \cref{ex:lrt_mle}, and let us introduce the \emph{Relative Likelihood} (\cite{kalbfleisch_probability_1985}) which is the ratio of the likelihood evaluated at a point $\kk$ to the maximal value of the likelihood:
\begin{equation}
  R(\theta) = \frac{\mathcal{L}(\theta;y)}{\mathcal{L}(\estimtxt{\theta}{MLE};y)} = \frac{\mathcal{L}(\theta;y)}{\sup_{\theta^{\prime} \in \Theta}\mathcal{L}(\theta^{\prime};y)}
\end{equation}
This ratio allows for comparing the plausibility of the value $\kk$, compared to the MLE.
The likelihood interval of level $p\in ]0,1]$ is defined as
\begin{equation}
  \label{eq:def_lik_interval}
  \mathcal{I}_{\mathrm{Lik}}(p) = \left\{\theta \mid R(\theta)=\frac{\mathcal{L}(\theta;y)}{\mathcal{L}(\estimtxt{\theta}{MLE};y)} \geq p\right\}
\end{equation}

$p$ can be set to an arbitrary threshold, but it can also be chosen specifically so that $\mathcal{I}_{\mathrm{Lik}}(p)$ is the complement of the rejection region of a likelihood ratio test with certain confidence.

Using the Likelihood ratio test, as in~\cref{ex:lrt_mle}, we can test whether a given value of the parameter $\kk$ performs compared to the MLE by comparing the models $(\mathcal{M}, \{\theta\})$ and $(\mathcal{M},\Theta)$. Let $R(\theta)$ be the corresponding likelihood ratio. The complement of the rejection region \cref{eq:rejreg_chi2} written using \cref{eq:def_lik_interval} is
\begin{equation}
  \label{eq:rel_lik_threshold}
  \mathcal{I}_{\mathrm{Lik}}\left(\exp\left(-\frac{1}{2}\chi^2_{\mathrm{dim}(\Theta)}(1-\nu)\right)\right) = \left\{\theta \mid R(\theta) \geq \exp\left(-\frac{1}{2}\chi^2_{\mathrm{dim}(\Theta)}(1-\nu)\right) \right\}
\end{equation}
The values of the calibrated parameters in this set generate models that are statistically equivalent to the model comprising the MLE as its calibrated parameter.


For $1$ dimensional models, and the confidence level of $.05$, the threshold of \cref{eq:rel_lik_threshold} is $\exp\left(-\frac{1}{2}\chi^2_{\mathrm{dim}(\Theta)}(1-\nu)\right) = \exp\left(-\frac{1}{2}\chi^2_{1}(.95)\right) \approx 0.15$, and at a level $.10$, $\exp\left(-\frac{1}{2}\chi^2_{1}(.90)\right) \approx 0.26$.

\begin{figure}[ht]
  \centering 
  \input{\imgpath relative_likelihood.pgf}
  \caption{\label{fig:relative_likelihood} Example of relative likelihood, and associated likelihood interval, for $p=0.15$}
\end{figure}

Due to the likelihood ratio test and the relative likelihood, we can see that even though $\estimtxt{\theta}{MLE}$ is the optimizer of the likelihood function, other values close to it may not be discarded, provided that the value of the log-likelihood does not drop off too much.

\subsection{Criteria for non-nested model comparison}
% \todo{revoir}
\label{sec:criteria_AIC}
The likelihood ratio test presented above is defined for nested models. In the more general case, we can also associate each model with a single numerical value that measures the balance between ``fit'' and complexity of the model. This takes usually the form of
\begin{equation}
  \mathrm{Crit}(\mathfrak{M})=-2 \log \mathcal{L} (\estimtxt{\theta}{MLE}) + \text{Complexity penalization}
\end{equation}
where $\mathcal{L}$ is the likelihood function for the model $\mathfrak{M} = (\mathcal{M},\Theta)$, and $\mathcal{L}(\estimtxt{\theta}{MLE}) = \max \mathcal{L}$. The role of the complexity penalization is to avoid overfitting, and is often directly linked to the dimension of the parameter space $\Theta$.
Two quite popular examples of criteria are the AIC (Akaike Information Criterion) introduced in~\cite{akaike_new_1974} and the BIC (Bayesian Information Criterion) in~\cite{schwarz_estimating_1978}.

Then, to compare two models $\mathfrak{M}_1$ and $\mathfrak{M}_2$, the magnitude of the difference $\mathrm{Crit}(\mathfrak{M}_1) - \mathrm{Crit}(\mathfrak{M}_2)$ is compared to some thresholds as shown in~\cite{burnham_multimodel_2004}.
The difference shows whether a model should be preferred, or if no substantial evidence exists for either model.
These criteria, as well as the likelihood ratio test, are based on the evaluation of the likelihood at its maximal value.


Another approach is to marginalize the likelihood with respect to the prior distribution of the calibration parameter, giving Bayesian model comparison. The criterion for a model $\mathfrak{M}_1$ is the evidence of model, \emph{i.e.} the pdf of the data given the model $\mathfrak{M}_1$, and $\mathrm{Crit}(\mathfrak{M}_1) = \log \int_\Kspace \mathcal{L}(\kk) p_\kk(\kk)\,\mathrm{d}\kk$.
% % \subsection{Bayesian model comparison}
% Given a model $\mathfrak{M}$, the model evidence is the likelihood marginalized over the parameter space as introduced in~\cref{def:model_evidence}, and will be written $p_{Y\mid \mathfrak{M}}$. This evidence represents how likely have the data $y$ been generated using the statistical model $\mathfrak{M}$.
% For two models $\mathfrak{M}_1$ and $\mathfrak{M}_2$ the Bayes' factor is defined as the ratio of the evidence of the two models:
% \begin{equation}
%   \label{eq:bayes_factor}
%   \mathop{\mathrm{BF}}(\mathfrak{M}_1,\mathfrak{M}_2)= \frac{p_{Y\mid \mathfrak{M}_1}(y \mid \mathfrak{M}_1) }{p_{Y\mid \mathfrak{M}_1}(y \mid \mathfrak{M}_2)}
% \end{equation}
% where
% \begin{equation}
%   p_{Y \mid \mathfrak{M}_i}(y \mid \mathfrak{M}_i)= \int_{\Theta_i} p_{Y, \theta \mid \mathfrak{M}_i}(y, \theta \mid \mathfrak{M}_i) \,\mathrm{d}\theta= \int_{\Theta_i} p_{Y \mid \theta,\mathfrak{M}_i}(y \mid \theta, \mathfrak{M}_i)p_{\theta \mid \mathfrak{M}_i}(\theta \mid \mathfrak{M}_i) \,\mathrm{d}\theta 
% \end{equation}
% % This factor, as described in~\cite{fragoso_bayesian_2018}, represents how much more evidence does the model $\mathfrak{M}_1$ provides of being ``true'' than $\mathfrak{M}_2$.
% Quite similarly to the BIC and AIC, the logarithm of the Bayes' factor
% Again, it is usually compared to specific values, allowing us to conclude roughly on how strong does the data favor $\mathfrak{M}_1$. 
Again, the logarithm of Bayes' factor is compared with specified thresholds~\cite{kass_bayes_1995,burnham_multimodel_2004}.
% Given a model, a criterion is derived either by maximization (for the likelihood ratio test, and criteria of \cref{sec:criteria_AIC}) and marginalization (for Bayes' factors).
% In both cases, the preference toward one or another model is directly linked to the difference on the logarithms of the densities.

\section{Parametric model misspecification}
\label{sec:model_misspecification}
We introduced earlier the mathematical model $(\mathcal{M},\Theta)$, and based our analysis on the fact that the ``target model'', i.e.\ the reality is $(\mathscr{M},\Theta_{\mathrm{real}} \supseteq \Theta)$, so the parameter spaces are the same for the two models. However, between the reality and the numerical model, various simplifications are introduced, thus the reality is not often completely \emph{representable} by the numerical model: we have then misspecified models.

\begin{definition}[Misspecified model]
  Let $Y$ be the random variable associated with the observations, and $p_Y$ its pdf. Let $\{p_{Y \mid \KK} ; \kk \in \Kspace\}$ the parametric family of densities, among which we are looking to find $p_Y$. The model is said to be misspecified, if $p_Y \notin \{p_{Y \mid \KK} ; \kk \in \Kspace\}$.

  We can also define this misspecification in terms of numerical models defined in this chapter:  
  let $(\mathscr{M},\Theta_{\mathrm{real}})$ be the physical system under study and $(\mathcal{M},\Theta)$ the numerical model that is to be calibrated with respect to the observations $y=\mathscr{M}(\vartheta)$. $(\mathcal{M}, \Theta)$ is said to be misspecified, if $\vartheta \notin \Theta$.

  Finally, when we have a family of models $\{(\mathcal{M}(\cdot, \uu),\Kspace) \mid \uu \in\Uspace\}$, where each element depends on some additional parameter, and for all $\uu\in \Uspace$, $\mathcal{M}(\cdot,\uu) \neq \mathscr{M}$, we talk about \emph{parameteric misspecification}.
\end{definition}

In practice, in addition to the simplifications, the parameter space $\Theta$ does not contain necessarily all the parameters needed to run the forward model, but represents the space of the parameters of interest, or calibration parameters. In addition to them, some other parameters are at play, that we are going to call the \emph{environmental parameters}, or \emph{uncertain parameters} written $u\in \Uspace$. These parameters come from instance from the external forcings. 



Bayesian framework and more specifically Bayesian update of the prior by the likehihood puts the emphasis on the update of the information on the \emph{parameter of interest}. However the environmental parameters are assumed to have an inherent variability. In that sense, it may not be worth spending time and resources to infer these parameter values, as they are bound to change.
Moreover, we can only get information on the environmental conditions used to generate the observations.

% We aim at letting this parameter stay free, and at finding a good value of the calibration parameter $\theta$, without doing inference on $\uu$.

In terms of models, each choice of $\uu \in \Uspace$ gives a different model $\mathfrak{M}(u) = \{\mathcal{M}(\cdot, u),\,\Theta\}$. Let us assume that we can model the uncertain parameters as a random variable $\UU$.
Let us consider that we chose a specific $\uu_0 \in \Uspace$, and that we are given some observation $y=\mathscr{M}(\vartheta)$. We can formulate an inverse problem, and an objective function $J:\kk \mapsto J(\kk,\uu_0)$, that we wish to minimize with respect to $\kk$.
Some estimators still carry nice properties. The MLE for instance, defined~\cref{sec:frequentist_inference_MLE} can still be written as the minimizer of the empirical KL--divergence. We assume that we can write the sampling distribution as $p_{Y \mid \KK, \UU}$, and
\begin{equation}
  \estimtxt{\theta}{MLE}(\uu_0)=  \argmin_{\kk \in \Kspace}\DKL[empirical]{p_Y}{p_{Y \mid \theta, \UU}(\cdot \mid \theta, \UU=\uu_0)}
\end{equation}
and can be seen as the ``best'' value given $\UU=\uu_0$. However, the asymptotic properties of the MLE are slightly different as described in~\cite{white_maximum_1982}.
So when the model is misspecified, minimizing the same objective function still makes sense.

However, the calibration will depend on the chosen $\uu_0$: $\estimtxt{\theta}{}(\uu_0) = \argmin_{\theta \in \Theta} J(\theta,\uu_0)$, and there is no guarantee that $\estimtxt{\theta}{}(\uu_0)$ will minimize $J(\cdot, \uu_1)$ for $\uu_0 \neq \uu_1$, as illustrated~\cref{fig:minimizer_misspecification}.

\begin{figure}[ht]
  \centering
  \input{\imgpath minimizer_misspecification.pgf}
  \caption{Effect of the misspecification on the minimizer.}
  \label{fig:minimizer_misspecification}
\end{figure}

$\uu_0$ and $\uu_1$ are close to each other, but ${\kk}(\uu_0)$ and ${\kk}(\uu_1)$ are not. However, as the objective function shows similar values at those points, choosing either one would lead to a satisfactory outcome given $\UU = \uu_0$ or $\UU=\uu_1$. If $\UU = \uu_2$ is considered as well, the modeller may have a preference and choose $\estimtxt{\kk}{}(\uu_1)$ as the final estimator.

In terms of model selection, the asymptotic distribution of the likelihood ratio statistic defined~\cref{sec:likelihood_ratio_test} is also slightly different. Instead of following a $\chi^2_{r}$ distribution, where $r$ is the number of dimensions for the test, $-2\log \Lambda$ will asymptotically have the same distribution as a weighted sum of $r$ random variables, where every one have a $\chi^2_1$ distribution and whose weights are the eigenvalues of a matrix involving the Jacobian and the Hessian of the log-likelihood~\cite{kent_robust_1982}.


 This random misspecification leads to some issues in the calibration of the model, and it asks for a notion of robustness with respect to the environmental parameters. 

\section{Partial conclusion}
\label{sec:chap2_partial}

In this chapter, starting from a probabilistic point of view, we established the usual tools encountered in model calibration: the misfit between the data and the numerical model is measured by an objective function $J$, that can be minimized using for instance gradient descent. From this optimization, we can define a ``acceptable'' region for the estimate. In other words, values in this set yield an misfit that is not different enough to be completely discarded.

Adding an environmental variable as a random parameter introduces a random parametric misspecification to the model: each realization of this underlying random variable will yield a different estimation. In \cref{chap:robust_estimators}, we will discuss the notion of robustness under this random misspecification, and introduce a family of robust estimators, inspired by model selection. 


% Robustness under parametric model 

%%%%%%%%%%%%%%%%%%%%%%%%%%%%%%%%%%%%%%s

%% BIB
%%%%%%%%%%%%%%%%%%%%%%%%%%%%%%%%%%%%%%
\subfileLocal{
  % \listoftodos
	\pagestyle{empty}
	\bibliographystyle{alpha}
	\bibliography{/home/victor/acadwriting/bibzotero}
}
\end{document}


%%% Local Variables:
%%% mode: latex
%%% TeX-master: "../../Main_ManuscritThese"
%%% End:
