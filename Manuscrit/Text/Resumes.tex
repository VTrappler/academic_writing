% 4e de couverture
\pagestyle{empty}


\footnotesize

The wrist is an essential joint, source of the large range of motion of the hand. It is also a complex joint, composed of eight small bones, connected to five metacarpal bones and two forearm bones. The complexity of the joint is not only due to the  high number of interconnected bones, but also to the small size of the carpal bones and their elaborate shapes interlocked with each other, that move in an intricate way around each other. 
In this thesis we are interested in modeling the 3D wrist bone shapes. Not many works on wrist bones modeling have been conducted yet and little data have been collected into databases exploitable for computer models. The latter can be used to take measurements, serve as basis for the creation of automated IT tools, or else be integrated into software for diagnosis support for example. The quality of the results of such applications depends on the quality of the model. We therefore attach a special importance to the validation of our work, while such assessment cannot directly be measured and must be proven by indirect metrics. 
Interest was taken in tools for the modeling of 3D shapes, especially in techniques of correspondence between 3D meshes. We propose a method to transform raw meshes extracted from CT scans into bones representations with correspondence relations.
 %It is not trivial to define a resampling procedure, and neither is its quality assessment, to which we pay particular attention. 
 The dense correspondence relations computed make possible many applications, that serve as further validation of the correspondence results. We propose several utilizations. Variability among bones is analyzed with statistical procedures such as the Principal Component Analysis (PCA) and another one based on Gaussian Processes.
The registration capacities of the first model are employed for defining correspondence with a second database. We propose a method to easily transfer systems of coordinates or other landmarks from a few example towards the rest of the database, a convenient function for biomechanical wrist motion study. In a last phase, we are concerned with modeling wrist bones motions with a parametric model based on meaningful and easily measurable predictors.

\etoile

\pagestyle{empty}

\footnotesize

\begin{singlespace}
	
Le poignet humain est une articulation essentielle, car il est à l'origine de la grande amplitude de mouvement de la main. C'est également une articulation complexe, composée de huit petits os carpiens, qui sont connectés aux cinq métacarpes et aux deux os de l'avant-bras. La complexité de l'articulation est non seulement due à ce grand nombre d'os, mais également à la petite taille des os carpiens et à leurs formes élaborées, qui rendent le mouvement des os les uns autour des autres également complexe. Dans cette thèse, nous nous sommes intéressés à la modélisation 3D de la forme des os. Peu de travaux ont été menés jusqu'à présent sur la modélisation des os du poignet, et peu de données exploitables pour des modèles informatiques ont été collectées. Or de tels modèles informatiques peuvent avoir de nombreuses applications : ils peuvent servir de base pour la création d'outils informatiques automatisés ou encore être intégrés dans des logiciels servant de support au diagnostic. La qualité des résultats de telles applications dépend de la qualité du modèle utilisé. C'est pourquoi nous attachons une attention particulière à la validation de notre travail, alors même qu'il n'existe pas de mesure directe pour l'évaluation, et qu'il faut utiliser des métriques indirectes.
Nous nous sommes intéressés à des outils pour la modélisation 3D, particulièrement aux techniques de correspondance entre maillages. Nous présentons une méthode pour transformer des maillages bruts directement créés à partir de tomodensitogrammes en nouveaux maillages représentant les mêmes os tout en définissant des relations de correspondance.
Une fois définies, ces relations rendent possibles de nombreuses applications, qui permettent une validation supplémentaire des correspondances. Nous présentons plusieurs applications. La variabilité de la forme des os est analysée à l'aide d'outils statistiques tels que l'Analyse en Composantes Principales (ACP) ainsi qu'un outil basé sur les Processus Gaussiens.
Les capacités d'adaptation du modèle ACP à de nouvelles formes sont utilisées pour définir des relations de correspondance avec une seconde base de données. Nous proposons également une méthode pour transférer simplement des systèmes de coordonnées ou tout autre point remarquable défini pour quelques exemples vers le reste de la base de données. Une telle application est utile pour des études biomécaniques de mouvement du poignet. Finalement, dans une dernière étape, nous nous sommes intéressés à la modélisation des mouvements des os du poignet à l'aide d'un modèle paramétrique basé sur des prédicteurs significatifs et facilement mesurables.

\end{singlespace}