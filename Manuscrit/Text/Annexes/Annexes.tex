\documentclass[../../Main_ManuscritThese.tex]{subfiles}

\subfileGlobal{
\renewcommand{\RootDir}[1]{./Text/Annexes/#1}
}

% For cross referencing
\subfileLocal{
\externaldocument{../../Text/Introduction/build/Introduction}
\externaldocument{../../Text/Chapter2/build/Chapter2}
\externaldocument{../../Text/Chapter3/build/Chapter3}
\externaldocument{../../Text/Chapter4/build/Chapter4}
\externaldocument{../../Text/Chapter5/build/Chapter5}
\externaldocument{../../Text/Conclusion/build/Conclusion}
}

%%%%%%%%%%%%%%%%%%%%%%%%%%%%%%%%%%%%%%
%% CHAPTER TITLE
%%%%%%%%%%%%%%%%%%%%%%%%%%%%%%%%%%%%%%

\begin{document}
%	
%\appendix
\pagestyle{conclusionStyle}


\begingroup

%% Le chapitre est numéroté A
\setcounter{chapter}{0}
\renewcommand{\thechapter}{\Alph{chapter}}%

%% Le titre doit être entre les traits de la 1ere page du chapitre
\TitleBtwLines

\specialchapter{Appendix}
\pagestyle{appendixStyle}

\section{Likelihood ratio test for misspecified nested models}
\cite{white_maximum_1982}
The observations $Y$ are following the law with true density $p_Y$, and $n$ samples are available $y=\{y_i\}$.
We have a family of distributions $\{p_{Y\mid \KK}\}$.
  The quasi log-likelihood is
  \begin{equation}
    \mathcal{L}(\kk; y) = \frac{1}{n}\sum_{i=1}^{n} \log p_{Y \mid \KK}(y;\kk)
  \end{equation}

  \begin{align}
    A_n(\kk) = \frac{1}{n} \sum_{i=0}^n \frac{\partial^2 \log p_{Y \mid \KK}(y\mid\kk)}{\partial \kk_i \partial \kk_j} \\
    B_n(\kk) = \frac{1}{n} \sum_{i=0}^n \pfrac{p_{Y\mid \KK}(y \mid \kk)}{\kk_i} \pfrac{p_{Y\mid \KK}(y \mid \kk)}{\kk_j}
  \end{align}

  \begin{align}
    A(\kk) &= \Ex\left[\frac{\partial^2 \log p_{Y\mid \KK}(y \mid \kk)}{\partial \kk^2}\right] _\\
    B(\kk) &= \Ex\left[\pfrac{\log p_{Y\mid \KK}(y \mid \kk)}{\kk}\pfrac{\log p_{Y\mid \KK}(y \mid \kk)}{\kk}^T\right]
  \end{align}

  If the model is well-specified, in other words if there exists $\kk_0$ such that $p_Y = p_{Y\mid \KK}(\cdot \mid\kk_0)$, $A(\kk) = -B(\kk)$:
  \begin{align}
    \frac{\partial^2 \log p_{Y\mid \KK}(y \mid \kk)}{\partial \kk^2} &= \pfrac{}{\kk}\left(\frac{1}{p}\pfrac{p}{\kk}\right) \\
    &= \frac{1}{p}\frac{\partial^2 p}{\partial \kk^2} - \frac{1}{p^2}\left(\pfrac{p}{\kk}\right)^2 = \frac{1}{p}\frac{\partial^2 p}{\partial \kk^2} - \left(\pfrac{\log p}{\kk}\right)^2
  \end{align}
  Taking the expectation, we have
  \begin{align}
    A(\kk) &= \Ex\left[\frac{\partial^2 \log p_{Y\mid \KK}(y \mid \kk)}{\partial \kk^2}\right] \\
           &= -\Ex\left[\left(\pfrac{\log p}{\kk}\right)^2\right] + \Ex\left[\frac{1}{p}\frac{\partial^2 p}{\partial \kk^2}\right]
  \end{align}
  and the last term is
  \begin{align}
     \Ex\left[\frac{1}{p}\frac{\partial^2 p}{\partial \kk^2}\right] = \int_{\Yspace} \frac{1}{p}\frac{\partial^2 p}{\partial \kk^2} p_{Y}(y) \,\mathrm{d}y
  \end{align}
  If the inverse exist,
  then
  \begin{align}
    C_n(\kk) &= A_n(\kk)^{-1} B_n(\kk) A_n(\kk) \\
    C(\kk) &= A(\kk)^{-1}B(\kk)A(\kk) \\
  \end{align}

\minitoc
\pagestyle{appendixStyle}

%%%%%%%%%%%%%%%%%%%%%%%%%%%%%%%%%%%%%%%%%%%%%%%%%%%%%%%


%%%%%%%%%%%%%%%%%%%%%%%%%%%%%%%%%%%%%%
%% BIB
%%%%%%%%%%%%%%%%%%%%%%%%%%%%%%%%%%%%%%
\subfileLocal{
	\pagestyle{empty}
	\bibliographystyle{alpha}
	\bibliography{/home/victor/acadwriting/bibzotero}

}

\endgroup

\end{document}

%%% Local Variables:
%%% mode: latex
%%% TeX-master: "../../Main_ManuscritThese"
%%% End:
