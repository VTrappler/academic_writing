%%%%%%%%%%%%%%%%%%%%%%%%%%%%%%%%%%%%%%%%%%%%%%%%
\section{Introduction}
\label{sec:4_Introduction}



In this chapter, we present a dense correspondence mapping method, which is applied to 8 carpal bones, 5 metacarpals and the radius of around forty wrists of the \db*. The ulna is left out, the pronation-supination mouvement is not a focus of the study, and the radius alone is used as reference. The carpal bones raise particular difficulties for correspondence mapping, as they have various complex shapes, which need to be precisely captured. Whether for bone shape analysis, study of shape influence over each other or detection of bone erosion, precision in bones descriptions and in inter-correspondence is of high importance and particular attention is paid to the results quantitative evaluation. 
We also endeavor to propose a method easily implementable and reproducible. 

We present a correspondence method based on the deformation of a template using a mesh deformation method: Laplacian Surface Edition \cite{sorkine_2004_laplacian}. The chosen templates are initially bones from the database. To reduce particular details effects these chosen meshes are smoothed and downsampled to speed up computations. The deformation of the template towards the target shapes is computed following a forward group-wise registration model. The template is registered to the database meshes, then it is updated by averaging every vertex positions. These steps are iterated, the update of the template allows to reduce its influence. Finally, we use a simple local projection along the normals of the vertices towards the target shape to refine the results. 

%Les templates sont progressivement construits en partant de meshes bruts de la BD 

%Faire un renvoi vers la section où je dis qu'il faut update les reference pour limiter leur influence

From all the existing methods for carpal bones correspondence definition, the method proposed by Joshi et al. \cite{joshi_2016_registration-based} is the closest to ours. Indeed, we also use templates selected from the database, that are deformed to match the target meshes. However important differences exist between both approaches. At first, we have chosen to select each template independently, while in \cite{joshi_2016_registration-based} they all come from the same wrist. We justify the independence of the templates choice with the mapping of each bone which is computed separately anyway, the most important parameter is that the bone should not be too specific. Indeed it has been proven that the template chosen has an impact on the following results. We use an iterative deformation, with visual inspections at every steps for control if need be. It provides a more local control on the results than the minimization of a cost function in an analytical way like in \cite{joshi_2016_registration-based}. Moreover if proved to be necessary, the deformation could be guided manually. We propose to use a forward group-wise registration model, that iteratively deforms the template towards the targets and then updates the target meshes. It's been proved that updating the template is important to limit the initial template influence. On the other hand, in \cite{joshi_2016_registration-based}, the templates are being deformed only once. Finally we have added a last step of projection along the normals to optimize surface similarity.

As illustrated in chapter \ref{chap:Wrist}, dense correspondence is a strong property for polygonal meshes, that is required for many applications. However, it is also a property that cannot be easily verified, as no objective metric exists to validate or evaluate it. In the absence of such an unbiased criteria, other tests must be used to prove that if the error can not be exactly measured, it lies nonetheless within a limited range. 


We have chosen to use several metrics to compare similarity between meshes. Two point-to-face distances between surfaces are used, one that measures the mean distance between the vertices and their closest point on the other mesh \eqref{eq:mesh_dist}, and the Hausdorff distance that measures the maximum distance between a vertex and its closest point on the opposite mesh \eqref{eq:mesh_hausdorff}. Both are symmetrical and evaluate both accuracy and completeness since we want to avoid both folds in the deformed template and missed details from the target shape. Two other volumetric metrics are used, the Dice coefficient and the Jaccard distance, that are based on the overlap between two volumes. Both are given, though they are related. These metrics values are intended for comparisons with future methods that would like to compare their results with our algorithm. 



In this chapter, we pay particular attention to assess every step of the process. We guarantee an upper bound on errors and show that our meshes can be trusted to be used for any application wanted. We prove that the output mesh outline very similar 3D volumes than the initial database meshes, and a statistical shape model is used in the next chapter to measure the 3 factors: generalization, specificity and compactness. 

\vspace*{2cm}



 
%A criticism towards this method in the case of parallel pair-wise alignments is the bias introduced by the choice of a template that is registered to the set of shapes. However iterations between pairwise correspondence and reference shape update solves this issue.
%\cite{rueckert_2003_automatic}


%% Contenu -> Donner envie de lire le chapitre

% Retour sur ce qui existe en dense correspondence pour les os carpiens
% Retour sur la méthode qu'on a choisi, pourquoi la déformation d'un template
% Retour sur les distances qu'on a choisi d'utiliser
% Explication que on va tout argumenter, tout mesurer, pour s'assurer de la justesse du truv
% Puis on va rendre le travail public, pour que cela serve à d'autres personnes

% Notre travail présente plusieurs intérets : nouvelle base de données, résultats publics, bien mesurés
% Joshi (2016) prouve que ça donne des meilleurs résultats de choisir un template que de faire une moyenne des shapes ou qqch comme ça
 
% Les gens utilisent les remiallage, mais ne prouvent jamais qu'ils sont bons. On prend un soin particulier à démontrer que nos résultats sont bons pour convaincre les futurs de s'en servir


 
% L'idée c'est qu'on a été très sobre dans la présentation des méthodes pour les os carpiens, on a juste expliqué ce qui existait. Là on argumente pourquoi on veut qqch par rapport à autre chose
 
% Exactement la meme pour les distances 
 
% Joshi : selection de 1 patient dont tous les os sont tous les templates, nous templates différents patients
 
% Pseudo-distance metric
 
% Les seuls à donner des résultats numériques sont Joshis et al (2016) qui mesurent la distance entre les meshes resampled et les meshes en correspondence mais la métrique de distance qu'ils utilisent n'est pas bonne. Ils ne divisent pas par le nombre d'éléments dans leur mesh, ce qui empêche de faire toute comparaison avec eux, nous n'avaons pas le même nombre d'éléments. et cela d'autant plus qu'ils ne donnent même pas le nombre précis d'éléments dans leurs meshes, ils disent simplement que c'est environ 5000. YA aussi ceux qui ont donné des résultats mais pas pour les os carpiens