\documentclass[../../Main_ManuscritThese.tex]{subfiles}

\subfileGlobal{
\renewcommand{\RootDir}[1]{./Text/Chapter3/#1}
}

% For cross referencing
\subfileLocal{
\externaldocument{../../Text/Introduction/Introduction}
\externaldocument{../../Text/Chapter2/Chapter2}
\externaldocument{../../Text/Chapter4/Chapter4}
\externaldocument{../../Text/Chapter5/Chapter5}
\externaldocument{../../Text/Conclusion/Conclusion}
\externaldocument{../../Text/Annexes/Annexes}
}
\newcommand\imgpath{/home/victor/acadwriting/Manuscrit/Text/Chapter3/img/} 
%%%%%%%%%%%%%%%%%%%%%%%%%%%%%%%%%%%%%%
%% CHAPTER TITLE
%%%%%%%%%%%%%%%%%%%%%%%%%%%%%%%%%%%%%%

\begin{document}
% \dominitoc
% \faketableofcontents
% \subfileLocal{\setcounter{chapter}{1}}
\chapter{Robust estimators in the presence of uncertainties} 
\label{chap:robust_estimators}

\minitoc
\newpage
\subfileLocal{\pagestyle{contentStyle}}
%%%%%%%%%%%%%%%%%%%%%%%%%%%%%%%%%%%%%%%%%%%%%%%%%%%%%%%
In the previous chapter, we introduced the problem of calibration of a
numerical model with respect to a \emph{calibration parameter}
$\kk$. This takes the form of the optimisation of an objective
function. We also raised the problem of parametric misspecification of
the numerical model with respect to the reality: $\uu \in
\Uspace$. Moreover, this misspecification is modelled by a random
variable $\UU$ with known distribution.  One desirable property is
that the calibrated model shows relatively good performances when the
environmental variables vary, or in other words, we want the
calibrated model to be \emph{robust} with respect to the varying
environmental parameters. In this chapter, we are going to introduce
some criteria that aim at solving this \emph{robust optimisation
  problem}. The actual computation of those estimates will be
discussed in the next chapter.

\section{Defining robustness}
\label{sec:def_robustness}
\subsection{Classifying the uncertainties}
In the Bayesian formulation of the problem, the uncertainty on the
calibration parameter is modelled through the prior distribution,
while the uncertain parameter, $u$ has its own distribution. While
mathematically similar, those two representations actually encompasses
a significant difference: we are actively trying to reduce the
uncertainty of the calibration parameter by Bayesian update, while the
uncertainty on the environmental parameter is seen as a nuisance.

In that context, the very notion of uncertainty can be roughly split
in two, as described in~\cite{walker_defining_2003}:
\begin{itemize}
\item Aleatoric uncertainties, coming from the inherent variability of
  a phenomenon, \emph{e.g.} intrinsic randomness of some environmental
  variables
\item Epistemic uncertainties coming from a lack of knowledge about
  the properties and conditions of the phenomenon underlying the
  behaviour of the system under study
\end{itemize}
According to this distinction, the epistemic uncertainty can be
reduced by investigating the effect of the calibration parameter $\kk$
upon the physical system, and choose it accordingly to an objective
function.  The uncertain variable $\uu$ on the other hand is uncertain
in the aleatoric sense, and cannot be controlled directly, as its
value is destined to change. This is why we model it using a random
variable $\UU$. This distinction, illustrated
in~\cref{fig:sources_uncertainties}, is a bit simplistic,
as~\cite{kiureghian_aleatory_2009} points out that deciding the type
of uncertainties is up to the modeller, who decides on which
parameters inference is worth doing.

\begin{figure}[ht]
  \begin{center}
  \resizebox{\linewidth}{!}
  {
       \usetikzlibrary{calc, arrows, decorations.markings}

\begin{tikzpicture}[decoration = {markings,
    mark = at position 0.5 with {\arrow[>=stealth]{>}}
  }, node distance= 40ex, text centered
  ]
\definecolor{mycolor}{rgb}{1,0.2,0.3}
\fill[gray!10] (-1.8,-.7) -- (1.8,-.7) -- (1.8,2.5) -- (-1.8,2.5);

 \draw  node[rectangle, fill=brewdark] (x) {Physical system} ;
  \draw node[rectangle, fill=brewdark, right of=x] (y) {Mathematical model};
 \draw  node[rectangle, fill=brewdark, right of = y] (z) {Computer code};
  \draw[postaction = decorate, very thick] (x) to node[midway, below, text width=20ex](az) {Simplifications, parametrizations} (y)  ;
  \draw[postaction = decorate, very thick] (y) to node[midway, below, text width=20ex] {Discretization implementation}(z);
\draw node[text width = 20ex, text centered](tune) at (40ex,-15ex) {Error on the parameters};
\draw node[text width = 20ex] (nature) at (0ex, 10ex) {Natural variability};
\draw[->] (nature) to (x);
\draw[->] (tune) to  (55ex,-7ex);
\draw[->] (tune) to  (25ex,-7ex);
\end{tikzpicture}
    }
    \end{center}
  \caption[Sources of uncertainties and errors in the modelling]{\label{fig:sources_uncertainties} Sources of uncertainties and errors in the modelling. The natural variability of the physical system can be seen as aleatoric uncertainties, and the errors on the parameters as epistemic uncertainties}
\end{figure}

\subsection{Robustness and/or reliability}
The notion of \emph{robustness} is dependent on the context in which
it is used. In this work, the term ``robust'' qualifies a model that
behaves still nicely under uncertainties, or to put it in an other
way, that is insensitive up to a certain extent to some perturbations.
Moreover, robustness is often linked and sometimes confused to the
semantically close notion of
\emph{reliability}. In~\cite{lelievre_consideration_2016} we can find
summarized in~\cref{tab:lelievre} the difference between these
notions, by defining optimality as the deterministic counterpart of
robustness, and admissibility as the counterpart of reliability.

\begin{table}[htb]
\centering
\resizebox{\textwidth}{!}{%
\begin{tabular}{@{}llll@{}}
\toprule
                          & No objective                   & Objective with deterministic inputs         & Objective with uncertain inputs                 \\ \midrule
Unconstrained             &                                & Optimal                                     & Robust                 \cellcolor{brewlight}    \\
Deterministic constraints & Admissible                     & Optimal and admissible                      & Robust and admissible     \cellcolor{brewlight} \\
Uncertain constraints     & Reliable \cellcolor{brewlight} & Optimal and reliable  \cellcolor{brewlight} & Robust and reliable  \cellcolor{brewlight}      \\ \bottomrule
\end{tabular}%
}
\caption[Nomenclature of robustness proposed in~\cite{lelievre_consideration_2016}]{Types of problems, depending on their deterministic nature for the constraints or the objective. Shaded cells correspond to problems comprising an uncertain part. Reproduced from~\cite{lelievre_consideration_2016}}
\label{tab:lelievre}
\end{table}
% \begin{table}[htb]
% \centering
% \resizebox{\textwidth}{!}{%
% \begin{tabular}{@{}clllll@{}}
% % \multicolumn{1}{l}{}         & \multicolumn{5}{c}{Robustness}                                                                                                \\ \cmidrule(l){2-6} 
% % \multirow{5}{*}{\rotatebox{90}{Reliability}}
%                                & \multicolumn{2}{l}{\multirow{2}{*}{}}        & \multirow{2}{*}{no objective} & \multicolumn{2}{c}{objective}                  \\ \cmidrule(l){5-6} 
%                              & \multicolumn{2}{l}{}                         &                               & \multicolumn{1}{c}{deterministic inputs}  & uncertain inputs      \\ \cmidrule(l){2-6} 
%                              & \multicolumn{2}{c}{Unconstrained}            & No problem                    & Optimal                & \cellcolor{brewlight} Robust \\ % \cmidrule(l){2-6} 
%                              & \multirow{2}{*}{Constrained} & \multicolumn{1}{r}{deterministic constraints} & Admissible                    &Optimal and admissible & \cellcolor{brewlight} Robust and admissible \\
%                              &                              & \multicolumn{1}{r}{uncertain constraints}     & \cellcolor{brewlight} Reliable &\cellcolor{brewlight} Optimal and reliable   & \cellcolor{brewlight} Robust and reliable   \\ \cmidrule(l){2-6} 
% \end{tabular}%
% }

Other definitions of robustness can be encountered in the literature, and will not be treated in this work: Bayesian approaches are sometimes criticized for their use of subjective probabilities that represent the state of beliefs, especially on the choice of prior distributions. In that sense, robust Bayesian analysis aims at quantifying the sensitivity of the choice of the prior distribution on the resulting inference and relative Bayesian quantities derived. In the statistical community, robustness is often implied as the non-sensitivity on the outliers in the sample set.



\subsection{Robustness under parameteric misspecification}

Given a family of models $\left\{\left(\mathcal{M}(\cdot, \uu), \Theta\right), \uu\in\Uspace\right\}$ and some observations $y\in \Yspace$ sampled from a random variable $Y$, we can derive a problem of parameter estimation for each $\uu \in \Uspace$. As detailed in~\cref{chap:inverse_problem}, we can formulate the likelihood $\mathcal{L}$ and the posterior distribution, and then compute the MLE and the MAP.\@

% More generally, we can assume the existence of an objective function $J$ that takes two distinct inputs where $\kk\in \Kspace$ is the calibration parameter, and $\uu \in \Uspace$ is the uncertain parameter.
% \begin{equation}
%   \label{eq:def_J}
%   \kk, \uu \longmapsto J(\kk,\uu)
% \end{equation}
% This uncertain parameter is modelled by a random variable $\UU$.


Not taking into account the uncertainty on $\uu$ may be an issue in
the modelling, especially if the influence of this variable is
non-negligible.  Choosing a specific $\uu \in \Uspace$ leads to
\emph{localized optimisation} \citep{huyse_free-form_2001} and
\emph{overcalibration}, that is choosing a value $\hat{\kk}$ that is
optimal for the given situation (which is induced by $\uu$). This
value does not carry the optimality to other situations, or in
Layman's term according to~\cite{andreassian_all_2012}, being lured by
``fool's gold''.  In geophysics and especially in hydrological models,
this overcalibration may lead to the appearance of aberrations in the
predictions as those uncertainties become prevalent sources of
errors. In hydrology, uncertainties are the principal culprit of the
existence of ``Hydrological monsters''~\citep{kuczera_there_2010},
that are calibrated models that perform really badly.


There are two main ways to tackle this problem. Since the
environmental parameter is random by nature with a known distribution,
we can introduce it directly in the probabilistic inference framework,
by appending $\uu$ to the calibration parameter and to consider
$(\kk, \uu)$ for the inference. This will be
treated in~\cref{sec:nuisance_parameters}. In this context, the
additional environmental variables are usually called \emph{nuisance
  parameters}.

Another way, that we are calling the \emph{variational} approach, is
to consider instead the objective function
$(\kk, \uu) \mapsto J(\kk, \uu)$ that we want to minimise, as
introduced in the previous chapter. Due to the uncertainty on $\uu$,
we can then study the family of random variables indexed by
$\kk \in \Kspace: \{J(\kk, \UU); \kk\in\Kspace\}$. This will be
addressed in~\cref{sec:J_rv}.

\section{Probabilistic inference}
\label{sec:nuisance_parameters}
In probabilistic inference, the environmental parameters are sometimes
called \emph{nuisance} parameters, and different ways have been
studied to remove their influence.  We will first detail
likelihood-based methods and then the extension to Bayesian framework.
\subsection{Frequentist approach}
From a frequentist approach, we define the joint likelihood
$\mathcal{L}(\theta, u ;y) = p_{Y\mid \KK,\UU}(y \mid \kk, \uu)$.
Under a Gaussian assumption, the sampling distribution,
$Y\mid \KK, \UU$ is
\begin{equation}
Y \mid \KK, \UU \sim \mathcal{N}(\mathcal{M}(\KK, \UU), \Sigma)
\end{equation}
where $\Sigma$ is a covariance matrix.

There are two common ways to get rid of the nuisance parameters: one
by \emph{profiling}, one by \emph{marginalization}.  Profiling implies
to perform first a maximisation of the likelihood with respect to the
nuisance parameters:
\begin{equation}
  \label{eq:def_profile_lik}
  \mathcal{L}_{\mathrm{profile}}(\theta;y) = \max_{\uu \in \Uspace} \mathcal{L}(\theta,u;y)
\end{equation}
and
\begin{equation}
  \estimtxt{\kk}{prMLE} = \argmax_{\kk\in\Kspace} \mathcal{L}_{\mathrm{profile}}(\theta;y)
\end{equation}
In other words, considering the most favorable case of the likelihood
given the nuisance parameters.  Comparing the MLE over
$\Kspace \times \Uspace$ for the original joint likelihood and the
profile MLE on $\Kspace$ for the profile likelihood, it is
straightforward to verify that their components on $\Kspace$ coincide
as
\begin{equation}
  \max_{(\kk ,\uu) \in \Kspace\times \Uspace} \mathcal{L}(\theta,\uu;y) = \max_{\kk\in \Kspace} \mathcal{L}_{\mathrm{profile}}(\theta;y)
\end{equation}

The resulting estimator does not take into account the uncertainty
upon $\uu$, and can perform quite badly when the likelihood presents
sharp ridges~\citep{berger_integrated_1999}.

Another alternative is to define the \emph{integrated}, or \emph{marginalized} likelihood as
\begin{align}
  \mathcal{L}_{\mathrm{integrated}}(\theta;y) &= \int_{\Uspace} \mathcal{L}(\theta,\uu;y) p_{\UU}(\uu) \,\mathrm{d}\uu \label{eq:def_int_lik}\\
                                              &=\int_{\Uspace} p_{Y|\theta,\UU}(y \mid \theta,u) p_{\UU}(\uu) \,\mathrm{d}\uu \\
                                              &=\int_{\Uspace} p_{Y,\UU|\theta}(y,u \mid \theta)\,\mathrm{d}\uu \\
  &= p_{Y \mid \kk}(y \mid \kk)
\end{align}
and by maximising this function,
\begin{equation}
  \estimtxt{\kk}{intMLE} = \argmax_{\kk\in\Kspace}   \mathcal{L}_{\mathrm{integrated}}(\theta;y)
\end{equation}


\begin{example}
  \label{ex:profile_int_lik}
In order to illustrate the difference between those two methods, the profile and integrated likelihood have been computed for the following likelihood:
\begin{align}
  Y \mid \KK, \UU &\sim \mathcal{N}(\kk + \uu^2, 2^2)
\end{align}
and the observations $y = (y_1,\cdots, y_{10})$ have been generated
using $\kk+\uu^2=1$. We set $\Kspace = \interval{-5}{5}$ and
$\Uspace = \interval{-2}{2}$. The likelihood evaluated on
$\Kspace \times \Uspace$ is
displayed~\cref{fig:profile_integrated_lik}, with the integrated and
profile likelihood.  We can see that there is not unicity of the
maximiser for the profile likelihood:
$\mathcal{L}_{\mathrm{profile}}(\kk;y)$ is constant for
$\kk \in \interval{-3}{1}$. This is due to the fact that the observations can
have been generated with any $\kk$ and $\uu$ verifying
$\kk + \uu^2=1$.  For the integrated likelihood however, there is a
unique maximum, attained for $\estimtxt{\kk}{intMLE} \approx 0.8$.
\end{example}

% Choosing a certain way of treating the nuisance parameters leads to different estimates in the end.

\subsection{Bayesian approach}
Similarly as in~\cref{sec:bayesian_inference_MAP}, we can incorporate
information on $\kk$ by introducing a prior distribution $p_{\kk}$,
and we can derive the posterior distribution using Bayes' theorem. We
assume that $\UU$ and $\KK$ are independent:
$p_{\KK, \UU} = p_{\KK}\cdot p_{\UU}$.  The likelihood of the data
given $\kk$ and $\uu$ is
\begin{equation}
  \mathcal{L}(\kk,\uu;y) = p_{Y \mid \KK, \UU}(y\mid \kk,\uu)
\end{equation}
The joint posterior distribution can be written as:
\begin{align}
  p_{\KK,\UU \mid Y}(\kk,\uu \mid y) &= \mathcal{L}(\kk,\uu;y)p_{\KK}(\kk)p_{\UU}(\uu) \frac{1}{p_Y(y)} \\
  &\propto \mathcal{L}(\kk,\uu;y)p_{\KK}(\kk)p_{\UU}(\uu)
\end{align}
Here, the posterior is used to do inference on $\kk$ and $\uu$
jointly. In order to suppress the dependency in $u$, we integrate with
respect to $\UU$ and get the marginalized posterior
$p_{\theta \mid Y}$:
\begin{align}
  p_{\KK \mid Y}(\kk \mid y) &= \int_{\Uspace} p_{\KK,\UU\mid Y}(\kk,\uu\mid y)\,\mathrm{d}u \label{eq:marg_MMAP}\\
                             &= \int_{\Uspace}p_{\KK \mid Y, \UU}(\kk \mid y, \uu) p_{\UU\mid Y}(\uu \mid y)\,\mathrm{d}u
\end{align}

We can then define the \emph{marginalized maximum a posteriori}
(MMAP)~\citep{doucet_marginal_2002} as the maximiser of this
marginalized posterior:
\begin{equation}
  \label{eq:def_MMAP}
  \estimtxt{\kk}{MMAP} = \argmax_{\kk\in\Kspace} p_{\KK \mid Y}(\kk\mid y)
\end{equation}
or, by taking the negative logarithm to get a minimisation problem,
another writing is
\begin{equation}
\estimtxt{\kk}{MMAP} = \argmin_{\kk\in\Kspace} -\log  p_{\KK\mid Y}(\kk\mid y)
\end{equation}
Unfortunately, neither the integration with respect to the nuisance
parameter in~\eqref{eq:marg_MMAP} nor the subsequent optimisation is
analytically easy.  Assuming that we are able to get i.i.d.\ samples
$\{(\kk_i, \uu_i)\}_{1\leq i \leq n_{\mathrm{samples}}}$ from the
posterior distribution using MCMC methods for instance, by discarding
the $\uu$ components, the samples
$\{\kk_i\}_{1 \leq i\leq n_{\mathrm{samples}}}$ are distributed
according to the marginal posterior $p_{\KK \mid Y}$, and thus can be
used to get the MMAP. More direct techniques, such
as~\cite{doucet_marginal_2002}, introduce methods in order to estimate
iteratively the MMAP, through sampling of the joint posterior.
\begin{example}
  Using the same data as in~\cref{ex:profile_int_lik}, we add a prior
  distribution of $\KK$ as a centered normal distribution, truncated
  on $\Kspace$. On~\cref{fig:profile_integrated_lik} we can see the
  influence of the prior distribution, as it nudges the MMAP
  $\estimtxt{\kk}{MMAP}$ toward $0$, compared to the integrated
  likelihood.
\end{example}
\begin{figure}[ht]
  \centering
  % \input{\imgpath profile_integrated_lik.pgf}
  \includegraphics{\imgpath profile_integrated_lik.png}
  \caption[Joint likelihood and posterior
  distribution]{\label{fig:profile_integrated_lik} Joint likelihood
    and posterior (left). Profile and integrated likelihood for an
    uniform nuisance parameter and marginal posterior distribution
    (right)}
\end{figure}
Those three estimators appear quite naturally in the probabilistic
formulations. The main difference between the MMAP and the integrated
likelihood is the presence of the prior distribution of $\kk$ in the
formulation, thus similarly as in the inference problem of the
previous chapter (without nuisance parameter), we can incorporate
information on the calibration parameter. Regarding the profile
likelihood however, this estimation relies on the optimisation with
respect to $\uu$, and thus does not really take into account its
random nature.

\clearpage
\section{Variational approach}
\label{sec:J_rv}
We discussed so far the calibration problem with nuisance parameters
in the formulation of the likelihood or the posterior
distribution. However, in data assimilation for instance, problems of
parameter estimation are often formulated directly by introducing a
cost function:
\begin{equation}
  \begin{array}{rrcl}
    J: &\Kspace \times \Uspace &\longrightarrow& \mathbb{R}^+ \\
    &(\kk, \uu) &\longmapsto &J(\kk, \uu)
    \end{array}
  \end{equation}
  
  This function in a calibration context is measuring the misfit
  between the data $y$ and the forward operator, and can be written as
  the negative log-likelihood, or the negative log-posterior
  distribution. Still, the objectives and criteria introduced in the
  following are not specific to this context, and $J$ can represent
  other properties that ought to be reduced such as a loss or some
  unwanted physical properties such as the drag in airfoil design
  optimisation. This general problem is sometimes quoted as
  \emph{Optimisation under uncertainties} (OUU)
  \citep{cook_effective_2018,seshadri_density-matching_2014}


All in all, $J(\kk, \uu)$ represents the cost of taking the decision
$\kk\in\Kspace$ when the environmental variable is equal to $\uu$.  We
are going to make several assumptions:
\begin{itemize}
\item $\Kspace$ is convex and bounded 
\item For all $\kk \in \Kspace$ and $\uu \in \Uspace$, $J(\kk, \uu)>0$
\item For all $\kk \in \Kspace$, $J(\kk, \cdot)$ is measurable
\item For all $\kk \in \Kspace$, $J(\kk, \UU) \in L^s(\Prob_{\UU})$
  and $s \geq 2$. So for each $\kk$, mean and variance exist and are
  finite.
\end{itemize}

As the function represents a cost, \emph{i.e.} an undesirable
property, we are interested in minimising in some sense this random
variable, which depends on $\kk$.  Most of existing methods rely on
the definition of a \emph{robust} counterpart of the minimisation
problem, which implies to remove the dependence on the uncertain
variable $\UU$. This counterpart being a deterministic optimisation
problem, we can solve it using classical methods of minimisation.

\subsection{Decision under deterministic uncertainty set}
We will first introduce some estimators that can be argued robust, even
though the random nature of $\UU$ is not directly taken into account.
The uncertainty is modelled here by assuming that no information is
available on $\uu$, except that $\uu \in \Uspace$. In this paradigm,
$\Uspace$ is called the uncertainty set~\citep{bertsimas_theory_2010}.

\subsubsection{Global optimisation}
A global optimisation criterion, as its name suggests, advocates for minimising the cost function over the whole space $\Kspace \times \Uspace$, giving this optimisation problem:
\begin{equation}
  \min_{(\kk,\uu) \in \Kspace \times \Uspace} J(\kk, \uu)
\end{equation}
Rearranging slightly this problem, the $\theta$-component of the minimiser can be written as
\begin{equation}
  \label{eq:kkglobal}
  \estimtxt{\kk}{global} = \argmin_{\kk \in \Kspace} \min_{\uu \in \Uspace} J(\kk, \uu)
\end{equation}

The global minimum is the equivalent of profile likelihood
maximisation defined~\cref{eq:def_profile_lik}, when $J$ is the negative
log-likelihood. This method exhibits some flaws: we are optimising the
cost function only over the most favourable cases of the environmental
parameter, thus there is no guarantee on the behaviour of $J$ outside
of those optimistic situations.  It then makes sense to ``separate''
$\kk$ and $\uu$ in the optimisation.

\subsubsection{Worst-case optimisation}
\label{sec:saddle_point}
As global optimisation is inhenrently optimistic, we can easily derive
a criterion which is pessimistic in the sense that we want to minimise
over the \emph{least favourable} cases, thus minimising the objective
in the worst-case scenarios. The optimisation problem in this case
becomes
\begin{equation}
  \min_{\kk\in\Kspace} \max_{\uu \in \Uspace} J(\kk,\uu)
\end{equation}
This criterion is sometimes called Wald's Minimax criterion~\citep{wald_statistical_1945}, and the associated estimator is
\begin{equation}
  \label{eq:kkwc}
  \estimtxt{\kk}{WC} =  \argmin_{\kk \in \Kspace} \max_{\uu \in \Uspace} J(\kk, \uu)
\end{equation}

Minimising in the worst-case sense also possesses some flaws,
especially from a computational point of view.  First, the maximum on
$\Uspace$ may not exist, especially if $\Uspace$ is unbounded: we
could make the model perform as badly as possible by taking extreme
values of $\uu$.  Additionally, if it exists, the resulting estimator
is most likely very conservative as only the worse cases are
considered.

\subsubsection{Regret maximin}
\label{ssec:regret_savage}
One other approach, called Savage's maximin
regret~\citep{savage_theory_1951} is to compare the current objective
to the best performance given the uncertain variable $\uu$. The
translated objective is called the \emph{regret} and is defined as
\begin{equation}
  \label{eq:def_regret_savage}
  r(\kk, \uu) = J(\kk, \uu) - \min_{\kk \in \Kspace} J(\kk, \uu)
\end{equation}
Using the regret as the new objective function, we can optimise it in the worst-case sense, as introduced in~\cref{sec:saddle_point}, and the minimum is attained at $\estimtxt{\kk}{rWC}$:
\begin{equation}
  \estimtxt{\kk}{rWC} = \argmin_{\kk\in\Kspace} \max_{\uu \in \Uspace} r(\kk, \uu)
\end{equation}

\begin{example}
\Cref{fig:decision_under_uncertainty} shows global, worst-case and regret optimisation for the analytical cost function
\begin{align}
  \label{eq:decision_under_unc}
  J(\kk, \uu) &= \left(1 + \uu(\kk + 0.1)^2\right)\left(1 + (\kk - \uu)^2\right)
\end{align}
We can see how the worst-case minimisation (in blue) and Savage's
maximin regret (in green) compare in this example. Maximin regret will
favour values of $\kk$ giving an objective that is never too far from
the optimal value available, in contrast to the worst-case that
focuses on the absolute objective.
\end{example}

\begin{figure}[ht]
  \centering
  \input{\imgpath decision_under_uncertainty.pgf}
  \caption[Robust optimisation under uncertainty
  set]{\label{fig:decision_under_uncertainty} Illustration of global
    optimisation, worst-case, and regret worst-case. The red points on
    the contour of the cost function correspond to the minimisers of
    $J$ at each $\uu$ fixed}
\end{figure}

So far, we did not use the fact that $\uu$ was a realisation of a
random variable, and did not take advantage of the knowledge we have
upon it. In the next sections, we will see how to incorporate the
knowledge of the distribution of $\UU$ in the estimations.

\subsection{Robustness based on the moments of an objective function}
In the presence of uncertainties, choosing a parameter value $\kk$ can
also be seen as making a choice under risk. Let
$J:\Kspace \times \Uspace\rightarrow \mathbb{R}^+$ be an objective
function, and assume that for all $\kk \in \Kspace$, $J(\kk, \cdot)$
is a measurable function.  $J$ can be seen as the opposite of the
\emph{utility} function, often encountered in game theory or
econometrics.  Because of the random nature of $\UU$, we can define a
family of real random variables $\{J(\kk,\UU) \mid \kk \in \Kspace\}$,
indexed by $\kk \in \Kspace$.  In~\cite{beyer_robust_2007}, the
authors define an \emph{aggregation approach}, based on the
integration with respect to the uncertain variable, in order to get an
\emph{aggregated objective}, which is a deterministic function that
depends only on $\kk$.  An example of this aggregation is the
integration of the successive powers of the cost function, in order to
get the moments of the associated random variable, that we will detail
in~\cref{sec:exp_loss_minimization,sec:multiobjective_optimization,sec:higher_moments}.
The aggregated objective is then minimised with respect to the control
variable.
\subsubsection{Expected loss minimisation, central tendency}
\label{sec:exp_loss_minimization}
One of the simplest approach when facing such a problem is to look to
optimise a central tendency of those random variables. The mean value
being an obvious candidate, we define the expected objective (or
expected loss) as
\begin{equation}
  \label{eq:mean_objective}
  \mu(\kk) = \Ex_\UU\left[J(\kk,\UU)\right] =\int_{\Uspace} J(\kk,\uu)p_\UU(\uu)\,\mathrm{d}\uu
\end{equation}
The expected objective $\mu(\kk)$ is sometimes called the conditional
mean given $\kk$. Taking the expectation of the objective function is
very common in many problems of classification and
regression~\citep{bishop_pattern_2006}.


The conditional mean is minimised, giving
$\estimtxt{\kk}{mean}$. Assuming that
$J(\kk, \uu) \propto - \log \mathcal{L}(\kk,\uu;y)$, we have
\begin{align}
  \estimtxt{\kk}{mean} = \argmin_{\kk\in\Kspace} \mu(\theta)&= \argmin_{\kk\in\Kspace}\int_\Uspace J(\kk,\uu) p_{\UU}(\uu)\,\mathrm{d}\uu \\
                                                            &= \argmin_{\kk\in\Kspace} - \int_\Uspace \log \mathcal{L}(\kk,\uu;y) p_{\UU}(\uu)\,\mathrm{d}\uu \\
                                                            &= \argmin_{\kk\in\Kspace} - \int_\Uspace\log\left(p_{Y \mid \KK,\UU}(y \mid \kk,\uu)\right)p_{\UU}(\uu)\,\mathrm{d}\uu 
\end{align}

Taking the average of an objective function is the basis of
\emph{stochastic programming}.  However, the integral of
\cref{eq:mean_objective} is often intractable analytically, so instead
of computing it exactly, one usually resorts to minimising the
empirical mean risk. For $1\leq i \leq n_U$, let $\uu_i$ be i.i.d.\
samples from $\UU$. We can then use those samples to approximate
$\mu$: the empirical mean is
\begin{equation}
  \label{eq:emp_mean_objective}
  \mu^{\mathrm{emp}}(\kk) = \frac{1}{n_U}\sum_{i=1}^{n_U} J(\kk, \uu_i)
\end{equation}
and the minimisation problem defined as
\begin{equation}
  \min_{\kk\in\Kspace} \frac{1}{n_U}\sum_{i=1}^{n_U} J(\kk, \uu_i)
\end{equation}
is called the \emph{sample average
  problem}~\citep{juditsky_stochastic_2009}, or \emph{empirical risk
  minimisation} problem in Machine Learning (see
e.g.~\cite{vapnik_principles_1992}) Other indicators of central
tendency can be considered for optimisation, such as the mode or the
median of the cost function.

Despite some similarities with the integrated likelihood introduced
\cref{eq:def_int_lik}, $\estimtxt{\kk}{mean}$ and
$\estimtxt{\kk}{intMLE}$ are not equal in general, as shown
\cref{fig:difference_arithmetic_geometric_mean} for the likelihood
introduced in~\cref{ex:profile_int_lik} and
\cref{fig:profile_integrated_lik}.

\begin{figure}[ht]
  \centering
  \input{\imgpath integrated_lik_average_costfunction.pgf}
  \caption[Difference between integrated likelihood and mean
  loss]{\label{fig:difference_arithmetic_geometric_mean} Difference
    between the negative logarithm of the integrated likelihood
    defined in~\cref{eq:def_int_lik}, and the mean loss of
    $J = -\log \mathcal{L}$ defined in~\cref{eq:mean_objective} and the
    subsequent difference in estimators}
\end{figure}



A low expected value is to be taken with caution, as it refers to a
behaviour \emph{in the long run}. Indeed, the mean value is equivalent
to averaging over all the outcomes, but there can be a compensation
effect, where ``good surprises'' balance the ``bad surprises''.  An
example is the following problem:
\begin{align}
  J(\kk_1, \UU) &\sim \mathcal{N}(2, 2^2) \\
  J(\kk_2, \UU) & \sim \mathcal{N}(3, 1^2)
\end{align}
and we have to choose either $\estimtxt{\kk}{}=\kk_1$ or
$\estimtxt{\kk}{} = \kk_2$.  It is clear that
$\Ex_{\UU}[J(\kk_1, \UU)] < \Ex_{\UU}[J(\kk_2, \UU)]$. However, making
the decision $\estimtxt{\kk}{} = \kk_2$ leads to less extreme values:
\begin{align}
  \Prob_{U}[J(\kk_1, \UU) > 5] = 0.06681> \Prob_{U}[J(\kk_2, \UU) > 5] = 0.02275
\end{align}
Depending on the application, such a behaviour could be prohibitive.
The difference in these probabilities is explained by the difference
in the variance of the random variable $J(\kk, \UU)$.  Accounting for
the variance in the objective function is discussed
in~\cref{sec:multiobjective_optimization}.

\subsubsection{Variance optimisation}
In~\cref{sec:exp_loss_minimization}, we used the mean as a measure of
the central tendency that we want to minimise. Jointly with the
central tendency, information about the dispersion of the random
variable may also be relevant, in order to predict how much deviation
should be expected around the mean.  Let us define the variance of the
objective function:
\begin{equation}
  \sigma^2(\kk) = \Var\left[J(\kk,\UU)\right]
\end{equation}
and minimising this variance yields
\begin{equation}
  \estimtxt{\kk}{var} = \min_{\kk \in \Kspace} \sigma^2(\kk) =  \min_{\kk \in \Kspace} \sigma(\kk)
\end{equation}
Depending on the application, the equivalent formulation using the
standard deviation may be used instead of the variance. In both
formulations, such a computation requires the evaluation
of an expensive integral, this problem can be tackled using sample
averaging, and the minimisation problem becomes
\begin{equation}
  \min_{\kk\in \Kspace} \frac{1}{n_{\UU}-1} \sum_{i=1}^{n_{\UU}} \left(J(\kk, \uu_i) - \mu^{\mathrm{emp}}(\kk)\right)^2
\end{equation}

\Cref{fig:mean_std} shows the conditional mean and conditional
standard deviation for the objective function $J$
defined~\cref{eq:decision_under_unc}.
\begin{figure}[ht]
  \centering
  \input{\imgpath mean_std_wc.pgf}
  \caption[Conditional mean and standard deviation]{Illustration of
    conditional mean and conditional standard deviation, as a function
    of $\kk$. Those quantities have been rescaled to share the same
    range on the right plot.}
  \label{fig:mean_std} 
\end{figure}

\subsubsection{Multiobjective optimisation}
\label{sec:multiobjective_optimization}
Minimising the variance is often irrelevant without additional
constraints, as it could just point toward really high values of the
objective function, but steady with respect to $\kk$. Taking both
objectives: low mean value and low variance together to the following
multiobjective optimisation problem:
\begin{align}
  \label{eq:multiobj_e_var}
  \min_{\kk\in\Kspace} \left(\mu(\kk),\sigma(\kk)\right)
\end{align}


This problem can be tackled in different ways using multiobjective optimisation.
% The literature is rich in methods to approach or even find the Pareto frontier.
To compare $\kk_1$ and $\kk_2$, we can compare component-wise the
objective vectors $(\mu(\kk_i),\sigma(\kk_i))$ for $i=1,2$. If
$\mu(\kk_1) \leq \mu(\kk_2)$ and $\sigma(\kk_1) \leq \sigma(\kk_2)$,
$\kk_2$ is said to be \emph{dominated} by $\kk_1$. The Pareto frontier
is defined as the set of points in $\Kspace$ that cannot be dominated
by any other points. For points on this front, one cannot decrease
further one of the objective without increasing the other.
On~\cref{fig:pareto} is illustrated the Pareto frontier for the
multiobjective problem of~\cref{eq:multiobj_e_var}. The red point
corresponding to $\kk_1$ is dominated by the green point $\kk_0$ on
the frontier, but not by the green point of $\kk_2$. A solution of the
multiobjective problem can then be chosen within the Pareto frontier.

\begin{figure}[ht]
  \centering
  \input{\imgpath pareto_frontier.pgf}
  \caption[Pareto frontier]{\label{fig:pareto} 
 Illustration of the Pareto frontier for the multiobjective problem of~\cref{eq:multiobj_e_var}. The shaded regions corresponds to the domain dominated by each points}
\end{figure}

Instead of finding the Pareto frontier, the multiobjective problem is
often ``scalarized'' by summing the weighted
objectives~\citep{marler_weighted_2010}, provided that such an
operation makes sense with regards to the units of the quantities,
justifying the use of the standard deviation instead of the variance.
\begin{equation}
  \min_{\kk\in\Kspace} \lambda \mu(\kk) + (1- \lambda)\sigma(\kk) =   \min_{\kk\in\Kspace} \lambda \Ex_{\UU}[J(\kk,\UU)] + (1- \lambda)\sqrt{\Var[J(\kk,\UU)]}
\end{equation}
where $\lambda \in [0,1]$ is chosen to reflect the preference toward one or another objective.

\subsubsection{Higher moments in optimisation}
\label{sec:higher_moments}
Higher moments can also be considered as additional criteria,
especially in Portfolio optimisation
\citep{lai_mean-variance-skewness-kurtosis-based_2006,briec_mean-variance-skewness_2007}.
For a real random variable $X \in L^3$ (with respect to Lebesgue's
measure on $\mathbb{R}$), the skewness coefficient measures the
asymmetry in the distribution, and is the (normalized) centered moment
of order $3$:
\begin{equation}
  \mathrm{sk}\left[X\right] = \Ex\left[\left(\frac{X - \mu}{\sigma}\right)^3\right]
\end{equation}
where $\mu = \Ex[X]$ and $\sigma = \sqrt{\Var[X]}$.

Adding the skewness in the optimisation translates to a preference
toward a risk-averse or a risk-seeking approach. Indeed, as the main
goal is the optimisation of an objective function, deviations of the
value of the random variable toward lower values is more desirable
than deviations toward larger values.

This is illustrated \cref{fig:skewness_example}: all three of the
random variables displayed have the same mean and variance.  If the
skewness coefficient is negative, the distribution presents a heavier
left tail than right. In other words, a sample taken from this
distribution has a higher probability of being a ``good surprise''. On
the other hand, if a big deviation occurs for a sample from a
right-skewed distribution, it is more probable to be a large deviation
toward large values of the sample space, hence the term ``bad
surprise''.

\begin{figure}[ht]
  \centering
  \input{\imgpath skewness_examples.pgf}
  \caption[Influence of the skewness]{\label{fig:skewness_example} Pdf and cdf of random variables with same mean, variance but different skewness}
\end{figure}

% In order to have a finer tuning on the ``risk-averse'' or
% ``risk-seeking'' properties of the wanted solution, some authors
% propose to directly minimise with respect to $\kk$ the difference
% between the cdf of the r.v. $J(\kk, \UU)$ and a target cdf, giving
% \emph{Horsetail
%   matching}~\citep{cook_extending_2017,cook_effective_2018}.

Other extensions have been developed around the cdf of the random
variable, especially in portfolio optimisation. Indeed, integrating
and comparing the cdf allows to introduce a \emph{domination order}
between random variables. These concepts of Stochastic
Dominance~\citep{ogryczak_stochastic_1997} are then used to take
decisions under uncertainties.


\section{Regret-based families of estimators}
\label{sec:rr_family}
All the methods introduced above required first to eliminate in some
sense the dependency on the environmental parameter, in order to
transform the random variable $J(\kk, \UU)$ into an objective that
depends solely on $\kk$, and to optimise this deterministic
counterpart.  For a given $\kk\in\Kspace$, this elimination is done by
aggregating all the possible outcomes $J(\kk, \uu)$ when $\uu$ is a
sample of $\UU$.


We propose now to reverse these steps, by first optimising the
objective function with respect to $\kk$, and, from the set of
minimisers that depend on $\uu$ obtained, derive an estimator. The
rationale behind this permutation is that every situation induced by a
realisation $\uu$ is to be taken separately, quite similarly as
Savage's regret introduced \cref{ssec:regret_savage}.  In turn, this
avoids aggregation (and in a sense compensation) between the different
$\uu$.

The work detailed in this section is largely based
on~\cite{trappler_robust_2020}.

% \begin{figure}[ht]
%   \centering
%   \input{\imgpath reversing.pgf}
%   \caption{\label{fig:reversing_steps} Principle of regret based estimators: the random nature of $\UU$ is kept after}
% \end{figure}


\subsection{Conditional minimum and minimiser}
\label{sec:MPE}
We assume that $\UU$ is a continuous random variable, with a compact
support.
\begin{definition}[Conditional minimum, minimiser]
  Let $J: \Kspace \times \Uspace$ be an objective function, and let us
  assume that for each $\uu \in \Uspace$,
  $\min_{\kk \in \Kspace} J(\kk,\uu)$ exists and is attained at a
  unique point.  We denote
  \begin{equation}
    J^*(\uu) = \min_{\kk \in \Kspace} J(\kk,\uu)
  \end{equation}
  the \emph{conditional minimum} of $J$ given $u$, and
  \begin{equation}
    \label{eq:def_kstar}
    \kk^*(\uu) = \argmin_{\kk\in\Kspace} J(\kk, \uu)
  \end{equation}
 is defined as the \emph{conditional minimiser}
\end{definition}
As $\uu$ is thought to be a realization of a random variable $\UU$, we
can consider the two random variables $\kk^*(\UU)$ and $J^*(\UU)$.
The conditional minimum $J^*(\UU)$ is then a random variable
describing the best performances of the calibration, if we could
optimise the objective function for each realization of $\UU$.

Similarly, let us assume that the conditional minimiser is well
defined for all $\uu\in \Uspace$. We can study the image of the random
variable through this mapping, that we will denote $\kk^*(\UU)$.  This
random variable in itself gives already information on the
``identifiability'' of a robust estimate, depending on the information
carried by its distribution.
\begin{example}
Let $\Kspace = \Uspace= [0, 1]$, and $\UU \sim \mathrm{Unif}(\Uspace)$, and the following objective functions:
\begin{align}
  J_1(\kk, \uu) &= (1+\uu)+(\kk - 0.5)^2\\
  J_2(\kk, \uu) &= (\kk-\uu)^2 + 1
\end{align}
We have
\begin{align}
  \kk^*_1(\uu) &= \argmin_{\kk \in \Kspace} J_1(\kk, \uu) = 0.5 \\
  \kk^*_2(\uu) &= \argmin_{\kk\in\Kspace} J_2(\kk, \uu) = \uu
\end{align}

In the first case, $\kk_0 = \argmin_{\kk} J(\kk, \UU)$, so
$\kk_1^*(\UU)$ is a degenerate random variable almost surely equals to
$\kk_0$. In other words, the minimiser is not dependent on the value
taken by the environmental parameter. The minimal value attained $J^*$
might be dependent though. On the other hand, for $J_2$, as
$\Kspace=\Uspace$ and $\UU \sim \mathrm{Unif}(\Uspace)$,
$\KK_2^*(\UU)$ is uniformly distributed on $\Kspace$, no value shows a
better affinity of being a minimiser than the other.
\end{example}

In general, this random variable cannot be classified as continuous or
discrete without further study. However, in the following, we are
going to assume that it is a \emph{continuous random variable}.  The
entropy of the random variable $\kk^*(\UU)$
(see~\cref{def:KL_entropy}) can be seen as a measure of the
sensitivity of the calibration when the environmental variable
varies. If the support of the random variable $\kk^*(\UU)$ is bounded,
the distribution with the highest entropy on this support is a uniform
distribution.  Per the continuity assumption, this entropy can be
estimated by various methods (see for
instance~\cite{beirlant_nonparametric_1997}).


This distribution of the minimisers and its entropy can be used for
global optimisation, as outlined
in~\cite{hennig_entropy_2011}. Furthermore, the authors provide an
analytical expression of the pdf of the minimisers, and the nature of
the infinite product is discussed:
\begin{equation}
  p_{\KK^*}(\kk) = \int_{\Uspace} p_{\UU}(\uu)\prod_{\substack{\tilde{\kk}\in \Kspace\\\tilde{\kk}\neq \kk}} \mathbbm{1}_{\{J(\tilde{\kk},\uu) > J(\kk, \uu)\}}\,\mathrm{d}\uu
\end{equation}

However, except for simple analytical problems, the pdf $p_{\KK^*}$
cannot be obtained analytically, and needs to be estimated. This can
be done by different methods, depending on the assumptions we can make
upon $\kk^*(\UU)$. Let $\{\uu_i\}_{1\leq i \leq n_{\UU}}$ be $n_{\UU}$
i.i.d.\ samples of $\UU$, and
$\{\kk^*(\uu_i)\}_{1\leq i \leq n_{\UU}}$ the corresponding
minimisers, as defined \cref{eq:def_kstar}. Among the methods of
density estimation, one of the easiest to implement and widespread
method is \emph{Kernel Density Estimation} (KDE).  Given the samples
$\uu_i$ and the minimisers $\kk^*(\uu_i)$ for $1\leq i \leq n_{\UU}$,
the isotropic KDE is given by
\begin{equation}
  \hat{p}_{\KK^*}(\kk^*) = \frac{1}{n_{\UU} h^{\dim \Kspace}} \sum_{i=1}^{n_{\UU}} \mathcal{K}\left(\frac{\kk^* - \kk^*(\uu_i)}{h}\right)
\end{equation}
where $h>0$ is the bandwidth (that measures the influence of each
sample), and $\mathcal{K}$ is a kernel of dimension $p=\dim \Kspace$,
usually defined as the product of one-dimensional kernels
$\mathcal{K}_{1\mathrm{D}}$:
$\mathcal{K}(\kk) = \prod_{j=1}^{\dim
  \Kspace}\mathcal{K}_{1\mathrm{D}}(\kk_j)$. Several choices of 1D
kernels are available, and one of the most common one is the Gaussian
Kernel:
$\mathcal{K}_{1\mathrm{D}}(\kk_j) = (2\pi)^{-1/2}\exp(-\kk_j^2
/2)$. \Cref{fig:theta_star_samples} shows the estimated density
$\hat{p}_{\kk^*}$ using KDE and Scott's rule for the
bandwidth~\citep{scott_optimal_1979}, along with the histogram of the
minimisers.


\begin{figure}[ht]
  \centering
  \input{\imgpath theta_star_samples.pgf}
  \caption{\label{fig:theta_star_samples} Density estimation of the minimisers of $J$}
\end{figure}

Finally, when we have an estimation of the density of $\KK^*$, and if
it exists, we can compute its mode, that we are going to call the
\emph{Most Probable estimator}:
\begin{equation}
  \estimtxt{\kk}{MPE} = \argmax_{\kk \in \Kspace} p_{\KK^*}(\kk)
\end{equation}
This mode can be sought directly using appropriate algorithms, such as
the Mean-shift algorithm~\citep{yizong_cheng_mean_1995}, based on the
KDE, or clustering methods, such as the Expectation-Maximisation
algorithm introduced in~\cite{dempster_maximum_1977}.

Choosing $\estimtxt{\kk}{MPE}$ means to select the value that is
``most often'' the minimiser of the objective function. However, we
have no indication on its performances when it is \emph{not} optimal,
and how often this non-optimality happens.  In \cref{sec:regret}, we
are going to introduce the notions of regret (additive and relative),
in order to try to be ``almost optimal'' with high probability.


\subsection{Regret and model selection}
\label{sec:regret}
In this section, we will first focus on objective functions defined as
the negative log-likelihood in order to link the additive regret and
the likelihood ratio test.
\subsubsection{Objective as the negative log-likelihood}

Let the objective function be the negative log-likelihood:
\begin{equation}
  J(\kk, \uu) = - \log p_{Y\mid \KK, \UU}(y \mid \kk, \uu) = -\log \mathcal{L}(\kk,\uu)
\end{equation}

We can then link the notion of Wald's regret introduced earlier
in~\cref{eq:def_regret_savage} to the likelihood-ratio test:
  \begin{align}
    r(\kk,\uu_0) &= J(\kk, \uu_0) - \min_{\kk^{\prime}\in\Kspace}J(\kk^{\prime}, \uu_0) = J(\kk,\uu_0) - J^*(\uu_0)  \\
                 &= \max_{\kk^{\prime}\in\Kspace} \log \mathcal{L}(\kk^{\prime}, \uu_0) - \log \mathcal{L}(\kk, \uu_0) \\
                 &= -\log \frac{\mathcal{L}(\kk, \uu_0)}{\max_{\kk^{\prime}\in\Kspace}\mathcal{L}(\kk^{\prime}, \uu_0)} = -\log \frac{\max_{\kk^{\prime}\in\{\kk\}}\mathcal{L}(\kk^{\prime}, \uu_0)}{\max_{\kk^{\prime}\in\Kspace}\mathcal{L}(\kk^{\prime}, \uu_0)}
  \end{align}

  
  As the model $(\mathcal{M}(\cdot, \uu_0), \Kspace)$ is misspecified,
  we can apply the misspecified likelihood-ratio test, and for a
  candidate $\kk\in \Kspace$ we can test for the following hypotheses:
 \begin{itemize}
 \item $\mathcal{H}_0$: the model $\left(\mathcal{M}(\cdot, \uu_0), \{\kk\}\right)$ is statistically equivalent to $\left\{\mathcal{M}(\cdot, \uu_0), \Kspace\right\}$
 \item $\mathcal{H}_1$: the models are statistically different
 \end{itemize}
 The statistic of the test, \emph{i.e.} the regret $r$ is to be
 compared with half the quantile of the r.v.\ $X(\uu_0)$ defined as
  \begin{equation}
    \label{eq:sumchi2}
X(\uu_0)=\sum_{i=1}^{\dim \Kspace} c_i(\uu_0) \Xi_i \quad \text{ with } \Xi_i \sim \chi^2_1 \text{ i.i.d.}
\end{equation}
where $\{c_i(\uu_0)\}_{1\leq i \leq \dim \Kspace}$ are coefficients
linked to the eigenvalues of the Fisher information matrix as evoked
in~\cref{sec:model_misspecification}.
 
 The null hypothesis $\mathcal{H}_0$ is rejected at a level
 $\eta \in ]0;1[$ if
  \begin{align}
    r(\kk, \uu_0)=J(\kk, \uu_0) - J^*(\uu_0) > \beta 
  \end{align}
  Where $\beta$ is half the $1-\eta$ quantile of the random variable $X(\uu_0)$ defined~\eqref{eq:sumchi2}.
  
 Using this rejection region, we can construct a likelihood interval (as defined~\cref{eq:def_lik_interval}), which depends on $\uu_0$:
  \begin{equation}
    \label{eq:lik_interval_add}
    \mathcal{I}_{\mathrm{Lik}}(\uu_0; \beta) = \left\{\kk\in \Kspace \mid  J(\kk, \uu_0) - J^*(\uu_0) \leq \beta\right\}
  \end{equation}
  So, for $\kk \in \mathcal{I}_{\mathrm{Lik}}(\uu_0;\beta)$, the model
  $(\mathcal{M}(\cdot, \uu_0),\{\kk\})$ is acceptable at the
  $\eta$-level, per the likelihood-ratio test.

  
  From a computational point of view, the coefficients
  $\{c_i(\uu_0)\}$ are hard to obtain, and depend on $\uu_0$. A first
  approximation would be to suppose that
  $X(\uu_0) \sim \chi^2_{\dim \Kspace}$, \emph{i.e.} to apply the
  ``well-specified'' likelihood-ratio test. In the more general case,
  we can choose $\beta>0$ in a more arbitrary way in order to avoid
  the computations of the coefficients $\{c_i(\uu_0)\}_i$, as we are
  going to see~\cref{ssec:general_cost_prob}.
  
  
  \subsubsection{Interval and probability of acceptability}
     \label{ssec:general_cost_prob}
     We assumed before that $J$ was the negative log-likelihood. In
     the more general case, $J$ represents a loss function, that we
     want to minimise. The generalization of
     \cref{eq:lik_interval_add} is what we are calling the
     \emph{interval of acceptability}.
  \begin{definition}[Interval of acceptability]
    By analogy with the likelihood interval
    defined~\cref{eq:def_lik_interval} and \cref{eq:lik_interval_add},
    we can construct a set for an arbitrary threshold $\beta \geq 0$
    such that
  \begin{equation}
    \mathcal{I}_{\beta}(\uu) = \left\{\kk \in \KK \mid J(\kk,\uu) \leq J^*(\uu) + \beta \right\}
  \end{equation}
  As $J$ may not stem from a likelihood, we call
  $\mathcal{I}_{\beta}(\uu)$ the \emph{interval of acceptability}. In
  other words, we say that $\kk\in \Kspace$ is $\beta$-acceptable for
  $\UU=\uu$ if $\kk \in \mathcal{I}_{\beta}(\uu)$.
\end{definition}

\Cref{fig:lik_interval_threshold} shows an objective function
evaluated for different fixed $u_i$. The $\beta$-acceptable intervals
for those environmental variables are plotted below the curves.
\begin{figure}[ht]
  \centering
  \input{\imgpath lik_interval_threshold.pgf}
  \caption{\label{fig:lik_interval_threshold} Different acceptable regions corresponding to different $\uu \in \Uspace$}
\end{figure}

Now, for a given $\kk\in\Kspace$, we can define the set of
$\uu\in \Uspace$ such that $\kk \in \mathcal{I}_{\beta}(\uu)$, i.e.\
  \begin{align}
    R_{\beta}(\kk) &= \left\{\uu\in \Uspace \mid \kk \in \mathcal{I}_{\beta}(\uu) \right\} \\
           &= \left\{\uu\in \Uspace \mid J(\kk, \uu)  \leq J^*(\uu) + \beta \right\}
  \end{align}
  This set is measurable, and by measuring this subset of $\Uspace$
  with respect to the distribution of $\UU$, we get
  \begin{equation}
    \label{eq:def_gamma_additive}
    \Gamma_\beta(\kk) = \Prob_U\left[R_{\beta}(\kk)\right]
  \end{equation}
  Loosely speaking, for a given $\kk$, $\Gamma_{\beta}(\kk)$ is the
  probability that the model $(\mathcal{M}(\cdot, \UU), \{\kk\})$ is
  ``statistically equivalent'' to the ``full model''
  $\left(\mathcal{M}(\cdot, \UU), \Kspace)\right)$, at a certain level
  linked to the value of $\beta$.~\Cref{fig:gamma_beta_increasing}
  shows the regions of acceptability for different $\beta$, and the
  associated $\Gamma_{\beta}$.

\begin{figure}[ht]
  \centering
  \input{\imgpath gamma_beta_increasing.pgf}
  \caption[Regions of $\beta$-acceptability]{\label{fig:gamma_beta_increasing} Boundaries of regions of acceptability for increasing $\beta$, and $\Gamma_{\beta}$. The coloured lines on the left plot are the boundaries of those regions}
\end{figure}


$\Gamma_{\beta}$ is then maximised with respect to its argument, in
order to get the value $\kk$ which has the highest probability of
being acceptable at the level $\beta$. For different $\beta \geq 0$,
we can define the family of additive-regret estimators.

  
  \begin{definition}[Additive-regret family of estimators]
    \label{def:AR_family}
    For $\beta \geq 0$, we define the family of robust estimators as
    the maximisers of \cref{eq:def_gamma_additive}:
    \begin{equation}
      \label{eq:def_AR_family}
      \left\{\estimtxt{\kk}{AR,\beta} = \argmax_{\kk\in\Kspace} \Gamma_{\beta}(\kk) \mid \beta > 0\right\} %= \Prob_U\left[R_{\beta}(\kk)\right]\mid \beta \geq 0\right\}
    \end{equation}
  \end{definition}
  Among this family of estimators, we can then choose a particular
  value, either by setting a threshold $\beta$ arbitrarily, or by
  choosing it so that the probability of being acceptable
  $\max \Gamma_{\beta}$ reaches a particular value. This will be
  discussed later in \cref{sec:choice_threshold}.


  \subsection{Relative-regret}
  \subsubsection{Absolute and relative error}
  \label{ssec:hyp_situation}
  We examined before regret that can be qualified as \emph{additive}
  as this is the difference between $J$ and $J^*$ that is compared to
  fixed thresholds.  However, we can argue that the relative magnitude
  of the objective function has an importance in the comparison. For
  illustration purposes, let us consider the situation
  described~\cref{tab:hyp_situation}, with
  $\Kspace = \{\kk_1, \kk_2\}$ and $\Uspace = \{\uu_1, \uu_2\}$, and
  $\Prob[\UU = \uu_1] =\Prob[\UU = \uu_2]= 1/2$.

\begin{table}[ht]
  \centering
  \begin{tabular}{rlrr}
    \toprule
    $J$                 & $\uu_1$ & $\uu_2$ & $\Ex[J(\cdot, \UU)]$ \\ \midrule
    $\kk_1$             & 10000   & 110     & 5055                 \\
    $\kk_2$             & 10100   & 10      & 5055                 \\ \bottomrule
  \end{tabular}
  \quad
  \begin{tabular}{rll}
    \toprule
    $J-J^*$             & $\uu_1$ & $\uu_2$                        \\ \midrule
    $\kk_1$             & 0       & 100                            \\
    $\kk_2$             & 100     & 0                              \\ \bottomrule
  \end{tabular}
    \quad
  \begin{tabular}{rll}
    \toprule
    $\frac{J-J^*}{J^*}$ & $\uu_1$ & $\uu_2$                        \\ \midrule
    $\kk_1$             & 0       & 10                             \\
    $\kk_2$             & 0.01    & 0                              \\ \bottomrule
  \end{tabular}  
  \caption[Objective function, expected loss, additive regret and
  relative error]{\label{tab:hyp_situation} Illustration of an
    objective function, expected loss, additive regret and relative
    error}
\end{table}
In this situation, for both $\uu_1$ and $\uu_2$, the maximal additive
regret is $\max_{\kk} J(\kk, \uu) - J^*(\uu) = 100$, so no clear
preference could be inferred toward one or another
value. % The situation $\UU=\uu_1$ seems less favorable than $\UU=\uu_2$, as the values of the objective function $J(\cdot, \uu_1)$ are larger than for all $\kk$.
However, choosing $\kk_1$ over $\kk_2$ means to choose to improve the
performance of an already pretty bad situation ($10000$ instead of
$10100$), while increasing tenfold the loss for the situation
$\UU=\uu_2$.


From the example developed~\cref{tab:hyp_situation}, an alternative to
the additive regret may be considered, as the difference in magnitude
of the objective function between $\UU = \uu_1$ and $\UU = \uu_2$ is
probably due to the effect of a misspecification. So to take into
account this difference, we are now going to consider the
\emph{relative regret} $J/J^*$, instead of the \emph{absolute regret}
$J - J^*$, and derive a family of estimators in a similar fashion.

\subsubsection{Relative-regret estimators family}
Analogously as the additive regret defined before, we are going first
to define the notions of acceptability in the case of the relative
regret.
\begin{definition}[$\alpha$-acceptability]
  Let $\alpha\geq 1$.  A point $(\kk, \uu)$ is said to be
  $\alpha$-acceptable if $J(\kk, \uu) \leq \alpha J^*(\uu)$.  We
  define the $\alpha$-acceptable interval as
  \begin{equation}
    \mathcal{I}_{\alpha}(\uu) = \left\{\kk \in \Kspace \mid J(\kk,\uu) \leq \alpha J^*(\uu) \right\}
  \end{equation}

  Then, for a given $\alpha$ and $\kk$, we can define the set of
  $\uu\in \Uspace$ such that $\kk$ is $\alpha$-acceptable:
\begin{equation}
  R_{\alpha}(\kk) = \left\{\uu \in \Uspace \mid \kk \in \mathcal{I}_{\alpha}(\uu) \right\} = \left\{\uu \in \Uspace \mid  J(\kk, \uu) \leq \alpha J^*(\uu) \right\}
\end{equation}
\end{definition}
$R_{\alpha}(\kk)$ is a measurable subset of $\Uspace$, and by integrating this set with respect to $\Prob_\UU$, we get
\begin{equation}
\Gamma_{\alpha}(\kk) = \Prob_{\UU}\left[R_\alpha(\kk)\right]
\end{equation}
the probability of being $\alpha$-acceptable. Using this function, we
can define an estimator as the value which maximises this
probability. And by varying the threshold $\alpha$, we get the family
of relative-regret estimators.


\begin{definition}[Relative-regret family of estimators]
  \label{def:RR_family}
  Given $\alpha$, the value of $\kk$ that maximises the probability of
  being $\alpha$-acceptable is called the relative-regret (RR)
  estimator ${\kk}_{\mathrm{RR},\alpha}$.
  
  We define the family of relative regret estimators as the set of
  those estimators:
\begin{equation}
  \label{eq:def_RR_family}
    \left\{\estimtxt{\kk}{RR,\alpha} = \argmax_{\kk\in\Kspace} \Gamma_{\alpha}(\kk)\mid \alpha > 1\right\}
    \end{equation}
    For different increasing $\alpha$, the corresponding regions of
    acceptability are represented in~\cref{fig:gamma_alpha_increasing},
    along with the functions $\Gamma_{\alpha}$.
  \end{definition}


  
  Among those estimators and the associated quantities, two limiting
  cases appear. One particular choice is to set $\alpha$ to $1$. In
  this case, we have
\begin{align}
  \mathcal{I}_{1}(\uu) &= \{\kk^*(\uu)\} \\
  R_1(\kk) &= \left\{\uu\in\Uspace \mid J(\kk, \uu) = J^*(\uu)\right\} = \left\{\uu\in\Uspace \mid \kk = \kk^*(\uu)\right\}
\end{align}

We have then that $\Gamma_1(\kk)$ is non-zero if the set $R_1(\kk)$
has non-zero measure with respect to $\Prob_{\UU}$. In other words,
$\Gamma_1(\kk)$ is non-zero if $\kk$ is the minimiser of
$J(\cdot,\uu)$ for a non-negligible subset of $\Uspace$.  If we
consider that $\Kspace$ is a discrete space (due to a discretization
for instance), $\KK^*(\UU)$ is a discrete random
variable. $\Gamma_1(\kk)$ is then the probability mass function
(discrete parallel of the pdf) of the discrete r.v.\ $\KK^*(\UU)$.  In
this case, $\estimtxt{\kk}{RR, \alpha=1}= \estimtxt{\kk}{MPE}$.  A
similar argument can be made for the additive regret and $\beta=0$.

We can see that the thresholds act like a relaxation of the optimality
condition, as for $\alpha=1$ and $\beta=0$, we are measuring the
probability of being optimal, while increasing those values means to
measure the probability of being nearly optimal.


Another choice is to set the threshold large enough so that the
probability of being acceptable reaches a unique maximum, being
$1$. In this situation, the regret is bounded almost surely.  Let
$\beta_{\inf{}}$ and $\alpha_{\inf{}}$, which verify
\begin{equation}
  \beta_{\inf{}}=\inf \left\{\beta \geq 0 \mid \max \Gamma_{\beta} = 1\right\} \text{ and }\alpha_{\inf{}}= \inf \left\{\alpha \geq 1 \mid \max \Gamma_{\alpha} = 1\right\}
\end{equation}
To put it differently, $\alpha_{\inf{}}$ and $\beta_{\inf{}}$ are the
smallest thresholds where there exists a value in $\Kspace$ acceptable
almost surely. This value shares similarities with the minimiser of
Savage's regret: $\estimtxt{\kk}{rWC}$
introduced~\cref{ssec:regret_savage}, as it minimises almost surely
(\emph{i.e.} for all $\uu$ in a non-negligible set) the regret, either
additive or relative.



% \begin{equation}
%   \Prob_{\UU}\left[J(\estimtxt{\kk}{AR, \beta_1}, \UU) - J^*(\UU) \leq \beta_1\right]=1 \quad \text{ and } \Prob_{\UU}\left[J(\estimtxt{\kk}{RR, \alpha_1}, \UU) \leq \alpha_1 J^*(\UU)\right]=1
% \end{equation}


\begin{figure}[ht]
  \centering
  \input{\imgpath gamma_alpha_increasing.pgf}
  \caption[Regions of
  $\alpha$-acceptability]{\label{fig:gamma_alpha_increasing} Regions
    of acceptability for relative regret and increasing $\alpha$. The
    coloured lines on the left plot are the boundaries of those
    regions}
\end{figure}

For both regret-based approaches, the interval of acceptability at a
fixed $\uu\in\Uspace$ grows with the threshold, and the sharper the minimum is
(in the sense of a large curvature), the faster it grows.  A more
telling illustration of the differences between relative and additive
regret is shown~\cref{fig:illustration_region_regret}.  The regions of
acceptability of an objective function $J$ have been plotted, along
with an interval $\mathcal{I}(\uu)$ and the region $R$ for a given
$\kk$ for both regrets.
\begin{figure}[ht]
  \centering
  % \input{\imgpath illustration_region_regret.pgf}
  \includegraphics{\imgpath illustration_region_regret.png}
  \caption{\label{fig:illustration_region_regret} Comparison of the regions of acceptability for additive and relative regret}
\end{figure}
For $\uu$ around $0$, $J$ is quite flat, but also has very low
values. In this case, the interval $\mathcal{I}_{\beta}(\uu)$ is also
large. But for $\uu$ around $1$, the objective function presents
higher values, and a sharper minimum (\emph{i.e.} a higher
curvature). Additive regret in this case puts stronger confidence on
the value of the parameter as indicated by the smaller interval of
acceptability.

For the relative regret, the situation is reversed. Although sharp,
the large value attained by the minimum $J^*(\uu)$, for $\uu$ around
$1$ leads to a large interval of acceptability, meaning that we can
deviate a bit more from this minimum, as the situation $\uu \approx 1$
is already pretty bad. For $\uu \approx 0$, which is close to the
global minimum of $J$, the interval is smaller, as this criterion does
not favour a large deviation from this global minimum.



\subsection{The choice of the threshold}
\label{sec:choice_threshold}
We are now going to focus exclusively on the relative-regret, and the
associated treshold $\alpha$, but similar arguments can be made for
the absolute regret and the threshold $\beta$.

In order to have an insight on the potential robustness of an
estimator $\estimtxt{\kk}{RR,\alpha}$ both values can be studied
together: the threshold $\alpha$ and the maximal probability reached
$\max \Gamma_\alpha$. Finding a relevant threshold can be a thorny
issue, especially with no further information on $J$. Setting it too
large will lead to large $\alpha$-acceptable intervals, and
$\Gamma_\alpha$ may reach $1$ for several different values. On the
other hand, choosing a threshold too small may give a maximal
acceptability probability too low to assess the robustness of the
chosen solution. We can contemplate three starting points:
 \begin{itemize}
 \item Set a relaxation parameter $\alpha > 1$. From this, the maximal
   probability is
   $p_{\alpha} = \max_{\kk\in\Kspace} \Gamma_{\alpha} =
   \max_{\kk\in\Kspace} \Prob_{\UU}\left[J(\kk, \UU) \leq \alpha
     J^*(\UU)\right]$. This task can be seen as the estimation and
   optimisation of a specific probability.

   
\item Set a probability $0< p\leq 1 $ we want to reach, and define
  $\alpha_p = \inf_{\alpha \geq 1}\{\max_{\kk \in
    \Kspace}\Gamma_{\alpha}(\kk)\geq p \}$, so the problem can be
  thought as the estimation and the optimisation of quantile of a
  particular random variable: we can define $\alpha_p$ as the solution
  of the following chance constrained problem:
  \begin{equation}
    \label{eq:opt_alpha_p_formulation}
  \left\{
  \begin{array}{rl}
    \min_{\kk\in\Kspace} &  q(\kk) \\
  \text{such that} & \Prob_{\UU}\left[r(\kk, \UU) \leq q(\kk) \right] \geq p
  \end{array}
  \right.
\end{equation}
where $r(\kk, \uu) = J(\kk, \uu)/J^*(\uu)$ or
$r(\kk, \uu)=J(\kk, \uu) - J^*(\uu)$ depending on the context, and
$q(\kk)$ is the quantile of order $p$ of the r.v. $r(\kk, \UU)$.



\item Study the evolution of one quantity with respect to the other,
  in order to find a balance between the probability of being
  acceptable, and the relaxation needed to reach it.
\end{itemize}

Specific techniques may be applied in order to perform the first two
approaches efficiently, which will be introduced in the next
chapter. The third one however requires the knowledge of the objective
function on the whole joint space $\Kspace \times \Uspace$, in order
to compute $\max \Gamma_{\alpha}$ for a various number of thresholds, thus may
not be adapted for costly computer simulations.


\section{Partial Conclusion}
\label{sec:ch3_partial_ccl}
As shown throughout this chapter, when optimising under uncertainties, a
lot of criteria can be defined in order to satisfy the idea of
\emph{robustness}, depending on the interpretation of this term. A
summary of those introduced here can be
found~\cref{tab:summary_robust}. Some criteria are commonly
encountered in optimisation under uncertainty, such as expected loss
minimisation. We introduced also new families of estimators, which aim
at maximising a probability based on the regret, either additive or
relative.
\begin{table}[ht]
  \centering
  \begin{tabular}{ll}
    \toprule
    Objective name                & Objective to \emph{minimise} with respect to  $\kk$                                                                                 \\ \midrule
    Profile Likelihood            & $-\log \left(\max_{\uu \in \Uspace} p_{Y \mid \KK, \UU}(y \mid \kk, \uu)\right)$                                       \\
    Integrated Likelihood         & $-\log \int_{\Uspace} p_{Y \mid \KK, \UU}(y \mid \kk, \uu)p_{\UU}(\uu) \,\mathrm{d}\uu=-\log p_{Y\mid \KK}(y\mid \kk)$ \\
    Marginal maximum a posteriori & $-\log p_{\KK\mid Y}(\kk \mid y)$                                                                                      \\ \midrule
    Global Optimum                & $\min_{\uu\in\Uspace} J(\kk, \uu)$                                                                                     \\
    Worst-case                    & $\max_{\uu \in \Uspace} J(\kk, \uu)$                                                                                   \\
    Regret worst-case             & $\max_{\uu\in\Uspace}\left\{J(\kk, \uu) - \min_{\kk^{\prime}\in\Kspace} J(\kk^{\prime}, \uu)\right\}$                  \\ \midrule
    Mean                          & $\Ex_{\UU}[J(\kk, \UU)]$                                                                                               \\
    Mean and variance             & $ \lambda \Ex_{\UU}[J(\kk, \UU)] + (1-\lambda) \sqrt{\Var_{\UU}[J(\kk, \UU)]}$                                         \\ \midrule
    Most Probable Estimate        & $-\log p_{\kk^*}(\kk)$                                                                                                 \\
    Additive-regret               & $-\Prob_{\UU}\left[J(\kk, \UU) \leq J^*(\UU) + \beta \right]=-\Gamma_{\beta}(\kk)$                                     \\
    Relative-regret               & $-\Prob_{\UU}\left[J(\kk, \UU) \leq \alpha J^*(\UU) \right]=-\Gamma_{\alpha}(\kk) $                                    \\ \bottomrule
  \end{tabular}
  \caption{\label{tab:summary_robust} Summary of single objective robust estimators}
\end{table}
 
Obviously, other criteria can be defined, that satisfy other
robustness requirements.  Furthermore, we did not treat the
possibility of combining some of those objectives by using them to set
constraints. An example is the minimisation of the variance, under the
constraint that the mean value does not exceed a certain threshold $T$
as in \cite{lehman_designing_2004}:
\begin{align*}
  \min \Var_{\UU}&\left[J(\kk, \UU)\right]  \\
  \text{s.t. } \Ex_{\UU}&\left[J(\kk, \UU)\right]   \leq T
\end{align*}

All of the criteria introduced above require costly numerical
procedures, such as integration and optimisation. Solving these robust
estimation problems is then expensive in term of computer resources,
as one would need to run the forward model a very large number of
times in order to get accurate numerical integration or
optimisation. In the next chapter we will discuss methods based on
surrogate modelling, that can be used to solve efficiently such
problems, in order to make the best of the evaluations of the
numerical model $\mathcal{M}$ on the space $\Kspace \times\Uspace$.

\markchapterend

\subfileLocal{
	\pagestyle{empty}
	\bibliographystyle{alpha}
	\bibliography{/home/victor/acadwriting/bibzotero}
      }
\end{document}


%%% Local Variables:
%%% mode: latex
%%% TeX-master: "../../Main_ManuscritThese"
%%% End:
