\documentclass[10pt]{beamer}
\usepackage[T1]{fontenc}
\usepackage[utf8]{inputenc}
\usepackage[english]{babel}
\usepackage{bm}
\usepackage{pgfplots}
\pgfplotsset{compat=newest}
\usepackage{booktabs}
%\usepackage{siunitx}
\newcommand{\Ex}{\mathbb{E}}
\newcommand{\Var}{\mathbb{V}\mathrm{ar}}
\newcommand{\Prob}{\mathbb{P}}
% Latin Modern
%\usepackage{lmodern}
% Verdana font type
%\usepackage{verdana}
% Helvetica
%\usepackage{helvet}
% Times (text and math)
%\usepackage{newtx, newtxmath}
\setlength{\leftmargini}{35pt}
%\usetheme[department=compute]{DTU}%

\title[Robust estimation of bottom friction]{Robust Estimation of bottom friction \\ \textit{Parameter control in the presence of uncertainties}}
\author{\textsc{Victor Trappler}}
\institute{Master Thesis Defence}
\date{\today}
	
\setcounter{tocdepth}{1}

\begin{document}
\frame{
\frametitle{Victor Trappler \\ \small{1ère année de thèse}}
\begin{itemize}
 \item[Formation :] École Centrale Lyon - DTU : MSc Mathematical Modelling and Computation \\ \,
 \item[Sujet :] \emph{Contrôle de paramètres en présence d'incertitudes} (sous la supervision de \textbf{Élise Arnaud}, \textbf{Laurent Debreu} et \textbf{Arthur Vidard}).\\ Allocation ministérielle avancée. \\ \,
Estimer un paramètre de friction de fond sur un modèle de shallow water 1D $\rightarrow$ Minimiser une fonction objective \\ En présence d'incertitudes, adopter une \textbf{approche robuste}.\\ \,
 \item[Mots-clés :] Shallow water 1D -- Indices de Sobol' -- Méta-modélisation -- Optimisation robuste
 \end{itemize}
}
\end{document}