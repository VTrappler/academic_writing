\message{ !name(slides_laurent_15_06.tex)}\documentclass[11pt]{beamer}
\usetheme{metropolis}
\usepackage[utf8]{inputenc}
\usepackage[english]{babel}
\usepackage[T1]{fontenc}
\usepackage{amsmath}
\usepackage{amsfonts}
\usepackage{amssymb}
\usepackage{bm}
\usepackage{tikz}
\usetikzlibrary{tikzmark,decorations.pathreplacing,calligraphy}
\usepackage{subfig}
\usepackage{pgfplots}
\pgfplotsset{compat=newest}
\usepackage{multimedia}
%\usepackage{booktabs}
\newcommand{\Ex}{\mathbb{E}}
\newcommand{\Var}{\mathbb{V}\mathrm{ar}}
\newcommand{\Prob}{\mathbb{P}}
\DeclareMathOperator*{\argmin}{arg\,min}
\DeclareMathOperator*{\argmax}{arg\,max}
\newcommand{\Cov}{\textsf{Cov}}
\newcommand{\tra}{\mathrm{tr}}
\newcommand{\yobs}{\bm{y}^{\mathrm{obs}}}
\newcommand{\kest}{\hat{\bm{k}}}
\usetikzlibrary{positioning}
\DeclareMathOperator*{\KL}{\textsf{KL}}
\graphicspath{{../Figures/}}

\definecolor{blkcol}{HTML}{E1E1EA}
\definecolor{blkcol2}{RGB}{209, 224, 224}

% \definecolor{BlueTOL}{HTML}{222255}
% \definecolor{BrownTOL}{HTML}{666633}
% \definecolor{GreenTOL}{HTML}{225522}
% \setbeamercolor{normal text}{fg=BlueTOL,bg=white}
% \setbeamercolor{alerted text}{fg=BrownTOL}
% \setbeamercolor{example text}{fg=GreenTOL}
\setbeamercolor{block title}{bg = blkcol}
\setbeamercolor{block body}{bg = blkcol!50}


\title{Parameter control in the presence of uncertainties}
\author{{\large Victor Trappler } \\ Supervisors: Élise Arnaud, Laurent Debreu, Arthur Vidard}
\institute{AIRSEA Research team (Inria)-- Laboratoire Jean Kuntzmann \\
 \begin{center}
\includegraphics[scale=0.20]{INRIA_SCIENTIFIQUE_UK_CMJN}
\includegraphics[scale=0.20]{ljk}
\end{center}
}

\date{\today}
\setcounter{tocdepth}{1}


\begin{document}

\message{ !name(slides_laurent_15_06.tex) !offset(59) }
\section{Estimation sous incertitudes}
\frame[t]{
\frametitle{Incertitudes dans un code déterministe}
$\bm{U}$ est maintenant une variable aléatoire (densité $\pi(\bm{u})$) \\ $\yobs$ a été générée avec $\bm{u}_{\mathrm{ref}}$, échantillon de $\bm{U}$
\vfill
% \only<1>{\usetikzlibrary{positioning}
% \tikzstyle{block} = [rectangle, draw, fill=blue!30, 
%     text centered, minimum width=3em] 
 \tikzstyle{block} = [rectangle, draw, fill=blkcol, 
      text centered, minimum width=3em]

\tikzstyle{block2} = [rectangle, draw, fill=blkcol2, 
     text centered, rounded corners, minimum width=3em]

% \tikzstyle{block2} = [rectangle, draw, fill=blkcol2, 
%      text centered, rounded corners, minimum width=3em]

\tikzstyle{LHS}=[rectangle, draw, text centered]

\begin{tikzpicture}

%\node [align = center] at (0,0) (input) {Control variable \\$\mathbf{k} \in \Kspace$};
\node [align = center] at (0,0) (input) {Control variable \\$\theta \in \Kspace$};
\node[block] at (4,0) (code){Direct Simulation};
%\node [align = center, above =of  code ] (envir) {Environmental variables \\$\mathbf{u} \in \Uspace$ fixed};
\node [align = center] at (4,1.5) (envir) {Environmental variables \\$\uu \in \Uspace$ fixed};




\node[align = center] at (8,0) (output) {$\mathcal{M}(\theta,\uu)$};
\node [align = center] at (8,-1) (obs) {$\yobs$};
\node[block] at (4,-1) (inv) {Inverse Problem};

\draw[->] (input) -- (code);
\draw[->] (envir) -- (code);
\draw[->] (code) -- (output);

 % \node [align = center] at (0,0) (input) {$Y = \mathbb{H}M(K_{\mathrm{ref}})$};
 % \node [align = center] at (4,1.5) (envir) {Environmental variables \\$X_e$ r.v.};

 % \node[block] at (4,0)(code){"Inverse Problem"};

% \node[align = center] at (8,0) (output) {$K$};

\draw[->] (input) -- (code);
% \draw[->] (envir) -- (code);
\draw[->] (code) -- (output);
\draw[->] (output) -- (obs) ;
\draw[->] (inv) -|(input) ;
\draw[->] (obs) -- (inv);
\end{tikzpicture}}
\only<1>{\input{../Figures/comp_code_unc_inv_noalert.pgf}}

\begin{itemize}
\item $M(\bm{k}) \quad \text{devient} \quad M(\bm{k},\bm{u})$ ($\bm{u}$ est une entrée du modèle)
\item Fonction coût: $J(\bm{k},\bm{u}) =  \frac12\|M(\bm{k},\bm{u}) - \yobs\|^2$ + \text{Régul}
\end{itemize}
}
\begin{frame}
  \frametitle{Estimateur robuste ?}
  On veut pouvoir trouver une valeur $\kest$, tel que $M(\kest,\bm{U})$ soit relativement semblable à $\yobs$

  

  Idéalement, $M(\kest,\cdot)$
  \begin{itemize}
  \item reste assez performant pour fournir des prédictions acceptables
  \item ne varie pas trop avec $\bm{U}$
  \end{itemize}
\end{frame}
\frame{
\frametitle{Approche variationnelle ou Bayésienne ?}
\begin{itemize}
\item<1-> \textbf{Variationnelle}: Variable aléatoire indexée par $\bm{k}$: $\bm{k} \mapsto J(\bm{k},\bm{U})$, \\ Extrema des moments ? $\Ex[J(\bm{K},\bm{U})|\bm{K} = \bm{k}] \dots$
\item<1-> \textbf{Bayésienne}:  $e^{-J(\bm{k},\bm{u})} \propto p(\yobs|\bm{k},\bm{u})$= Vraisemblance  \\ Inférence bayésienne, marginalisation, Estimation bayésienne
\end{itemize}
\onslide<1-> Mais
\begin{itemize}
\item Estimer efficacement moments ?
\item Quelle connaissance de $\bm{U}$ ?
\item Gérer le coût de calcul du modèle
\end{itemize}
}
%\frame{
%\frametitle{Issues raised}
%%Random variable : $J(\bm{X}_e,K)$
%\begin{tabular}{ll}
%Influence of $\bm{X}_e$ ? & \onslide<2->{ $\longrightarrow$ Sensitivity analysis}\\
%Computational cost ? & \onslide<2->{ $\longrightarrow$ Use of surrogate}
%\end{tabular}
%}



\message{ !name(slides_laurent_15_06.tex) !offset(266) }

\end{document}